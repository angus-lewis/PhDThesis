%!TEX root = ../thesis.tex
\chapter{Properties of closing operators}\label{appendix: sec: 2}
This appendix is dedicated to proving that the closing operators in (\ref{eqn: density approx aug}), (\ref{eqn: density approx 3+}), and (\ref{eqn: density approx 4+}) have the Properties~\ref{properties: some props}, which we recall below, for convenience.

\renewcommand{\thedefn}{\arabic{chapter}.\arabic{defn}}
\addtocounter{defn}{1}
\setcounter{chapter}{4}
\begin{property}\label{properties: some props2}
	Let \(\{\bs v^{(p)}(x)\}_{p\geq 1}\) be a sequence of closing operators such that they may be decomposed into \(\bs v^{(p)}(x)=\bs w^{(p)}(x) + \widetilde{\bs w}^{(p)}(x)\), where; \\
	\subproperty \label{properties: -12} for \(x\in[0,\Delta),u,v\geq 0\),  
        \begin{align*}
        		\bs \alpha^{(p)} e^{\bs S^{(p)}(u+v)}(-\bs S^{(p)})^{-1} \widetilde{\bs w}^{(p)}(x) &\leq \bs \alpha^{(p)} e^{\bs S^{(p)}u}(-\bs S^{(p)})^{-1} \widetilde{\bs w}^{(p)}(x).
		\end{align*}
	\subproperty \label{properties: 02} for \(x\in[0,\Delta),u\geq 0\),
		\begin{align*}
			\bs \alpha^{(p)} e^{\bs S^{(p)}u}(-\bs S^{(p)})^{-1} \widetilde{\bs w}^{(p)}(x) &=\widetilde G_{\bs v}^{(p)} \to 0,\, \mbox{ as }p \to \infty.  
		\end{align*}
	\subproperty \label{properties: 12} for \(x\in[0,\Delta),u\geq 0\),  
        \begin{align*}
        		\bs \alpha^{(p)} e^{\bs S^{(p)}u}(-\bs S^{(p)})^{-1} \bs w^{(p)}(x) &\leq \bs \alpha^{(p)} e^{\bs S^{(p)}u} \bs e G_{\bs v},
	\end{align*}
	for some \(0\leq G_{\bs v}<\infty\) independent of \(p\) for \(p>p_0\) where \(p_0<\infty\). \\
	\subproperty \label{properties: -22} for \(\bs a \in\mathcal A,\,u\geq 0\),  
	\begin{align*}
			\int_{x\in[0,\Delta)}\bs a^{(p)} e^{\bs S^{(p)}u} {\bs v}^{(p)}(x) \wrt x&\leq \bs a^{(p)} e^{\bs S^{(p)}u} \bs e.
	\end{align*}
	\subproperty \label{properties: 22} Let \(g\) be a function satisfying the Assumptions~\ref{asu: g}. For \(u\leq \Delta-\varepsilon^{(p)}\), \(v\in[0,\Delta)\), then
	\[\left|\int_{x=0}^\infty \cfrac{\bs \alpha^{(p)} e^{\bs{S}^{(p)}(u+x)} }{\bs \alpha^{(p)} e^{\bs{S}^{(p)}u} \bs e} {\bs v}^{(p)}(v)g(x)\wrt x -g(\Delta-u-v) 1(u+v\leq\Delta-\varepsilon^{(p)})\right| =  |r_{\bs v}^{(p)}(u,v)|,\]
	where 
	\[ \int_{u=0}^{\Delta}\left| r_{\bs v}^{(p)}(u,v)\right| \wrt u  \leq R_{{\bs v},1}^{(p)} \to 0\]
	and 
	\[ \int_{v=0}^{\Delta}\left| r_{\bs v}^{(p)}(u,v)\right| \wrt v  \leq R_{{\bs v},2}^{(p)} \to 0\]
	as \(\var(Z^{(p)})\to 0\). 
\end{property} 
\addtocounter{defn}{-2}
\setcounter{chapter}{2}
\renewcommand{\thedefn}{\Alph{chapter}.\arabic{defn}}
% yuck 

\section{The closing operator \(\bs v(x) = e^{\bs S x}\bs s\)}
For the closing operator \(\bs v(x) = e^{\bs Sx}\bs s\) we may set \(\widetilde{\bs w}(x) = \bs 0\), so Properties~\ref{properties: -1} and \ref{properties: 0} hold trivially.

\begin{lem}\label{lem: akc}
	The closing operator \(\bs v(x) = e^{\bs Sx}\bs s\) has Property~\ref{properties: 1}.
	
	For any valid orbit, \(\bs a\in\mathcal A\), \(x,u\geq 0\), 
		\begin{align*}
			\int_{x_n=0}^\infty \bs a e^{\bs{S}(x_n+u)}\bs s = \bs a e^{\bs{S}(u)}\bs e. 
		\end{align*}
	\end{lem}
	\begin{proof}
		For any valid orbit, \(\bs a\in\mathcal A\), 
			\begin{align*}
					\bs a e^{\bs{S}(x+u)}\bs e &= \mathbb P(Z>x+u) \leq\mathbb P(Z>u) = \bs a e^{\bs Su} \bs e. 
		\end{align*}
	\end{proof}

\begin{cor}
	The closing operator \(\bs v(x) = e^{\bs S x}\bs s\) has Property~\ref{properties: -2}.
	
	For \(\bs a\in\mathcal A,\, u\geq 0, \)
	\[\int_{x=0}^\Delta \bs a e^{\bs Su}e^{\bs Sx}\bs v(x)\wrt x = \int_{x=0}^\Delta \bs a e^{\bs Su}e^{\bs Sx}\bs s\wrt x \leq \bs a e^{\bs Su}\bs e.\]	
\end{cor}
\begin{proof}
	\[\int_{x=0}^\Delta \bs a e^{\bs Su}e^{\bs Sx}\bs s\wrt x = \bs a e^{\bs Su}\bs e-\bs a e^{\bs S(u+\Delta)}\bs e \leq \bs a e^{\bs Su}\bs e,\]
	since \(0\leq \bs a e^{\bs S(u+\Delta)}\bs e\) as it is a probability. 
\end{proof}

\begin{cor}\label{cor: cond bnd 2 V}
	The closing operator \(\bs v(x)=e^{\bs S x}\bs s\) has Property~\ref{properties: 2}.

	Let \(g\) be a function satisfying the Assumptions \ref{asu: g} and consider the closing operator \({\bs v}(x)=e^{\bs Sx}\bs s\). For \(u\leq \Delta-\varepsilon \), \(v\geq 0\), 
	\[\int_{x=0}^\infty \cfrac{\bs \alpha  e^{\bs{S} (u+x)} }{\bs \alpha  e^{\bs{S} u} \bs e} {\bs v}(v)g(x)\wrt x = g(\Delta-u-v) 1(u+v\leq\Delta-\varepsilon) + r_{\bs v} (u,v),\]
	where \[\displaystyle R_{\bs v,1} = \int_{u=0}^{\Delta-\varepsilon}|r_{\bs v}(u,v)|\wrt u \leq r_2\Delta + 2\varepsilon G + \Delta G \cfrac{\var(Z)/\varepsilon^2}{1-\var(Z)/\varepsilon^2}\]
	and \[R_{\bs v,2} = \int_{u=0}^\Delta |r_{\bs v}(u,v)| \wrt u\leq R_{{\bs v},1}\leq r_2\Delta + 2\varepsilon G + \Delta G \cfrac{\var(Z)/\varepsilon^2}{1-\var(Z)/\varepsilon^2}.\] 
\end{cor}
\begin{proof}
	By Corollary~\ref{cor: cond bnd 2}, 
	\[\int_{x=0}^\infty \cfrac{\bs \alpha  e^{\bs{S} (u+x)} }{\bs \alpha  e^{\bs{S} u} \bs e} {\bs v}(v)g(x)\wrt x = g(\Delta-u-v) 1(u+v\leq\Delta-\varepsilon) + r_3 (u+v),\]
	so \(r_{\bs v}(u,v)=r_3(u+v)\). All that remains to be shown are the bounds. To this end, observe 
	\begin{align*}
		R_{{\bs v},1} \leq \displaystyle\int_{u=0}^{\Delta}|r_{\bs v}(u,v)|\wrt u 
		 &= \displaystyle\int_{u=0}^{\Delta} r_3(u+v)\wrt u 
		 %
%		 \\&\leq \displaystyle\int_{u=0}^{2\Delta} r_3(u+v)\wrt u
		 %
		 \leq r_2\Delta + 2\varepsilon G + \Delta G \cfrac{\var(Z)/\varepsilon^2}{1-\var(Z)/\varepsilon^2},
	\end{align*}
	since \(r_{3}(u,v)\geq 0\) for all \(u,v\geq 0\). Similarly,
	\begin{align*}
		 R_{{\bs v},2} \leq \displaystyle\int_{v=0}^{\Delta}r_{\bs v}(u,v)\wrt v 
		 &= \displaystyle\int_{v=0}^{\Delta} r_3(u+v)\wrt v 
		 %
%		 \\&\leq \displaystyle\int_{u=0}^{2\Delta} r_3(u+v)\wrt u
		 %
		 \leq r_2\Delta + 2\varepsilon G + \Delta G \cfrac{\var(Z)/\varepsilon^2}{1-\var(Z)/\varepsilon^2}.
	\end{align*}
\end{proof}

% \begin{lem}\label{lem:macmnm}
% 	For any valid orbit, \(\bs a\in\mathcal A\), \(x\geq 0\), 
%         \begin{align*}
%         		\bs a\bs D e^{\bs{S}x}\bs e &\leq 1. 
% 	\end{align*}
% \end{lem}
% \begin{proof}
% By the definition of \(\bs D\) and Lemma~\ref{lem: akc},
% 	\begin{align*}
%         		\bs a\bs D e^{\bs{S}x}\bs e &= \bs a \int_{u=0}^\infty e^{\bs Su}\bs s\cfrac{\bs \alpha e^{\bs Su}}{\bs \alpha e^{\bs Su}\bs e}\wrt ue^{\bs{S}x}\bs e
% 		%
% 		\\& \leq \bs a \int_{u=0}^\infty e^{\bs Su}\bs s\cfrac{\bs \alpha e^{\bs Su}\bs e}{\bs \alpha e^{\bs Su}\bs e}\wrt u
% 		%
% 		\\& = \bs a \int_{u=0}^\infty e^{\bs Su}\bs s\wrt u
% 		%
% 		\\& = \bs a \bs e = 1.
% 	\end{align*}
% \end{proof}

\section{The closing operator \(\widehat{\bs v}(x) =  \left(e^{\bs{S} x}  + e^{\bs{S} (2\Delta-x)}  \right)\bs s\)}

Let \(\widehat{\bs v}(x)\) be the closing operator, 
\[  \widehat{\bs v}(x) =   \left(e^{\bs{S} x}\bs s  + e^{\bs{S} (2\Delta-x)}\bs s\right),\]
for \(x\in[0,\Delta)\).

For the closing operator \(\widehat{\bs v}(x)\) we may set \(\widetilde{\bs w}(x) = \bs 0\), so Properties~\ref{properties: -1} and \ref{properties: 0} hold trivially.

\begin{lem}\label{lem: akxnj}
	The closing operator \(\widehat{\bs v}(x)\) has the Property~\ref{properties: 1}.

	For \(x\in[0,\Delta),u\geq 0\),  
        \begin{align*}
        		\bs a   e^{\bs Su}(-\bs S)^{-1} {\bs u}(x) &\leq 2 \bs a e^{\bs Su} \bs e.
	\end{align*}
\end{lem}
\begin{proof}
Let \(\bs a   \in \mathcal A\) be arbitrary. By definition 
	\begin{align*}
        		\bs a  e^{\bs Su}(-\bs S)^{-1} {\bs u}(x) & = \bs a  e^{\bs Su}(-\bs S)^{-1}  \left(e^{\bs{S}x}\bs s + e^{\bs{S}(2\Delta-x)}\bs s\right)
				\\& = \bs a  e^{\bs Su}  \left(e^{\bs{S}x}\bs e + e^{\bs{S}(2\Delta-x)}\bs e\right)
	\end{align*}
	since \((-\bs S)^{-1}\) and \(e^{\bs Sx}\) commute and \(\bs s = -\bs S e\). 
	By Lemma~\ref{lem: akc} this is less than or equal to, 
	\begin{align}
        		& \bs a   e^{\bs Su} \left(\bs e + \bs e\right) = 2 \bs a   e^{\bs Su} \bs e. \label{eqn:mzm}
	\end{align}
\end{proof}

\begin{lem}
	The closing operator \(\widehat{\bs v}(x)\) has the property \ref{properties: -2}. 
	
	For \(\bs a\in\mathcal A\), \(u\geq 0\), 
	\[\int_{x=0}^\Delta \bs a e^{\bs Su}\widehat{\bs v}(x)\wrt x\leq\bs ae^{\bs Su}\bs e\]
\end{lem}
\begin{proof}
	\[\int_{x=0}^\Delta \bs a e^{\bs Su}\widehat{\bs v}(x)\wrt x=\bs ae^{\bs Su}\bs e-\bs ae^{\bs S(u+2\Delta)}\bs e\leq \bs ae^{\bs Su}\bs e.\]
\end{proof}

\begin{cor}\label{cor: cond bnd 2 U}
	The closing operator \(\widehat{\bs v}(x)\) has the Property~\ref{properties: 2}.

	Let \(g\) be a function satisfying the Assumptions \ref{asu: g}. For \(u\leq \Delta-\varepsilon \), \(v\in[ 0,\Delta)\), 
	\[\int_{x=0}^\infty \cfrac{\bs \alpha  e^{\bs{S} (u+x)} }{\bs \alpha  e^{\bs{S} u} \bs e} {\widehat{\bs v}}(v)g(x)\wrt x = g(\Delta-u-v) 1(u+v\leq\Delta-\varepsilon) + r_{\widehat{\bs v}} (u,v),\]
	where 
	\[\left|r_{\widehat{\bs v}} (u,v)\right|\leq r_3 (u+v) + r_3 (u+2\Delta - v).\]
	Furthermore,  
	\begin{align*}
		&\int_{u=0}^{\Delta}| r_{\widehat{\bs v}}(u,v)|\wrt u
		\leq R_{{\widehat{\bs v}},1},
	\end{align*}
	and
	\begin{align*}
		&\int_{v=0}^{\Delta}| r_{\widehat{\bs v}}(u,v)|\wrt u
		\leq R_{{\widehat{\bs v}},2},
	\end{align*}
	where 
	\[R_{{\widehat{\bs v}},1},\, R_{{\widehat{\bs v}},2} \leq 2\left(\Delta r_2 + 2\varepsilon G + \Delta\cfrac{\var(Z)/\varepsilon^2}{1-\var(Z)/\varepsilon^2}\right).\]
\end{cor}
\begin{proof}
	By the definition of the operator \(\widehat{\bs v}(x)\), 
	\begin{align}
		\int_{x=0}^\infty \cfrac{\bs \alpha  e^{\bs{S} (u+x)} }{\bs \alpha  e^{\bs{S} u} \bs e} \widehat{\bs v}(v)g(x)\wrt x 		
		%
		&=
			\displaystyle\int_{x=0}^\infty 
				\cfrac{
					\bs \alpha e^{\bs S(u+x)}
					}{
					\bs \alpha e^{Su}\bs e
					} 
				e^{\bs{S}v}\bs s g(x)
				+
				\cfrac{
					\bs \alpha e^{\bs S(u+x)}
					}{
					\bs \alpha e^{Su}\bs e
					} 
				e^{\bs{S}(2\Delta-v)}\bs sg(x) \wrt x. \label{eqn: ghi is this a}
	\end{align}
	By Corollary~\ref{cor: cond bnd 2} 
	\begin{align}
		\int_{x=0}^\infty 
				\cfrac{
					\bs \alpha e^{\bs S(u+x)}
					}{
					\bs \alpha e^{Su}\bs e
					} 
				e^{\bs{S}v}\bs s g(x)\wrt x 
				&= g(\Delta-u-v) 1(u+v\leq\Delta-\varepsilon) + r_3 (u+v), \label{eqn: dkskkk2}
		\\
		\int_{x=0}^\infty\cfrac{
					\bs \alpha e^{\bs S(u+x)}
					}{
					\bs \alpha e^{Su}\bs e
					} 
				e^{\bs{S}(2\Delta-v)}\bs sg(x) \wrt x 
				&= r_3 (u+2\Delta - v).
	\end{align}
	Therefore, (\ref{eqn: ghi is this a}) is, 
	\begin{align}
		&g(\Delta-u-v) 1(u+v\leq\Delta-\varepsilon) + r_3 (u+v) + r_3 (u+2\Delta - v).
	\end{align}	
	
	Now,
	\begin{align*}
		R_{{{\bs u}},1}&\leq \int_{u=0}^{\Delta}| r_{{\bs u}}(u,v)|\wrt u
		\\& \leq \int_{u=0}^{\Delta} r_3 (u+v) + r_3 (u+2\Delta - v) \wrt u
		\\& \leq \int_{u=0}^{2\Delta} 2r_3 (u+v)\wrt u
		\\&\leq 2\left(\Delta r_2 + 2\varepsilon G + \Delta\cfrac{\var(Z)/\varepsilon^2}{1-\var(Z)/\varepsilon^2}\right).
	\end{align*}
	Similarly, 
	\begin{align*}
		R_{{{\bs u}},2}&\leq \int_{v=0}^{\Delta}| r_{{\bs u}}(u,v)|\wrt v
		\\&= \int_{v=0}^{\Delta} r_3(u+v) + r_3(u+2\Delta-v) 
		\\& \leq 2\left(\Delta r_2 + 2\varepsilon G + \Delta\cfrac{\var(Z)/\varepsilon^2}{1-\var(Z)/\varepsilon^2}\right).
	\end{align*}
\end{proof}

% The error term \(r_{{\bs u}}(u,v)\) depends on \(p\) so we should write \(r_{{\bs u}}^{(p)}(u,v)\). The error term \(r_{{\bs u}}^{(p)}(u,v)\) has similar properties to \(r_3^{(p)}(u+v)\); we can prove that it converges point-wise to \(0\) only on some areas of its domain, however when we integrate the error against bounded functions on bounded domains, then the resulting integral tends to \(0\). 

\section{The closing operator \(\left(e^{\bs{S} x} + e^{\bs{S} (2\Delta-x)}\right)\left[I-e^{\bs S  2\Delta}\right]^{-1}\bs s \)}

Let \(\overline{\bs v} (x)\) be the closing operator 
\[\overline{\bs v}(x) = \left(e^{\bs{S}x} + e^{\bs{S}(2\Delta-x)}\right)\left[I-e^{\bs S 2\Delta}\right]^{-1}\bs s,\]
for \(x\in[0,\Delta)\).
Notice that 
\[\bs a \overline{\bs v}(x) = \bs a \left(e^{\bs{S}x} + e^{\bs{S}(2\Delta-x)}\right)\sum_{n=0}^\infty e^{\bs S2n\Delta}\bs s.\]
We decompose the closing operator \(\displaystyle \overline{\bs v}(x)=\bs w(x)+\widetilde{\bs w}(x),\) where \(\bs w(x) = \widehat{\bs v}(x)\) and 
\[\widetilde{\bs w}(x) = \left(e^{\bs{S}x} + e^{\bs{S}(2\Delta-x)}\right)\sum_{n=1}^\infty e^{\bs S2n\Delta}\bs s.\]

\begin{lem}
	The closing operator \(\overline{\bs v}(x)\) has Property~\ref{properties: -1}.
	
	For \(\bs a \in\mathcal A\), \(u\geq 0\), 
	\[\bs a e^{\bs S(u+v)}(-\bs S)^{-1}\overline{\bs v}(x) \leq \bs a e^{\bs Su}(-\bs S)^{-1}\overline{\bs v}(x) . \]
\end{lem}
\begin{proof}
	\begin{align*}
		 \bs a e^{\bs S(u+v)}(-\bs S)^{-1}\overline{\bs v}(x)\wrt x &= \sum_{n=0}^\infty\bs a e^{\bs S(x+u+v+2n\Delta)}\bs e + \bs a  e^{\bs S(2\Delta-x + u + v + 2n\Delta)} \bs e
		\\ &\leq \sum_{n=0}^\infty \bs a e^{\bs S(x+u+2n\Delta)}\bs e + \bs a  e^{\bs S(2\Delta-x + u + 2n\Delta)} \bs e
		\\ &= \bs a e^{\bs Su}(-\bs S)^{-1}\overline{\bs v}(x)\wrt x.
	\end{align*}
\end{proof}

\begin{lem}\label{lem: akxnj2}
	The closing operator \(\overline{\bs v}(x)\) has Property~\ref{properties: 0}.

	For \(x\in[0,\Delta),u\geq 0\),  
        \begin{align}
        		\bs \alpha   e^{\bs Su}(-\bs S)^{-1} \widetilde{\bs w}(x) &\leq \cfrac{\var(Z)}{\Delta^2}\cfrac{\pi^2}{4}.\label{eqn: weha cjcweyugeiow}
	\end{align}
\end{lem}
\begin{proof}
	The expression on the left-hand side of (\ref{eqn: weha cjcweyugeiow}) is 
	\begin{align*}
		&\bs \alpha e^{\bs Su}(-\bs S)^{-1} \left(e^{\bs{S} x} + e^{\bs{S} (2\Delta-x)}\right)\sum_{n=1}^\infty e^{\bs S  2n\Delta} \bs s
		\\&= \bs \alpha e^{\bs Su} \left(e^{\bs{S} x} + e^{\bs{S} (2\Delta-x)}\right)\sum_{n=1}^\infty e^{\bs S  2n\Delta} \bs e
		\\&=\sum_{n=1}^\infty \mathbb P(Z>u+x+2n\Delta) + \mathbb P(Z>u+2\Delta-x+2n\Delta)
		\\&\leq 2 \sum_{n=1}^\infty \mathbb P(Z>2n\Delta).
	\end{align*}
	By Chebyshev's inequality, \(\mathbb P(Z>2n\Delta)\leq \cfrac{\var(Z)}{\Delta^2(1+2(n-1))^2}\). Therefore
	\begin{align}
		2 \sum_{n=1}^\infty \mathbb P(Z>2n\Delta)
		& \leq 2 \cfrac{\var(Z)}{\Delta^2} \sum_{n=0}^\infty \cfrac{1}{(1+2n)^2}.
	\end{align}
	Now, consider the sum 
	\begin{align}
		\sum_{n=1}^\infty \cfrac{1}{n^2} = \sum_{n=1}^\infty \cfrac{1}{(2n)^2} + \sum_{n=0}^\infty \cfrac{1}{(1+2n)^2} = \cfrac{1}{4}\sum_{n=1}^\infty \cfrac{1}{n^2} + \sum_{n=0}^\infty \cfrac{1}{(1+2n)^2}.
	\end{align}
	The solution to the \emph{Basel problem} states that \(\displaystyle\sum_{n=1}^\infty 1/n^2=\pi^2/6.\) Hence 
	\[\cfrac{\pi^2}{6} = \cfrac{1}{4}\cfrac{\pi^2}{6} + \sum_{n=0}^\infty \cfrac{1}{(1+2n)^2}\]
	and therefore 
	\[\sum_{n=0}^\infty \cfrac{1}{(1+2n)^2} = \cfrac{\pi^2}{8}.\]
	Thus, (\ref{eqn: weha cjcweyugeiow}) is less than or equal to 
	\[\cfrac{\var(Z)}{\Delta^2}\cfrac{\pi^2}{4}.\]
\end{proof}

Since \(\bs w(x)=\widehat{\bs v}(x)\) then, from the results of the previous section, \(\overline{\bs v}(x)\) has Property~\ref{properties: 1}.

\begin{lem}
	The closing operator \(\overline{\bs v}(x)\) has Property~\ref{properties: -2}.
	
	For \(\bs a \in\mathcal A\), \(u\geq 0\), 
	\[\int_{x=0}^\Delta \bs a e^{\bs Su}\overline{\bs v}(x)\wrt x =\bs a e^{\bs Su}\bs e. \]
\end{lem}
\begin{proof}
	\begin{align*}
		\int_{x=0}^\Delta \bs a e^{\bs Su}\overline{\bs v}(x)\wrt x &= \bs a  e^{\bs Su} (\bs S)^{-1} \left( e^{\bs S\Delta} - I + e^{\bs S2\Delta} - e^{\bs S\Delta}\right)\left[I-e^{\bs S  2\Delta}\right]^{-1}\bs s
		\\ &= \bs a  e^{\bs Su}(-\bs S)^{-1}\bs s
		\\ &= \bs a  e^{\bs Su}\bs e.
	\end{align*}
\end{proof}

\begin{cor}\label{cor: cond bnd 2 U2}
	The closing operator \(\overline{\bs v}(x)\) has Property~\ref{properties: 2}.

	Let \(g\) be a function satisfying the Assumptions \ref{asu: g}. For \(u\leq \Delta-\varepsilon \), \(v\in[ 0,\Delta)\), 
	\[\int_{x=0}^\infty \cfrac{\bs \alpha  e^{\bs{S} (u+x)} }{\bs \alpha  e^{\bs{S} u} \bs e} \overline{\bs v}(v)g(x)\wrt x = g(\Delta-u-v) 1(u+v\leq\Delta-\varepsilon) + r_{\overline{\bs v}} (u,v),\]
	where 
	\[\left|r_{\overline{\bs v}} (u,v)\right|\leq r_{\bs u} (u,v) + \cfrac{G \varepsilon^2 \pi^2}{4\Delta^2}.\]
	Furthermore,  
	\begin{align*}
		R_{{\overline{\bs v}},1}&=\int_{u=0}^{\Delta}| r_{\overline{\bs v}}(u,v)|\wrt u\leq R_{{\widehat{\bs v}},1} + \cfrac{G \varepsilon^2 \pi^2}{4\Delta},
	\end{align*}
	and
	\begin{align*}
		R_{{\overline{\bs v}},2}&=\int_{v=0}^{\Delta}| r_{\overline{\bs v}}(u,v)|\wrt u\leq R_{{\widehat{\bs v}},2} + \cfrac{G \varepsilon^2 \pi^2}{4\Delta}.
	\end{align*}
\end{cor}
\begin{proof}
	By the definition of the operator \(\overline{\bs v}(v)\), 
	\begin{align}
		&\int_{x=0}^\infty \cfrac{\bs \alpha  e^{\bs{S} (u+x)} }{\bs \alpha  e^{\bs{S} u} \bs e} \overline{\bs v}(v)g(x)\wrt x \nonumber		
		%
		\\&= \int_{x=0}^\infty \cfrac{\bs \alpha  e^{\bs{S} (u+x)} }{\bs \alpha  e^{\bs{S} u} \bs e} {\widehat{\bs v}}(v)g(x)\wrt x + \int_{x=0}^\infty \cfrac{\bs \alpha  e^{\bs{S} (u+x)} }{\bs \alpha  e^{\bs{S} u} \bs e} \left(e^{\bs{S} v} + e^{\bs{S} (2\Delta-v)}\right)\sum_{n=1}^\infty e^{\bs S2n\Delta}\bs s g(x)\wrt x. \label{eqn: ghi is this a2}
	\end{align}
	By Lemma \ref{cor: cond bnd 2 U} the first term is 
	\(g(\Delta-u-v) 1(u+v\leq\Delta-\varepsilon) + r_{{\widehat{\bs v}}} (u,v),\)
	where 
	\(\left|r_{\widehat{\bs v}} (u,v)\right|\leq r_3 (u+v) + r_3 (u+2\Delta - v).\)

	Since \(g\leq G\), the second term is less than or equal to 
	\begin{align*}
		G\int_{x=0}^\infty \cfrac{\bs \alpha  e^{\bs{S} (u+x)} }{\bs \alpha  e^{\bs{S} u} \bs e} \left(e^{\bs{S} v} + e^{\bs{S} (2\Delta-v)}\right)\sum_{n=1}^\infty e^{\bs S2n\Delta}\bs s \wrt x
		%
		&= G \cfrac{\bs \alpha  e^{\bs{S} u} }{\bs \alpha  e^{\bs{S} u} \bs e} \left(e^{\bs{S} v} + e^{\bs{S} (2\Delta-v)}\right)\sum_{n=1}^\infty e^{\bs S2n\Delta}\bs e.
	\end{align*}
	By similar arguments to those used in the proof of Lemma \ref{lem: akxnj2} we can show that this is less than or equal to 
	\begin{align*}
		\cfrac{G}{\bs \alpha  e^{\bs{S} u} \bs e} \cfrac{\var(Z)}{\Delta^2}\cfrac{\pi^2}{4}.
	\end{align*}
	Now, as \(u\leq \Delta-\varepsilon\), then \(\bs\alpha e^{\bs Su}\bs e\geq \var(Z)/\varepsilon^2\) by Chebyshev's inequality, hence 
	\begin{align*}
		\cfrac{G}{\bs \alpha  e^{\bs{S} u} \bs e} \cfrac{\var(Z)}{\Delta^2}\cfrac{\pi^2}{4} \leq \cfrac{G}{\var(Z)/\varepsilon^2} \cfrac{\var(Z)}{\Delta^2}\cfrac{\pi^2}{4} = \cfrac{G \varepsilon^2 \pi^2}{4\Delta^2}.
	\end{align*}

	Putting it all together, we have shown 
	\begin{align}
		\int_{x=0}^\infty \cfrac{\bs \alpha  e^{\bs{S} (u+x)} }{\bs \alpha  e^{\bs{S} u} \bs e} \overline{\bs v}(v)g(x)\wrt x = g(\Delta-u-v) 1(u+v\leq\Delta-\varepsilon) + r_{\overline{\bs v}} (u,v)
	\end{align}
	where 
	\[|r_{\overline{\bs v}} (u,v)| \leq \left|r_{\widehat{\bs v}} (u,v) + \cfrac{G \varepsilon^2 \pi^2}{4\Delta^2}\right|.\]
	
	Lastly, observe 
	\begin{align*}
		R_{\overline{\bs v},1} = \int_{u=0}^\Delta |r_{\overline{\bs v}} (u,v) |\wrt u 
		&\leq \int_{u=0}^\Delta \left|r_{{\widehat{\bs v}}} (u,v) \right| + \left|\cfrac{G \varepsilon^2 \pi^2}{4\Delta^2}\right| \wrt u
		\\& = R_{{{\widehat{\bs v}}},1} + \cfrac{G \varepsilon^2 \pi^2}{4\Delta}
	\end{align*}
	and similarly, 
	\[R_{\overline{\bs v},2} = \int_{v=0}^\Delta |r_{\overline{\bs v}} (u,v) |\wrt v \leq R_{{{\widehat{\bs v}}},2} + \cfrac{G \varepsilon^2 \pi^2}{4\Delta}\]
	where we have used Lemma \ref{cor: cond bnd 2 U}. 
\end{proof}
