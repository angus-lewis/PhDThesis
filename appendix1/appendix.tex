%!TEX root = ../thesis.tex
\chapter{Technical results for covergence\label{app:tand}}
\section{Some bounds for integrating against matrix exponential distributions.}\label{appendix: bounds}
This appendix proves some technical Lemmas about integrating functions with respect to matrix exponential distributions. 

As with Section~\ref{sec: conv}, we use the superscript \((p)\) to denote dependence on the underlying choice of matrix exponential random variable \(Z^{(p)}\). However, to simplify notation, we may omit the super script \((p)\) where it is not explicitly needed. We show results for an arbitrary parameter \(\varepsilon>0\). Keep in mind the ultimate intention is to show convergence, for which we choose this parameter to be \(\varepsilon^{(p)}=\var\left(Z^{(p)}\right)^{1/3}\). Other notations, previously defined and which depend on \(p\) are \(\bs\alpha^{(p)},\) \(\bs \alpha_0^{i,\ell_0,(p)}(x_0),\) \(\bs S^{(p)},\) \(\bs s^{(p)},\) \(\varepsilon^{(p)},\) \(V^{(p)}(x),\) \(R_{V,1}^{(p)},\) \(R_{V,2}^{(p)},\) \(\bs D^{(p)},\) \( \mathcal A^{(p)}\).

\subsection{One integral}\label{appendix: int one}
 Here we show a collection of results that show that integrating functions against a ME density function, or against the density function of a ME conditional on the ME-life-time surviving until some time \(u<\Delta-\varepsilon\) where \(\Delta = \mathbb E[Z]\) is the mean of the matrix exponential distribution, approximates integrating said function against a Kronecker delta situated at the \(\Delta\), provided the variance of the ME is sufficiently low. %(Lemma~\ref{lemma:bound}, Corollaries \ref{cor: cond bnd}, \ref{cor: cond bnd 2}). 
  
\begin{lem}\label{lemma:bound}
	Let \(g\) be a function satisfying Assumptions \ref{asu: g}, then, for \(u \leq \Delta - \varepsilon\), 
	\begin{align*}
		\int_{x=0}^\infty g\left(x\right)\bs \alpha e^{\bs{S}\left(x+u\right)} \bs s \wrt x = g\left(\Delta-u\right) + r_1,
	\end{align*}
	where 
	\[\left|r_1\right|\leq 2G\cfrac{\var \left(Z\right)}{\varepsilon^2} + 2L\varepsilon.\]
\end{lem}
The proof follows closely that of \cite[Appendix A, Theorem 4]{hht2020}.
\begin{proof}
	By a change of variables, 
	\begin{align*}
		\\&\left|\int_{x=0}^\infty g\left(x\right)\bs \alpha  e^{\bs{S} \left(x+u\right)} \bs s \wrt x - g\left(\Delta-u\right)\right| 
		%
		\\&= \left|\int_{x=u}^\infty g\left(x-u\right)\bs \alpha  e^{\bs{S} x} \bs s \wrt x - g\left(\Delta-u\right)\right| 
		%
		\\&= \Bigg|\int_{x=u}^\infty g\left(x-u\right)\bs \alpha  e^{\bs{S} x} \bs s \wrt x - \int_{x=u}^\infty g\left(\Delta-u\right)\bs\alpha  e^{\bs{S} x}\bs s\wrt x - g\left(\Delta-u\right)\left(1-\bs\alpha  e^{\bs{S} u}\bs e \right)\Bigg|.
		%
	\end{align*}
	{By the triangle inequality this is less than or equal to}
	\begin{align*}
		&\left|\int_{x=u}^\infty \left(g\left(x-u\right)- g\left(\Delta-u\right)\right)\bs \alpha  e^{\bs{S} x} \bs s \wrt x \right| 
		{}+ \left|g\left(\Delta-u\right)\left(1-\bs\alpha  e^{\bs{S} u}\bs e \right)\right|
		%
		\\&= \left|\int_{x=u}^\infty \left(g\left(x-u\right)- g\left(\Delta-u\right)\right)\bs \alpha  e^{\bs{S} x} \bs s \wrt x\right| 
		%
		+ \left|\int_{x=0}^u g\left(\Delta-u\right)\bs \alpha  e^{\bs{S} x} \bs s \wrt x \right| 
		%
		\\&\leq d_1 +{d_2} 
	\end{align*}
	where 
	\begin{align*}
		d_1 &= \left|\int_{x=0}^u g\left(\Delta-u\right)\bs \alpha  e^{\bs{S} x} \bs s \wrt x\right| + \left|\int_{x=u}^{\Delta-\varepsilon } \left(g\left(x-u\right)- g\left(\Delta-u\right)\right)\bs \alpha  e^{\bs{S} x} \bs s \wrt x \right| 
		\\&\quad{}+ \left|\int_{x=\Delta+\varepsilon }^{\infty} \left(g\left(x-u\right)- g\left(\Delta-u\right)\right)\bs \alpha  e^{\bs{S} t} \bs s \wrt x \right|,
	\\{d_2} &= \left|\int_{x=\Delta-\varepsilon }^{\Delta+\varepsilon } \left(g\left(t-u\right)- g\left(\Delta-u\right)\right)\bs \alpha  e^{\bs{S} x} \bs s \wrt x\right| .
	\end{align*}
	
 	Applying the triangle inequality to \(d_1\),
	\begin{align}
		d_1  &\leq \int_{x=u}^{\Delta-\varepsilon } \left|g\left(x-u\right)- g\left(\Delta-u\right)\right|\bs \alpha  e^{\bs{S} x} \bs s \wrt x
		+ \int_{x=\Delta+\varepsilon }^{\infty} \left|g\left(x-u\right)- g\left(\Delta-u\right)\right|\bs \alpha  e^{\bs{S} x} \bs s \wrt x \nonumber
		\\&\quad{}+ \left|\int_{x=0}^u g\left(\Delta-u\right)\bs \alpha  e^{\bs{S} x} \bs s \wrt x \right| .\label{eqn: kkkka}
		%
		\end{align}
		{Since \(|g\left(x\right)|\leq G\), then (\ref{eqn: kkkka}) is less than or equal to}
		\begin{align}
		& 2G\Bigg( \int_{x=u}^{\Delta-\varepsilon }\bs \alpha  e^{\bs{S} x} \bs s \wrt x
		+ \int_{x=\Delta+\varepsilon }^{\infty}\bs \alpha  e^{\bs{S} x} \bs s \wrt x
		+ \int_{x=0}^u \bs \alpha  e^{\bs{S} x} \bs s \wrt x \Bigg)
		%
		=2G\mathbb P\left(|Z -\Delta|>\varepsilon \right).
		%
		\intertext{By Chebyshev's inequality,}
		&2G\mathbb P\left(|Z -\Delta|>\varepsilon \right)\leq 2G\cfrac{\var \left(Z \right)}{\varepsilon ^2}.
		%
%		\\&= \left(2GL^2\var \left(Z_1 \right)\right)^{1/3}\Delta^2
	\end{align}
	For the term \({d_2} \) we have 
	\begin{align*}
		{d_2}  &= \left|\int_{x=\Delta-\varepsilon }^{\Delta+\varepsilon } \left(g\left(x-u\right)- g\left(\Delta-u\right)\right)\bs \alpha  e^{\bs{S} x} \bs s \wrt x\right| 
		\\&\leq \int_{x=\Delta-\varepsilon }^{\Delta+\varepsilon } \left|g\left(x-u\right)- g\left(\Delta-u\right)\right|\bs \alpha  e^{\bs{S} x} \bs s \wrt x
		\\&\leq \int_{x=\Delta-\varepsilon }^{\Delta+\varepsilon } 2L\varepsilon \bs \alpha  e^{\bs{S} x} \bs s \wrt x
		%
		\\&=2L\varepsilon \mathbb P(Z\in(\Delta-\varepsilon, \Delta+\varepsilon))
		%
		\\&\leq 2L\varepsilon ,
	\end{align*}
	where the first inequality is the triangle inequality and the second inequality is from the Lipschitz property of \(g\) in Assumption \ref{asu: lipschitz}. 
	Hence there is some \(r_1\) such that 
	\[\left|\int_{x=0}^\infty g\left(x\right)\bs \alpha  e^{\bs{S} \left(x+u\right)} \bs s \wrt x - g\left(\Delta-u\right)\right| = |r_1| \leq 2G\cfrac{\var(Z)}{\varepsilon^2} + 2 L \varepsilon,\]
	and this completes the proof. 
\end{proof}

The error term \(r_1^{(p)}\) depends on \(p\), as it is defined by \(Z^{(p)}\) and \(\varepsilon^{(p)}\), but we have omitted the superscript \(p\) here. Note that, upon choosing \(\varepsilon=\var(Z^{(p)})^{1/3}\), the error term \(|r_1^{(p)}|\) is at most \(O\left(\var\left(Z^{(p)}\right)^{1/3}\right)\), which tends to 0 as \(p\to\infty\).

\begin{cor}\label{cor: cond bnd}
	Let \(g\) be a function satisfying the Assumptions \ref{asu: g}. For \(u\leq \Delta-\varepsilon \), 
	\[\int_{x=0}^\infty \cfrac{\bs \alpha  e^{\bs{S} (x+u)} \bs s}{\bs \alpha  e^{\bs{S} u} \bs e} g(x)\wrt x = g(\Delta-u) + r_2 ,\]
	where 
	\[\left|r_2 \right|\leq 3G\cfrac{\var \left(Z \right)}{\varepsilon ^2} + 2L\varepsilon .\]
\end{cor}
\begin{proof}
	Observe that Chebyshev's inequality gives
	\begin{align*}
		\bs \alpha e^{\bs Su}\bs e&=\mathbb P\left(Z >u\right) 
		\\&\geq \mathbb P\left(|Z -\Delta|\leq \varepsilon \right) 
		%
		\\&\geq 1 - \cfrac{\var\left(Z \right)}{\varepsilon ^2} 
%		%
%		\\&= 1-\Delta^2\left(\cfrac{L^2\var\left(Z_1 \right)}{4G^2}\right)^{1/3}
		%
		\\&=: 1-\delta .
	\end{align*}
	
	Now, since \(1-\delta\leq\bs \alpha e^{\bs Su}\bs e\leq 1\), then
	\begin{align*}
		\int_{x=0}^\infty \bs \alpha  e^{\bs{S} (x+u)} \bs s g(x)\wrt x
		\leq \int_{x=0}^\infty \cfrac{\bs \alpha  e^{\bs{S} (x+u)} \bs s}{\bs \alpha  e^{\bs{S} u} \bs e} g(x)\wrt x
		%
		&\leq \frac{1}{1-\delta }\int_{x=0}^\infty \bs \alpha  e^{\bs{S} (x+u)} \bs s g(x)\wrt x.
	\end{align*}
	By Lemma~\ref{lemma:bound} we have 
	\begin{align*}
		g(\Delta-u)+r _1
		&\leq \int_{x=0}^\infty \cfrac{\bs \alpha  e^{\bs{S} (x+u)} \bs s}{\bs \alpha  e^{\bs{S} u} \bs e} g(x)\wrt x
		%
		\leq \frac{g(\Delta-u)+r _1}{1-\delta }. 
	\end{align*}
	Multiplying by \(1-\delta \), then subtracting \(g(\Delta-u)\) and adding \(\displaystyle\int_{x=0}^\infty \cfrac{\bs \alpha  e^{\bs{S} (x+u)} \bs s}{\bs \alpha  e^{\bs{S} u} \bs e} g(x)\wrt x\delta \) gives 
	\begin{align*}
		&r _1(1-\delta ) - g(\Delta-u)\delta +\int_{x=0}^\infty \cfrac{\bs \alpha  e^{\bs{S} (x+u)} \bs s}{\bs \alpha  e^{\bs{S} u} \bs e} g(x)\wrt x\delta 
		\\&\leq \int_{x=0}^\infty \cfrac{\bs \alpha  e^{\bs{S} (x+u)} \bs s}{\bs \alpha  e^{\bs{S} u} \bs e} g(x)\wrt x -g(\Delta-u)
		%
		\\&\leq r _1+\int_{x=0}^\infty \cfrac{\bs \alpha  e^{\bs{S} (x+u)} \bs s}{\bs \alpha  e^{\bs{S} u} \bs e} g(x)\wrt x\delta .
	\end{align*}
	The right-hand side is bounded above as 
	\begin{align*}
		r _1+\int_{x=0}^\infty \cfrac{\bs \alpha  e^{\bs{S} (x+u)} \bs s}{\bs \alpha  e^{\bs{S} u} \bs e} g(x)\wrt x\delta 
		%
		&\leq r _1 + G \delta .
	\end{align*}
	The left-hand side is bounded below as 
	\begin{align*}
		r_1 (1-\delta ) - g(\Delta-u)\delta +\int_{x=0}^\infty \cfrac{\bs \alpha  e^{\bs{S} (x+u)} \bs s}{\bs \alpha  e^{\bs{S} u} \bs e} g(x)\wrt x\delta 
		%
		&\geq r _1(1-\delta ) - g(\Delta-u)\delta .
	\end{align*}
	Hence, 
	\begin{align}
		\left|\int_{x=0}^\infty \cfrac{\bs \alpha  e^{\bs{S} (x+u)} \bs s}{\bs \alpha  e^{\bs{S} u} \bs e} g(x)\wrt x -g(\Delta-u)\right| \leq \max\left(r _1(1-\delta )+g(\Delta-u)\delta , r _1 + G \delta \right)
	\end{align}
	and therefore 
	\begin{align}
		\int_{x=0}^\infty \cfrac{\bs \alpha  e^{\bs{S} (x+u)} \bs s}{\bs \alpha  e^{\bs{S} u} \bs e} g(x)\wrt x  = g(\Delta-u) + r_2 ,
	\end{align}
	where 
	\begin{align}
		\nonumber\left|r_2 \right| 
		&\leq \max\left(r _1(1-\delta ) + g(\Delta-u)\delta , r _1 + G \delta \right) 
		%
		\\\nonumber&\leq  r_1 + G\delta%\max\left(G\cfrac{\var \left(Z \right)}{\varepsilon^2}, 3G\cfrac{\var \left(Z \right)}{\varepsilon^2} + 2L\varepsilon  \right) 
		%
		\\&=3G\cfrac{\var \left(Z \right)}{\varepsilon^2} + 2L\varepsilon .
	\end{align}
	This completes the proof. 
\end{proof}

The error term \(r_2^{(p)}\) also depends on \(p\), as it is defined by \(Z^{(p)}\) and \(\varepsilon^{(p)}\), but we have omitted the superscript \(p\) here. Choosing \(\varepsilon=\var(Z^{(p)})\), the error term \(|r_2^{(p)}|\) is at most \(O\left(\var\left(Z^{(p)}\right)^{1/3}\right)\), which tends to 0 as \(p\to\infty\).

\begin{cor}\label{cor: cond bnd 2}
	Let \(g\) be a function satisfying the Assumptions \ref{asu: g}. For \(u\leq \Delta-\varepsilon \), \(v\geq 0\), 
	\[\int_{x=0}^\infty \cfrac{\bs \alpha  e^{\bs{S} (x+u+v)} \bs s}{\bs \alpha  e^{\bs{S} u} \bs e} g(x)\wrt x = g(\Delta-u-v) 1(u+v\leq\Delta-\varepsilon) + r_3 (u+v),\]
	where 
	\[\left|r_3 (u+v)\right|\leq \begin{cases} 
		r_2  & u+v\leq \Delta-\varepsilon,\\
		G & u+v\in(\Delta-\varepsilon,\Delta+\varepsilon), \\
		G\cfrac{\var(Z)/\varepsilon^2}{1-\var(Z)/\varepsilon^2} & u+v \geq \Delta + \varepsilon.
		\end{cases}\]
\end{cor}
\begin{proof}
	For \(u+v \leq \Delta - \varepsilon\) we use similar arguments those as in the proof of Corollary~\ref{cor: cond bnd}. We have 
	\begin{align*}
		\int_{x=0}^\infty \bs \alpha  e^{\bs{S} (x+u+v)} \bs s g(x)\wrt x
		\leq \int_{x=0}^\infty \cfrac{\bs \alpha  e^{\bs{S} (x+u+v)} \bs s}{\bs \alpha  e^{\bs{S} u} \bs e} g(x)\wrt x
		%
		&\leq \frac{1}{1-\delta }\int_{x=0}^\infty \bs \alpha  e^{\bs{S} (x+u+v)} \bs s g(x)\wrt x.
	\end{align*}
	By Lemma~\ref{lemma:bound}  
	\begin{align*}
		g(\Delta-u-v)+r _1
		&\leq \int_{x=0}^\infty \cfrac{\bs \alpha  e^{\bs{S} (x+u+v)} \bs s}{\bs \alpha  e^{\bs{S} u} \bs e} g(x)\wrt x
		%
		\leq \frac{g(\Delta-u-v)+r _1}{1-\delta }. 
	\end{align*}
	Multiplying by \(1-\delta \), then subtracting \(g(\Delta-u-v)\) and adding \(\displaystyle\int_{x=0}^\infty \cfrac{\bs \alpha  e^{\bs{S} (x+u+v)} \bs s}{\bs \alpha  e^{\bs{S} u} \bs e} g(x)\wrt x\delta \) gives
	\begin{align*}
		&r _1(1-\delta ) - g(\Delta-u-v)\delta +\int_{x=0}^\infty \cfrac{\bs \alpha  e^{\bs{S} (x+u+v)} \bs s}{\bs \alpha  e^{\bs{S} u} \bs e} g(x)\wrt x\delta 
		\\&\leq \int_{x=0}^\infty \cfrac{\bs \alpha  e^{\bs{S} (x+u+v)} \bs s}{\bs \alpha  e^{\bs{S} u} \bs e} g(x)\wrt x -g(\Delta-u-v)
		%
		\\&\leq r _1+\int_{x=0}^\infty \cfrac{\bs \alpha  e^{\bs{S} (x+u+v)} \bs s}{\bs \alpha  e^{\bs{S} u} \bs e} g(x)\wrt x\delta .
	\end{align*}
	The right-hand side is bounded above by 
	\begin{align*}
		r _1+\int_{x=0}^\infty \cfrac{\bs \alpha  e^{\bs{S} (x+u+v)} \bs s}{\bs \alpha  e^{\bs{S} u} \bs e} g(x)\wrt x\delta 
		%
		&\leq r _1 + G \delta .
	\end{align*}
	The left-hand side is bounded below by 
	\begin{align*}
		r_1 (1-\delta ) - g(\Delta-u-v)\delta +\int_{x=0}^\infty \cfrac{\bs \alpha  e^{\bs{S} (x+u+v)} \bs s}{\bs \alpha  e^{\bs{S} u} \bs e} g(x)\wrt x\delta 
		%
		&\geq r _1(1-\delta ) - g(\Delta-u-v)\delta .
	\end{align*}
	and ultimately, 
	\begin{align}
		\left|\int_{x=0}^\infty \cfrac{\bs \alpha  e^{\bs{S} (x+u+v)} \bs s}{\bs \alpha  e^{\bs{S} u} \bs e} g(x)\wrt x -g(\Delta-u-v)\right| \leq 3G\cfrac{\var \left(Z \right)}{\varepsilon^2} + 2L\varepsilon .
	\end{align}
	
	For \(u+v\in (\Delta-\varepsilon, \Delta + \varepsilon)\),
	\begin{align}
		\int_{x=0}^\infty \cfrac{\bs \alpha  e^{\bs{S} (x+u+v)} \bs s}{\bs \alpha  e^{\bs{S} u} \bs e} g(x)\wrt x & \leq G \mathbb P(Z>u+v\mid Z>u) \leq G
%		%
%		\\& = \int_{x=u}^\infty \cfrac{\bs \alpha  e^{\bs{S} (x+v)} \bs s}{\bs \alpha  e^{\bs{S} u} \bs e} g(x)\wrt x
%		%
%		\\& \leq \int_{x=0}^\infty \cfrac{\bs \alpha  e^{\bs{S} (x+v)} \bs s}{\bs \alpha  e^{\bs{S} u} \bs e} g(x)\wrt x
%		%
%		\\& \leq \cfrac{1}{\bs \alpha  e^{\bs{S}(\Delta-\varepsilon) } \bs e} \int_{x=0}^\infty {\bs \alpha  e^{\bs{S} (x+\Delta-\varepsilon+v)} \bs s} g(x)\wrt x
%		%
%		\\& \leq \cfrac{1}{\bs \alpha  e^{\bs{S}(\Delta-\varepsilon) } \bs e} \int_{x=0}^\infty {\bs \alpha  e^{\bs{S} (x+\Delta-\varepsilon)} \bs s} g(x)\wrt x 1(v< 2 \varepsilon)
%		\\&\qquad{} + \cfrac{1}{\bs \alpha  e^{\bs{S}(\Delta-\varepsilon) } \bs e} \int_{x=0}^\infty {\bs \alpha  e^{\bs{S} (x+\Delta+\varepsilon)} \bs s} g(x)\wrt x 1(v\geq 2 \varepsilon)
%		%
%		\\&\leq \cfrac{G}{1-\var(Z)/\varepsilon^2}\left(1(v<2\varepsilon) + \var(Z)1(v\geq2\varepsilon)\right)
	\end{align}
	
	For \(u+v \geq \Delta + \varepsilon\),
	\begin{align}
		\int_{x=0}^\infty \cfrac{\bs \alpha  e^{\bs{S} (x+u+v)} \bs s}{\bs \alpha  e^{\bs{S} u} \bs e} g(x)\wrt x & \leq G \cfrac{\mathbb P(Z>u+v)}{\mathbb P( Z>u)} 
		%
		 \leq G\cfrac{\var(Z)/\varepsilon^2}{1-\var(Z)/\varepsilon^2} .
	\end{align}
\end{proof}

The error term \(r_3^{(p)}\) depends on \(p\), as it is defined by \(Z^{(p)}\) and \(\varepsilon^{(p)}\), but we have omitted the superscript \(p\) here. Choosing \(\varepsilon=\var(Z^{(p)})^{1/3}\) then, outside of the vanishingly small interval \(u\in(\Delta-\varepsilon^{(p)},\Delta+\varepsilon^{(p)})\), the error term \(|r_3^{(p)}(u)|\) is bounded by \(O\left(\var\left(Z^{(p)}\right)^{1/3}\right)\), which tends to 0 as \(p\to\infty\). On \(u\in(\Delta-\varepsilon^{(p)},\Delta+\varepsilon^{(p)})\) the error term \(|r_3^{(p)}(u)|\) is bounded by a constant which does not tend to \(0\) as \(p \to \infty\). However, when we integrate a bounded function against \(r_3^{(p)}(u)\), then the resulting integral tends to \(0\), i.e.~for \(|\psi(x)|\leq F, \, M<\infty\), \(\displaystyle \int_{0}^M f(u) r_3^{(p)}(u)\wrt u\leq F\Delta |r_2^{(p)}| + 2GF\varepsilon^{(p)} + (M-\Delta)GF\cfrac{\var(Z^{(p)})/\left(\varepsilon^{(p)}\right)^2}{1-\var\left(Z^{(p)}\right)/\left(\varepsilon^{(p)}\right)^2}=O\left(\var\left(Z^{(p)}\right)^{1/3}\right)\to 0 \) as \(p\to\infty\). This is the context in which we we apply Corollary~\ref{cor: cond bnd 2} and thus the error bound is sufficient. See, for example, Corollary~\ref{cor: cond bnd 2 V}. 

%\begin{lem}\label{lem: Dcoajc}
%	Let \(\psi:[0,\Delta)\to \mathbb R\) be bounded, \(\psi(x)\leq F\), and Lipschitz. Then, for \(x\in\calD_{\ell_0,j}\), \(\ell_0\in\mathcal K\setminus\{-1,K+1\}\), \(\lambda > 0\),
%	\begin{align}
%            	\left|\int_{x=0}^\Delta \widehat f^{\ell_0,(p)}_{0,+,+}(\lambda)(x,j; x_0,i)\psi(x)\wrt x - \int_{x=0}^\Delta\widehat \mu^{\ell_0}_{0,+,+}(\lambda)(x,j; x_0,i)\psi(x)\wrt x\right| \leq R_{V,2}^{(p)} GF + \varepsilon^{(p)} GF, \label{eqn: anue}
%            \end{align}
%            and 
%            \begin{align}
%            	\left|\int_{x=0}^\Delta \widehat f^{\ell_0,(p)}_{0,-,-}(\lambda)(x,j; x_0,i)\psi(x)\wrt x - \int_{x=0}^\Delta\widehat \mu^{\ell_0}_{0,-,-}(\lambda)(x,j; x_0,i)\psi(x)\wrt x\right| \leq R_{V,2}^{(p)} GF + \varepsilon^{(p)} GF. \label{eqn: anue2}
%            \end{align} 
%\end{lem}
\begin{lem}\label{lem: Dcoajc}
	Let \(\psi:[0,\Delta)\to \mathbb R\) be bounded, \(\psi(x)\leq F\), and Lipschitz. Then, for \(x\in\calD_{\ell_0,j}\), \(\ell_0\in\mathcal K\setminus\{-1,K+1\}\), \(\lambda > 0\), \(q\in\{+,-\}\), 
	\begin{align}
		\left|\int_{x=0}^\Delta \widehat f^{\ell_0,(p)}_{0,q,q}(\lambda)(x,j; x_0,i)\psi(x)\wrt x - \int_{x=0}^\Delta\widehat \mu^{\ell_0}_{0,q,q}(\lambda)(x,j; x_0,i)\psi(x)\wrt x\right| \leq \left(R_{V,2}^{(p)} + \varepsilon^{(p)}\right) GF.
		\label{eqn: anue}
	\end{align}
            % and 
            % \begin{align}
            % 	\left|\int_{x=0}^\Delta \widehat f^{\ell_0,(p)}_{0,-,-}(\lambda)(x,j; x_0,i)\psi(x)\wrt x - \int_{x=0}^\Delta\widehat \mu^{\ell_0}_{0,-,-}(\lambda)(x,j; x_0,i)\psi(x)\wrt x\right| \leq \left(R_{V,2}^{(p)} + \varepsilon^{(p)}\right) GF. \label{eqn: anue2}
            % \end{align} 
\end{lem}
\begin{proof} 
                Property \ref{properties: 2} states
                \begin{align}
                	\left|\int_{x=0}^\infty \cfrac{\bs \alpha e^{\bs{S}(u+x)} }{\bs \alpha e^{\bs{S}u} \bs e} V(v)g(x)\wrt x -g(\Delta-u-v) 1(u+v\leq\Delta-\varepsilon)\right| =  |r_V(u,v)|.
                \end{align}
                Setting \(g(x) = h_{ij}^{++}(\lambda,x)\), 
                \begin{align}
                	\left|\int_{x=0}^\infty \cfrac{\bs \alpha e^{\bs{S}(u+x)} }{\bs \alpha e^{\bs{S}u} \bs e} V(v)h_{ij}^{++}(\lambda,x)\wrt x -h_{ij}^{++}(\lambda,\Delta-u-v) 1(u+v\leq\Delta-\varepsilon)\right| =  |r_V(u,v)|. \label{eqn: akv}
                \end{align}
                We recognise the left-most term as 
                \[\widehat f^{\ell_0,(p)}_{0,+,+}(\lambda)(y_{\ell_0+1}-v,j; y_{\ell_0}+u,i) = \int_{x=0}^\infty \cfrac{\bs \alpha e^{\bs{S}(u+x)} }{\bs \alpha e^{\bs{S}u} \bs e} V(v)h_{ij}^{++}(\lambda,x)\wrt x.\]
                
                Now consider 
                \begin{align}
                	&\Bigg|\int_{v=0}^\Delta \int_{x=0}^\infty \cfrac{\bs \alpha e^{\bs{S}(u+x)} }{\bs \alpha e^{\bs{S}u} \bs e} V(v)h_{ij}^{++} (\lambda,x)\wrt x \psi(v)\wrt v  \nonumber 
		\\&\qquad{}- \int_{v=0}^\Delta h_{ij}^{++}(\lambda,\Delta-u-v) 1(\Delta-u-v\geq 0) \psi(v)\wrt v\Bigg| \nonumber 
                	%
                	\\&\leq \int_{v=0}^\Delta \left|  \int_{x=0}^\infty \cfrac{\bs \alpha e^{\bs{S}(u+x)} }{\bs \alpha e^{\bs{S}u} \bs e} V(v)h_{ij}^{++} (\lambda,x)\wrt x  - h_{ij}^{++}(\lambda,\Delta-u-v) 1(\Delta-u-v\geq 0) \right| \left| \psi(v) \right| \wrt v\nonumber 
                	%
                	\\&\leq \int_{v=0}^\Delta \left|  \int_{x=0}^\infty \cfrac{\bs \alpha e^{\bs{S}(u+x)} }{\bs \alpha e^{\bs{S}u} \bs e} V(v)h_{ij}^{++} (\lambda,x)\wrt x  - h_{ij}^{++}(\lambda,\Delta-u-v) 1(\Delta-u-v\geq \varepsilon) \right| \left| \psi(v) \right| \wrt v\nonumber 
                	\\&\qquad {}+ \int_{v=0}^\Delta \left| h_{ij}^{++}(\lambda,\Delta-u-v) 1(\varepsilon \geq \Delta-u-v\geq 0) \right| \left| \psi(v) \right| \wrt v. \label{eqn: ALllllLsdnn}
                \end{align}
                Recognising the first term as the left-hand side of (\ref{eqn: akv}), then (\ref{eqn: ALllllLsdnn}) is less than or equal to 
                \begin{align}
                	& \int_{v=0}^\Delta |r_V(u,v)| \left| \psi(v) \right| \wrt v 
                	+ \int_{v=0}^\Delta \left| h_{ij}^{++}(\lambda,\Delta-u-v) 1(\varepsilon \geq \Delta-u-v\geq 0) \right| \left| \psi(v) \right| \wrt v\nonumber 
                	\\&\leq R_{V,2} GF + \varepsilon GF. \nonumber 
                \end{align}
                Finally, noting \(h_{ij}^{++}(\lambda,\Delta-u-v) 1(\Delta-u-v\geq 0)=\widehat \mu_{0,+,+}^{\ell_0}(\lambda)(y_{\ell_0+1}-v,j;y_{\ell_0}+u,i)\), then we have shown (\ref{eqn: anue}) for \(q=+\). 
                
                Using analogous arguments we can show  (\ref{eqn: anue}) for \(q=-\).
\end{proof}


\subsection{Many integrals.}
%In this section we first show bounds for tails of the integrals in expressions of the form (\ref{eqn: approx final end 2}) (Lemmas \ref{lem: lh bnd}, \ref{lem: rh bnd} and Corollary~\ref{cor: lh and rh}); these bounds are in terms of the variance of the ME. We then combine these results with the Properties \ref{properties: 1}-\ref{properties: 2} to prove Corollary~\ref{cor: a cor}. 

Define the column vectors 
\begin{align}
	\mathcal I_{m,k}(u_k) = \left[\prod_{\ell=m}^{k-1}\int_{x_\ell=0}^\infty g_\ell(x_\ell) e^{\bs{S}x_\ell}\wrt x_\ell \bs{D} \right]
%            	\int_{x_{m+1}=0}^\infty g_{m+1}(x_{m+1}) e^{\bs{S}x_{m+1}} \wrt x_{m+1} \bs{D} \nonumber
%            	\dots 
            	\int_{x_k=0}^\infty g_{k}(x_k) e^{\bs{S}x_k} \wrt x_k e^{\bs{S}u_k}\bs s
\end{align}
for \(m,k\in\{1,2,\dots\}\), \(m\leq k\), where a product over an empty set is equal to 1.
Also define the row vectors 
\begin{align}
	\mathcal J_{k+1,k+1}(u_k,x_{k+1}) &:= g_{k+1}(x_{k+1})\cfrac{\bs \alpha e^{\bs{S}u_{k}}}{\bs \alpha e^{\bs{S}u_{k}}\bs e}e^{\bs{S}x_{k+1}} 
	\intertext{and}
	\mathcal J_{k+1,n}(u_k,x_{k+1}) &:= g_{k+1}(x_{k+1})\cfrac{\bs \alpha e^{\bs{S}u_{k}}}{\bs \alpha e^{\bs{S}u_{k}}\bs e} e^{\bs{S}x_{k+1}} \bs{D} \left[\prod_{m=k+2}^{n-1}\int_{x_{m}=0}^\infty g_{m}(x_{m}) e^{\bs{S}x_{m}} \wrt x_{m} \bs{D} \right]\nonumber
%		\\&\quad\hdots 
%		\int_{x_{n-1}=0}^\infty g_{n-1}(x_{n-1}) e^{\bs{S}x_{n-1}} \wrt x_{n-1}  
            	\\&\qquad\times\int_{x_n=0}^\infty g_{n}(x_n) e^{\bs{S}x_n} \wrt x_n
\end{align}
for \(k,n\in\{0,1,2,\dots\}\), \(k+1<n\). Also recall the row vector function \(\bs k(x): [0,\infty)\to \mathcal A \subset \mathbb R^p\),
\[\bs k(x) = \cfrac{\bs \alpha e^{\bs Sx}}{\bs \alpha e^{\bs Sx}\bs e}.\]

\begin{lem}\label{lem: lh bnd}
	Let \(g_1, g_2, \dots,\) be functions satisfying Assumptions \ref{asu: g}, then, for \(k\in\{1,2,\dots\}\), 
	\begin{align}
		\bs k(x_0)\mathcal I_{1,k}(u_k) 
            	&\leq \cfrac{1}{\bs \alpha e^{\bs{S}x_0}\bs e}G\widehat G^{k-1} \bs \alpha e^{\bs{S}u_k}\bs e.\label{eqn: in here}
	\end{align}
\end{lem}
\begin{proof}
	Recall the definition of \(\bs{D}:=\displaystyle\int_{u=0}^\infty e^{\bs{S}u}\bs s \cfrac{\bs \alpha e^{\bs{S}u}}{\bs \alpha e^{\bs{S}u}\bs e}\wrt u\) and substitute it into the left-hand side of (\ref{eqn: in here}), 
	\begin{align*}
		\bs k(x_0) \mathcal I_{1,k}(u_k) &=\bs k(x_0) \int_{x_1=0}^\infty g_1(x_1) e^{\bs{S}x_1} \bs{D} \mathcal I_{2,k}(u_k)
		\\&=\bs k(x_0)\int_{x_1=0}^\infty g_1(x_1) e^{\bs{S}x_1} \int_{u_1=0}^\infty e^{\bs{S}u_1}\bs s \cfrac{\bs \alpha e^{\bs{S}u_1}}{\bs \alpha e^{\bs{S}u_1}\bs e}\wrt u_1 \mathcal I_{2,k}(u_k).
	%
		%\\&\int_{x_1=0}^\infty g_1(x_1) \cfrac{\bs \alpha e^{\bs{S}(w-y_{\ell_0})}}{\bs \alpha e^{\bs{S}(w-y_{\ell_0})}\bs e} e^{\bs{S}x_1}\wrt x_1 \int_{u_1=0}^\infty e^{\bs{S}u_1}\bs s \cfrac{\bs \alpha e^{\bs{S}u_1}}{\bs \alpha e^{\bs{S}u_1}\bs e}\wrt u_1
            	%\int_{x_2=0}^\infty g_2(x_2) e^{\bs{S}x_2}\wrt x_2 
%		\\&\quad\times\int_{u_2=0}^\infty e^{\bs{S}u_2}\bs s \cfrac{\bs \alpha e^{\bs{S}u_2}}{\bs \alpha e^{\bs{S}u_2}\bs e}\wrt u_2
%		\hdots 
%            	\int_{x_{k-1}=0}^\infty g_{k-1}(x_{k-1}) e^{\bs{S}x_{k-1}} \wrt x_{k-1} \int_{u_{k-1}=0}^\infty e^{\bs{S}u_{k-1}}\bs s \cfrac{\bs \alpha e^{\bs{S}u_{k-1}}}{\bs \alpha e^{\bs{S}u_{k-1}}\bs e}\wrt u_{k-1}
%            	\\&\quad\times\int_{x_k=0}^\infty g_{k}(x_k) e^{\bs{S}x_k} \wrt x_k e^{\bs{S}u_k}\bs s
	\end{align*}
	
	Now, since \(|g_1|\leq G\), then this is less than or equal to
	\begin{align}
		&\bs k(x_0) \int_{x_1=0}^\infty G  e^{\bs{S}x_1} \int_{u_1=0}^\infty e^{\bs{S}u_1}\bs s \cfrac{\bs \alpha e^{\bs{S}u_1}}{\bs \alpha e^{\bs{S}u_1}\bs e}\wrt u_1 \mathcal I_{2,k}(u_k).\label{eqn: int this}
%		\\&G \int_{x_1=0}^\infty  \cfrac{\bs \alpha e^{\bs{S}(w-y_{\ell_0})}}{\bs \alpha e^{\bs{S}(w-y_{\ell_0})}\bs e} e^{\bs{S}x_1}\wrt x_1 \int_{u_1=0}^\infty e^{\bs{S}u_1}\bs s \cfrac{\bs \alpha e^{\bs{S}u_1}}{\bs \alpha e^{\bs{S}u_1}\bs e}\wrt u_1 \nonumber 
%            	\int_{x_2=0}^\infty g_2(x_2) e^{\bs{S}x_2} \wrt x_2 
%		\\&\quad\times\int_{u_2=0}^\infty e^{\bs{S}u_2}\bs s \cfrac{\bs \alpha e^{\bs{S}u_2}}{\bs \alpha e^{\bs{S}u_2}\bs e}\wrt u_2
%            	\hdots\int_{x_{k-1}=0}^\infty g_{k-1}(x_{k-1}) e^{\bs{S}x_{k-1}} \wrt x_{k-1} 
%		\int_{u_{k-1}=0}^\infty e^{\bs{S}u_{k-1}}\bs s \cfrac{\bs \alpha e^{\bs{S}u_{k-1}}}{\bs \alpha e^{\bs{S}u_{k-1}}\bs e}\wrt u_{k-1} \nonumber 
%            	\\&\quad\times\int_{x_k=0}^\infty g_{k}(x_k) e^{\bs{S}x_k} \wrt x_k e^{\bs{S}u_k}\bs s \label{eqn: int this end}
	\end{align}
	Computing the integral with respect to \(x_1\) in (\ref{eqn: int this}) gives 
	\begin{align}
		 &G  \bs k(x_0)(-\bs{S})^{-1} \int_{u_1=0}^\infty e^{\bs{S}u_1}\bs s \cfrac{\bs \alpha e^{\bs{S}u_1}}{\bs \alpha e^{\bs{S}u_1}\bs e}\wrt u_1 \mathcal I_{2,k}(u_k) \nonumber 
		%
		\\&= G \bs k(x_0)\int_{u_1=0}^\infty e^{\bs{S}u_1}\bs e \cfrac{\bs \alpha e^{\bs{S}u_1}}{\bs \alpha e^{\bs{S}u_1}\bs e}\wrt u_1\mathcal I_{2,k}(u_k) \nonumber 
		%
		\\&=  \cfrac{G}{\bs \alpha e^{\bs{S}x_0}\bs e} \int_{u_1=0}^\infty \bs \alpha e^{\bs{S}(x_0+u_1)}\bs e \cfrac{\bs \alpha e^{\bs{S}u_1}}{\bs \alpha e^{\bs{S}u_1}\bs e}\wrt u_1\mathcal I_{2,k}(u_k), \label{eqn: yet another label}
	\end{align}
	since \((-\bs{S})^{-1}\) and \(e^{\bs{S}t}\) commute, \(\bs s = - \bs{S} \bs e \) and \(e^{\bs{S}(t+u)} = e^{\bs{S}t}e^{\bs{S}u}\). 
	
	Now, as \( \bs \alpha e^{\bs{S}(x_0 +u_1)}\bs e \leq \bs \alpha e^{\bs{S}u_1}\bs e \), then (\ref{eqn: yet another label}) is less than or equal to 
	\begin{align}
		%&G  \cfrac{1}{\bs \alpha e^{\bs{S}(w-y_{\ell_0})}\bs e} \int_{u_1=0}^\infty \bs \alpha e^{\bs{S}(w-y_{\ell_0}+u_1)}\bs e \cfrac{\bs \alpha e^{\bs{S}u_1}}{\bs \alpha e^{\bs{S}u_1}\bs e}\wrt u_1
%            	\int_{x_2=0}^\infty g_2(x_2) e^{\bs{S}x_2} \wrt x_2 \int_{u_2=0}^\infty e^{\bs{S}u_2}\bs s \cfrac{\bs \alpha e^{\bs{S}u_2}}{\bs \alpha e^{\bs{S}u_2}\bs e}\wrt u_2
%		\\&\quad\hdots 
%            	\int_{x_{k-1}=0}^\infty g_{k-1}(x_{k-1}) e^{\bs{S}x_{k-1}} \wrt x_{k-1} \int_{u_{k-1}=0}^\infty e^{\bs{S}u_{k-1}}\bs s \cfrac{\bs \alpha e^{\bs{S}u_{k-1}}}{\bs \alpha e^{\bs{S}u_{k-1}}\bs e}\wrt u_{k-1}
%            	\int_{x_k=0}^\infty g_{k}(x_k) e^{\bs{S}x_k} \wrt x_k e^{\bs{S}u_k}\bs s
%		%
		&G  \cfrac{1}{\bs \alpha e^{\bs{S}x_0}\bs e} \int_{u_1=0}^\infty \bs \alpha e^{\bs{S}u_1}\bs e \cfrac{\bs \alpha e^{\bs{S}u_1}}{\bs \alpha e^{\bs{S}u_1}\bs e}\wrt u_1 \mathcal I_{2,k}(u_k) \nonumber
%            	\int_{x_2=0}^\infty g_2(x_2) e^{\bs{S}x_2} \wrt x_2 \int_{u_2=0}^\infty e^{\bs{S}u_2}\bs s \cfrac{\bs \alpha e^{\bs{S}u_2}}{\bs \alpha e^{\bs{S}u_2}\bs e}\wrt u_2
%		\\&\quad\hdots 
%            	\int_{x_{k-1}=0}^\infty g_{k-1}(x_{k-1}) e^{\bs{S}x_{k-1}} \wrt x_{k-1} \int_{u_{k-1}=0}^\infty e^{\bs{S}u_{k-1}}\bs s \cfrac{\bs \alpha e^{\bs{S}u_{k-1}}}{\bs \alpha e^{\bs{S}u_{k-1}}\bs e}\wrt u_{k-1}
%            	\int_{x_k=0}^\infty g_{k}(x_k) e^{\bs{S}x_k}\wrt x_k e^{\bs{S}u_k}\bs s,
%	\end{align*}
%	This is equal to 
%	\begin{align}
		=G  \cfrac{1}{\bs \alpha e^{\bs{S} x_0 }\bs e} \int_{u_1=0}^\infty \bs \alpha e^{\bs{S}u_1}\wrt u_1 \mathcal I_{2,k}(u_k) . %G  \cfrac{1}{\bs \alpha e^{\bs{S}(w-y_{\ell_0})}\bs e} \int_{u_1=0}^\infty  \bs \alpha e^{\bs{S}u_1} \wrt u_1
%            	\int_{x_2=0}^\infty g_2(x_2) e^{\bs{S}x_2} \wrt x_2 \int_{u_2=0}^\infty e^{\bs{S}u_2}\bs s \cfrac{\bs \alpha e^{\bs{S}u_2}}{\bs \alpha e^{\bs{S}u_2}\bs e}\wrt u_2 \label{eqn: rep from here}
%		\\&\quad\hdots 
%            	\int_{x_{k-1}=0}^\infty g_{k-1}(x_{k-1}) e^{\bs{S}x_{k-1}} \wrt x_{k-1} \int_{u_{k-1}=0}^\infty e^{\bs{S}u_{k-1}}\bs s \cfrac{\bs \alpha e^{\bs{S}u_{k-1}}}{\bs \alpha e^{\bs{S}u_{k-1}}\bs e}\wrt u_{k-1}
%            	\int_{x_k=0}^\infty g_{k}(x_k) e^{\bs{S}x_k} \wrt x_k e^{\bs{S}u_k}\bs s.  \nonumber 
	\end{align}
	Now integrate with respect to \(u_1\) and use the facts that \((-\bs{S})^{-1}\) and \(e^{\bs{S}x}\) commute, and \(\bs s = - \bs{S} \bs e \), to get 
	\begin{align}
		\nonumber & G  \cfrac{1}{\bs \alpha e^{\bs{S}x_0}\bs e} \bs \alpha (-\bs{S})^{-1}  \mathcal I_{2,k}(u_k) 
		\\& = G  \cfrac{1}{\bs \alpha e^{\bs{S}x_0}\bs e}  \bs \alpha (-\bs{S})^{-1} \int_{x_2=0}^\infty g_2(x_2)  e^{\bs{S}x_2} \wrt x_2 \int_{u_2=0}^\infty e^{\bs{S}u_2}\bs s \cfrac{\bs \alpha e^{\bs{S}u_2}}{\bs \alpha e^{\bs{S}u_2}\bs e}\wrt u_2 \mathcal I_{3,k}(u_k)\label{eqn: rep from here}
		\\& = G  \cfrac{1}{\bs \alpha e^{\bs{S}x_0}\bs e}  \int_{x_2=0}^\infty g_2(x_2) \bs \alpha e^{\bs{S}x_2} \wrt x_2 \int_{u_2=0} ^\infty e^{\bs{S}u_2}\bs e \cfrac{\bs \alpha e^{\bs{S}u_2}}{\bs \alpha e^{\bs{S}u_2}\bs e}\wrt u_2 \mathcal I_{3,k}(u_k)\label{eqn: anoth ref here}
	\end{align}
	Since \(\bs \alpha e^{\bs{S}x_2}e^{\bs{S}u_2}\bs e \leq \bs \alpha e^{\bs{S}u_2}\bs e \), then (\ref{eqn: anoth ref here}) is less than or equal to 
	\begin{align}
		& G  \cfrac{1}{\bs \alpha e^{\bs{S}x_0}\bs e}  \int_{x_2=0}^\infty g_2(x_2) \wrt x_2 \int_{u_2=0}^\infty \bs \alpha e^{\bs{S}u_2}\bs e \cfrac{\bs \alpha e^{\bs{S}u_2}}{\bs \alpha e^{\bs{S}u_2}\bs e}\wrt u_2 \mathcal I_{3,k}(u_k) \nonumber
		\\& =G  \cfrac{1}{\bs \alpha e^{\bs{S}x_0 }\bs e}  \widehat G  \int_{u_2=0}^\infty \bs \alpha e^{\bs{S}u_2}\bs e \cfrac{\bs \alpha e^{\bs{S}u_2}}{\bs \alpha e^{\bs{S}u_2}\bs e}\wrt u_2 \mathcal I_{3,k}(u_k) \nonumber
		\\& =G  \cfrac{1}{\bs \alpha e^{\bs{S} x_0 }\bs e}  \widehat G  \int_{u_2=0}^\infty \bs \alpha e^{\bs{S}u_2} \mathcal I_{3,k}(u_k) \nonumber
		\\& =G  \cfrac{1}{\bs \alpha e^{\bs{S} x_0 }\bs e}  \widehat G \bs \alpha (-\bs S)^{-1} \mathcal I_{3,k}(u_k).  \label{eqn: rep to here}
%		\\&G  \cfrac{1}{\bs \alpha e^{\bs{S}(w-y_{\ell_0})}\bs e}
%            	\int_{x_2=0}^\infty g_2(x_2)  \bs \alpha \wrt x_2 \int_{u_2=0}^\infty e^{\bs{S}u_2}\bs e \cfrac{\bs \alpha e^{\bs{S}u_2}}{\bs \alpha e^{\bs{S}u_2}\bs e}\wrt u_2\hdots 
%            	\int_{x_{k-1}=0}^\infty g_{k-1}(x_{k-1}) e^{\bs{S}x_{k-1}}\wrt x_{k-1}  \nonumber 
%		%
%		\\&\int_{u_{k-1}=0}^\infty e^{\bs{S}u_{k-1}}\bs s \cfrac{\bs \alpha e^{\bs{S}u_{k-1}}}{\bs \alpha e^{\bs{S}u_{k-1}}\bs e}\wrt u_{k-1}
%		\int_{x_k=0}^\infty g_{k}(x_k) e^{\bs{S}x_k} \wrt x_k e^{\bs{S}u_k}\bs s \nonumber 
%		%
%		\\&=G  \cfrac{1}{\bs \alpha e^{\bs{S}(w-y_{\ell_0})}\bs e}  
%            	\widehat G  \int_{u_2=0}^\infty \bs \alpha e^{\bs{S}u_2}\wrt u_2  \int_{x_3=0}^\infty e^{\bs{S}x_3} g_3(x_3) \wrt x_3 \int_{u_3=0}^\infty e^{\bs{S}u_3}\bs e \cfrac{\bs \alpha e^{\bs{S}u_3}}{\bs \alpha e^{\bs{S}u_3}\bs e}\wrt u_3 \nonumber 
%		\\&\quad\hdots 
%            	\int_{x_{k-1}=0}^\infty g_{k-1}(x_{k-1}) e^{\bs{S}x_{k-1}} \wrt x_{k-1} 
%		\int_{u_{k-1}=0}^\infty e^{\bs{S}u_{k-1}}\bs s \cfrac{\bs \alpha e^{\bs{S}u_{k-1}}}{\bs \alpha e^{\bs{S}u_{k-1}}\bs e}\wrt u_{k-1}
%            	\int_{x_k=0}^\infty g_{k}(x_k) e^{\bs{S}x_k} \wrt x_k e^{\bs{S}u_k}\bs s \label{eqn: rep to here}
	\end{align}
	Repeating the arguments which got us from (\ref{eqn: rep from here}) to (\ref{eqn: rep to here}) another \(k-2\) times gives the result.
\end{proof}

\begin{lem}\label{lem: rh bnd}
	Let \(g_1, g_2, \dots,\) be functions satisfying the Assumptions \ref{asu: g} and let \(V(x)\) be a closing operator with the Properties \ref{properties: some props}, then, for \(k,n\in\{1,2,\dots\},\, k+1 < n\), 
	\begin{align*}
%		&g_{k+1}(x_{k+1})\cfrac{\bs \alpha e^{\bs{S}u_{k}}}{\bs \alpha e^{\bs{S}u_{k}}\bs e} e^{\bs{S}x_{k+1}} \bs{D} \int_{x_{k+2}=0}^\infty g_{k+2}(x_{k+2}) e^{\bs{S}x_{k+2}} \wrt x_{k+2} \bs{D} \hdots 
%		\int_{x_{n-1}=0}^\infty g_{n-1}(x_{n-1}) e^{\bs{S}x_{n-1}} \wrt x_{n-1}  
%		\\&
%            	\times \bs{D}\int_{x_n=0}^\infty g_{n}(x_n) e^{\bs{S}x_n} \wrt x_n e^{\bs{S}(y_{\ell_0}+\Delta-x)}\bs s
            	\mathcal J_{k+1,n}(u_k,x_{k+1})  V(x) \leq  g_{k+1}(x_{k+1})\widehat G^{n-k-2} G G_V.
	\end{align*}
\end{lem}
\begin{proof}
	Starting with the left-hand side, upon substituting \(\bs{D}\), 
	\begin{align}
		& \mathcal J_{k+1,n}(u_k,x_{k+1})  V(x)  \nonumber 
		\\&= \mathcal J_{k+1,n-1}(u_k,x_{k+1})  \bs{D}
		\int_{x_n=0}^\infty g_{n}(x_n) e^{\bs{S}x_n} \wrt x_nV(x) \nonumber
		\\&= \mathcal J_{k+1,n-1}(u_k,x_{k+1})  \int_{u_{n-1}=0}^\infty e^{\bs{S}u_{n-1}}\bs s \cfrac{\bs \alpha e^{\bs{S}u_{n-1}}}{\bs \alpha e^{\bs{S}u_{n-1}}\bs e}\wrt  u_{n-1}
		\int_{x_n=0}^\infty g_{n}(x_n) e^{\bs{S}x_n} \wrt x_nV(x) \nonumber
		\\&\leq \mathcal J_{k+1,n-1}(u_k,x_{k+1})  \int_{u_{n-1}=0}^\infty e^{\bs{S}u_{n-1}}\bs s \cfrac{\bs \alpha e^{\bs{S}u_{n-1}}}{\bs \alpha e^{\bs{S}u_{n-1}}\bs e}\wrt  u_{n-1}
		\int_{x_n=0}^\infty G e^{\bs{S}x_n} \wrt x_nV(x). \label{eqn: bnd again}
%		\\& g_{k+1}(x_{k+1}) \cfrac{\bs \alpha e^{\bs{S}u_{k}}}{\bs \alpha e^{\bs{S}u_{k}}\bs e} e^{\bs{S}x_{k+1}}\int_{u_{k+1}=0}^\infty e^{\bs{S}u_{k+1}}\bs s \cfrac{\bs \alpha e^{\bs{S}u_{k+1}}}{\bs \alpha e^{\bs{S}u_{k+1}}\bs e}\wrt u_{k+1} 
%		 \int_{x_{k+2}=0}^\infty g_{k+2}(x_{k+2}) e^{\bs{S}x_{k+2}} \wrt x_{k+2} \nonumber 
%		 \\& \int_{u_{k+2}=0}^\infty e^{\bs{S}u_{k+2}}\bs s \cfrac{\bs \alpha e^{\bs{S}u_{k+2}}}{\bs \alpha e^{\bs{S}u_{k+2}}\bs e}\wrt u_{k+2} 
%            	\hdots 
%		\int_{x_{n-1}=0}^\infty g_{n-1}(x_{n-1}) e^{\bs{S}x_{n-1}} \wrt x_{n-1}   \nonumber 
%		\\&\quad\times\int_{u_{n-1}=0}^\infty e^{\bs{S}u_{n-1}}\bs s \cfrac{\bs \alpha e^{\bs{S}u_{n-1}}}{\bs \alpha e^{\bs{S}u_{n-1}}\bs e}\wrt  u_{n-1}
%		\int_{x_n=0}^\infty g_{n}(x_n) e^{\bs{S}x_n} \wrt x_n e^{\bs{S}(y_{\ell_0}+\Delta-x)}\bs s.
	\end{align}
	By the Property \ref{properties: 1} of \(V(x)\), \(\displaystyle\int_{x_n=0}^\infty \bs \alpha e^{\bs{S}u_{n-1}}e^{\bs{S}x_n} V(x) \wrt x_n  \leq \bs \alpha e^{\bs{S}u_{n-1}}\bs eG_V\). Therefore (\ref{eqn: bnd again}) is less than or equal to 
	\begin{align}
		&\mathcal J_{k+1,n-1}(u_k,x_{k+1})  \int_{u_{n-1}=0}^\infty e^{\bs{S}u_{n-1}}\bs s \cfrac{\bs \alpha e^{\bs{S}u_{n-1}}\bs e}{\bs \alpha e^{\bs{S}u_{n-1}}\bs e}\wrt  u_{n-1} G  G_V\nonumber 
		%
		\\& = \mathcal J_{k+1,n-1}(u_k,x_{k+1})  \int_{u_{n-1}=0}^\infty e^{\bs{S}u_{n-1}}\bs s \wrt  u_{n-1} G  G_V\nonumber
		%
		\\& = \mathcal J_{k+1,n-1}(u_k,x_{k+1})  \bs e G  G_V\label{eqn: mid ref} 
		%
		\\& = \mathcal J_{k+1,n-2}(u_k,x_{k+1}) \int_{u_{n-2}=0}^\infty e^{\bs{S}u_{n-2}}\bs s \cfrac{\bs \alpha e^{\bs{S}u_{n-2}}}{\bs \alpha e^{\bs{S}u_{n-2}}\bs e}\wrt u_{n-2}  \int_{x_{n-1}=0}^\infty g_{n-1}(x_{n-1}) e^{\bs{S}x_{n-1}} \wrt x_{n-1} \bs e \nonumber 
		\\&\qquad {} \times G  G_V.\label{eqn: this}
	\end{align}
	Now, since \(\bs\alpha e^{\bs{S}(x_{n-1}+u_{n-2})}\bs e\leq  \bs\alpha e^{\bs{S}(u_{n-2})}\bs e\), then (\ref{eqn: this}) is less than or equal to
	\begin{align}
		&\mathcal J_{k+1,n-2}(u_k,x_{k+1}) \int_{u_{n-2}=0}^\infty e^{\bs{S}u_{n-2}}\bs s \cfrac{\bs \alpha e^{\bs{S}u_{n-2}}\bs e}{\bs \alpha e^{\bs{S}u_{n-2}}\bs e}\wrt u_{n-2}  \int_{x_{n-1}=0}^\infty g_{n-1}(x_{n-1})\wrt x_{n-1} G  G_V\nonumber
		\\& = \mathcal J_{k+1,n-2}(u_k,x_{k+1}) \int_{u_{n-2}=0}^\infty e^{\bs{S}u_{n-2}}\bs s \wrt u_{n-2} \widehat G G G_V \nonumber
		%
		\\& = \mathcal J_{k+1,n-2}(u_k,x_{k+1}) \bs e \widehat G G G_V. \label{eqn: ref here too}
		%
%		\\&g_{k+1}(x_{k+1})\cfrac{\bs \alpha e^{\bs{S}u_{k}}}{\bs \alpha e^{\bs{S}u_{k}}\bs e} e^{\bs{S}x_{k+1}} \int_{u_{k+1}=0}^\infty e^{\bs{S}u_{k+1}}\bs s \cfrac{\bs \alpha e^{\bs{S}u_{k+1}}}{\bs \alpha e^{\bs{S}u_{k+1}}\bs e}\wrt u_{k+1} 
%		\int_{x_{k+2}=0}^\infty g_{k+2}(x_{k+2}) e^{\bs{S}x_{k+2}} \wrt x_{k+2} 
%		\\&\int_{u_{k+2}=0}^\infty e^{\bs{S}u_{k+2}}\bs s \cfrac{\bs \alpha e^{\bs{S}u_{k+2}}}{\bs \alpha e^{\bs{S}u_{k+2}}\bs e}\wrt u_{k+2} 
%            	\hdots \times
%		 \int_{u_{n-2}=0}^\infty e^{\bs{S}u_{n-2}}\bs s \cfrac{\bs \alpha e^{\bs{S}u_{n-2}}\bs e}{\bs \alpha e^{\bs{S}u_{n-2}}\bs e}\wrt u_{n-2}  
%		 \\&\quad\times\int_{x_{n-1}=0}^\infty g_{n-1}(x_{n-1}) \wrt x_{n-1}
%            	G_{n},
%	%
%		\\&=g_{k+1}(x_{k+1})\cfrac{\bs \alpha e^{\bs{S}u_{k}}}{\bs \alpha e^{\bs{S}u_{k}}\bs e} e^{\bs{S}x_{k+1}} \int_{u_{k+1}=0}^\infty e^{\bs{S}u_{k+1}}\bs s \cfrac{\bs \alpha e^{\bs{S}u_{k+1}}}{\bs \alpha e^{\bs{S}u_{k+1}}\bs e}\wrt u_{k+1} 
%		\int_{x_{k+2}=0}^\infty g_{k+2}(x_{k+2}) e^{\bs{S}x_{k+2}} \wrt x_{k+2} 
%		\\&\int_{u_{k+2}=0}^\infty e^{\bs{S}u_{k+2}}\bs s \cfrac{\bs \alpha e^{\bs{S}u_{k+2}}}{\bs \alpha e^{\bs{S}u_{k+2}}\bs e}\wrt u_{k+2} 
%		\hdots \times 
%		 \int_{u_{n-2}=0}^\infty e^{\bs{S}u_{n-2}}\bs s \wrt u_{n-2}  \widehat G_{n-1}
%            	G_{n}.
	\end{align} 
	This is of the same form as (\ref{eqn: mid ref}), hence repeating the same arguments which got us from (\ref{eqn: mid ref}) to (\ref{eqn: ref here too}) another \(n-k-3\) more times gives
	 \begin{align*}
		\mathcal J_{k+1,k+1}(u_k,x_{k+1}) \bs e  \widehat G^{n-k-2}G G_V
		%
		&\leq g_{k+1}(x_{k+1}) \cfrac{\bs\alpha e^{\bs{S}(u_k+x_{k+1})}}{\bs \alpha e^{\bs{S}u_k}\bs e} \bs e\widehat G^{n-k-2}G G_V
		%
		\\& \leq g_{k+1}(x_{k+1}) \widehat G^{n-k-2}G G_V.
	\end{align*} 
\end{proof}	
\begin{cor}\label{cor: ksjkd}
	Let \(g_1, g_2, \dots,\) be functions satisfying the Assumptions \ref{asu: g} and let \(V(x)\) be a closing operator with the Properties \ref{properties: some props}, then,
	\begin{align}
		&\int_{x_1=0}^\infty g_1(x_1) \bs k(x_0) e^{\bs{S}x_1}\wrt x_1\bs D 
            	\left[\prod_{k=2}^{n-1}\int_{x_k=0}^\infty g_k(x_k) e^{\bs{S}x_k} \wrt x_k \bs D\right] \int_{x_n=0}^\infty g_{n}(x_n) e^{\bs{S}x_n} \wrt x_n V(x) \nonumber 
		\\&\leq \widehat{G}^{n-1}GG_V \label{eqn :mmmm}
	\end{align}
\end{cor}
\begin{proof}
	The left-hand side of (\ref{eqn :mmmm}) can be seen to be equivalent to \(\mathcal J_{1,n+1}(x_0,x_1),\) with \(g_1(x_1)=1\), and the integrability condition on \(g_1\) is not required to prove the bound. 
\end{proof}

\begin{cor}\label{cor: lh and rh}
	Let \(g_1, g_2, \dots,\) be functions satisfying the Assumptions \ref{asu: g} and let \(V(x)\) be a closing operator with the Properties \ref{properties: some props}, then, for \(k,n \in \{1,2,\dots\}\), \(k+1\leq n\),
	\begin{align}
		&\int_{x_{k+1}=0}^\infty \int_{u_k=\Delta-\varepsilon}^\infty \bs k(x_0)\mathcal I_{1,k}(u_k) \mathcal J_{k+1,n}(u_k,x_{k+1})V(x) \nonumber
		%
            	\\&\leq \left(2\varepsilon + \cfrac{\var(Z)}{\varepsilon}\right) \cfrac{1}{\bs \alpha e^{\bs{S}x_0}\bs e} G \widehat G^{n-2} G G_V =: |r_4(n)|. \label{eqn: the result tail}
	\end{align}
\end{cor}
\begin{proof}
	Consider first \(k+1<n\). Combining Lemmas \ref{lem: lh bnd} and \ref{lem: rh bnd} the left-hand side of (\ref{eqn: the result tail}) is less than or equal to 
	\begin{align}
		&\cfrac{1}{\bs \alpha e^{\bs{S} x_0 }\bs e}G \widehat G^{k-1}
		\int_{x_{k+1}=0}^\infty \int_{u_k=\Delta-\varepsilon}^\infty \bs \alpha e^{\bs{S}u_k}\bs e g_{k+1}(x_{k+1}) \wrt u_k \wrt x_{k+1}\widehat G^{n-k-2} G G_V \label{eqn: yet yet another label}
		%
		\\&=\cfrac{1}{\bs \alpha e^{\bs{S} x_0}\bs e}G \widehat G^{k-1}  
		\int_{u_k=\Delta-\varepsilon}^\infty \bs \alpha e^{\bs{S}u_k}\bs e \wrt u_k \widehat G_{k+1} \widehat G^{n-k-2} G G_V. \label{eqnL afejhm789}
	\end{align}
	Now 
	\begin{align}
		\int_{u_k=\Delta-\varepsilon}^\infty \bs \alpha e^{\bs{S}u_k}\bs e \wrt u_k &= \int_{u_k=\Delta-\varepsilon}^{\Delta+\varepsilon} \mathbb P(Z> u_k) \wrt u_k + \int_{u_k=\Delta+\varepsilon}^\infty \mathbb P(Z> u_k) \wrt u_k\nonumber
		%
		\\&\leq \int_{u_k=\Delta-\varepsilon}^{\Delta+\varepsilon} \wrt u_k + \int_{u_k=\Delta+\varepsilon}^\infty \cfrac{\var(Z)}{(u_k-\Delta)^2} \wrt u_k\nonumber
		% 
		\\&= 2\varepsilon + \cfrac{\var(Z)}{\varepsilon},\label{eqn:kdjf55}
	\end{align}
	where we have used Chebyshev's inequality to bound the tail probability, 
	\[\mathbb P(Z> u_k) \leq \mathbb P(|Z-\Delta|> |u_k-\Delta|) \leq \cfrac{\var(Z)}{(u_k-\Delta)^2},\]
	for \(u_k \geq \Delta +\varepsilon\). Hence (\ref{eqnL afejhm789}) is less than or equal to 
	\[\cfrac{1}{\bs \alpha e^{\bs{S} x_0}\bs e}G \widehat G^{k-1}  
		\left(2\varepsilon + \cfrac{\var(Z)}{\varepsilon}\right) \widehat G_{k+1} \widehat G^{n-k-2} G G_V.\]
	
	Now consider \(k+1=n\). By Lemma~\ref{lem: lh bnd} we have 
	\begin{align}
		&\cfrac{1}{\bs \alpha e^{\bs{S} x_0 }\bs e}G \widehat G^{k-1}
		\int_{x_{k+1}=0}^\infty \int_{u_k=\Delta-\varepsilon}^\infty \bs \alpha e^{\bs{S}u_k}\bs e g_{k+1}(x_{k+1}) \cfrac{\bs\alpha e^{\bs{S}(u_k+x_{k+1})}}{\bs \alpha e^{\bs{S}u_k}\bs e}V(x)\wrt u_k \wrt x_{k+1}. \label{eqn: yet yet another label 2}
		%
		\end{align}
		{Since \(g_{k+1}\leq G\), and upon integrating over \(x_{k+1}\), then (\ref{eqn: yet yet another label 2}) is less than or equal to }
		\begin{align}
		& \cfrac{1}{\bs \alpha e^{\bs{S} x_0 }\bs e}G \widehat G^{k-1}  
		\int_{u_k=\Delta-\varepsilon}^\infty G \bs\alpha e^{\bs{S}u_k}(-\bs S)^{-1}V(x) \wrt u_k\nonumber 
		%
		\\&\leq \cfrac{1}{\bs \alpha e^{\bs{S} x_0 }\bs e}G \widehat G^{k-1}  
		\int_{u_k=\Delta-\varepsilon}^\infty G \bs \alpha e^{\bs S u_k} \bs e G_V \wrt u_k , \label{eqn: aksgm}
	\end{align}
	where we have used Property \ref{properties: 1} to get the upper bound on the right-hand side of (\ref{eqn: aksgm}). Using (\ref{eqn:kdjf55}) again, then (\ref{eqn: aksgm}) is less than or equal to
	\begin{align}
		\cfrac{1}{\bs \alpha e^{\bs{S} x_0 }\bs e}G\widehat G^{n-2}   G G_V\left(2\varepsilon + \cfrac{\var\left(Z\right)}{\varepsilon}\right).
	\end{align}
	This completes the proof.  
\end{proof}

The error term \(r_4(n)\) depends on \(p\) and we write \(r_4^{(p)}(n)\) when we need to make this dependence explicit, otherwise it is omitted from the notation. Upon choosing \(\varepsilon = \var(Z^{(p)})^{1/3}\), then for fixed \(n<\infty\) the error term \(|r_4^{(p)}(n)|= O\left(\var\left(Z^{(p)}\right)^{1/3}\right)\to 0\) as \(p\to\infty\). 

Define \(\displaystyle\bs D(b) = \int_{u=0}^be^{\bs Su}\bs s \cfrac{\bs\alpha e^{\bs S u}}{\bs \alpha e^{\bs S u}\bs e} \wrt u.\) Also define 
	\begin{align}
		g^*_{2,n}(u_1,x) &:= \int_{u_2=0}^{\Delta-u_1}g_2(\Delta - u_2 - u_1)\wrt u_1 \dots \nonumber 
            	\int_{u_{n-1}=0}^{\Delta-u_{n-2}} g_{n-1}(\Delta - u_{n-1} - u_{n-2}) \wrt u_{n-2}
            	\\&\qquad{}g_{n}(\Delta - x-u_{n-1})1(\Delta-x-u_{n-1}\geq0)\wrt u_{n-1},
		%
		\\g^*_{1,n}(x_0,x) &:= \int_{u_1=0}^{\Delta-x_0}g_1(\Delta - u_1 - x_0)g^*_{2,n}(u_1,x)\wrt u_1,
	\end{align}
	and 
	\begin{align}
		g^{*,\varepsilon}_{1,n}(x_0,x) &:= \int_{u_1=0}^{\Delta-\varepsilon-x_0}g_1(\Delta - u_1 - x_0)
		\int_{u_2=0}^{\Delta-\varepsilon-u_1}g_2(\Delta - u_2 - u_1)\wrt u_1  \nonumber 
		\\&\quad\hdots 
            	\int_{u_{n-1}=0}^{\Delta-\varepsilon-u_{n-2}} g_{n-1}(\Delta - u_{n-1} - u_{n-2}) \wrt u_{n-2}
            	g_{n}(\Delta-x-u_{n-1}) \nonumber 
		\\&\quad{}\times 1(\Delta-x-u_{n-1}\geq\varepsilon).
	\end{align}
	For later, observe that 
	\begin{align}
		g^*_{2,n}(u_1,x) &= \int_{u_2=0}^{\Delta-u_1}g_2(\Delta - u_2 - u_1)\wrt u_1 \dots \nonumber 
            	\int_{u_{n-1}=0}^{\Delta-u_{n-2}} g_{n-1}(\Delta - u_{n-1} - u_{n-2}) \wrt u_{n-2}
            	\\&\qquad{}g_{n}(\Delta - x-u_{n-1})1(\Delta-x-u_{n-1}\geq0)\wrt u_{n-1} \nonumber
	%
		\\&\leq G^{n-1}\int_{u_2=0}^{\Delta-u_1}\wrt u_1 \dots\nonumber
            	\int_{u_{n-1}=0}^{\Delta-u_{n-2}}  \wrt u_{n-1}
	%
		\\&\leq G^{n-1}\Delta^{n-2}:=G^*_n.
	\end{align}

\begin{lem}\label{lem: lst convergence}
	Let \(g_1,g_2,\dots,\) be functions satisfying the Assumptions \ref{asu: g} and let \(V(x)\) be a closing operator with the Properties \ref{properties: some props}. Then, for \(n\geq 2\),  
	\begin{align}
		&\int_{x_1=0}^\infty g_1(x_1) \bs k(x_0) e^{\bs{S}x_1}\wrt x_1 \bs D(\Delta-\varepsilon)
            	\left[\prod_{k=2}^{n-1}\int_{x_k=0}^\infty g_k(x_k) e^{\bs{S}x_k} \wrt x_k \right]
		\bs D(\Delta-\varepsilon) \nonumber 
%		\hdots
%            	\int_{x_{n-1}=0}^\infty g_{n-1}(x_{n-1}) e^{\bs{S}x_{n-1}} \wrt x_{n-1} \int_{u_{n-1}=0}^{\Delta-\varepsilon} e^{\bs{S}u_{n-1}}\bs s \cfrac{\bs \alpha e^{\bs{S}u_{n-1}}}{\bs \alpha e^{\bs{S}u_{n-1}}\bs e}\wrt u_{n-1} \nonumber 
		\\&\qquad\times\int_{x_n=0}^\infty g_{n}(x_n) e^{\bs{S}x_n} \wrt x_n V(x) \nonumber 
	%
		\\& =g^{*}_{1,n}(x_0,x) + r_5(n) + r_6(n), \label{eqn: rhs g 2}
	\end{align}
	where  
	\begin{align*}
		|r_5(n)|&= O\left(\max\left\{G^{n-1}\Delta^{n-2}\left(\frac{1}{2}\Delta|r_2 |+ 2\varepsilon G 
		%
		+ \cfrac{1}{2}\Delta G\cfrac{\var(Z)/\varepsilon^2}{1-\var(Z)/\varepsilon^2}\right),
		G^{n-1}\Delta^{n-2}R_{V,1}\right\}\right),
		%
		\\|r_6(n)| &\leq \varepsilon^{n-2}G^{n-1}
	\end{align*}
\end{lem}
\begin{proof}
	Rewrite the left-hand side of (\ref{eqn: rhs g 2}) as 
	\begin{align*}
		& \int_{u_1=0}^{\Delta-\varepsilon} \int_{x_1=0}^\infty \cfrac{\bs \alpha e^{\bs{S}(x_{0}+x_1+u_1)}\bs s}{\bs \alpha e^{\bs{S}x_0}\bs e}g_1(x_1) \wrt u_1\wrt x_1 \nonumber 
		\\&{}\quad\times\left[\prod_{\ell=2}^{n-1}\int_{u_\ell=0}^{\Delta-\varepsilon} \int_{x_\ell=0}^\infty \cfrac{\bs \alpha e^{\bs{S}(u_{\ell-1}+x_\ell+u_\ell)}\bs s}{\bs \alpha e^{\bs{S}u_{\ell-1}}\bs e}g_\ell(x_\ell) \wrt u_\ell\wrt x_\ell \right]
            	\\&{}\quad\times\int_{x_n=0}^\infty \cfrac{\bs \alpha e^{\bs{S}(u_{n-1}+x_n )}}{\bs \alpha e^{\bs{S}u_{n-1}}\bs e} V(x)g_{n}(x_n)\wrt u_{n-1} \wrt x_n,
	\end{align*}
	then we see that we can apply Corollary~\ref{cor: cond bnd 2} to all of the integrals over \(x_k,\, k=1,\dots,n-1\) and use Property \ref{properties: 2} of \(V(x)\) to get  
	\begin{align*}
		& \int_{u_1=0}^{\Delta-\varepsilon}\left[g_1(\Delta - u_1 - x_0)1(u_1 + x_0\leq \Delta - \varepsilon) + r_3 (u_1 + x_0)\right]
		\\&\quad\times\int_{u_2=0}^{\Delta-\varepsilon}\left[g_2(\Delta - u_2 - u_1)1(u_2 + u_1\leq \Delta - \varepsilon) + r_3 (u_2 + u_1)\right]\wrt u_1
		\\&\quad\hdots 
            	 \int_{u_{n-1}=0}^{\Delta-\varepsilon}  \left[g_{n-1}(\Delta - u_{n-1} - u_{n-2}) 1(u_{n-1} + u_{n-2}\leq \Delta - \varepsilon) +   r_3 (u_{n-1} + u_{n-2})\right] \wrt u_{n-2}
            	\\&\quad\times\left[g_{n}(\Delta-u_{n-1}-x)1(u_{n-1}+x\leq \Delta - \varepsilon) + r_V (u_{n-1},x)\right]\wrt u_{n-1}
		%
		\\&=g^{*,\varepsilon}_{1,n}(x_0,x) + r_5(n)
	\end{align*}
	where \(r_5(n)\) is an error term. The leading terms of \(r_5(n)\) are of the form 
	\begin{align*}
		&\int_{u_1=0}^{\Delta-\varepsilon-x_0}g_1(\Delta - u_1 - x_0)
		\int_{u_2=0}^{\Delta-\varepsilon-u_1}g_2(\Delta - u_2 - u_1)\wrt u_1
		\\&\quad\hdots\int_{u_{k-1}=0}^{\Delta-\varepsilon-u_{k-2}}g_{k-1}(\Delta - u_{k-1} - u_{k-2}) \wrt u_{k-2}
		\int_{u_k=0}^{\Delta-\varepsilon}r_3(u_{k}+u_{k-1}) \wrt u_{k-1}
		\\&\quad\times\int_{u_{k+1}=0}^{\Delta-\varepsilon-u_{k}}g_{k+1}(\Delta - u_{k+1} - u_{k}) \wrt u_{k}
		\hdots
            	\int_{u_{n-1}=0}^{\Delta-\varepsilon-u_{n-2}} g_{n-1}(\Delta - u_{n-1} - u_{n-2}) \wrt u_{n-2}
            	\\&\quad\times g_{n}(\Delta-u_{n-1}-x)1(u_{n-1}+x\leq \Delta -\varepsilon)\wrt u_{n-1} 
		%
		\\&\leq G^{k-1} \Delta^{k-2} \int_{u_{k-1}=0}^{\Delta-\varepsilon}
		\int_{u_k=0}^{\Delta-\varepsilon }r_3(u_{k}+u_{k-1}) \wrt u_k \wrt u_{k-1} G^{n-k}\Delta^{n-k-1},
	\end{align*} 
	and 
	\begin{align*}
		&\int_{u_1=0}^{\Delta-\varepsilon-x_0}g_1(\Delta - u_1 - x_0)
		\int_{u_2=0}^{\Delta-\varepsilon-u_1}g_2(\Delta - u_2 - u_1)\wrt u_1
		\\&{}\hdots
            	\int_{u_{n-1}=0}^{\Delta-\varepsilon-u_{n-2}} g_{n-1}(\Delta - u_{n-1} - u_{n-2}) \wrt u_{n-2}
            	 r_V(u_{n-1},x)\wrt u_{n-1} 
		%
		\\&\leq G^{n-1} \Delta^{n-2} \int_{u_{n-1}=0}^{\Delta-\varepsilon}
		 r_V(u_{n-1},x) \wrt u_{n-1}.
	\end{align*} 
	Now, 
	\begin{align*}
		 &\left|\int_{u_{k-1}=0}^{\Delta-\varepsilon}\int_{u_k=0}^{\Delta-\varepsilon}r_3(u_{k}+u_{k-1}) \wrt u_k \wrt u_{k-1} \right|
		\\& \leq \int_{u_{k-1}=0}^{\Delta-\varepsilon}\Bigg[ \int_{u_k=u_{k-1}}^{\Delta-\varepsilon} | r_3(u_k) |\wrt u_k + \int_{u_k=\Delta-\varepsilon}^{\Delta+\varepsilon} |r_3(u_k)|\wrt u_k 
		%
		\\&\qquad{}+ \int_{u_k = \Delta+\varepsilon}^{\Delta-\varepsilon+u_{k-1}} |r_3(u_{k})|\wrt u_k1(u_{k-1}>2\varepsilon)\Bigg] \wrt u_{k-1}
		%
		\intertext{}& \leq \Bigg[\int_{u_{k-1}=0}^{\Delta-\varepsilon} (\Delta-\varepsilon-u_{k-1})|r_2 |+ 2\varepsilon G 
		%
		+ G\cfrac{\var(Z)/\varepsilon^2}{1-\var(Z)/\varepsilon^2}(u_{k-1}-2\varepsilon)1(u_{k-1}>2\varepsilon) \Bigg]\wrt u_{k-1}
		% 
		\\& \leq \frac{1}{2}\Delta^2|r_2 |+ 2\Delta\varepsilon G 
		%
		+ \cfrac{1}{2}\Delta^2G\cfrac{\var(Z)/\varepsilon^2}{1-\var(Z)/\varepsilon^2},
	\end{align*}
	and, by Property \ref{properties: 2}, 
	\begin{align*}
		&\int_{u_{n-1}=0}^{\Delta-\varepsilon}| r_V(u_{n-1},x)|\wrt u_{n-1} 
		\leq R_{V,1}
	\end{align*}
	Therefore, the error term \(|r_5(n)|\) is less than or equal to the larger of these two terms, 
	\begin{align*}
		|r_5(n)|= O\left(\max\left\{G^{n-1}\Delta^{n-2}\left(\frac{1}{2}\Delta|r_2 |+ 2\varepsilon G 
		%
		+ \cfrac{1}{2}\Delta G\cfrac{\var(Z)/\varepsilon^2}{1-\var(Z)/\varepsilon^2}\right),
		G^{n-1}\Delta^{n-2}R_{V,1}\right\}\right).
	\end{align*}
	
	Now, 
	\begin{align*}
		&\Bigg|g_{1,n}^{*,\varepsilon}(x_0,x) - g_{1,n}^{*}(x_0,x)
		%
		\Bigg|\nonumber
		%
		\\&= \int_{u_1=\Delta-\varepsilon-x_0}^{\Delta-x_0}g_1(\Delta - u_1 - x_0)
		\int_{u_2=\Delta-\varepsilon-u_1}^{\Delta-u_1}g_2(\Delta - u_2 - u_1)\wrt u_1  \nonumber 
		\\&\quad\hdots 
            	\int_{u_{n-1}=\Delta-\varepsilon-u_{n-2}}^{\Delta-u_{n-2}} g_{n-1}(\Delta - u_{n-1} - u_{n-2}) \wrt u_{n-2}
            	g_{n}(\Delta - x-u_{n-1})\nonumber 
		\\&\quad\times 1(\Delta - x-u_{n-1}\geq 0)\wrt u_{n-1}\nonumber
		\\&\leq \int_{u_1=\Delta-\varepsilon-x_0}^{\Delta-x_0}G 
		\int_{u_2=\Delta-\varepsilon-u_1}^{\Delta-u_1}G \wrt u_1  \hdots 
            	\int_{u_{n-1}=\Delta-\varepsilon-u_{n-2}}^{\Delta-u_{n-2}} G \wrt u_{n-2}\nonumber
            	G\wrt u_{n-1} 
		\\& = \varepsilon^{n-1}G ^n 
	\end{align*}
	Therefore, the left-hand side of (\ref{eqn: rhs g 2}) is equal to 
	\begin{align*}
		g^{*,\varepsilon}_{1,n}(x_0,x) + r_5(n) + r_6(n),
	\end{align*}
	where \(r_6(n) = g_{1,n}^{*}(x_0,x) - g_{1,n}^{*,\varepsilon}(x_0,x) \), and \(|r_6(n)|\leq \varepsilon^{n-1}G ^n\).
\end{proof}

The error terms \(r_5(n)\) depend on \(p\) as they are functions of \(r_2^{(p)},\, \varepsilon^{(p)},\, \var\left(Z^{(p)}\right)\) and \(R_{V,1}^{(p)}\). We write \(r_5^{(p)}(n)\) when this dependence is explicitly needed, otherwise this dependence it omitted from the notation. Choosing \(\varepsilon = \var(Z^{(p)})^{1/3}\), the error term \(|r_5^{(p)}(n)|= O\left(\var\left(Z^{(p)}\right)^{1/3}\right)\to 0\) as \(p\to\infty\). Similarly, \(r_6(n)\) depends on \(p\) as it is a function of \(\varepsilon^{(p)}\) and we write \(r_6^{(p)}(n)\) when we need to denote this explicitly. 

%\begin{cor}\label{cor: yet another}
%	Let \(g_1,g_2,\dots,\) be functions satisfying the Assumptions \ref{asu: g} and let \(V(x)\) be a closing operator with the Properties \ref{properties: some props}. Then, for \(n\geq 2\), 
%	\begin{align}
%		&\Bigg| \int_{x_1=0}^\infty g_1(x_1) \bs k(x_0) e^{\bs{S}x_1}\wrt x_1\bs D(\Delta-\varepsilon)
%            	\left[\prod_{k=2}^{n-1}\int_{x_k=0}^\infty g_k(x_k) e^{\bs{S}x_k} \wrt x_k \nonumber 
%		\bs D(\Delta-\varepsilon)\right]
%%		\hdots
%%            	\int_{x_{n-1}=0}^\infty g_{n-1}(x_{n-1}) e^{\bs{S}x_{n-1}} \wrt x_{n-1} \int_{u_{n-1}=0}^{\Delta-\varepsilon} e^{\bs{S}u_{n-1}}\bs s \cfrac{\bs \alpha e^{\bs{S}u_{n-1}}}{\bs \alpha e^{\bs{S}u_{n-1}}\bs e}\wrt u_{n-1} \nonumber 
%            	\int_{x_n=0}^\infty g_{n}(x_n) e^{\bs{S}x_n} \wrt x_n V(x) \nonumber 
%	%
%		\\&{}- g_{1,n}^*(x_0,x)\Bigg|%\int_{u_1=0}^{\Delta-x_0}g_1(\Delta - u_1 - x_0)
%%		\int_{u_2=0}^{\Delta-u_1}g_2(\Delta - u_2 - u_1)\wrt u_1  \nonumber 
%%		\\&\quad\hdots 
%%            	\int_{u_{n-1}=0}^{\Delta-u_{n-2}} g_{n-1}(\Delta - u_{n-1} - u_{n-2}) \wrt u_{n-2}
%%            	g_{n}(\Delta - x-u_{n-1})1(\Delta - x-u_{n-1}\geq 0)\wrt u_{n-1} \Bigg| \nonumber
%		\\&\leq |r_5(n)| + |r_6(n)|, \label{eqn: rhs g 3}
%	\end{align}
%	where \(|r_6(n)| \leq\varepsilon^{n-1}G^n \).
%\end{cor}
%\begin{proof}
%	First observe that the difference 
%	\begin{align}
%		&\Bigg|\int_{u_1=0}^{\Delta-\varepsilon-x_0}g_1(\Delta - u_1 - x_0)
%		\int_{u_2=0}^{\Delta-\varepsilon-u_1}g_2(\Delta - u_2 - u_1)\wrt u_1  \nonumber 
%		\\&\quad\hdots 
%            	\int_{u_{n-1}=0}^{\Delta-\varepsilon-u_{n-2}} g_{n-1}(\Delta - u_{n-1} - u_{n-2}) \wrt u_{n-2}
%            	g_{n}(\Delta - x-u_{n-1})1(\Delta - x-u_{n-1}\geq\varepsilon)\wrt u_{n-1}\nonumber
%		%
%		\\&\quad - \int_{u_1=0}^{\Delta-x_0}g_1(\Delta - u_1 - x_0)
%		\int_{u_2=0}^{\Delta-u_1}g_2(\Delta - u_2 - u_1)\wrt u_1  \nonumber 
%		\\&\quad\hdots 
%            	\int_{u_{n-1}=0}^{\Delta-u_{n-2}} g_{n-1}(\Delta - u_{n-1} - u_{n-2}) \wrt u_{n-2}
%            	g_{n}(\Delta-x-u_{n-1})1(\Delta - x-u_{n-1}\geq 0)\wrt u_{n-1} \Bigg|\nonumber
%		%
%		\\&= \int_{u_1=\Delta-\varepsilon-x_0}^{\Delta-x_0}g_1(\Delta - u_1 - x_0)
%		\int_{u_2=0}^{\Delta-\varepsilon-u_1}g_2(\Delta - u_2 - u_1)\wrt u_1  \nonumber 
%		\\&\quad\hdots 
%            	\int_{u_{n-1}=\Delta-\varepsilon-u_{n-2}}^{\Delta-u_{n-2}} g_{n-1}(\Delta - u_{n-1} - u_{n-2}) \wrt u_{n-2}
%            	g_{n}(\Delta - x-u_{n-1})1(\Delta - x-u_{n-1}\geq 0)\wrt u_{n-1}\nonumber
%		\\&\leq \int_{u_1=\Delta-\varepsilon-x_0}^{\Delta-x_0}G 
%		\int_{u_2=0}^{\Delta-\varepsilon-u_1}G \wrt u_1  \hdots 
%            	\int_{u_{n-1}=\Delta-\varepsilon-u_{n-2}}^{\Delta-u_{n-2}} G_{n-1} \wrt u_{n-2}\nonumber
%            	G_{n}\wrt u_{n-1} 
%		\\& = \varepsilon^{n-1}G \dots G \label{eqn: this pnn}
%	\end{align}
%	
%	Adding and subtracting the integral on the right-hand side of (\ref{eqn: rhs g 2}) to the left-hand side of (\ref{eqn: rhs g 3}) (within the absolute value), applying the triangle inequality, Lemma~\ref{lem: lst convergence} gives the required bound. 
%\end{proof}
%
%The error term \(r_6(n)\) depends on \(p\). We write \(r_6^{(p)}(n)\) when this dependence needs to be made explicit. Also note that \(|r_6^{(p)}(n)|\to 0 \) as \(p\to\infty\). 

\begin{cor}\label{cor: a cor}
        Let \(g_1,g_2,\dots,\) be functions satisfying Assumptions \ref{asu: g} and let \(V(x)\), \(x\in[0,\Delta)\), be a closing operator with Properties \ref{properties: some props}. Then, for \(n\geq 2\), \(x_0\in[0,\Delta)\), 
	\begin{align}
		&\Bigg| \int_{x_1=0}^\infty g_1(x_1) \bs k(x_0) e^{\bs{S}x_1}\wrt x_1\bs D 
            	\left[\prod_{k=2}^{n-1}\int_{x_k=0}^\infty g_k(x_k) e^{\bs{S}x_k} \wrt x_k \bs D\right] \int_{x_n=0}^\infty g_{n}(x_n) e^{\bs{S}x_n} \wrt x_n V(x) \nonumber 
	%
		\\&{}- \int_{u_1=0}^{\Delta-x_0}g_1(\Delta - u_1 - x_0)
%		\int_{u_2=0}^{\Delta-u_1}g_2(\Delta - u_2 - u_1)\wrt u_1  \nonumber 
		\left[\prod_{k=2}^{n-1} \int_{u_k=0}^{\Delta-u_{k-1}} g_k(\Delta-u_k-u_{k-1})\wrt u_{k-1}\right] \nonumber 
		%\\&{}\nonumber
            	%\int_{u_{n-1}=0}^{\Delta-u_{n-2}} g_{n-1}(\Delta - u_{n-1} - u_{n-2}) \wrt u_{n-2}
            	g_{n}(\Delta - x-u_{n-1})
	\\&\qquad{} 1(\Delta-x-u_{n-1}\geq0) \wrt u_{n-1} \Bigg| \nonumber
		\\&\leq |r_5(n)| + |r_6(n)| + (n-1)|r_4(n)|, \label{eqn: rhs g 4}
	\end{align}
	where 
	\begin{align*}
		|r_4(n)| &= \left(2\varepsilon + \cfrac{\var(Z)}{\varepsilon}\right) \cfrac{1}{1-\var(Z)/(\Delta-x_0)} G \widehat G^{n-2} G ,
		\\|r_5(n)|&= O\left(\max\left\{G^{n-1}\Delta^{n-2}\left(\frac{1}{2}\Delta|r_2 |+ 2\varepsilon G 
		%
		+ \cfrac{1}{2}\Delta G\cfrac{\var(Z)/\varepsilon^2}{1-\var(Z)/\varepsilon^2}\right),
		G^{n-1}\Delta^{n-2}R_{V,1}\right\}\right)
		\\|r_6(n)| &\leq\varepsilon^{n-1}G^n.
	\end{align*}
\end{cor}
\begin{proof}
	The left-most term on the left-hand side of (\ref{eqn: rhs g 4}) can be written as 
	\begin{align}
		&\int_{x_1=0}^\infty g_1(x_1) \bs k(x_0)e^{\bs{S}x_1}\wrt x_1\bs D(\Delta-\varepsilon)
            	\left[\prod_{k=2}^{n-1}\int_{x_k=0}^\infty g_k(x_k) e^{\bs{S}x_k} \wrt x_k \bs D(\Delta-\varepsilon)\right]
%		\hdots
%            	\int_{x_{n-1}=0}^\infty g_{n-1}(x_{n-1}) e^{\bs{S}x_{n-1}} \wrt x_{n-1} \int_{u_{n-1}=0}^{\Delta-\varepsilon} e^{\bs{S}u_{n-1}}\bs s \cfrac{\bs \alpha e^{\bs{S}u_{n-1}}}{\bs \alpha e^{\bs{S}u_{n-1}}\bs e}\wrt u_{n-1} \nonumber 
            	\\&\qquad \times\int_{x_n=0}^\infty g_{n}(x_n) e^{\bs{S}x_n} \wrt x_n V(x) \nonumber 
	%
	+\sum_{k=1}^{n-1} \int_{x_{k+1}=0}^\infty \int_{u_k=\Delta-\varepsilon}^\infty \bs k(x_0)\mathcal I_{1,k}(u_k) \mathcal J_{k+1,n}(u_k,x_{k+1})V(x). \nonumber
	\end{align}
	 Now, substitute this into the left-hand side of (\ref{eqn: rhs g 4}), apply the triangle inequality and Lemmas \ref{cor: lh and rh} and \ref{lem: lst convergence} to get the result.  
\end{proof}

Define 
\begin{align}
		w_n(x_0,x) &= \int_{x_1=0}^\infty g_1(x_1) \bs k(x_0) e^{\bs{S}x_1}\wrt x_1\bs D 
            	\left[\prod_{k=2}^{n-1}\int_{x_k=0}^\infty g_k(x_k) e^{\bs{S}x_k} \wrt x_k \bs D\right] \nonumber
		\\&\qquad{}\int_{x_n=0}^\infty g_{n}(x_n) e^{\bs{S}x_n} \wrt x_n V(x), \nonumber 
\end{align}
where \(g_1,g_2,\dots,\) are functions satisfying the Assumptions \ref{asu: g} and \(V(x)\) is a closing operator with the Properties \ref{properties: some props}. %By Lemma~\ref{cor: ksjkd}, for any \(x_0,x\in [0,\Delta)\) 
%\begin{align}
%		w_n(x_0,x) &\leq \widehat G^{n-2}GG_V.
%\end{align}

\begin{cor}
	 Let \(g_1,g_2,\dots,\) be functions satisfying Assumptions \ref{asu: g} and let \(V(x)\), \(x\in[0,\Delta)\), be a closing operator with Properties \ref{properties: some props}. Then, for \(n\geq 2\), \(x_0\in[0,\Delta)\), 
	\begin{align}
		&\Bigg| \int_{x=0}^\Delta w_n(x_0,x) \psi(x) \wrt x \nonumber 
	%
		\\&{}- \int_{x=0}^\Delta \int_{u_1=0}^{\Delta-x_0}g_1(\Delta - u_1 - x_0)
%		\int_{u_2=0}^{\Delta-u_1}g_2(\Delta - u_2 - u_1)\wrt u_1  \nonumber 
		\left[\prod_{k=2}^{n-1} \int_{u_k=0}^{\Delta-u_{k-1}} g_k(\Delta-u_k-u_{k-1})\wrt u_{k-1}\right] \nonumber 
		%\\&{}\nonumber
            	%\int_{u_{n-1}=0}^{\Delta-u_{n-2}} g_{n-1}(\Delta - u_{n-1} - u_{n-2}) \wrt u_{n-2}
            	g_{n}(\Delta - x-u_{n-1})
	\\&\qquad{} 1(\Delta-x-u_{n-1}\geq0) \wrt u_{n-1}\psi(x) \wrt x \Bigg| \nonumber
		\\&\leq (|r_5(n)| + |r_6(n)| + (n-1)|r_4(n)|)\Delta F. \label{eqn: rhs g 4dvfklsmv}
	\end{align}
\end{cor}
\begin{proof}
	The left-hand side of (\ref{eqn: rhs g 4dvfklsmv}) is less than or equal to 
	\begin{align}
		&\int_{x=0}^\Delta \Bigg| w_n(x_0,x)  \nonumber 
	%
		\\&{}- \int_{u_1=0}^{\Delta-x_0}g_1(\Delta - u_1 - x_0)
%		\int_{u_2=0}^{\Delta-u_1}g_2(\Delta - u_2 - u_1)\wrt u_1  \nonumber 
		\left[\prod_{k=2}^{n-1} \int_{u_k=0}^{\Delta-u_{k-1}} g_k(\Delta-u_k-u_{k-1})\wrt u_{k-1}\right] \nonumber 
		%\\&{}\nonumber
            	%\int_{u_{n-1}=0}^{\Delta-u_{n-2}} g_{n-1}(\Delta - u_{n-1} - u_{n-2}) \wrt u_{n-2}
            	g_{n}(\Delta - x-u_{n-1})
	\\&\qquad{} 1(\Delta-x-u_{n-1}\geq0) \wrt u_{n-1}\Bigg| \left|\psi(x)\right| \wrt x. \label{eqn: rhs g 4dvfklsmsssv}
	\end{align}
	Apply Corollary~\ref{cor: a cor} to bound the first absolute value so that (\ref{eqn: rhs g 4dvfklsmsssv}) is less than or equal to 
	\begin{align}
		&\int_{x=0}^\Delta (|r_5(n)| + |r_6(n)| + (n-1)|r_4(n)|) \left|\psi(x)\right| \wrt x \nonumber
		\\&\leq \int_{x=0}^\Delta(|r_5(n)| + |r_6(n)| + (n-1)|r_4(n)|) F \wrt x \nonumber 
		\\&= (|r_5(n)| + |r_6(n)| + (n-1)|r_4(n)|)\Delta F 
	\end{align}
\end{proof}

We have assumed throughout the appendix that the functions \(g\) are scalar functions, however, we are ulitmately interested in expressions of the form (\ref{eqn: approx final end 2}), which contain matrix functions. 
\begin{lem}\label{lem: boobies}
	Let \(\bs G_k(x)\), \(k\in\{1,2,...\}\), be matrix functions with dimensions \(N_k \times N_{k+1}\). Further, suppose \([\bs G_k(x)]_{ij}\), \(k\in\{1,2,...\}\) satisfy Assumptions \ref{asu: g}. Then, 
	\begin{align}
		&\Bigg| \int_{x=0}^\Delta \int_{x_1=0}^\infty \bs G_1(x_1) \otimes \bs k(x_0) e^{\bs{S}x_1} \bs D (x_1) \wrt x_1
		\left[\prod_{k=2}^{n-1}\int_{x_k=0}^\infty \bs G_{k }(x_k) \otimes e^{\bs{S}x_k} \wrt x_k \bs D\right] \nonumber
\\&\qquad{}\int_{x_n=0}^\infty \bs G_{n }(x_n)\otimes e^{\bs{S}x_n} \wrt x_n V(x) \psi(x) \wrt x \nonumber 
	%
		\\&{}- \int_{x=0}^\Delta \int_{u_1=0}^{\Delta-x_0}\bs G_1(\Delta - u_1 - x_0)
	%		\int_{u_2=0}^{\Delta-u_1}g_2(\Delta - u_2 - u_1)\wrt u_1  \nonumber 
		\left[\prod_{k=2}^{n-1} \int_{u_k=0}^{\Delta-u_{k-1}} \bs G_{k}(\Delta-u_k-u_{k-1})\wrt u_{k-1}\right] \nonumber 
		%\\&{}\nonumber
				%\int_{u_{n-1}=0}^{\Delta-u_{n-2}} g_{n-1}(\Delta - u_{n-1} - u_{n-2}) \wrt u_{n-2}
				\\&\qquad{} \bs G_{n }(\Delta - x-u_{n-1})
			1(\Delta-x-u_{n-1}\geq0) \wrt u_{n-1}\psi(x) \wrt x \Bigg| \nonumber
		\\&\leq (|r_5(n)| + |r_6(n)| + (n-1)|r_4(n)|)\Delta F \prod_{k=2}^{n}N_{k}. \label{eqn: rhs g 4dvfklsmv2G}
	\end{align}
	Moreover, choosing \(\varepsilon=\var(Z)\), then, for fixed \(n\), the error term \((|r_5(n)| + |r_6(n)| + (n-1)|r_4(n)|)\Delta F \prod_{k=2}^{n}N_{k}\) is \(\mathcal O(\var(Z)^{1/3})\). 
\end{lem}
\begin{proof}
	By the \ref{eqn:mpr} the \((i,j)\)th element, \(i\in\{1,\dots,N_1\}\), \(j\in\{1,\dots,N_{n+1}\}\), of the first term on the left-hand side of (\ref{eqn: rhs g 4dvfklsmv2G}) is 
	\begin{align}
		&\int_{x=0}^\Delta \int_{x_1=0}^\infty \dots \int_{x_n=0}^\infty \left[\bs G_1(x_1)\dots \bs G_n(x_n)\right]_{i,j} \bs k(x_0) e^{\bs{S}x_1} \bs D (x_1) 
		e^{\bs{S}x_k} \bs D \nonumber
		\\&\qquad{} e^{\bs{S}x_n} \wrt x_n \dots \wrt x_1 V(x) \psi(x) \wrt x \nonumber 
		% 
		\\&= \int_{x=0}^\Delta \int_{x_1=0}^\infty \dots \int_{x_n=0}^\infty \sum_{j_1=1}^{N_2}[\bs G_1(x_1)]_{i,j_1}\sum_{j_2=1}^{N_3}[\bs G_2(x_2)]_{j_1,j_2}\dots \sum_{j_{n-1}=1}^{N_n} [\bs G_n(x_n)]_{j_{n-1},j} \nonumber
		\\&\qquad{} \bs k(x_0) e^{\bs{S}x_1} \bs D (x_1) 
		e^{\bs{S}x_k} \bs D 
		e^{\bs{S}x_n} \wrt x_n \dots \wrt x_1 V(x) \psi(x) \wrt x, \label{eqn: s}
	\end{align}
	from which we see that (\ref{eqn: s}) is a linear combination of the scalar function case. Applying the bound for the scalar case to each term in the linear combination, then summing the bounds gives the bounds. 

	The fact that the error bound is \(\mathcal O(\var(Z)^{1/3})\) follows by substituting \(\varepsilon=\var(Z)\) into each term and observing that each term is at most \(\mathcal O(\var(Z)^{1/3})\). 
\end{proof}
Lemma \ref{lem: boobies} effectively shows that, as \(p \to \infty\), then 
\begin{align}
	& \int_{x=0}^\Delta \int_{x_1=0}^\infty \bs G_1(x_1) \otimes \bs k^{(p)} (x_0) e^{\bs{S}^{(p)}x_1} \bs D^{(p)} (x_1) \wrt x_1
	\left[\prod_{k=2}^{n-1}\int_{x_k=0}^\infty \bs G_{k }(x_k) \otimes e^{\bs{S}^{(p)}x_k} \wrt x_k \bs D^{(p)}\right] \nonumber
\\&\qquad{}\int_{x_n=0}^\infty \bs G_{n }(x_n)\otimes e^{\bs{S}^{(p)}x_n} \wrt x_n V^{(p)}(x) \psi(x) \wrt x \nonumber 
%
	\\&{}\to \int_{x=0}^\Delta \int_{u_1=0}^{\Delta-x_0}\bs G_1(\Delta - u_1 - x_0)
%		\int_{u_2=0}^{\Delta-u_1}g_2(\Delta - u_2 - u_1)\wrt u_1  \nonumber 
	\left[\prod_{k=2}^{n-1} \int_{u_k=0}^{\Delta-u_{k-1}} \bs G_{k}(\Delta-u_k-u_{k-1})\wrt u_{k-1}\right] \nonumber 
	%\\&{}\nonumber
			%\int_{u_{n-1}=0}^{\Delta-u_{n-2}} g_{n-1}(\Delta - u_{n-1} - u_{n-2}) \wrt u_{n-2}
			\\&\qquad{} \bs G_{n }(\Delta - x-u_{n-1})
		1(\Delta-x-u_{n-1}\geq0) \wrt u_{n-1}\psi(x) \wrt x \nonumber.
\end{align}


\section{Properties of closing operators}\label{appendix: sec: 2}
\begin{cor}\label{cor: cond bnd 2 V}
	Let \(g\) be a function satisfying the Assumptions \ref{asu: g} and consider the closing operator \(V(x)=e^{\bs Sx}\bs s\). For \(u\leq \Delta-\varepsilon \), \(v\geq 0\), 
	\[\int_{x=0}^\infty \cfrac{\bs \alpha  e^{\bs{S} (u+x)} }{\bs \alpha  e^{\bs{S} u} \bs e} V(v)g(x)\wrt x = g(\Delta-u-v) 1(u+v\leq\Delta-\varepsilon) + r_V (u,v),\]
	where \[\displaystyle\int_{u=0}^{\Delta-\varepsilon}r_V(u,v)\wrt u = \displaystyle\int_{u=0}^{\Delta-\varepsilon} r_3(u+v)\wrt u \leq r_2\Delta + 2\varepsilon G + \Delta G \cfrac{\var(Z)/\varepsilon^2}{1-\var(Z)/\varepsilon^2}\]
	and \[\int_{u=0}^\Delta r_V(u,v) \wrt u\leq R_{V,1}\] where \[R_{V,1}=r_2\Delta + 2\varepsilon G + \Delta G \cfrac{\var(Z)/\varepsilon^2}{1-\var(Z)/\varepsilon^2}.\] 
\end{cor}
\begin{proof}
	By Corollary~\ref{cor: cond bnd 2}, 
	\[\int_{x=0}^\infty \cfrac{\bs \alpha  e^{\bs{S} (u+x)} }{\bs \alpha  e^{\bs{S} u} \bs e} V(v)g(x)\wrt x = g(\Delta-u-v) 1(u+v\leq\Delta-\varepsilon) + r_3 (u+v),\]
	so \(r_V(u,v)=r_3(u+v)\). All that remains to be shown are the bounds \(R_{V,1}\) and \(R_{V,2}\). To this end, observe 
	\begin{align*}
		R_{V,1}= \displaystyle\int_{u=0}^{\Delta}r_V(u,v)\wrt u 
		 &= \displaystyle\int_{u=0}^{\Delta} r_3(u+v)\wrt u 
		 %
%		 \\&\leq \displaystyle\int_{u=0}^{2\Delta} r_3(u+v)\wrt u
		 %
		 \leq r_2\Delta + 2\varepsilon G + \Delta G \cfrac{\var(Z)/\varepsilon^2}{1-\var(Z)/\varepsilon^2}.
	\end{align*}
	\begin{align*}
		 R_{V,2}= \displaystyle\int_{v=0}^{\Delta}r_V(u,v)\wrt v 
		 &= \displaystyle\int_{v=0}^{\Delta} r_3(u+v)\wrt v 
		 %
%		 \\&\leq \displaystyle\int_{u=0}^{2\Delta} r_3(u+v)\wrt u
		 %
		 \leq r_2\Delta + 2\varepsilon G + \Delta G \cfrac{\var(Z)/\varepsilon^2}{1-\var(Z)/\varepsilon^2}.
	\end{align*}
\end{proof}

\begin{lem}\label{lem: akc}
For any valid orbit, \(\bs a\in\mathcal A\), \(x,u\geq 0\), 
        \begin{align*}
        		\int_{x_n=0}^\infty \bs a e^{\bs{S}(x+x_n+u)}\bs s = \bs a e^{\bs{S}(x+u)}\bs e &\leq \bs a e^{\bs{S}u}\bs e. 
	\end{align*}
\end{lem}
\begin{proof}
	For any valid orbit, \(\bs a\in\mathcal A\), 
        \begin{align*}
        		\bs a e^{\bs{S}(x+u)}\bs e &= \mathbb P(Z>x+u) \leq\mathbb P(X>u) = \bs a e^{\bs Su} \bs e. 
	\end{align*}
\end{proof}
\begin{lem}\label{lem:macmnm}
	For any valid orbit, \(\bs a\in\mathcal A\), \(x\geq 0\), 
        \begin{align*}
        		\bs a\bs D e^{\bs{S}x}\bs e &\leq 1. 
	\end{align*}
\end{lem}
\begin{proof}
By the definition of \(\bs D\) and Lemma~\ref{lem: akc},
	\begin{align*}
        		\bs a\bs D e^{\bs{S}x}\bs e &= \bs a \int_{u=0}^\infty e^{\bs Su}\bs s\cfrac{\bs \alpha e^{\bs Su}}{\bs \alpha e^{\bs Su}\bs e}\wrt ue^{\bs{S}x}\bs e
		%
		\\& \leq \bs a \int_{u=0}^\infty e^{\bs Su}\bs s\cfrac{\bs \alpha e^{\bs Su}\bs e}{\bs \alpha e^{\bs Su}\bs e}\wrt u
		%
		\\& = \bs a \int_{u=0}^\infty e^{\bs Su}\bs s\wrt u
		%
		\\& = \bs a \bs e = 1.
	\end{align*}
\end{proof}

Let \(U^{(p)}(x)\) be the closing operator such that, for \(\bs k^{(p)} \in\mathcal A^{(p)}\), \(x\in[0,\Delta)\),
\[\bs k^{(p)} U^{(p)}(x) = \bs k^{(p)} \left(e^{\bs{S}^{(p)}x}\bs s^{(p)} + e^{\bs{S}^{(p)}(2\Delta-x)}\bs s^{(p)}\right).\]
\begin{lem}\label{lem: akxnj}
	For \(x\in[0,\Delta),u\geq 0\),  
        \begin{align*}
        		\bs a   e^{\bs Su}(-\bs S)^{-1} U(x) &\leq 2 \bs a e^{\bs Su} \bs e.
	\end{align*}
\end{lem}
\begin{proof}
Let \(\bs a   \in \mathcal A\) be arbitrary. By definition 
	\begin{align*}
        		\bs a  e^{\bs Su}(-\bs S)^{-1} U(x) & = \bs a  e^{\bs Su}(-\bs S)^{-1}  \left(e^{\bs{S}x}\bs s + e^{\bs{S}(2\Delta-x)}\bs s\right)
				\\& = \bs a  e^{\bs Su}  \left(e^{\bs{S}x}\bs e + e^{\bs{S}(2\Delta-x)}\bs e\right)
	\end{align*}
	since \((-\bs S)^{-1}\) and \(e^{\bs Sx}\) commute and \(\bs s = -\bs S e\). 
	By Lemma~\ref{lem: akc} this is less than or equal to, 
	\begin{align}
        		& \bs a   e^{\bs Su} \left(\bs e + \bs e\right) = 2 \bs a   e^{\bs Su} \bs e \label{eqn:mzm}
	\end{align}
\end{proof}

%\begin{rem}\label{cor: lajd}
%	For any valid orbit, \(\bs a\in\mathcal A\), \(x\in[0,\Delta),u\geq 0\), 
%	\[\int_{x_n=0}^\infty \bs \alpha e^{\bs{S}u}e^{\bs{S}x_n} \wrt x_n U(x)  \leq 2\bs \alpha e^{\bs{S}u}\bs e.\]
%	To see this compute the integral, then
%	\begin{align}
%		\int_{x_n=0}^\infty \bs \alpha e^{\bs{S} }e^{\bs{S}x_n} \wrt x_n U(x)  &= \bs \alpha e^{\bs{S} } (-\bs S)^{-1} U(x).
%	\end{align}
%	Now apply Lemma~\ref{lem: akxnj} to the right-hand side. 
%\end{rem}

\begin{cor}\label{cor: cond bnd 2 U}
	Let \(g\) be a function satisfying the Assumptions \ref{asu: g}. For \(u\leq \Delta-\varepsilon \), \(v\in[ 0,\Delta)\), 
	\[\int_{x=0}^\infty \cfrac{\bs \alpha  e^{\bs{S} (u+x)} }{\bs \alpha  e^{\bs{S} u} \bs e} U(v)g(x)\wrt x = g(\Delta-u-v) 1(u+v\leq\Delta-\varepsilon) + r_V (u,v),\]
	where 
	\[\left|r_V (u,v)\right|\leq r_3 (u+v) + r_3 (u+2\Delta - v).\]
	Furthermore,  
	\begin{align*}
		&\int_{u=0}^{\Delta}| r_V(u,x)|\wrt u
		\leq R_{V,1},
	\end{align*}
	and
	\begin{align*}
		&\int_{v=0}^{\Delta}| r_V(u,x)|\wrt u
		\leq R_{V,2},
	\end{align*}
	where 
	\[R_{V,1},\, R_{V,2} \leq 2\left(\Delta r_2 + 2\varepsilon G + \Delta\cfrac{\var(Z)/\varepsilon^2}{1-\var(Z)/\varepsilon^2}\right).\]
\end{cor}
\begin{proof}
	By the definition of the operator \(U(x)\), 
	\begin{align}
		\int_{x=0}^\infty \cfrac{\bs \alpha  e^{\bs{S} (u+x)} }{\bs \alpha  e^{\bs{S} u} \bs e} U(v)g(x)\wrt x 		
		%
		&=
			\displaystyle\int_{x=0}^\infty 
				\cfrac{
					\bs \alpha e^{\bs S(u+x)}
					}{
					\bs \alpha e^{Su}\bs e
					} 
				e^{\bs{S}v}\bs s g(x)
				+
				\cfrac{
					\bs \alpha e^{\bs S(u+x)}
					}{
					\bs \alpha e^{Su}\bs e
					} 
				e^{\bs{S}(2\Delta-v)}\bs sg(x) \wrt x. \label{eqn: ghi is this a}
	\end{align}
	By Corollary~\ref{cor: cond bnd 2} 
	\begin{align}
		\int_{x=0}^\infty 
				\cfrac{
					\bs \alpha e^{\bs S(u+x)}
					}{
					\bs \alpha e^{Su}\bs e
					} 
				e^{\bs{S}v}\bs s g(x)\wrt x 
				&= g(\Delta-u-v) 1(u+v\leq\Delta-\varepsilon) + r_3 (u+v), \label{eqn: dkskkk2}
		\\
		\int_{x=0}^\infty\cfrac{
					\bs \alpha e^{\bs S(u+x)}
					}{
					\bs \alpha e^{Su}\bs e
					} 
				e^{\bs{S}(2\Delta-v)}\bs sg(x) \wrt x 
				&= r_3 (u+2\Delta - v).
	\end{align}
	Therefore, (\ref{eqn: ghi is this a}) is, 
	\begin{align}
		&g(\Delta-u-v) 1(u+v\leq\Delta-\varepsilon) + r_3 (u+v) + r_3 (u+2\Delta - v).
	\end{align}	
	
	Now,
	\begin{align*}
		R_{V,1}&\leq \int_{u=0}^{\Delta}| r_V(u,v)|\wrt u
		\\& \leq \int_{u=0}^{\Delta} |r_3 (u+v)| + |r_3 (u+2\Delta - v) |
		\\&\leq 2\left(\Delta r_2 + 2\varepsilon G + \Delta\cfrac{\var(Z)/\varepsilon^2}{1-\var(Z)/\varepsilon^2}\right).
	\end{align*}
	Similarly 
	\begin{align*}
		R_{V,2}&\leq \int_{v=0}^{\Delta}| r_V(u,v)|\wrt v
		\\&= \int_{v=0}^{\Delta} r_3(u+v) + r_3(u+2\Delta-v) 
		\\& \leq 2\left(\Delta r_2 + 2\varepsilon G + \Delta\cfrac{\var(Z)/\varepsilon^2}{1-\var(Z)/\varepsilon^2}\right).
	\end{align*}
\end{proof}

The error term \(r_V(u,v)\) depends on \(p\) so we should write \(r_V^{(p)}(u,v)\). The error term \(r_V^{(p)}(u,v)\) has similar properties to \(r_3^{(p)}(u+v)\); we can prove that it converges point-wise to \(0\) only on some areas of its domain, however when we integrate the error against bounded functions on bounded domains, then the resulting integral tends to \(0\). 

% Let \(W^{(p)}(x)\) be the closing operator such that, for \(\bs k^{(p)} \in\mathcal A^{(p)}\), \(x\in[0,\Delta)\),
% \[\bs k^{(p)} W^{(p)}(x) = \bs k^{(p)} \left(e^{\bs{S}^{(p)}x} + e^{\bs{S}^{(p)}(2\Delta-x)}\right)\left[I-e^{\bs S^{(p)} 2\Delta}\right]^{-1}\bs s^{(p)}.\]
% \begin{lem}\label{lem: akxnj2}
% 	For \(x\in[0,\Delta),u\geq 0\),  
%         \begin{align*}
%         		\bs \alpha   e^{\bs Su}(-\bs S)^{-1} W(x) &\leq 2 \bs \alpha e^{\bs Su} \bs e.
% 	\end{align*}
% \end{lem}
% \begin{proof}
% 	First observe that 
% 	\begin{align}
% 		\sum_{k=1}^\infty \bs \alpha e^{\bs S(u+x+k2\Delta)}\bs e
% 		&= \sum_{k=1}^\infty \mathbb P(Z>u+x+k2\Delta)
% 		\\&= \mathbb P(Z>u+x+2\Delta)\sum_{k=1}^\infty \mathbb P(Z>u+x+k2\Delta \mid Z>u+x+2\Delta)
% 		\\&= \mathbb P(Z>u+x+2\Delta)\sum_{k=1}^\infty \cfrac{ \mathbb P(Z>u+x+k2\Delta)}{\mathbb P(Z>u+x+2\Delta)}
% 		\\&= \mathbb P(Z>u+x+2\Delta)\sum_{k=1}^\infty \cfrac{ \var(Z)/(\Delta + (k-1)2\Delta)^2}{\var(Z)/(\Delta + (k-1)2\Delta)^2}
% 	\end{align}
% 	Let \(\bs \alpha \in \mathcal A\) be arbitrary. By definition 
% 	\begin{align*}
%         		\bs \alpha  e^{\bs Su}(-\bs S)^{-1} W(x) & = \bs \alpha e^{\bs Su}(-\bs S)^{-1}  \left(e^{\bs{S} x} + e^{\bs{S} (2\Delta-x)}\right)\left[I-e^{\bs S 2\Delta}\right]^{-1}\bs s 
% 				\\&=\bs \alpha  e^{\bs Su}(-\bs S)^{-1}  \left(e^{\bs{S} x} + e^{\bs{S} (2\Delta-x)}\right)\sum_{m=0}^\infty e^{\bs S 2m\Delta}\bs s 
% 				\\&=\bs \alpha  e^{\bs Su} \left(e^{\bs{S} x} + e^{\bs{S} (2\Delta-x)}\right)\sum_{m=0}^\infty e^{\bs S 2m\Delta}\bs e 
% 	\end{align*}
% 	since \((-\bs S)^{-1}\) and \(e^{\bs Sx}\) commute and \(\bs s = -\bs S e\). 
% 	By Lemma~\ref{lem: akc} this is less than or equal to, 
% 	\begin{align}
%         		& \bs a   e^{\bs Su} \left(\bs e + \bs e\right) = 2 \bs a   e^{\bs Su} \bs e \label{eqn:mzm2}
% 	\end{align}
% \end{proof}

% %\begin{rem}\label{cor: lajd}
% %	For any valid orbit, \(\bs a\in\mathcal A\), \(x\in[0,\Delta),u\geq 0\), 
% %	\[\int_{x_n=0}^\infty \bs \alpha e^{\bs{S}u}e^{\bs{S}x_n} \wrt x_n U(x)  \leq 2\bs \alpha e^{\bs{S}u}\bs e.\]
% %	To see this compute the integral, then
% %	\begin{align}
% %		\int_{x_n=0}^\infty \bs \alpha e^{\bs{S} }e^{\bs{S}x_n} \wrt x_n U(x)  &= \bs \alpha e^{\bs{S} } (-\bs S)^{-1} U(x).
% %	\end{align}
% %	Now apply Lemma~\ref{lem: akxnj} to the right-hand side. 
% %\end{rem}

% \begin{cor}\label{cor: cond bnd 2 U2}
% 	Let \(g\) be a function satisfying the Assumptions \ref{asu: g}. For \(u\leq \Delta-\varepsilon \), \(v\in[ 0,\Delta)\), 
% 	\[\int_{x=0}^\infty \cfrac{\bs \alpha  e^{\bs{S} (u+x)} }{\bs \alpha  e^{\bs{S} u} \bs e} U(v)g(x)\wrt x = g(\Delta-u-v) 1(u+v\leq\Delta-\varepsilon) + r_V (u,v),\]
% 	where 
% 	\[\left|r_V (u,v)\right|\leq r_3 (u+v) + r_3 (u+2\Delta - v).\]
% 	Furthermore,  
% 	\begin{align*}
% 		&\int_{u=0}^{\Delta}| r_V(u,x)|\wrt u
% 		\leq R_{V,1},
% 	\end{align*}
% 	and
% 	\begin{align*}
% 		&\int_{v=0}^{\Delta}| r_V(u,x)|\wrt u
% 		\leq R_{V,2},
% 	\end{align*}
% 	where 
% 	\[R_{V,1},\, R_{V,2} \leq 2\left(\Delta r_2 + 2\varepsilon G + \Delta\cfrac{\var(Z)/\varepsilon^2}{1-\var(Z)/\varepsilon^2}\right).\]
% \end{cor}
% \begin{proof}
% 	By the definition of the operator \(U(x)\), 
% 	\begin{align}
% 		\int_{x=0}^\infty \cfrac{\bs \alpha  e^{\bs{S} (u+x)} }{\bs \alpha  e^{\bs{S} u} \bs e} U(v)g(x)\wrt x 		
% 		%
% 		&=
% 			\displaystyle\int_{x=0}^\infty 
% 				\cfrac{
% 					\bs \alpha e^{\bs S(u+x)}
% 					}{
% 					\bs \alpha e^{Su}\bs e
% 					} 
% 				e^{\bs{S}v}\bs s g(x)
% 				+
% 				\cfrac{
% 					\bs \alpha e^{\bs S(u+x)}
% 					}{
% 					\bs \alpha e^{Su}\bs e
% 					} 
% 				e^{\bs{S}(2\Delta-v)}\bs sg(x) \wrt x. \label{eqn: ghi is this a2}
% 	\end{align}
% 	By Corollary~\ref{cor: cond bnd 2} 
% 	\begin{align}
% 		\int_{x=0}^\infty 
% 				\cfrac{
% 					\bs \alpha e^{\bs S(u+x)}
% 					}{
% 					\bs \alpha e^{Su}\bs e
% 					} 
% 				e^{\bs{S}v}\bs s g(x)\wrt x 
% 				&= g(\Delta-u-v) 1(u+v\leq\Delta-\varepsilon) + r_3 (u+v), \label{eqn: dkskkk22}
% 		\\
% 		\int_{x=0}^\infty\cfrac{
% 					\bs \alpha e^{\bs S(u+x)}
% 					}{
% 					\bs \alpha e^{Su}\bs e
% 					} 
% 				e^{\bs{S}(2\Delta-v)}\bs sg(x) \wrt x 
% 				&= r_3 (u+2\Delta - v).
% 	\end{align}
% 	Therefore, (\ref{eqn: ghi is this a}) is, 
% 	\begin{align}
% 		&g(\Delta-u-v) 1(u+v\leq\Delta-\varepsilon) + r_3 (u+v) + r_3 (u+2\Delta - v).
% 	\end{align}	
	
% 	Now,
% 	\begin{align*}
% 		R_{V,1}&\leq \int_{u=0}^{\Delta}| r_V(u,v)|\wrt u
% 		\\& \leq \int_{u=0}^{\Delta} |r_3 (u+v)| + |r_3 (u+2\Delta - v) |
% 		\\&\leq 2\left(\Delta r_2 + 2\varepsilon G + \Delta\cfrac{\var(Z)/\varepsilon^2}{1-\var(Z)/\varepsilon^2}\right).
% 	\end{align*}
% 	Similarly 
% 	\begin{align*}
% 		R_{V,2}&\leq \int_{v=0}^{\Delta}| r_V(u,v)|\wrt v
% 		\\&= \int_{v=0}^{\Delta} r_3(u+v) + r_3(u+2\Delta-v) 
% 		\\& \leq 2\left(\Delta r_2 + 2\varepsilon G + \Delta\cfrac{\var(Z)/\varepsilon^2}{1-\var(Z)/\varepsilon^2}\right).
% 	\end{align*}
% \end{proof}
