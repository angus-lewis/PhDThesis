%!TEX root = ../thesis.tex
\chapter{Properties of DG operator \(  \bs B\)}\label{appendix:properties}
\begin{center}
    \begin{minipage}{0.8\textwidth}
        \textit{This appendix has been taken from Appendix~1 of \cite{blnos2022} with only minor changes, such as notations, so that this chapter is consistent with the rest of the thesis. I am a co-author of the paper \cite{blnos2022}. The conceptualisation of \cite{blnos2022} was originally by Vikram Sunkara, Nigel Bean and Giang Nguyen, and the original coding was done by Vikram Sunkara. I made significant contributions to Section~3 of the paper, expressing the operator-theoretic expressions to use the same partition as the approximation scheme. I contributed Sections~4.4 and 5.1. I extended the numerical experiments in Section~6 to higher orders and made all the plots in Section~6. Appendix~A is also my original work. I did a significant proportion of the writing of the manuscript and addressed the reviewers comments and also developed code for the numerical experiments.
        }
    \end{minipage}
    \end{center}
Recall that the coefficients \({\boldsymbol a}_i^k(t)\) can be used to construct an approximate solution to a differential equation at time \(t\) as \(u_i^k(x,t)={\boldsymbol a}_i^k(t)\boldsymbol \phi^k(x)\tr{}.\) For \(i\in\calS,\,k\in\{\nabla,1,...,K,\Delta\},\,r\in\mathcal N_k\), define \(\alpha_{i,r}^k(t) := a_{i,r}^k(t)\displaystyle\int_{x\in\calD_k}\phi_r^k(x)\wrt x,\) and row-vectors \(\boldsymbol \alpha_i^k(t)=(\alpha_{i,r}^k(t))_{r\in\mathcal N_k}\). Motivated by the fact that we may be interested in approximations of the probabilities \(\mathbb P(X(t)\in\calD_k,\varphi(t)=i)\) rather than the function \(u_i^k\) itself, we can pose the problem equivalently in terms of the integrals 
\[\mathbb P\left(X(t)\in\calD_k,\varphi(t)=i\right)\approx{\boldsymbol a}_i^k(t)\int_{x\in\calD_k}\boldsymbol \phi^k(x)\tr{}\wrt x={\boldsymbol \alpha}_i^k(t)\boldsymbol 1.\] 
Define
\[\boldsymbol \alpha^k(t) = (\boldsymbol \alpha_i^k(t))_{i\in\calS},\mbox{ and } \boldsymbol \alpha(t)=(\boldsymbol \alpha^k(t)))_{k\in\{\nabla,1,...,K,\Delta\}},\]
and matrices 
\begin{align*}
\bs P_k &= \diag\left(\int_{x\in\calD_k}\phi_r^k(x)\wrt x\right)_{ r\in\mathcal N_k},\, k\in\{1,...,K\},
\\\bs P &=\left[\begin{array}{ccc}
		\bs I_{N_{|\calS|}}\otimes \bs P_1 & &\\
		& \ddots &\\
		& & \bs I_{N_{|\calS|}}\otimes \bs P_K
	\end{array}\right].
\end{align*}
By choosing the basis \(\{\phi_{r}^k\}_{r\in\mathcal N_k,k\in\{1,...,K\}}\) such that \(\int_{x\in\calD_k}\phi_r^k(x)\wrt x\neq 0\) for all \(r,k\), then \(\bs P\) is invertible. This is the case for the Lagrange polynomials, but not, for example, for the Legendre polynomials. We can (loosely) interpret the new coefficients \(\alpha_{i,r}^k(t)\) as representing the amount of probability captured by the basis function \(\phi_r^k(x)\) in phase \(i\).

The differential equation (\ref{eqn: DG ODE w BCs}) can be equivalently expressed as 
\(\label{eqn: DG alpha ODE w BCs}
	\cfrac{\wrt}{\wrt t} \boldsymbol \alpha(t)
	% 
	= \boldsymbol \alpha(t)  { \mathfrak{\bs B}},
\)
where 
\[ {\mathfrak{\bs B}} = \left[\begin{array}{ccc}
		\bs I_{|\calS_\nabla|} & & \\
		& \bs P^{-1} &  \\
		& & \bs I_{|\calS_\Delta|}
	\end{array}\right]
	\bs B
	\left[\begin{array}{ccc}
		\bs I_{|\calS_\nabla|} & & \\
		& \bs P&  \\
		& & \bs I_{|\calS_\Delta|}
	\end{array}\right].\] 
Let
\begin{align*}
 {  {\mathfrak{\bs B}}}^{\nabla1} &:= \bs T_{\nabla+}\otimes\left( \boldsymbol\phi^1(0)\bs M_1^{-1}\bs P_1\right),
 %
 \\ {  {\mathfrak{\bs B}}}^{1\nabla} &:=-\diag(c_i\mathbb 1_{(c_i<0)})_{i\in\calS}\otimes \bs P_1^{-1}\boldsymbol \phi^1(0)\tr{},
 %
 \\ {  {\mathfrak{\bs B}}}^{\Delta K} &:= \bs T_{\Delta-}\otimes \left(\boldsymbol\phi^K(\mathcal I)\bs M_K^{-1}\bs P_K\right)
 %
 \\ {  {\mathfrak{\bs B}}}^{K,\Delta} &:= \diag(c_i\mathbb 1_{(c_i>0)})_{i\in\calS} \otimes  \bs P_K^{-1} \boldsymbol\phi^K(\mathcal I)\tr{},
 %
 \intertext{}\mathfrak{\bs B}^{kk}_{ij} &:= \begin{cases}
    	T_{ii}\bs I_{N_k} + c_i\bs P_k^{-1}(\bs F_i^{kk}+\bs G_k)\bs M_k^{-1}\bs P_k & i = j, \\
	T_{ij}\bs I_{N_k} & i \neq j,
    \end{cases}\quad \mbox{ for \(k=1,\dots K\),}
 %
 \\\mathfrak{\bs B}^{k,k+1}_{ij} &:= \begin{cases}
	c_i\bs P_k^{-1}\bs F_i^{k,k+1}\bs M_{k+1}^{-1}\bs P_{k+1} & i = j, \\
	\bs 0_{N_k} & i \neq j,
    \end{cases}\quad \mbox{ for \(k=1,\dots K-1\),}
 %
 \\\mathfrak{\bs B}^{k-1,k}_{ij} &:= \begin{cases}
    	c_i\bs P_k^{-1}\bs F_i^{k,k-1}\bs M_{k-1}^{-1}\bs P_{k-1} & i = j, \\
	T_{ij}\bs I_{N_k} & i \neq j,
    \end{cases},\quad \mbox{ for \(k=2,\dots K\),}
\\
     {  {\mathfrak{\bs B}}}^{kk}%&=\left[\begin{array}{ccc}T_{11}I_{N_k} + c_1P_k^{-1}(F_1^{kk}+G_k)M_k^{-1}P_k & T_{12}I_{N_k} & T_{1N_\calS}I_{N_k}  \\ T_{21}I_{N_k} & & \\ \vdots &\ddots & \vdots \\ & &   T_{N_\calS-1,N_\calS}I_{N_k} \\  T_{N_\calS1}I_{N_k} &  T_{N_\calS,N_\calS-1}I_{N_k} & T_{N_\calS,N_\calS}I_{N_k} +c_{N_\calS}P_k^{-1}(F_{N_\calS}^{kk}+G_k)M_k^{-1}P_k\end{array}\right]
%    %
%    &=:\left[\begin{array}{ccc} {  {\mathfrak B}}_{11}^{kk} &  {  {\mathfrak B}}^{kk}_{12} &  {  {\mathfrak B}}^{kk}_{1N_\calS}  \\  {  {\mathfrak B}}^{kk}_{21} & & \\ \vdots &\ddots & \vdots \\ & &    {  {\mathfrak B}}^{kk}_{N_\calS-1,N_\calS} \\   {  {\mathfrak B}}^{kk}_{N_\calS1} &   {  {\mathfrak B}}^{kk}_{N_\calS,N_\calS-1} &  {  {\mathfrak B}}^{kk}_{N_\calS,N_\calS} \end{array}\right], 
    %
    &=:\left[\begin{array}{ccc} {  {\mathfrak{\bs B}}}_{11}^{kk} &  \hdots &  {  {\mathfrak{\bs B}}}^{kk}_{1N_\calS}  \\ \vdots &\ddots & \vdots \\   {  {\mathfrak{\bs B}}}^{kk}_{N_\calS1} &  \hdots &  {  {\mathfrak{\bs B}}}^{kk}_{N_\calS,N_\calS} \end{array}\right], \mbox{ for \(k=1,\dots K\),}
    %
\\ {  {\mathfrak{\bs B}}}^{k,k+1}%&=\left[\begin{array}{ccc}c_1P_k^{-1}F_1^{k,k+1}M_{k+1}^{-1}P_{k+1}&  & \\  &\ddots &  \\ & &   \\  &  &c_{N_\calS}P_k^{-1}F_{N_\calS}^{k,k+1}M_{k+1}^{-1}P_{k+1} \end{array}\right] 
%
&=:\left[\begin{array}{ccc} {  {\mathfrak{\bs B}}}^{k,k+1}_{1,1}& \hdots &  {  {\mathfrak{\bs B}}}^{k,k+1}_{1,N_\calS}\\  \vdots &\ddots & \vdots \\  {  {\mathfrak{\bs B}}}^{k,k+1}_{N_\calS,1} & \hdots & {  {\mathfrak{\bs B}}}^{k,k+1}_{N_\calS,N_\calS} \end{array}\right],\mbox{ for \(k=1,\dots K-1\),}
%
\\ {  {\mathfrak{\bs B}}}^{k,k-1}%&=\left[\begin{array}{ccc}c_1P_k^{-1}F_1^{k,k-1}M_{k-1}^{-1}P_{k-1}&  & \\  &\ddots &  \\  &  &c_{N_\calS}P_k^{-1}F_{N_\calS}^{k,k-1}M_{k-1}^{-1}P_{k-1} \end{array}\right]
%
&=:\left[\begin{array}{ccc} {  {\mathfrak{\bs B}}}^{k,k-1}_{1,1}& \hdots &  {  {\mathfrak{\bs B}}}^{k,k-1}_{1,N_\calS}\\  \vdots &\ddots & \vdots \\  {  {\mathfrak{\bs B}}}^{k,k-1}_{N_\calS,1} & \hdots & {  {\mathfrak{\bs B}}}^{k,k-1}_{N_\calS,N_\calS} \end{array}\right], \mbox{ for \(k=2,\dots K\).}
\end{align*} 
Then
\begin{align*}
{  {\mathfrak{\bs B}}} &= \left[\begin{array}{llllll}
	\bs T_{\nabla\nabla}&  {  {\mathfrak{\bs B}}}^{\nabla1} & & & & \\
	 {  {\mathfrak{\bs B}}}^{1\nabla} &  {  {\mathfrak{\bs B}}}^{11} &  {  {\mathfrak{\bs B}}}^{12} & & & \\
	&  {  {\mathfrak{\bs B}}}^{21} &  {  {\mathfrak{\bs B}}}^{22} &  {  {\mathfrak{\bs B}}}^{23} & & \\
	& & \ddots & \ddots & \ddots & \\
	& &  {   {\mathfrak{\bs B}}}^{K-1,K-2} & {   {\mathfrak{\bs B}}}^{K-1,K-1} &  {  {\mathfrak{\bs B}}}^{K-1,K} & \\
	& & & {   {\mathfrak{\bs B}}}^{K,K-1} &  {  {\mathfrak{\bs B}}}^{K,K} &  {  {\mathfrak{\bs B}}}^{K,\Delta} \\
	& & & &  {  {\mathfrak{\bs B}}}^{\Delta, K} & \bs T_{\Delta\Delta}
\end{array}\right].\end{align*}

	\begin{rem}
	One may recognise the structure of \( {  {\mathfrak{\bs B}}}\) as the structure of a quasi-birth-and-death process (QBD), with levels \(k=1,...,K\). This raises the question of whether \( {  {\mathfrak{\bs B}}}\) is indeed a representation of the generator matrix of a QBD, or QBD-like process. In the case of a constant basis function on each cell, i.e.~\(N_k=1\) and \(\phi_1^k(x)\propto1\), \(k=1,...,K\), then \( {  {\mathfrak{\bs B}}}\) is the generator of a QBD: it has zero row-sums, negative diagonal entires, and non-negative off-diagonal entries, the QBD-phase variable is \(\{\varphi_t\}\) and the level is \(k=\nabla,1,...,K,\Delta\). In fact, if \(h_k\) is the same for every \(k=1,...,K\), then this is the same QBD discretisation of a stochastic fluid process analysed by \cite{bo2013}. However, for higher-degree polynomials \(\mathfrak{\bs B}\) is not necessarily the generator of a QBD process. We conjecture that, using polynomial basis functions, then \(N_k=1\) and \(\phi_1^k(x)\propto1\), \(k=1,...,K\) is the only DG approximation which has an interpretation as a QBD-like process -- not even as a QBD-RAP \citep{bn2010}.
	\end{rem}
	
	In the following lemma, we use the following properties of the Lagrange interpolating polynomials defined by the Gauss-Lobatto quadrature nodes. 
	\paragraph{Property 1} \(\displaystyle\sum_{s=1}^{N_k} \phi_s^k(x) = \begin{cases} 1&x\in\calD_k,\\ 0 & x\notin \calD_k.\end{cases}\)

	For \(k\in\{1,...,K\}\), let \(\boldsymbol e_n^k\) be a row-vector of length \(N_k\) with a 1 in the \(n\)th position and zeros elsewhere.
	\paragraph{Property 2} At the cell edges, \(\bs \phi^k(x_k) = \bs e_1^k\) and \(\bs\phi^k(x_{k+1})=\bs e_{N_k}^k\), \(k=1,...,K\). 

\begin{lem}
	If \(\{\phi_r^k(x)\}_{r\in\mathcal N_k}\), are chosen as the Lagrange interpolating polynomials on \(\mathcal D_k\), \(k\in\{1,\dots,K\}\), then the matrix \( {  {\mathfrak{\bs B}}}\) has zero row-sums. 
\end{lem}
\begin{proof}
	Let \(\boldsymbol 1\)\, and \(\boldsymbol 0\) be column vectors of ones and zeros, respectively, with an appropriate length depending on the context. Using Property 1, observe that 
	\begin{align*}
        \bs M_k\boldsymbol 1 &= \left(\sum_{s=1}^{N_k} \int_{x\in\calD_k}\phi_r^k(x)\phi_s^k(x)\wrt x\right)_{r\in\mathcal N_k}\tr{}
	%
        \\&= \left(\int_{x\in\calD_k}\phi_r^k(x)\sum_{s=1}^{N_k} \phi_s^k(x)\wrt x \right)_{r\in\mathcal N_k} \tr{}
        %
        \\&= \left(\int_{x\in\calD_k}\phi_r^k(x)\wrt x\right)_{r\in\mathcal N_k} \tr{}
        %
        \\&= \bs P_k\boldsymbol 1,
	\end{align*}
	hence \(\bs M_k^{-1}\bs P_k\boldsymbol 1=\boldsymbol 1\). Also 
	\begin{align*}
        \bs G_k\boldsymbol 1 &= \left(\sum_{s=1}^{N_k} \int_{x\in\calD_k}\phi_r^k(x)\cfrac{\wrt }{\wrt x}\phi_s^k(x)\wrt x\right)_{r\in\mathcal N_k} \tr{}
	%
        \\&= \left(\int_{x\in\calD_k}\phi_r^k(x)\cfrac{\wrt }{\wrt x}\sum_{s=1}^{N_k} \phi_s^k(x)\wrt x \right)_{r\in\mathcal N_k} \tr{}
	%
        \\&= \left(\int_{x\in\calD_k}\phi_r^k(x)\cfrac{\wrt }{\wrt x}1\wrt x\right)_{r\in\mathcal N_k} \tr{}
	%
        \\&= \boldsymbol 0,
	\end{align*}
	where we have again used Property 1. 
	
	Consider first \(c_i>0\). Let \(\boldsymbol b\) and \(\boldsymbol d\) be arbitrary row-vectors of length \(N_k\) and \(N_{k+1}\), respectively. By Property 2, for \(k=1,...,K-1,\)
	\begin{align*}
        \bs F_i^{kk}\boldsymbol b &= -\boldsymbol \phi^k(x_{k+1})\tr{} \boldsymbol \phi^k(x_{k+1})\boldsymbol b \\&= -(\bs e^k_{N_k})\tr{} \bs e_{N_k}^k\bs b \\&= -b_{N_k}(\boldsymbol e_{N_k}^k)\tr{},
        %
        \\\bs F_i^{k,k+1}\boldsymbol d &= \boldsymbol \phi^{k}(x_{k+1})\tr{} \boldsymbol \phi^{k+1}(x_{k+1})\boldsymbol d \\&= (\bs e^{k}_{N_k})\tr{} \bs e_{1}^{k+1}\bs d \\&= d_{1}(\boldsymbol e_{N_{k}}^{k})\tr{}.
	\end{align*}
	Therefore, for \(c_i>0\), we claim
	\begin{align*}
		&\sum_{j\in\calS}T_{ij}\bs I_{N_k}\bs 1 + c_i \bs P_k^{-1}(\bs F_i^{kk}+\bs G_k)\bs M_k^{-1}\bs P_k\boldsymbol 1 + c_i \bs P_k^{-1}\bs F_i^{k,k+1}\bs M_k^{-1}\bs P_k\boldsymbol 1 = \bs 0.
	\end{align*}
	The first sum is zero since \(\bs T\) is a generator of a continuous-time Markov chain. This leaves the other two terms, which, using our previous observations, we get 
	\begin{align*}
		& c_i \bs P_k^{-1}(\bs F_i^{kk}+\bs G_k)\bs M_k^{-1}\bs P_k\boldsymbol 1 + c_i \bs P_k^{-1}\bs F_i^{k,k+1}\bs M_{k+1}^{-1}\bs P_{k+1}\boldsymbol 1
		\\&= c_i \bs P_k^{-1}(\bs F_i^{kk}+\bs G_k)\boldsymbol 1 + c_i \bs P_k^{-1}\bs F_i^{k,k+1}\boldsymbol 1
		\\&= c_i \bs P_k^{-1}\bs F_i^{kk}\boldsymbol 1 + c_i \bs P_k^{-1}\bs G_k\boldsymbol 1 +  c_i \bs P_k^{-1}F_i^{k,k+1}\boldsymbol 1
		\\&= c_i \bs P_k^{-1}(-\boldsymbol e_{N_k}^k)\tr{} + \boldsymbol 0 + c_i \bs P_k^{-1}(\boldsymbol e_{N_k}^k)\tr{}
		\\&= \boldsymbol 0.
	\end{align*}
	
	Similarly, for \(c_i<0\), and row-vectors \(\boldsymbol b\) and \(\boldsymbol d\) of length \(N_k\) and \(N_{k-1}\), respectively, 
	\begin{align*}
		\bs F_i^{kk}\boldsymbol b &= \boldsymbol \phi^k(x_{k})\tr{} \boldsymbol \phi^k(x_{k})\boldsymbol b \\&= (\bs e^{k}_{1})\tr{} \bs e_{1}^{k}\bs b  \\&= b_{1}(\boldsymbol e_{1}^k)\tr{} 
		%
		\\\bs F_i^{k,k-1}\boldsymbol d &= -\boldsymbol \phi^{k}(x_{k})\tr{} \boldsymbol \phi^{k-1}(x_{k})\boldsymbol d \\&= -(\bs e^{k}_{1})\tr{} \bs e_{N_{k-1}}^{k-1}\bs d \\&= -d_{N_{k-1}}(\boldsymbol e_{1}^k)\tr{}.
		\end{align*} 
		Therefore, for \(c_i<0\) and \(k=2,...,K\), using the same arguments as before we have
	\begin{align*}
		&\sum_{j\in\calS}T_{ij}\bs I_{N_k}\bs 1 + c_i \bs P_k^{-1}(\bs F_i^{kk}+\bs G_k)\bs M_k^{-1}\bs P_k\boldsymbol 1 + c_i \bs P_k^{-1}\bs F_i^{k,k-1}\bs M_k^{-1}\bs P_k\boldsymbol 1
		\\&= \boldsymbol 0 + c_i \bs P_k^{-1}(\bs F_i^{kk}+\bs G_k)\bs M_k^{-1}\bs P_k\boldsymbol 1 + c_i \bs P_k^{-1}\bs F_i^{k,k+1}\bs M_{k-1}^{-1}\bs P_{k-1}\boldsymbol 1
		\\&= c_i \bs P_k^{-1}(\bs F_i^{kk}+\bs G_k)\boldsymbol 1 + c_i \bs P_k^{-1}\bs F_i^{k,k-1}\boldsymbol 1
		\\&= c_i \bs P_k^{-1}\bs F_i^{kk}\boldsymbol 1 + c_i \bs P_k^{-1}\bs G_k\boldsymbol 1 + c_i \bs P_k^{-1}\bs F_i^{k,k-1}\boldsymbol 1
		\\&= c_i \bs P_k^{-1}(\boldsymbol e_{1}^k)\tr{} + \boldsymbol 0 + c_i \bs P_k^{-1}(-\boldsymbol e_{1}^k)\tr{}
		\\&= \boldsymbol 0.
	\end{align*}
	
	For the lower boundary,
	\begin{align*}
		\bs T_{\nabla\nabla}\boldsymbol 1 +  {  {\mathfrak{\bs B}}}^{\nabla1}\boldsymbol 1 
		&= \bs T_{\nabla\nabla}\boldsymbol 1 + \left[\bs T_{\nabla+}\otimes\left( \boldsymbol\phi^1(0)\bs M_1^{-1}\bs P_1\right)\right]\boldsymbol 1.
		\intertext{Swapping the order of summation and recalling \(\bs M_k^{-1}\bs P_k\boldsymbol 1=\boldsymbol 1\) then this is equal to}
		&\bs T_{\nabla\nabla}\boldsymbol 1 + \left[\bs T_{\nabla+}\otimes\left( \boldsymbol\phi^1(0)\bs M_1^{-1}\bs P_1\right)\boldsymbol 1\right]\boldsymbol 1 
		%
		\\&= \bs T_{\nabla\nabla}\boldsymbol 1 + \left[\bs T_{\nabla+}\otimes \boldsymbol e_1^1\bs 1\right]\bs 1
		\\&= \bs T_{\nabla\nabla}\boldsymbol 1 + \left[\bs T_{\nabla+}\otimes \boldsymbol 1\right]\bs 1
		\\&= \bs T_{\nabla\nabla}\boldsymbol 1 + \bs T_{\nabla+}\bs 1
		%
		\\&= \boldsymbol 0.
	\end{align*}
	Also, for \(c_i<0\), 
	\begin{align*}
		-c_i \bs P_1^{-1}\boldsymbol \phi^1(0)\tr{} + c_i \bs P_1^{-1}(\bs F_i^{1,1}+\bs G_k)\bs M_1^{-1}\bs P_1\boldsymbol 1
		&= -c_i \bs P_1^{-1}\boldsymbol \phi^1(0)\tr{} + c_i \bs P_1^{-1}(\bs F_i^{1,1}+\bs G_k)\boldsymbol 1
		%
		\\&=-c_i \bs P_1^{-1}\boldsymbol \phi^1(0)\tr{} + c_i \bs P_1^{-1}\bs F_i^{1,1}\boldsymbol 1
		%
		\\&=-c_i \bs P_1^{-1}(\boldsymbol e_1^1)\tr{} + c_i \bs P_1^{-1}(\boldsymbol e_1^1)\tr{} 
		\\&= \bs 0.
	\end{align*}
	
	For the upper boundary,
	\begin{align*}
		\bs T_{\Delta\Delta}\boldsymbol 1 +  {  {\mathfrak{\bs B}}}^{\Delta K}\boldsymbol 1 
		&= \bs T_{\Delta\Delta}\boldsymbol 1 + [\bs T_{\Delta-}\otimes( \boldsymbol\phi^K(\mathcal I)\bs M_K^{-1}\bs P_K)]\boldsymbol 1.
		\intertext{Swapping the order of summation and recalling \(\bs M_k^{-1}\bs P_k\boldsymbol 1=\boldsymbol 1\) then this is equal to}
        \bs T_{\Delta\Delta}\boldsymbol 1 + [\bs T_{\Delta-}\otimes( \boldsymbol\phi^K(\mathcal I)\bs M_K^{-1}\bs P_K)\bs 1]\boldsymbol 1 
		%
		%\\&%= T_{\Delta\Delta}\boldsymbol 1 + [T_{\Delta-}\otimes \boldsymbol\phi^K(\mathcal I)\bs 1]\boldsymbol 1
		&= \bs T_{\Delta\Delta}\boldsymbol 1 + [\bs T_{\Delta-}\otimes\bs e^K_{N_K}\bs 1]\boldsymbol 1
		\\&= \bs T_{\Delta\Delta}\boldsymbol 1 + [\bs T_{\Delta-}\otimes \bs e^K_{N_K}]\bs 1
		\\&= \bs T_{\Delta\Delta}\boldsymbol 1 + \bs T_{\Delta-}\bs 1
		%
		\\&= \boldsymbol 0.
	\end{align*}
	Also, for \(c_i>0\), 
	\begin{align*}
		c_i \bs P_K^{-1}\boldsymbol \phi^K(\mathcal I)\tr{} + c_i \bs P_K^{-1}(\bs F_i^{K,K}+\bs G_K)\bs M_K^{-1}\bs P_K\boldsymbol 1
		&= c_i \bs P_K^{-1}\boldsymbol \phi^K(\mathcal I)\tr{} + c_i \bs P_K^{-1}(\bs F_i^{K,K}+\bs G_K)\boldsymbol 1
		%
		\\&=c_i \bs P_K^{-1}\boldsymbol \phi^K(\mathcal I)\tr{} + c_i \bs P_K^{-1}\bs F_i^{K,K}\boldsymbol 1
		%
		\\&=c_i \bs P_K^{-1}(\boldsymbol e_{N_K}^K)\tr{} + c_i \bs P_K^{-1}(-\boldsymbol e_{N_K}^K)\tr{} 
		\\&= \bs 0.
	\end{align*}
	
	Combining all of the above we have shown that the row sums of \( {  {\mathfrak B}}\) are zero. 
\end{proof}

\begin{cor}
	The DG approximation to the generator \(  \bs B\) conserves probability. That is, for all \(t\geq 0\), 
	\begin{align*}
	&\sum_{i\in\calS_\nabla}q_{\nabla,i}(t)+\sum_{i\in\calS_\Delta}q_{\Delta,i}(t)+\sum_{i\in\calS} \int_{x\in[0,\mathcal I]}u_i(x,t)\wrt x 
	%
	\\&= \sum_{i\in\calS_\nabla}q_{\nabla,i}(0)+\sum_{i\in\calS_\Delta}q_{\Delta,i}(0)+\sum_{i\in\calS} \int_{x\in[0,\mathcal I]}u_i(x,0)\wrt x.
	\end{align*}
\end{cor}
\begin{proof}
%For a basis of Lagrange polynomials the DG operator \( {  {\mathfrak B}}\) has zero row-sums, therefore 
%\begin{align*}
%	\boldsymbol \alpha(t)\boldsymbol 1 
%	%
%	=  \boldsymbol \alpha(0)\exp( {  {\mathfrak B}}t)\boldsymbol 1 
%	%
%	= \boldsymbol \alpha(0)\boldsymbol 1. 
%\end{align*}
Let \(\{\psi_r^k(x)\}_{r\in\mathcal N_k,k\in\{1,...,K\}},\) be a basis for \(span(\phi_r^k(x),r\in\mathcal N_k,k\in\{1,...,K\})\), where \(\{\phi_r^k(x)\}_{r\in\mathcal N_k,k\in\{1,...,K\}}\) are the Lagrange polynomials. Also define \(\psi_1^\nabla(x)=\delta(x)\) and \(\psi_1^\Delta(x)=\delta(x-\mathcal I)\) to capture the point masses at the boundaries. Let us use the same vector notation for the basis \(\psi_r^k(x)\) as we do for \(\phi_r^k(x)\). For \(k\in\{1,...,K\}\), since \(\{\psi_r^k(x)\}_{r\in\mathcal N_k}\) and \(\{\phi_r^k(x)\}_{r\in\mathcal N_k}\) have the same span, then there is a matrix \(\bs V^k\) such that  \(\bs \psi^k(x)\tr{} = \bs V^k\bs \phi^k(x)\tr{}\). Trivially, this also holds for \(k=\nabla,\Delta\). 

Let 
\[ \bs W = \left[\begin{array}{ccc}
		\bs I_{|\calS_\nabla|} & & \\
		& \bs P &  \\
		& & \bs I_{|\calS_\Delta|}
	\end{array}\right]
\mbox{ and }
  \bs V = \left[\begin{array}{ccccc}
		\bs I_{|\calS_\nabla|} & & & &  \\
		& \bs V^1 & & &  \\
		& & \ddots & & \\
		& & & \bs V^K \\
		& & & & \bs I_{|\calS_\Delta|}
	\end{array}\right]. \]
For a DG approximation, \(\bs B\), constructed from \(\{\psi_r^k\}_{r\in\mathcal N_k,k\in\{\nabla,1,...,K,\Delta\}}\), it can be shown that \(\bs B\) is similar to \( {  {\mathfrak{\bs B}}}\) with similarity matrix, \(\bs V\bs W\), such that \(  \bs B_{ij} =  \bs V\bs W{ \mathfrak{\bs B}}_{ij}\bs W^{-1}\bs V^{-1},\,i,j\in\calS\).
Therefore 
\begin{align*} 
	\int_{x\in[0,\mathcal I]}  \bs B_{ij}\bs\psi(x)\tr{} \wrt x &= \bs V\bs W { \mathfrak{\bs B}}_{ij}\bs W^{-1}\bs V ^{-1}\int_{x\in[0,\mathcal I]} \bs V\bs \phi(x)\tr{}\wrt x 
	%
	\\&=\bs V\bs W { \mathfrak{\bs B}}_{ij}\bs W^{-1} \bs W\bs 1 \\&= \bs V \bs W { \mathfrak{\bs B}}_{ij}\bs 1,
\end{align*}
since \(\displaystyle\int_{x\in[0,\mathcal I]} \bs \phi(x)\tr{}\wrt x =\bs W\bs 1\).
The row sums of \(\mathfrak{\bs B}\) are 0, hence 
\begin{align}\label{eqn:Bsums0} 
	\int_{x\in[0,\mathcal I]}  \sum_{j\in\calS}\bs B_{ij}\bs\psi(x)\tr{} \wrt x  &=  \bs V\bs W \sum_{j\in\calS}{ \mathfrak{\bs B}}_{ij}\bs 1 \\&=  \bs V\bs W \bs 0 \\&= \bs 0.
\end{align}

Let \( {}^\psi \bs a_i(t)\), \(i\in\calS\) denote the coefficients related to the DG approximation constructed with the basis \(\{\psi_r^k\}_{r\in\mathcal N_k,k\in\{\nabla,1,...,K,\Delta\}}\) (to distinguish them from \(\bs a\) and \(\bs \alpha\) used above). The DE constructed by the DG method is
\begin{align*}
	\cfrac{\wrt}{\wrt t}\left({}^\psi \boldsymbol a_j (t)\right)\boldsymbol \psi(x) = \sum_{i\in\calS}\left({}^\psi \boldsymbol a_i (t)\right)\bs B_{ij}\bs\psi(x).
\end{align*}
Integrating over \(x\in [0,\mathcal I]\) and summing over \(j\in\calS\) we get
\begin{align*}
	\int_{x\in[0,\mathcal I]}\sum_{j\in\calS}\cfrac{\wrt}{\wrt t}\left({}^\psi \boldsymbol a_j (t)\right)\boldsymbol \psi(x)\wrt x = \int_{x\in[0,\mathcal I]}\sum_{j\in\calS}\sum_{i\in\calS}\left({}^\psi \boldsymbol a_i (t)\right)\bs B_{ij}\bs\psi(x).
\end{align*}
Exchanging the order of operations gives 
\begin{align}\label{eqn:totalprobDE}
	\cfrac{\wrt}{\wrt t}\sum_{j\in\calS} \left({}^\psi \boldsymbol a_j (t)\right)\int_{x\in[0,\mathcal I]} \boldsymbol \psi(x)\wrt x =  \sum_{i\in\calS}\left({}^\psi \boldsymbol a_i (t)\right)\int_{x\in[0,\mathcal I]}\sum_{j\in\calS}\bs B_{ij}\bs\psi(x)\wrt x = 0,
\end{align}
where the right-hand side is \(0\) due to Equation (\ref{eqn:Bsums0}). This holds for all \(t\geq 0\). The left-hand side of (\ref{eqn:totalprobDE}) is the rate of change (with respect to time) of the total mass of the system. Since this is \( 0\) for all \(t\geq 0\), there is no change in the total mass of the system and thus probability is conserved. 
%\begin{align*}
%	&\sum_{i\in\calS_\nabla}q_{\nabla,i}(t)+\sum_{i\in\calS_\Delta}q_{\Delta,i}(t)+\sum_{i\in\calS}\int_{x\in[0,\mathcal I]} u_i(x,t)\wrt x 
%	%
%	\\&=\sum_{i\in\calS_\nabla}q_{\nabla,i}(t)+\sum_{i\in\calS_\Delta}q_{\Delta,i}(t)+\sum_{i\in\calS}\int_{x\in[0,\mathcal I]}\boldsymbol a_i(t) \boldsymbol \psi(x)\wrt x 
%	%
%	\\&=\sum_{j\in\calS}\int_{x\in[0,\mathcal I]}\left[\vligne{\bs q_{\nabla}(0) & \boldsymbol a(0) & \bs q_{\Delta}(0)} \exp(  Bt)\right]_{j}\boldsymbol \psi(x)\wrt x 
%	%
%	\\&=\sum_{j\in\calS}\int_{x\in[0,\mathcal I]}\left[\vligne{\bs q_{\nabla}(0) & \boldsymbol a(0) & \bs q_{\Delta}(0)} \left(I+  Bt + \cfrac{  B^2t^2}{2!}+...\right)\right]_{j}\boldsymbol \psi(x)\wrt x
%	%
%	\\&=\sum_{i\in\calS_\nabla}q_{\nabla,i}(0)+\sum_{i\in\calS_\Delta}q_{\Delta,i}(0)+\sum_{i,j\in\calS}\int_{x\in[0,\mathcal I]}\boldsymbol a_i(0)\boldsymbol \psi(x)\wrt x
%	%
%	\\&=\sum_{i\in\calS_\nabla}q_{\nabla,i}(0)+\sum_{i\in\calS_\Delta}q_{\Delta,i}(0)+\sum_{i\in\calS}\int_{x\in[0,\mathcal I]} u_i(x,0)\wrt x.
%\end{align*}
\end{proof}
%\begin{remark}
%	For any choice of basis \(\{\phi_r^k(x)\}_{r\in\{1,...,N_k\},k\in\{1,...,K\}}\) which spans the set of polynomials of degree \(N_k-1\), the DG approximation constructed with this basis will conserve probability. 
%\end{remark}