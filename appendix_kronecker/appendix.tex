%!TEX root = ../thesis.tex
\chapter{Algebraic manipulations of certain Kronecker products}\label{appendix: kronecker}
%\section{Matrix exponentials}\label{appendix: matrix exponentials}
%This appendix contains some technical results regarding the matrix exponential.

\begin{lem}[\cite{ln2015}]\label{lem: sfkjgn}
	Let \(\bs B\) be the block-partitioned matrix
	\[\bs B = \left[\begin{array}{cc} \bs B_{11} & \bs B_{12} \\ \bs B_{21} & \bs B_{22} \end{array}\right]\]
	where \(\bs B_{11}\) and \(\bs B_{22}\) are matrices of order \(m_1\) and \(m_2\), respectively. Denote by \(\bs H_{11}(t)\) the top-left quadrant of order \(m_1\) of \(e^{\bs Bt}\):
	\[\bs H_{11}(t) = \vligne{\bs I_{m_1\times m_1} & \bs 0} e^{\bs Bt}\left[\begin{array}{c}\bs I_{m_1\times m_1} \\ \bs 0\end{array}\right].\]
	
	The matrix \(\bs H_{11}(t)\) is the solution of 
	\begin{align}\label{eqn: akg987LKJ}
		\bs H_{11}(t) = e^{\bs B_{11}t} + \int_{v=0}^t\int_{u=v}^t e^{\bs B_{11}(t-u)}\bs B_{12}e^{\bs B_{22}(u-v)}\bs B_{21}\bs H_{11}(v)\wrt u\wrt v.
	\end{align}
\end{lem}
\begin{proof}
	See \cite{ln2015}.
\end{proof}

Let \(\bs H_{12}(t)\) be the top-right quadrant of \(e^{\bs Bt}\) of size \(m_1\times m_2\), i.e.
\begin{align}
	\bs H_{12}(t) &= \vligne{  \bs I_{m_1\times m_1} & \bs 0}e^{\bs Bt} \left[\begin{array}{c} \bs 0 \\ \bs I_{m_2\times m_2}\end{array}\right]. \label{eqn: p03j}
\end{align}
Denote by \(\widehat{\bs H}_{11}(\lambda):=\displaystyle \int_{t=0}^\infty e^{-\lambda t}\bs H_{11}(t)\wrt t\) and by \(\widehat{\bs H}_{12}(\lambda):=\displaystyle \int_{t=0}^\infty e^{-\lambda t}\bs H_{12}(t)\wrt t\), the Laplace transforms of \(\bs H_{11}(t)\) and \(\bs H_{12}(t)\), respectively. Using Lemma~\ref{lem: sfkjgn} we can show the following result. 
\begin{lem} Assuming \(\lambda >0\) and \(\lambda\bs I-\bs B_{22}\) is invertible, then 
	\begin{align}
		\widehat{\bs H}_{11}(\lambda) &= \int_{x=0}^\infty e^{\left(\bs B_{11} -\lambda \bs I_{m_1\times m_1} + \bs B_{12}(\lambda \bs I_{m_2\times m_2}- \bs B_{22})^{-1}\bs B_{21}\right)x} \wrt x,\label{eqn: 202}
		%
		\\\widehat{\bs H}_{12}(\lambda) &= \int_{x=0}^\infty e^{\left(\bs B_{11} -\lambda \bs I_{m_1\times m_1} + \bs B_{12}(\lambda \bs I_{m_2\times m_2}- \bs B_{22})^{-1}\bs B_{21}\right)x} \bs B_{12}(\lambda \bs I_{m_2\times m_2}-\bs B_{22})^{-1}\wrt x.\label{eqn: 203}
	\end{align}
\end{lem}
\begin{proof}
	First we show the result for \(\widehat{\bs H}_{11}(\lambda)\). Taking the Laplace transform of (\ref{eqn: akg987LKJ}) shows that \(\widehat{\bs H}_{11}(\lambda)\) is equal to 
	\begin{align}
		&\int_{t=0}^\infty \int_{v=0}^t\int_{u=v}^t e^{-\lambda (t-u)}e^{\bs B_{11}(t-u)}\bs B_{12}e^{-\lambda (u-v)}e^{\bs B_{22}(u-v)}\bs B_{21}e^{-\lambda v}\bs H_{11}(v)\wrt u\wrt v \nonumber 
		\\&\qquad{}+ (\lambda \bs I_{m_1\times m_1} - \bs B_{11})^{-1}\nonumber 
		%
		\\& = (\lambda \bs I_{m_1\times m_1} - \bs B_{11})^{-1} + (\lambda \bs I_{m_1\times m_1} - \bs B_{11})^{-1}\bs B_{12}(\lambda \bs I_{m_2\times m_2}-\bs B_{22})^{-1}\bs B_{21}\widehat{\bs H}_{11}(\lambda),
	\end{align}
	by the convolution theorem for Laplace transforms. This implies
	\begin{align*}
		&\left[\bs I_{m_1\times m_1} -  (\lambda \bs I_{m_1\times m_1} - \bs B_{11})^{-1}\bs B_{12}(\lambda \bs I_{m_2\times m_2}-\bs B_{22})^{-1}\bs B_{21}\right]\widehat{\bs H}_{11}(\lambda) 
		\\&= (\lambda \bs I_{m_1\times m_1} - \bs B_{11})^{-1},
	\end{align*}
	and therefore 
	\begin{align*}
		&\widehat{\bs H}_{11}(\lambda) 
		\\&= \left[\bs I_{m_1\times m_1} -  (\lambda \bs I_{m_1\times m_1} - \bs B_{11})^{-1}\bs B_{12}(\lambda \bs I_{m_2\times m_2}-\bs B_{22})^{-1}\bs B_{21}\right]^{-1} (\lambda \bs I_{m_1\times m_1} - \bs B_{11})^{-1}
		%
		\\&= \left[(\lambda \bs I_{m_1\times m_1} - \bs B_{11})\left(\bs I_{m_1\times m_1} -  (\lambda \bs I_{m_1\times m_1} - \bs B_{11})^{-1}\bs B_{12}(\lambda \bs I_{m_2\times m_2}-\bs B_{22})^{-1}\bs B_{21}\right)\right]^{-1} 
		%
		\\&= \left[\lambda \bs I_{m_1\times m_1} - \bs B_{11} - \bs B_{12}(\lambda \bs I_{m_2\times m_2}-\bs B_{22})^{-1}\bs B_{21}\right]^{-1}
		%
		\\&= \int_{t=0}^\infty e^{\left(\bs B_{11} - \lambda \bs I_{m_1\times m_1} + \bs B_{12}(\lambda \bs I_{m_2\times m_2}-\bs B_{22})^{-1}\bs B_{21}\right)t}\wrt t,
	\end{align*}
	which is (\ref{eqn: 202}). 
	
	Now, to show (\ref{eqn: 203}), differentiate (\ref{eqn: p03j})
	\begin{align}
		\cfrac{\wrt}{\wrt t}\bs H_{12}(t) &= \vligne{  \bs I_{m_1\times m_1} & \bs 0}e^{\bs Bt} \left[\begin{array}{cc} \bs B_{11} & \bs B_{12} \\ \bs B_{21} & \bs B_{22} \end{array}\right] \left[\begin{array}{c} \bs 0 \\ \bs I_{m_2\times m_2}\end{array}\right] \nonumber
		%
		\\&= \vligne{  \bs I_{m_1\times m_1} & \bs 0}e^{\bs Bt} \left[\begin{array}{c} \bs B_{12} \\ \bs B_{22} \end{array}\right] \nonumber
		%
		\\&= \bs H_{11}(t) \bs B_{12} + \bs H_{12}(t)\bs B_{22}.
	\end{align}
	Now take the Laplace transform 
	\begin{align}
		\lambda \widehat{\bs H}_{12}(\lambda) - \bs H_{12}(0) = \widehat{\bs H}_{11}(\lambda) \bs B_{12} + \widehat{\bs H}_{12}(\lambda)\bs B_{22}.
	\end{align}
	Since \(\bs H_{12}(0)=\bs 0\) and after rearranging we get 
	\begin{align}
		\widehat{\bs H}_{12}(\lambda) = \widehat{\bs H}_{11}(\lambda) \bs B_{12} (\lambda \bs I_{m_2\times m_2} -\bs B_{22})^{-1},
	\end{align}
	which gives (\ref{eqn: 203}) upon substituting (\ref{eqn: 202}).
\end{proof}

Now, recall the matrix-functions
\begin{align*}
	% \bs Q_{+0}(\lambda) &= \bs C_+^{-1}\bs T_{+0}\left[\lambda \bs I - \bs T_{00}\right]^{-1},
	% %
	% \\\bs Q_{-0}(\lambda) &= \bs C_-^{-1}\bs T_{-0}\left[\lambda \bs I - \bs T_{00}\right]^{-1},
	%
	\bs Q_{++}(\lambda) &= \bs C_+^{-1} \left(\bs T_{++} - \lambda \bs I + \bs T_{+0}\left[\lambda \bs I - \bs T_{00}\right]^{-1}\bs T_{0+}\right),
	%
	\\\bs Q_{+-}(\lambda) &= \bs C_+^{-1} \left(\bs T_{+-} + \bs T_{+0}\left[\lambda \bs I - \bs T_{00}\right]^{-1}\bs T_{0-} \right) ,
	%
	\\\bs Q_{--}(\lambda) &= \bs C_-^{-1} \left(\bs T_{--}  - \lambda \bs I + \bs T_{-0}\left[\lambda \bs I - \bs T_{00}\right]^{-1}\bs T_{0-}\right),
	%
	\\\bs Q_{-+}(\lambda) &=\bs C_-^{-1} \left(\bs T_{-+}+ \bs T_{-0}\left[\lambda \bs I - \bs T_{00}\right]^{-1}\bs T_{0+}\right) ,
\end{align*}
from Chapter~\ref{sec: conv}.
\begin{cor}\label{cor: mpr B}
	For \(m\in\{+,-\}\) the top-left quadrant of size \(m_1\times m_1=|\calS_m|  p \times |\calS_m|  p\) of \(e^{\bs B_{mm}t}\), 
	\begin{align}
		&\vligne{\bs I_{|\calS_m|  p\times |\calS_m|  p} & \bs 0_{|\calS_m|  p\times |\calS_0|  p}}\int_{t=0}^\infty e^{-\lambda t} \exp\left\{\left[\begin{array}{cc} \bs T_{mm}\otimes \bs I + \bs C_m\otimes \bs S & \bs T_{m0}\otimes \bs I \\ \bs T_{0m} \otimes \bs I & \bs T_{00}\otimes \bs I \end{array}\right]t\right\} \wrt t
		\\&\quad{}\times \left[\begin{array}{c} \bs I_{|\calS_m|  p\times |\calS_m|  p} \\ \bs 0_{|\calS_0|  p\times |\calS_m|  p} \end{array}\right], \nonumber
	\end{align}
	is given by 
	\begin{align}
		\int_{x=0}^\infty e^{\bs Q_{mm} (\lambda)x}\otimes  e^{\bs S x}\wrt x(\bs C_m^{-1}\otimes \bs I).\label{eqn: skagh87} 
	\end{align}
	For \(m\in\{+,-\}\) the top-right quadrant of size \(m_1\times m_2=|\calS_m|  p\times |\calS_0|  p\) of  \(e^{\bs B_{mm}t}\), 
	\begin{align}
		&\vligne{\bs I_{|\calS_m|  p\times |\calS_m|  p} & \bs 0_{|\calS_m|  p\times |\calS_0|  p}}\int_{t=0}^\infty e^{-\lambda t} \exp\left\{\left[\begin{array}{cc} \bs T_{mm}\otimes \bs I + \bs C_m\otimes \bs S & \bs T_{m0}\otimes \bs I \\ \bs T_{0m} \otimes \bs I & \bs T_{00}\otimes \bs I \end{array}\right]t\right\} \wrt t 
		\\&\quad{}\times \left[\begin{array}{c}\bs 0_{|\calS_m|  p\times |\calS_m|  p} \\ \bs I_{|\calS_0|  p\times |\calS_0|  p}\end{array}\right] ,\nonumber
	\end{align}
	is given by
	\begin{align}
		\int_{x=0}^\infty e^{\bs Q_{mm}(\lambda)x}\otimes  e^{\bs S x}\wrt x((\bs C_m^{-1}\bs T_{m0}(\lambda \bs I - \bs T_{00})^{-1})\otimes \bs I).\label{eqn: skagh873} 
	\end{align}
	Also, 
	\begin{align}
		&\vligne{\bs I_{|\calS_m|  p\times |\calS_m|  p} & \bs 0_{|\calS_m|  p\times |\calS_0|  p} }\int_{t=0}^\infty e^{-\lambda t} e^{\bs{B}_{mm}t} \wrt t \bs{B}_{m{n}} %= \vligne{\bs I & \bs 0 }\int_{t=0}^\infty e^{-\lambda t} e^{\bs{B}_{mm}t} \wrt t  \left[\begin{array}{cc} \bs{T}_{mn}\otimes \bs{D} & \bs 0 \\ \bs T_{0n}\otimes \bs D & \bs 0 \end{array}\right]
		\\&= \int_{x=0}^\infty \bs H^{mn}(\lambda,x)  \otimes  e^{\bs S x}\bs D\wrt x\left(\vligne{\bs I_{|\calS_n|p} & \bs 0_{|\calS_n|p\times |\calS_0|p}}\right), \label{eqn: akgj987adKLDJ}
\end{align}
for \(m,n\in\{+,-\}\), \(m\neq n\).
\end{cor}
\begin{proof}
	From Lemma~\ref{lem: sfkjgn} the top-left quadrant of size \(m_1\times m_1=|\calS_m|  p \times |\calS_m|  p\) of the integral with respect to \(t\) on the left-hand side of (\ref{eqn: skagh87}) is 
	\begin{align}
		\int_{t=0}^\infty e^{\left(\bs T_{mm}\otimes \bs I + \bs C_m\otimes \bs S - \lambda \bs I+ (\bs T_{m0}\otimes \bs I)(\lambda \bs I-\bs T_{00}\otimes \bs I)^{-1}(\bs T_{0m}\otimes \bs I)\right)t}\wrt t. \label{eqn: akjf768}
	\end{align}
	By Lemma~\ref{lem: lst mpr}, (\ref{eqn: akjf768}) is equal to 
	\begin{align}
		&\int_{x=0}^\infty e^{\bs C_m^{-1}\left(\bs T_{mm} - \lambda \bs I+ \bs T_{m0} (\lambda \bs I-\bs T_{00})^{-1}\bs T_{0m} \right)x}\otimes e^{\bs Sx}\wrt x(\bs C_m\otimes \bs I)^{-1}\nonumber
		%
		\\&=\int_{x=0}^\infty e^{\bs Q_{mm} ( \lambda )x}\otimes e^{\bs Sx}\wrt x(\bs C_m\otimes \bs I)^{-1} , \label{eqn: akjf7623988}
	\end{align}
	from the definition of \(\bs Q_{mm}(\lambda)\). This proves (\ref{eqn: skagh87}). 
	
	Now, from Lemma~\ref{lem: sfkjgn} the top-right quadrant of size \(m_1\times m_2=|\calS_m|  p\times |\calS_0|  p\) of the integral with respect to \(t\) on the left-hand side of (\ref{eqn: skagh873}) is 
	\begin{align}
		&\int_{t=0}^\infty e^{\left(\bs T_{mm}\otimes \bs I + \bs C_m\otimes \bs S - \lambda \bs I+ (\bs T_{m0}\otimes \bs I)(\lambda \bs I-\bs T_{00}\otimes \bs I)^{-1}(\bs T_{0m}\otimes \bs I)\right)t}(\bs T_{m0}\otimes \bs I)(\lambda \bs I - \bs T_{00}\otimes \bs I)^{-1} \nonumber 
		\\&\qquad\times (\bs T_{0m}\otimes \bs I)\wrt t. \label{eqn: akjf768f}
	\end{align}
	By Lemma~\ref{lem: lst mpr}, (\ref{eqn: akjf768f}) is equal to 
	\begin{align}
		\int_{x=0}^\infty e^{\bs Q_{mm}(\lambda)x}\otimes e^{\bs Sx}\wrt x(\bs C_m\otimes \bs I)^{-1}(\bs T_{m0}\otimes \bs I)(\lambda \bs I - \bs T_{00}\otimes \bs I)^{-1}(\bs T_{0m}\otimes \bs I) . \label{eqn: akjf7623988g}
	\end{align}
	Now, 
	\begin{align*}
		\displaystyle (\lambda \bs I - \bs T_{00}\otimes \bs I)^{-1} &= \int_{u=0}^\infty e^{-(\lambda \bs I - \bs T_{00}\otimes \bs I)u}\wrt u \\&= \int_{u=0}^\infty e^{-\lambda u} e^{(\bs T_{00}\otimes \bs I)u}\wrt u \\&= \int_{u=0}^\infty e^{-\lambda u} e^{\bs T_{00}u}\otimes \bs I \wrt u,
	\end{align*} by (\ref{eqn:09ksdjgah}). Using this and the~\ref{eqn:mpr} we can write  
	\begin{align}
		&(\bs C_m\otimes \bs I)^{-1}(\bs T_{m0}\otimes \bs I)(\lambda \bs I - \bs T_{00}\otimes \bs I)^{-1}(\bs T_{0m}\otimes \bs I) \nonumber
		\\&= (\bs C_m^{-1}\otimes \bs I)(\bs T_{m0}\otimes \bs I)\int_{u=0}^\infty e^{-\lambda u} e^{\bs T_{00}u}\otimes \bs I \wrt u(\bs T_{0m}\otimes \bs I) \nonumber
		\\&= \left(\bs C_m^{-1}\bs T_{m0}(\lambda \bs I-\bs T_{00}u)^{-1}\bs T_{0m}\right)\otimes \bs I).\label{eqn: q092}
	\end{align}
	Substituting (\ref{eqn: q092}) into (\ref{eqn: akjf7623988g}) completes the proof of (\ref{eqn: skagh873}). 

Now, using (\ref{eqn: skagh87}) and (\ref{eqn: skagh873}) we can write 
\begin{align}
	\nonumber&\vligne{\bs I_{|\calS_m|p\times |\calS_m|p} & \bs 0_{|\calS_m|p\times |\calS_0|p} }\int_{t=0}^\infty e^{-\lambda t} e^{\bs{B}_{mm}t} \wrt t \bs{B}_{m{n}} 
	\\\nonumber &= \vligne{\bs I_{|\calS_m|p\times |\calS_m|p} & \bs 0_{|\calS_m|p\times |\calS_0|p} }\int_{t=0}^\infty e^{-\lambda t} e^{\bs{B}_{mm}t} \wrt t  \left[\begin{array}{cc} \bs{T}_{mn}\otimes \bs{D} & \bs 0 \\ \bs T_{0n}\otimes \bs D & \bs 0 \end{array}\right]
	%
%	\\\nonumber& =\left[ \begin{array}{cc} \displaystyle \int_{x=0}^\infty e^{\bs Q_{mm}(\lambda)x}\otimes  e^{\bs S x}\wrt x(\bs C_m^{-1} \otimes \bs I) & \displaystyle \int_{x=0}^\infty e^{\bs Q_{mm}(\lambda)x}\otimes  e^{\bs S x}\wrt x((\bs C_m^{-1}\bs T_{m0}(\lambda \bs I - \bs T_{00})^{-1})\otimes \bs I)  \end{array}\right]
%	\\\nonumber&\quad\times\left[\begin{array}{cc} \bs{T}_{mn}\otimes \bs{D} & \bs 0 \\ \bs T_{0n}\otimes \bs D & \bs 0 \end{array}\right]
	%
	\\&= \int_{x=0}^\infty e^{\bs Q_{mm}(\lambda)x}\otimes  e^{\bs S x}\wrt x (\bs C_m^{-1}(\bs{T}_{mn} + \bs T_{m0}(\lambda \bs I - \bs T_{00})^{-1}\bs T_{0n}) \vligne{\bs I_{|\calS_n|\times |\calS_n|} & \bs 0_{|\calS_n|\times |\calS_0|}} \otimes \bs D) \nonumber
	%
	\\&= \int_{x=0}^\infty e^{\bs Q_{mm}(\lambda)x}\otimes  e^{\bs S x}\wrt x \left( \left( \bs{Q}_{mn} (\lambda) \vligne{\bs I_{|\calS_n|} & \bs 0_{|\calS_n|\times |\calS_0|}} \right) \otimes \bs D\right)  \nonumber
	%
	\\&= \int_{x=0}^\infty \left(\bs H^{mn}(\lambda,x) \vligne{\bs I_{|\calS_n|} & \bs 0_{|\calS_n|\times |\calS_0|}}\right) \otimes  e^{\bs S x}\bs D\wrt x,\nonumber
	%
	\\&= \int_{x=0}^\infty \bs H^{mn}(\lambda,x)  \otimes  e^{\bs S x}\bs D\wrt x,\vligne{\bs I_{|\calS_m|p\times |\calS_n|p} & \bs 0_{|\calS_n|p\times |\calS_0|p}}, \label{eqn: akgj987ad}
\end{align}
for \(m,n\in\{+,-\}\), \(m\neq n\) which is (\ref{eqn: akgj987adKLDJ}), where the last line holds from the~\ref{eqn:mpr}. 
\end{proof}