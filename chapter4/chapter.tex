%!TEX root = ../thesis.tex
\chapter{Convergence of the QBD-RAP: an analysis on no change of level\label{sec: conv}}
This chapter details convergence of the approximation scheme constructed in Chapter~\ref{sec: construction and modelling} on the event that there is no change of level. Later, in Chapter \ref{ch: global results}, we use the main results of this chapter to prove a global convergence result about the QBD-RAP scheme. 

Recall that the QBD-RAP is constructed using \emph{matrix-exponential distributions} to model, approximately, the sojourn time of the fluid queue in a given interval. The main result of this chapter shows a type of convergence of the approximation scheme, under the assumption that the variance of the matrix-exponential distribution(s) used in the construction tends to 0. %Given that there is no known simple closed form for the transient distributions of a fluid queue, and that we make minimal assumptions about the matrix-exponential distributions used in the construction of the approximation, it is somewhat remarkable that we are able to establish a convergence result. 
The result applies to any sequence of matrix exponential distributions such that the variance tends to zero. The generality of the result is necessitated by the fact that, in practice, we use the class of \emph{concentrated matrix-exponential distributions} found numerically in \citep{hht2020}, for which there is relatively little known about their properties.

In this chapter we analyse the distribution of the QBD-RAP scheme up to the first change of level. The structure of the analysis is to first partition the distribution (before the first change of level) on the number of changes of phase from \(\calS_+\cup\calS_{+0}\) to \(\calS_-\) or \(\calS_-\cup\calS_{-0}\) to \(\calS_+\). We will refer to changes of phase from \(\calS_+\cup\calS_{+0}\) to \(\calS_-\) as an \emph{up-down} transition and changes of phase from \(\calS_-\cup\calS_{-0}\) to \(\calS_+\) as a \emph{down-up} transition. For each term in the partition we then take the Laplace transform with respect to time. This is convenient as it enables an algebraic manipulation of the expression such that we can seprates the expression into one term soley about the orbit process of the QBD-RAP, \(\{\bs A(t)\}\), and one expression about the phase process, \(\{\varphi(t)\}\) and associate rates \(c_i,\,i\in\mathcal S\). Once we have established a convenient algebraic form, we then turn our attention to bounds and convergence, establishing bounds for the difference between the Laplace transforms of the QBD-RAP just derived and the corresponding Laplace transforms of the fluid queue. Thus we establish a convergence result for the Laplace transforms with respect to time for each of the distributions in the partition. We then wish to undo the partitioning and establish a convergence result for the Laplace transform with resepct time of the distribution of the QBD-RAP to the corresponding Laplace transform of the fluid queue, on the event that there is no change of level. 

The main steps of the convergence result of this chapter are show in the list below.
\begin{enumerate}
	\item\label{step 1} Define the sample paths of the QBD-RAP and fluid queue with which we will work. Compute the distributions of these processes on theses sample paths and take the Laplace transform with respect to time. (Sections~\ref{sec: qbd dists}, \ref{sec: no change} and \ref{sec: lst on no change}).
	\item\label{step 2} Show error bounds for the difference between the Laplace transforms of the QBD-RAP and the fluid queue for each term in the partition. (Section~\ref{sec: no change convergence}).
	\item\label{step 3} Show a geometric domination condition so that we may apple the Dominated Convergence Theorem to prove convergence of the Laplace transforms up to the first change of level. (Section~\ref{sec: dom 1}).
\end{enumerate}

First we give some preliminaries describing the sequence of matrix exponential distributions as well as some general technical assumptions we will use in the proofs. %The following two sections are dedicated to defining the distributions, partition and Laplace transforms with which we will work. First, in Sections~\ref{sec: qbd dists} and \ref{sec: no change} we describe the behaviour of the fluid queue and QBD-RAP on the event that there is no change of level. Then, in Section~\ref{sec: lst on no change}, we derive expressions for the Laplace transforms with respect to time of the QBD-RAP and fluid queue on the event that there is no change of level. Section~\ref{sec: no change convergence} provides error bounds between the Laplace transforms of the QBD-RAP and the fluid queue that were derived in Section~\ref{sec: lst on no change}. Section~\ref{sec: dom 1} shows a domination condition so that we may apply the Dominated Convergence Theorem. 

\section*{Preliminaries}
Suppose we have a sequence \(\{Z^{(p)}\}_{p\geq 1}\) of matrix exponential distributions, \(Z^{(p)}\sim ME(\bs\alpha^{(p)},\bs S^{(p)}, \bs s^{(p)})\), such that \(\var\left(Z^{(p)}\right)\to 0 \) as \(p\to \infty\). We use the superscript \((p)\) to denote dependence on the underlying choice of matrix exponential distribution that is used in the construction of the QBD-RAP scheme. To simplify notation, we omit the super script \((p)\) where possible. In the following we show error bounds for an arbitrary parameter \(\varepsilon>0\). However, keep in mind the ultimate intention is to show convergence, for which we choose this parameter to be \(\varepsilon^{(p)}=\var\left(Z^{(p)}\right)^{1/3}\). Other notations which have been defined, which functions of \(Z^{(p)}\) and therefore also implicitly depend on \(p\), are \(\bs\alpha^{(p)},\,\bs S^{(p)},\,\bs s^{(p)},\, \bs S_i^{(p)},\, \bs s_i^{(p)},\, \bs D^{(p)},\,\mathcal A^{(p)},\, \bs Y^{(p)}(t) = (L^{(p)}(t),\bs A^{(p)}(t), \phi^{(p)}(t)),\, \Yd(t).\)

In the following we show various results which involve integrating a function \(g\), or a sequence of functions \(g_1,g_2,\dots\). We make the following assumptions about such functions, 
\begin{asu}\label{asu: g}
	Let \(g\) be a function \(g:[0,\infty)\to [0,\infty)\) which is \\
	\subasu \label{asu: g non-neg} non-negative, 
	\[g(x) \geq 0 \mbox{ for all } x \geq 0,\]
	\subasu bounded, 
	\[g(x) \leq G < \infty \mbox{ for all } x \geq 0,\]
	\subasu integrable, 
	\[\int_{x=0}^\infty g(x)\wrt x \leq \widehat G < \infty,\]
	\subasu \label{asu: lipschitz} and Lipschitz continuous 
	\begin{align}
		|g(x) - g(u)|&\leq L|x - u| \mbox{ for all } x,\, u \geq 0,\, 0<L<\infty.
	\end{align}
\end{asu}

We also need a corresponding sequence of closing operators which we denote by \({\bs v}^{(p)}\). For the convergence results, we require the following properties of the closing operators \({\bs v}^{(p)}(x),\, x \in[0,\Delta)\).
\begin{property}\label{properties: some props}
	Let \(\{\bs v^{(p)}(x)\}_{p\geq 1}\) be a sequence of closing operators such that they may be decomposed in to \(\bs v^{(p)}(x)=\bs w^{(p)}(x) + \widetilde{\bs w}^{(p)}(x)\), where; \\
	\subproperty \label{properties: -1} For \(x\in[0,\Delta),u,v\geq 0\),  
        \begin{align*}
        		\bs \alpha^{(p)} e^{\bs S^{(p)}(u+v)}(-\bs S^{(p)})^{-1} \widetilde{\bs w}^{(p)}(x) &\leq \bs \alpha^{(p)} e^{\bs S^{(p)}u}(-\bs S^{(p)})^{-1} \widetilde{\bs w}^{(p)}(x).
		\end{align*}
	\subproperty \label{properties: 0} For \(x\in[0,\Delta),u\geq 0\),
		\begin{align*}
			\bs \alpha^{(p)} e^{\bs S^{(p)}u}(-\bs S^{(p)})^{-1} \widetilde{\bs w}^{(p)}(x) &=\widetilde G_{\bs v}^{(p)} \to 0,\, \mbox{ as }p \to \infty.  
		\end{align*}
	\subproperty \label{properties: 1} For \(x\in[0,\Delta),u\geq 0\),  
        \begin{align*}
        		\bs \alpha^{(p)} e^{\bs S^{(p)}u}(-\bs S^{(p)})^{-1} \bs w^{(p)}(x) &\leq \bs \alpha^{(p)} e^{\bs S^{(p)}u} \bs e G_{\bs v},
	\end{align*}
	for some \(0\leq G_{\bs v}<\infty\) independent of \(p\) for \(p>p_0\) and some \(p_0<\infty\). \\
	\subproperty \label{properties: -2} For \(\bs a \in\mathcal A,\,u\geq 0\),  
	\begin{align*}
			\int_{x\in[0,\Delta)}\bs a^{(p)} e^{\bs S^{(p)}u} {\bs v}^{(p)}(x) \wrt x&\leq \bs a^{(p)} e^{\bs S^{(p)}u} \bs e.
	\end{align*}
	\subproperty \label{properties: 2} Let \(g\) be a function satisfying the Assumptions~\ref{asu: g}. For \(u\leq \Delta-\varepsilon^{(p)}\), \(v\in[0,\Delta)\), then
	\[\left|\int_{x=0}^\infty \cfrac{\bs \alpha^{(p)} e^{\bs{S}^{(p)}(u+x)} }{\bs \alpha^{(p)} e^{\bs{S}^{(p)}u} \bs e} {\bs v}^{(p)}(v)g(x)\wrt x -g(\Delta-u-v) 1(u+v\leq\Delta-\varepsilon^{(p)})\right| =  |r_{\bs v}^{(p)}(u,v)|,\]
	where 
	\[ \int_{u=0}^{\Delta}\left| r_{\bs v}^{(p)}(u,v)\right| \wrt u  \leq R_{{\bs v},1}^{(p)} \to 0\]
	and 
	\[ \int_{v=0}^{\Delta}\left| r_{\bs v}^{(p)}(u,v)\right| \wrt v  \leq R_{{\bs v},2}^{(p)} \to 0\]
	as \(\var(Z^{(p)})\to 0\). 
\end{property}

Though it is a slight abuse of notation, for convenience, let us write 
\begin{align*}
	& \mathbb P({\bs Y}^{(p)}(t) \in (\ell,\wrt x, j)\mid \bs Y^{(p)}(0)=\bs y_0^{(p)} )
\end{align*}
in place of 
\begin{align}
	\int_{\bs a \in\mathcal A}\mathbb P({\bs Y}^{(p)}(t) \in (\ell,\wrt \bs a, j)\mid \bs Y^{(p)}(0)=\bs y_0^{(p)})\bs a\bs v(x)\wrt x \label{eqn:GSW}
\end{align}
where \(\bs v(x)\) is a closing operator. The expression (\ref{eqn:GSW}) is an approximation to 
\begin{align}
	\mathbb P(\bs X(t)\in(\wrt x, j)\mid \bs X(0)=(x_0, i)),
\end{align}
\(x\in \calD_{\ell,j}\), \(x_0\in\calD_{\ell_0,i}\).
Further, let us write 
\begin{align*}
	& \mathbb P({\bs Y}^{(p)}(t) \in (\ell,E, j)\mid \bs Y^{(p)}(0)=\bs y_0^{(p)})
\end{align*}
in place of 
\begin{align}
	\int_{x\in E}\int_{\bs a \in\mathcal A}\mathbb P({\bs Y}^{(p)}(t) \in (\ell,\wrt \bs a, j)\mid \bs Y^{(p)}(0)=\bs y_0^{(p)} )\bs a\bs v(x)\wrt x \label{eqn:GSW2}
\end{align}
for some measurable set \(E\subseteq \mathcal D_{\ell,j}\). 


In Appendix~\ref{appendix: sec: 2} we provide results which show that the closing operators (\ref{eqn: alal}) - (\ref{eqn:5462}) satisfy Properties~\ref{properties: some props}. 

Ultimately, in Chapter~\ref{ch: global results}, we will apply the Extended Continuity Theorem for Laplace transforms \cite[Chapter XIII, Theorem 2a]{feller1957} to claim convergence. The Extended Continuity Theorem for Laplace transforms requires us to show convergence of the Laplace transform pointwise with respect to the transform parameter, \(\lambda\). Therefore, we can fix \(\lambda\in\mathbb R,\, \lambda>0\) in the following sections. 



%\section{What we are approximating}
%We now make clear exactly what we are approximating. First, partition the state space of \(\left\{X(t)\right\}\), \((-\infty,\infty)\), into intervals \(\mathcal D_k\) of width \(\Delta\). Let \(y_k = k\Delta\), \(k\in\mathbb Z\), and define intervals \(\mathcal D_k = \left[y_k,y_{k+1}\right]\). The process \(\left\{\left(\displaystyle \sum_{k\in\mathbb Z}k 1(X(t)\in\mathcal D_k),\varphi(t)\right)\right\}_{t\geq0}\) is approximated by \(\left\{(L(t),\varphi(t))\right\}_{t\geq0}\). From the discretised fluid process we can obtain quantities of interest of the original SFM, such as
%\begin{align*}
%	&\mathbb P(L(t) = \ell, \varphi(t) = j) 
%	= \mathbb P(X(t)\in \mathcal D_\ell, \varphi(t) = j).
%\end{align*} 
%
%We approximate \(\left\{\left(\displaystyle \sum_{k\in\mathbb Z}k 1(X(t)\in\mathcal D_k),\varphi(t)\right)\right\}_{t\geq0}\) by the joint process \(\left\{(L(t),{\varphi}(t))\right\}_{t\geq0}\). Note that the process \(\left\{(L(t),{\varphi}(t))\right\}_{t\geq0}\) in not Markovian (in fact, neither is the process \(\left\{\left(\displaystyle \sum_{k\in\mathbb Z}k 1(X(t)\in\mathcal D_k),\varphi(t)\right)\right\}_{t\geq0}\)). However, the process \(\left\{(L(t),\bs A(t),{\varphi}(t))\right\}_{t\geq0}\) is Markovian. 

%Recall that our goal is to approximate quantities relating to the SFM \(\left\{(X(t),\varphi(t))\right\}\). By utilising the orbit, \(\bs A(t)\), as well as \(\left\{(L(t),{\varphi}(t))\right\}_{t\geq0}\), an approximation to the level and phase of the fluid can be obtained. 


\section{The distribution of the QBD-RAP}\label{sec: qbd dists}
In this chapter we are interested in the QBD-RAP up to the first change of level. Let \(\tau_1^{(p)}\) be the random (stopping) time at which the QBD-RAP changes level, or hits the boundary, or exits a boundary, for the first time.
\[\tau_1^{(p)} = \inf\left\{t>0\mid L^{(p)}(t)\neq L^{(p)}(0),\mbox{ or }(L(t),\bs A(t),\varphi(t))\mbox{ hits a boundary}\right\}.\]

Consider the initial condition \(L^{(p)}(0)=\ell_0\in\mathcal K\setminus \{-1,K+1\}\), \(\bs A^{(p)}(0)=\bs a^{(p)}_{\ell_0,i}(x_0),\,\varphi(0)=i\), which is the approximation to the initial condition \(\bs X(0)=(x_0,i)\). We are interested in the quantity, \(f^{\ell_0,(p)}(t)(x,j;x_0,i)\wrt x\) given by
\begin{align}
	&\int_{\bs a \in \mathcal A^{(p)}}\mathbb P\Big(\bs A^{(p)}(t)\in \wrt \bs a, t<\tau_1^{(p)}, \varphi(t) = j \nonumber
	\mid \bs A^{(p)}(0) = \bs   a_{\ell_0,i}^{(p)}(x_0), \varphi(0) = i\Big)
	\bs a{\bs v}_{\ell_0,j}^{(p)}(x)\wrt x \nonumber 
	\\&= (\bs e_i\otimes \bs  a_{\ell_0,i}^{(p)}(x_0)) \exp\left(\bs{B}^{(p)}t\right)(\bs e_j \otimes {\bs v}_{\ell_0,j}^{(p)}(y_{\ell_0+1}-x))\wrt x, \label{eqn: ldll}
\end{align}
\(j\in\calS_+\cup\calS_{+0}\), which is QBD-RAP approximation to the distribution 
\[\mathbb P(\bs X(t)\in (\wrt x,j), t<\tau_1^X \mid \bs X(0) = (x_0, i)).\]

Now, we introduce a partition into the number of up-down and down-up transitions of the sample paths. Denote by \(\Sigma_m,\, m\geq 1\) the sequence of (stopping) times at which \(\{\varphi(t)\}\) jumps from \(\mathcal S_+\cup\calS_{+0}\) to \(\mathcal S_- \) for the \(m\)th time. Denote by \(\Gamma_m, m\geq 1\) the sequence of (stopping) times at which \(\{\varphi(t)\}\) jumps from \(\mathcal S_-\cup\calS_{-0}\) to \(\mathcal S_+\) for the \(m\)th time. More precisely, for sample paths with \(\varphi(0)\in\mathcal S_+\), let \(\Gamma_0=0\), then for \(m\geq 1\), 
\begin{align}
	\Sigma_m &:=\inf\{t > \Gamma_{m-1} \mid \varphi(t)\in\mathcal S_-\}, 
	\\ \Gamma_m &:=\inf\{t > \Sigma_{m} \mid \varphi(t)\in\mathcal S_+\}.
\end{align}
Similarly, for sample paths with \(\varphi(0)\in\mathcal S_-\), let \(\Sigma_0=0\), then for \(m\geq 1\), 
\begin{align}
	\Gamma_m &:=\inf\{t > \Sigma_{m-1} \mid \varphi(t)\in\mathcal S_+\}.
	\\\Sigma_m &:=\inf\{t > \Gamma_{m} \mid \varphi(t)\in\mathcal S_-\}.
\end{align}

With these stopping times, partition the sample paths of the QBD-RAP by the number of up-down and down-up transition which occur as follows. For \(x_0\in\calD_{\ell_0,i},\) \(x\in\calD_{\ell_0,j},\) \(t\geq0,\) \(\ell_0\in\{0,\dots,K\},\) \(m\geq 0\), and for \(i\in\calS_+,j\in\mathcal S_+\cup\mathcal S_{+0}\) let \(f^{\ell_0,(p)}_{m,+,+}(t)( x,j; x_0,i) ,\) be 
\begin{align}
	&\int_{\bs a \in\mathcal A^{(p)}}\mathbb P\Big(\bs A^{(p)}(t)\in \wrt \bs a, \varphi(t) = j, t<\tau_1^{(p)}, \Gamma_m\leq t<\Sigma_{m+1} \nonumber
	\mid \bs A^{(p)}(0) = \bs   a_{\ell_0,i}^{(p)}(x_0), \varphi(0) = i\Big)
	\\&\quad \times \bs a{\bs v}_{\ell_0,j}^{(p)}(x) \nonumber
	%
	\\
	&=\int_{\sigma_1=0}^t (\bs e_i\otimes \bs  a_{\ell_0,i}^{(p)}(x_0)) e^{\bs{B}^{(p)}_{++}\sigma_1}\bs{B}^{(p)}_{+-}	\nonumber
	\int_{\gamma_1=\sigma_1}^{t} e^{\bs{B}^{(p)}_{--}(\gamma_1-\sigma_1)}\bs{B}^{(p)}_{-+}
	\hdots 
	 \int_{\gamma_m=\sigma_{m}}^{t} e^{\bs{B}^{(p)}_{--}(\gamma_m-\sigma_{m})}\bs{B}^{(p)}_{-+}\\&\quad\times
	e^{\bs{B}^{(p)}_{++}(t-\gamma_m)}\left(\bs e_j  \otimes {\bs v}^{(p)}(y_{\ell_0+1}-x)\right) 
	\wrt \gamma_m\wrt \sigma_m\dots \wrt \gamma_1\wrt \sigma_1 . \label{eqn: approx end conv}
\end{align}
Analogously, for \(i\in\mathcal S_+ ,\,j\in\calS_{-}\cup\mathcal S_{-0}\), let 
\begin{align}
	&f^{\ell_0,(p)}_{m,+,-}(t)(  x, j; x_0,i) \nonumber
	\\&= \int_{\bs a \in\mathcal A^{(p)}}\mathbb P\Big(\bs A^{(p)}(t)\in \wrt \bs a, \varphi(t) = j, t<\tau_1^{(p)}, \Sigma_{m+1}\leq t<\Gamma_{m+1}\mid 
	%
	\bs A^{(p)}(0) = \bs   a_{\ell_0,i}^{(p)}(x_0), \nonumber
	\\&\qquad \varphi(0) = i\Big)
	\bs a{\bs v}_{\ell_0,j}^{(p)}( x)  \nonumber 
	%
	\\&:=\int_{\sigma_1=0}^t (\bs e_i\otimes \bs  a_{\ell_0,i}^{(p)}(x_0)) e^{\bs{B}^{(p)}_{++}\sigma_1}\bs{B}^{(p)}_{+-}	\nonumber
	\int_{\gamma_1=\sigma_1}^{t} e^{\bs{B}^{(p)}_{--}(\gamma_1-\sigma_1)}\bs{B}^{(p)}_{-+}
	\hdots 
	%\int_{\gamma_m=\sigma_{m}}^{t} e^{\bs{B}^{(p)}_{--}(\gamma_m-\sigma_{m})}\bs{B}^{(p)}_{-+}%
	\int_{\sigma_{m+1}=\gamma_m}^t e^{\bs{B}^{(p)}_{++}(\sigma_{m+1}-\gamma_m)}
	\\&\quad\times \bs B^{(p)}_{+-}e^{\bs{B}^{(p)}_{--}(t-\sigma_{m+1})}\left(\bs e_j  \otimes {\bs v}_{\ell_0,j}^{(p)}(x)\right) 
	\wrt \sigma_1\wrt \gamma_1\dots \wrt \sigma_m\wrt \gamma_m\wrt \sigma_{m+1}\label{eqn:gljagj}
	 %
	\intertext{for \( i\in\mathcal S_-  ,\, j\in\mathcal S_{+}\cup\calS_{+0} let\)}
	%
	&f^{\ell_0,(p)}_{m,-,+}(t)( x, j; x_0,i) \nonumber
	\\&= \int_{\bs a \in\mathcal A^{(p)}}\mathbb P\Big(\bs A^{(p)}(t)\in \wrt \bs a, \varphi(t) = j, t<\tau_1^{(p)},  \Gamma_{m+1}\leq t<\Sigma_{m+1} \mid 
	%
	\bs A^{(p)}(0) = \bs   a_{\ell_0,i}^{(p)}(x_0), \nonumber 
	\\& \varphi(0) = i\Big)\nonumber
	  \bs a{\bs v}_{\ell_0,j}^{(p)}(x) 
	%
	\\&:=\int_{\gamma_1=0}^t (\bs e_i\otimes \bs  a_{\ell_0,i}^{(p)}(x_0)) e^{\bs{B}^{(p)}_{--}\gamma_1}\bs{B}^{(p)}_{-+}	\nonumber
	\int_{\sigma_1=\gamma_1}^{t} e^{\bs{B}^{(p)}_{++}(\sigma_1-\gamma_1)}\bs{B}^{(p)}_{+-}
	\hdots 
	%\int_{\gamma_m=\sigma_{m}}^{t} e^{\bs{B}^{(p)}_{--}(\gamma_m-\sigma_{m})}\bs{B}^{(p)}_{-+}%
	\int_{\gamma_{m+1}=\sigma_m}^t e^{\bs{B}^{(p)}_{--}(\gamma_{m+1}-\sigma_m)}\bs B^{(p)}_{-+}
	\\&\quad\times e^{\bs{B}^{(p)}_{++}(t-\gamma_{m+1})}\left(\bs e_j  \otimes {\bs v}_{\ell_0,j}^{(p)}(x)\right) 
	\wrt \gamma_1\wrt \sigma_1\dots \wrt \gamma_m\wrt \sigma_m\wrt \gamma_{m+1}
	 %
	\intertext{and for \( i\in\calS_-,j\in\mathcal S_-\cup\calS_{-0}\) let}
	%
	&f^{\ell_0,(p)}_{m,-,-}(t)( x, j; x_0, i) \nonumber
	\\&= \int_{\bs a \in\mathcal A^{(p)}}\mathbb P\Big(\bs A^{(p)}(t)\in \wrt \bs a, \varphi(t) = j,  t<\tau_1^{(p)}, \Sigma_{m}\leq t<\Gamma_{m+1} \mid \bs A^{(p)}(0) = \bs   a_{\ell_0,i}^{(p)}(x_0), \nonumber 
	\\&\quad {}\varphi(0) = i\Big)\bs a{\bs v}_{\ell_0,j}^{(p)}(x) \nonumber 
	%
	\\&:=\int_{\gamma_1=0}^t (\bs e_i\otimes \bs  a_{\ell_0,i}^{(p)}(x_0)) e^{\bs{B}^{(p)}_{--}\gamma_1}\bs{B}^{(p)}_{-+}	\nonumber
	\int_{\sigma_1=\gamma_1}^{t} e^{\bs{B}^{(p)}_{++}(\sigma_1-\gamma_1)}\bs{B}^{(p)}_{+-}
	\hdots 
	 \int_{\sigma_m=\gamma_{m}}^{t} e^{\bs{B}^{(p)}_{++}(\sigma_m-\gamma_{m})}\bs{B}^{(p)}_{+-}\\&\quad\times
	e^{\bs{B}^{(p)}_{--}(t-\sigma_m)}\left(\bs e_j  \otimes {\bs v}_{\ell_0,j}^{(p)}(x)\right) 
	\wrt \sigma_m\wrt \gamma_m\dots \wrt \sigma_1 \wrt \gamma_1.\label{eqn: 67}
\end{align}
Now define 
\begin{align*}
	f^{\ell_0,(p)}_{+,+}(t)(x,j;x_0,i)  &:= \sum_{m=0}^\infty f^{\ell_0,(p)}_{m,+,+}(t)(x,j;x_0,i)  & i\in\calS_+,\,j\in\calS_+\cup\calS_{+0},
	\\ f^{\ell_0,(p)}_{+,-}(t)(x,j;x_0,i)  &:= \sum_{m=1}^\infty f^{\ell_0,(p)}_{m,+,-}(t)(x,j;x_0,i)  & i\in\mathcal S_+,\,\,j\in\mathcal S_{0}\cup\calS_{-0},
	\\ f^{\ell_0,(p)}_{-,+}(t)(x,j;x_0,i) &:= \sum_{m=1}^\infty f^{\ell_0,(p)}_{m,-,+}(t)(x,j;x_0,i)  & i\in\mathcal S_-,\,\,j\in\mathcal S_{+}\cup\calS_{+0},
	\\ f^{\ell_0,(p)}_{-,-}(t)(x,j;x_0,i)  &:= \sum_{m=0}^\infty f^{\ell_0,(p)}_{m,-,-}(t)(x,j;x_0,i)  & i\in\calS_-,\,j\in\mathcal S_-\cup\calS_{-0}.
\end{align*}
Then 
\begin{align}
	f^{\ell_0,(p)}(t)(x,j;x_0,i)  = \begin{cases}
		 f^{\ell_0,(p)}_{+,+}(t)(x,j;x_0,i)  & i\in\calS_+,\,j\in\calS_+\cup\calS_{+0},
	\\    f^{\ell_0,(p)}_{+,-}(t)(x,j;x_0,i)  & i\in\mathcal S_+,\,\,j\in\mathcal S_{0}\cup\calS_{-0},
	\\  f^{\ell_0,(p)}_{-,+}(t)(x,j;x_0,i)  & i\in\mathcal S_-,\,\,j\in\mathcal S_{+}\cup\calS_{+0},
	\\     f^{\ell_0,(p)}_{-,-}(t)(x,j;x_0,i)  & i\in\calS_-,\,j\in\mathcal S_-\cup\calS_{-0}.
	\end{cases}\label{eqn:ghghghghggg}
\end{align}

For this analysis we suppose the QBD-RAP approximation uses ephemeral states \(\calS_0^*\) to model the fluid queue whenever the phase starts in \(k\in\calS_0\). In general, for \(k\in\calS_0^*\), \(p\in \{+,-\}, \, q\in\{+,-\}\), \(m\geq 0\),
\begin{align}
	f_{m,0,q}^{\ell_0,(p)}(t)(x,j;x_0,k)  
	&:= \sum_{r\in\{+,-\}}\sum_{i\in\calS_r}\int_{t_0=0}^t \bs e_ke^{\bs T_{00}t_0} \bs T_{0i} f_{m+1(r\neq q),r,q}^{\ell_0,(p)}(t-t_0)(x,j;x_0,i)\wrt t_0 . \label{eqn: vma0}
\end{align}
Upon taking the Laplace transform, the convolution in (\ref{eqn: vma}) becomes a product, so the Laplace transform of \(f_{m,0,q}^{\ell_0}(t)(x,j;x_0,k)\) is a linear combination of the Laplace transforms of (\ref{eqn: approx end conv})-(\ref{eqn: 67}). Thus, once we show convergence for the Laplace transforms of (\ref{eqn: approx end conv})-(\ref{eqn: 67}) we get convergence of the Laplace transform for starting in \(\calS_0\) too. 

% \[\int_{\bs a \in\mathcal A^{(p)}}\mathbb P\Big(\bs A^{(p)}(t)\in \wrt \bs a, t<\tau_1^{(p)}, \varphi(t) = j, \Gamma_m\leq t<\Sigma_{m+1} \nonumber
% 	\mid \bs A^{(p)}(0) = \bs   a_{\ell_0,i}^{(p)}(x_0), \varphi(0) = i\Big)
% 	\\&\quad \times \bs a{\bs v}^{(p)}(y_{\ell_0+1}-x) .\]


% According to the approximation described in Chapter~\ref{sec: inspiration}, the summands in (\ref{eqn: sojourn partition}) are approximated by \[f^{\ell_0,(p)}_{m,+,+}(t)( x,j; x_0,i) ,\] 
% for \(i\in\calS_+,\) \(j\in\calS_+\cup\calS_{+0},\) \(x_0\in\calD_{\ell_0,i},\) \(x\in\calD_{\ell_0,j},\) \(t\geq0,\) \(\ell_0\in\{0,\dots,K\},\) \(m\geq 0\), which are given by
% \begin{align}
% 	&\int_{\bs a \in\mathcal A^{(p)}}\mathbb P\Big(\bs A^{(p)}(t)\in \wrt \bs a, t<\tau_1^{(p)}, \varphi(t) = j, \Gamma_m\leq t<\Sigma_{m+1} \nonumber
% 	\mid \bs A^{(p)}(0) = \bs   a_{\ell_0,i}^{(p)}(x_0), \varphi(0) = i\Big)
% 	\\&\quad \times \bs a{\bs v}^{(p)}(y_{\ell_0+1}-x) \nonumber
% 	%
% 	\\
% 	&=\int_{\sigma_1=0}^t (\bs e_i\otimes \bs  a_{\ell_0,i}^{(p)}(x_0)) e^{\bs{B}^{(p)}_{++}\sigma_1}\bs{B}^{(p)}_{+-}	\nonumber
% 	\int_{\gamma_1=\sigma_1}^{t} e^{\bs{B}^{(p)}_{--}(\gamma_1-\sigma_1)}\bs{B}^{(p)}_{-+}
% 	\hdots 
% 	 \int_{\gamma_m=\sigma_{m}}^{t} e^{\bs{B}^{(p)}_{--}(\gamma_m-\sigma_{m})}\bs{B}^{(p)}_{-+}\\&\quad\times
% 	e^{\bs{B}^{(p)}_{++}(t-\gamma_m)}\left(\bs e_j  \otimes {\bs v}^{(p)}(y_{\ell_0+1}-x)\right) 
% 	\wrt \sigma_1\wrt \gamma_1\dots \wrt \sigma_m\wrt \gamma_m. \label{eqn: approx end conv}
% \end{align}
% Analogously, approximations to (\ref{eqn: fkl})-(\ref{eqn: many eqns mu}) are 
% \begin{align}
% 	&f^{\ell_0,(p)}_{m,+,-}(t)(  x, j; x_0,i) \nonumber
% 	% \\&= \int_{\bs a \in\mathcal A^{(p)}}\mathbb P\Big(\bs A^{(p)}(t)\in \wrt \bs a, t<\tau_1^{(p)}, \varphi(t) = j, \Sigma_{m+1}\leq t<\Gamma_{m+1}\mid 
% 	% %
% 	% \bs A^{(p)}(0) = \bs   a_{\ell_0,i}^{(p)}(x_0), \nonumber
% 	% \\&\qquad \times \varphi(0) = i\Big)
% 	% \bs a{\bs v}^{(p)}(y_{\ell_0+1}-x)  \nonumber 
% 	%
% 	\\&:=\int_{\sigma_1=0}^t (\bs e_i\otimes \bs  a_{\ell_0,i}^{(p)}(x_0)) e^{\bs{B}^{(p)}_{++}\sigma_1}\bs{B}^{(p)}_{+-}	\nonumber
% 	\int_{\gamma_1=\sigma_1}^{t} e^{\bs{B}^{(p)}_{--}(\gamma_1-\sigma_1)}\bs{B}^{(p)}_{-+}
% 	\hdots 
% 	%\int_{\gamma_m=\sigma_{m}}^{t} e^{\bs{B}^{(p)}_{--}(\gamma_m-\sigma_{m})}\bs{B}^{(p)}_{-+}%
% 	\int_{\sigma_{m+1}=\gamma_m}^t e^{\bs{B}^{(p)}_{++}(\sigma_{m+1}-\gamma_m)}\bs B^{(p)}_{+-}
% 	\\&\quad\times e^{\bs{B}^{(p)}_{--}(t-\sigma_{m+1})}\left(\bs e_j  \otimes {\bs v}^{(p)}(x-y_{\ell_0})\right) 
% 	\wrt \sigma_1\wrt \gamma_1\dots \wrt \sigma_m\wrt \gamma_m\wrt \sigma_{m+1}\label{eqn:gljagj}
% 	 %
% 	\intertext{for \(i\in\mathcal S_+ ,\,j\in\calS_{-}\cup\mathcal S_{-0}\);}
% 	%
% 	&f^{\ell_0,(p)}_{m,-,+}(t)( x, j; x_0,i) \nonumber
% 	% \\&= \int_{\bs a \in\mathcal A^{(p)}}\mathbb P\Big(\bs A^{(p)}(t)\in \wrt \bs a, t<\tau_1^{(p)}, \varphi(t) = j,  \Gamma_{m+1}\leq t<\Sigma_{m+1},\varphi(\Gamma_{m+1})=k_{m+1}, \nonumber
% 	% \nonumber
% 	%  \\&\qquad\varphi(\Sigma_\ell) = j_\ell , \varphi(\Gamma_\ell) = k_\ell, \ell = 1,\dots,m \mid 
% 	% %
% 	% \bs A^{(p)}(0) = \bs   a_{\ell_0,i}^{(p)}(x_0), \varphi(0) = i\Big)\nonumber
% 	%  \\&\qquad\times \bs a{\bs v}^{(p)}(y_{\ell_0+1}-x) 
% 	%
% 	\\&:=\int_{\gamma_1=0}^t (\bs e_i\otimes \bs  a_{\ell_0,i}^{(p)}(x_0)) e^{\bs{B}^{(p)}_{--}\gamma_1}\bs{B}^{(p)}_{-+}	\nonumber
% 	\int_{\sigma_1=\gamma_1}^{t} e^{\bs{B}^{(p)}_{++}(\sigma_1-\gamma_1)}\bs{B}^{(p)}_{+-}
% 	\hdots 
% 	%\int_{\gamma_m=\sigma_{m}}^{t} e^{\bs{B}^{(p)}_{--}(\gamma_m-\sigma_{m})}\bs{B}^{(p)}_{-+}%
% 	\int_{\gamma_{m+1}=\sigma_m}^t e^{\bs{B}^{(p)}_{--}(\gamma_{m+1}-\sigma_m)}\bs B^{(p)}_{-+}
% 	\\&\quad\times e^{\bs{B}^{(p)}_{++}(t-\gamma_{m+1})}\left(\bs e_j  \otimes {\bs v}^{(p)}(y_{\ell_0+1}-x)\right) 
% 	\wrt \gamma_1\wrt \sigma_1\dots \wrt \gamma_m\wrt \sigma_m\wrt \gamma_{m+1}
% 	 %
% 	\intertext{for \( i\in\mathcal S_-  ,\, j\in\mathcal S_{+}\cup\calS_{+0};\)}
% 	%
% 	&f^{\ell_0,(p)}_{m,-,-}(t)( x, j; x_0, i) \nonumber
% 	% \\&= \int_{\bs a \in\mathcal A^{(p)}}\mathbb P\Big(\bs A^{(p)}(t)\in \wrt \bs a, t<\tau_1^{(p)}, \varphi(t) = j, \Sigma_{m}\leq t<\Gamma_{m+1}, \nonumber
% 	% \varphi(\Sigma_\ell) = j_\ell , \varphi(\Gamma_\ell) = k_\ell, 
% 	% \\&\qquad\ell = 1,\dots,m \mid \bs A^{(p)}(0) = \bs   a_{\ell_0,i}^{(p)}(x_0), \varphi(0) = i\Big)\bs a{\bs v}^{(p)}(y_{\ell_0+1}-x) 
% 	%
% 	\\&:=\int_{\gamma_1=0}^t (\bs e_i\otimes \bs  a_{\ell_0,i}^{(p)}(x_0)) e^{\bs{B}^{(p)}_{--}\gamma_1}\bs{B}^{(p)}_{-+}	\nonumber
% 	\int_{\sigma_1=\gamma_1}^{t} e^{\bs{B}^{(p)}_{++}(\sigma_1-\gamma_1)}\bs{B}^{(p)}_{+-}
% 	\hdots 
% 	 \int_{\sigma_m=\gamma_{m}}^{t} e^{\bs{B}^{(p)}_{++}(\sigma_m-\gamma_{m})}\bs{B}^{(p)}_{+-}\\&\quad\times
% 	e^{\bs{B}^{(p)}_{--}(t-\sigma_m)}\left(\bs e_j  \otimes {\bs v}^{(p)}(x-y_{\ell_0})\right) 
% 	\wrt \gamma_1\wrt \sigma_1\dots \wrt \gamma_m\wrt \sigma_m\label{eqn: 67}
% \end{align}
% for \( i\in\calS_-,j\in\mathcal S_-\cup\calS_{-0}\), respectively. %Expressions analgous to that on the right-hand side of (\ref{eqn: approx end conv}) can be written down for (\ref{eqn:gljagj})-(\ref{eqn: 67}). 

%To justify the last exponential in the expressions above. The whole expression is in terms of \emph{time}. Consider \(f^{\ell_0}_{m,+}(t)( x, j_1,k_1,\dots,j_m,k_m, j; x_0,i)\). Assume that \(X(t)\in \mathcal D_{\ell_0}\), if the process were to continue in phase \(j\) for another \((y_{\ell_0}+\Delta-x)/c_j = (y_{\ell_0+1}-x)/c_j\) units of time, at which time the ME life time expired, then, since the expiry of the ME is most likely to occur when \(X=y_{\ell_0}+\Delta\), we would approximate the position of \(X(t)\) as \(y_{\ell_0+1} - c_j(y_{\ell_0+1}-x)/c_j  = x\); the probability density associated with the description above is to take the position of the orbit at time \(t\), \(\bs \alpha(t)\) and post-multiply by \(e^{c_j\bs{S}(y_{\ell_0+1}-x)/c_j}c_j\bs s\), i.e. the probability density associated with \(X(t)\) being at \(x\) is \(\bs \alpha(t)e^{c_j\bs{S}(y_{\ell_0+1}-x)/c_j}c_j\bs s\).

% For sample paths with \(\varphi(0)\in\mathcal S_+\), (\(\varphi(0)\in\mathcal S_-\)) the events \(\{\Gamma_m\leq t< \Sigma_{m+1}\} , \mbox{ and } \{\Sigma_{m+1}\leq t< \Gamma_{m+1}\}, \) (respectively, \(\{\Sigma_m\leq t< \Gamma_{m+1}\} , \mbox{ and } \{\Gamma_{m+1}\leq t< \Sigma_{m+1}\}\)) \(m\geq 0\), form a partition of the sample paths of the QBD-RAP. With this partition the phase of the QBD-RAP changes from \(\mathcal S_+\) to \(j_m\in\mathcal S_-\) at times \(\Sigma_{m}\) and from \(\mathcal S_-\) to \(k_m\in\mathcal S_+\) at times \(\Gamma_m\). %Hence for \(i\notin\calS_{+0}\cup\calS_{-0}\), the probability (\ref{eqn: ldll}) is approximated by 
% %\begin{align}\label{eqn: approx to sojourn}
% %	f^{\ell_0}(t)(x,j;x_0,i)  &:=\begin{cases}
% %		 f^{\ell_0}_{+,+}(t)(x,j;x_0,i)  & i\in\calS_+,\,j\in\mathcal S_+\cup\calS_{+0},
% %		\\ f^{\ell_0}_{+,-}(t)(x,j;x_0,i)  & i\in\mathcal S_+,\,j\in\mathcal S_{-}\cup\calS_{-0},
% %		\\ f^{\ell_0}_{-,+}(t)(x,j;x_0,i)  & i\in\mathcal S_-,\,j\in\mathcal S_{+}\cup\calS_{+0},
% %		\\ f^{\ell_0}_{-,-}(t)(x,j;x_0,i)  & i\in\calS_-,\,j\in\mathcal S_{-}\cup\calS_{-0},
% %	\end{cases}
% %\end{align}
% %where 

% Define
% \begin{align}
% 	&f_{m,+,+}^{\ell_0,(p)}(t)(x,j;x_0,i)   \nonumber 
% 	\\&\qquad= \sum_{j_1\in\mathcal S_-} \sum_{k_1\in\mathcal S_+}\dots \sum_{j_m\in\mathcal S_-}\sum_{k_m\in\mathcal S_+} f_{m,+,+}^{\ell_0,(p)}(t)(x, j_1,k_1,\dots,j_m,k_m, j; x_0,i)  ,\label{eqn: oihg87576}
% 	%
% 	\\&f^{\ell_0,(p)}_{m,+,-}(t)(x, j; x_0,i)   \nonumber
% 	\\&= \sum_{j_1\in\mathcal S_-} \sum_{k_1\in\mathcal S_+}\dots \sum_{k_m\in\mathcal S_+}\sum_{j_{m+1}\in\mathcal S_-}f^{\ell_0,(p)}_{m+1,+,-}(t)(x, j_1,k_1,\dots,k_m,j_{m+1}, j; x_0,i)  ,
% 	\\&f^{\ell_0,(p)}_{m,-,+}(t)(x, j; x_0,i)   \nonumber
% 	\\&= \sum_{k_1\in\mathcal S_+} \sum_{j_1\in\mathcal S_-}\dots \sum_{j_m\in\mathcal S_-}\sum_{k_{m+1}\in\mathcal S_+}f^{\ell_0,(p)}_{m,-,+}(t)(x, k_1,j_1,\dots,j_m,k_{m+1}, j; x_0,i)   ,
% 	\\&f^{\ell_0,(p)}_{m,-,-}(t)(x, j; x_0,i)   \nonumber 
% 	\\&\qquad = \sum_{k_1\in\mathcal S_+} \sum_{j_1\in\mathcal S_-}\dots \sum_{k_m\in\mathcal S_+}\sum_{j_m\in\mathcal S_-}f^{\ell_0,(p)}_{m,-,-}(t)(x, k_1, j_1,\dots,k_m,j_m, j; x_0,i)  .  \label{eqn: many eqns}
% \end{align}
% and

% which are the corresponding approximations of (\ref{eqn: loop mu})-(\ref{eqn: ljg97skg2}).
% The sum (\ref{eqn: oihg87576}) is 
% \begin{align}
% 	&\int_{\sigma_1=0}^t (\bs e_i\otimes \bs  a_{\ell_0,i}^{(p)}(x_0)) e^{\bs{B}^{(p)}_{++}\sigma_1}\bs{B}^{(p)}_{+-}	\nonumber
% 	\int_{\gamma_1=\sigma_1}^{t} e^{\bs{B}^{(p)}_{--}(\gamma_1-\sigma_1)}\bs{B}^{(p)}_{-+}
% 	\hdots 
% 	 \int_{\gamma_m=\sigma_{m}}^{t} e^{\bs{B}^{(p)}_{--}(\gamma_m-\sigma_{m})}\\&\quad\times\bs{B}^{(p)}_{-+}
% 	e^{\bs{B}^{(p)}_{++}(t-\gamma_m)}\left(\bs e_j \otimes {\bs v}^{(p)}(y_{\ell_0+1}-x)\right) 
% 	\wrt \sigma_1\wrt \gamma_1\dots \wrt \sigma_m\wrt \gamma_m. \label{eqn: looper}
% \end{align}
% Analogous expressions can be written down for (\ref{eqn: oihg87576})-(\ref{eqn: many eqns}). For simplicity, we may drop the superscript \((p)\). 
%
%Let \(\psi:[0,\Delta)\to \mathbb R\) be bounded and Lipschitz continuous. To prove weak convergence we consider the expectations 
%\begin{align}
%	&\mathbb E[\psi(\widetilde X(t))1( t<\tau_1, \varphi(t) = j, \Sigma_{m}\leq t<\Gamma_{m+1}, \nonumber
%	\varphi(\Sigma_\ell) = j_\ell , \varphi(\Gamma_\ell) = k_\ell, \ell = 1,\dots,m )\mid 
%	%
%	\\&\qquad \bs A(0) = \bs   a_{\ell_0,i}(x_0), \varphi(0) = i)]. \label{eqn: skdjfq}
%\end{align}
%For example, when \(i\in\calS_+\), \(j\in\calS_+\cup\calS_{0+}\), the expectation (\ref{eqn: skdjfq}) is given by 
%\begin{align}
%	& \int_{x=0}^\Delta \int_{\sigma_1=0}^t (\bs e_i\otimes \bs  a_{\ell_0,i}(x_0)) e^{\bs{B}_{++}\sigma_1}\bs{B}_{+-}	\nonumber
%	\int_{\gamma_1=\sigma_1}^{t} e^{\bs{B}_{--}(\gamma_1-\sigma_1)}\bs{B}_{-+}
%	\hdots 
%	 \int_{\gamma_m=\sigma_{m}}^{t} e^{\bs{B}_{--}(\gamma_m-\sigma_{m})}\\&\quad\times\bs{B}_{-+}
%	e^{\bs{B}_{++}(t-\gamma_m)}\left(\bs e_j \otimes {\bs v}(y_{\ell_0+1}-x)\right) 
%	\wrt \sigma_1\wrt \gamma_1\dots \wrt \sigma_m\wrt \gamma_m f(x-y_{\ell_0}) .
%\end{align}

%The goal in this section is to show that (\ref{eqn: sojourn}) can be approximated arbitrarily closely by (\ref{eqn: approx to sojourn}) by choosing \(Z^{(p)}\sim ME(\bs \alpha^{(p)},\bs S^{(p)},\bs s^{(p)})\) with sufficiently small variance in the construction of the approximation scheme. 
%Specifically we show weak convergence of the measures defined by the density function (\ref{eqn: approx to sojourn}) to the measure (\ref{eqn: sojourn}). We do this by showing that for an arbitrary bounded and Lipschitz continuous function \(f:[0,\Delta)\to\mathbb R\), the expected value of \(\psi(X^{(p)}(t))\) converges to the expected value of \(\psi(X(t)\). To do so, we show that the Laplace transforms with respect to time of expectations with respect to (\ref{eqn: approx end conv})-(\ref{eqn: 67}) converge to the Laplace transforms with respect to time of expectations with respect to (\ref{eqn: density part +})-(\ref{eqn: 63}), respectively. This implies that the Laplace transforms with respect to time of expectations with respect to (\ref{eqn: oihg87576})-(\ref{eqn: many eqns}) converge to the Laplace transforms of expectations with respect to (\ref{eqn: loop mu})-(\ref{eqn: many eqns mu}), since the sums are finite and the Laplace transform is a linear operator. Using the Dominated Convergence Theorem, we then show that the Laplace transform with respect to time of expectations with respect to (\ref{eqn: approx to sojourn}) converge to the Laplace transform with respect to time of expectations with respect to (\ref{eqn: sojourn}).

\section{The distribution of the fluid queue}\label{sec: no change}
Now we have established the expressions of the QBD-RAP with which we will work, we now define the corresponding expressions for the fluid queue. Let \(\tau_1^X\) be the minimum of the time at which \(\{X(t)\}\) hits a boundary or exits \(\mathcal D_{\ell_0}\), where \(X(0)=x_0\in\mathcal D_{\ell_0}\), or \(\{X(t)\}\) exits a boundary. More precisely, 
\[\tau_1^X = \min\left\{\begin{array}{c}\inf\left\{t>0\mid X(t)=y_{\ell}, \ell\in\mathcal K\right\}, \\ \inf\left\{t>0 \mid X(t) \neq 0, X(0)=0\right\}, \\ \inf\left\{t>0 \mid X(t) \neq y_{K+1}, X(0)=y_{K+1}\right\} \end{array} \right\}.\]
Consider the measures 
\begin{align}\label{eqn: sojourn}
	\mu^{\ell_0}(t)(\cdot,j; x_0,i) := \mathbb P(\bs X(t)\in (\cdot ,j), t<\tau_1^X \mid \bs X(0) = (x_0, i)),
\end{align}
\(\ell_0\in\{0,\dots,K\}\), \(x_0 \in\mathcal D_{\ell_0,i}, i,j\in\mathcal S, t \geq 0. \)
In words, this is the distribution of the fluid queue at time \(t\) on the event that the fluid level remains within \(\mathcal D_{\ell_0}\) up to and including time \(t\) and is in phase \(j\) at time \(t\), given that is started at \(X(0)=x_0\in\mathcal D_{\ell_0,i}\) in phase \(i\). %The QBD-RAP approximation to (\ref{eqn: sojourn}) is 
% \begin{align}
% 	(\bs e_i\otimes \bs  a_{\ell_0,i}^{(p)}(x_0)) \exp\left(\bs{B}^{(p)}t\right)(\bs e_j \otimes {\bs v}^{(p)}(y_{\ell_0+1}-x))\wrt x,\,j\in\calS_+\cup\calS_{+0}. \label{eqn: ldll}
% \end{align}
% where 
% \[\bs{B}^{(p)} = \left[\begin{array}{cccc}
% 	\bs{C}_+\otimes \bs{S}^{(p)} + \bs{T}_{++}\otimes \bs{I} & \bs{T}_{+-}\otimes \bs{D}^{(p)} & \bs{T}_{+0}\otimes \bs{I} & \bs 0 \\
% 	\bs{T}_{-+}\otimes \bs{D}^{(p)} & \bs{C}_-\otimes \bs{S}^{(p)} + \bs{T}_{--}\otimes \bs{I} &\bs 0 & \bs{T}_{-0}\otimes \bs{I} \\
% 	\bs{T}_{0+}\otimes \bs{I} & \bs{T}_{0-}\otimes \bs{D}^{(p)} & \bs{T}_{00}\otimes \bs{I} & \bs 0 \\
% 	\bs{T}_{0+}\otimes \bs{D}^{(p)} & \bs{T}_{0-}\otimes \bs{I} &\bs 0 & \bs{T}_{00}\otimes \bs{I} 
% 	\end{array}\right].
% \] 
%\begin{align}
%	(\bs e_i\otimes \bs  a_{\ell_0,i}(x_0)) \exp\left(\bs{B}t\right)(\bs e_j \otimes e^{|c_j|\bs{S}(y_{{\ell_0}+1}-x)/|c_j|}|c_j|\bs s)\wrt x/|c_j|,
%\end{align}
%where 
%\[\bs{B} = \left[\begin{array}{cc}
%	\bs{C}_+\otimes \bs{S} + \bs{T}_{++}\otimes \bs{I} & \bs{T}_{+-}\otimes \bs{D} \\
%	\bs{T}_{-+}\otimes \bs{D} & \bs{C}_-\otimes \bs{S} + \bs{T}_{--}\otimes \bs{I} 
%	\end{array}\right],\] 
%and \(\bs   a_{\ell_0,i}(x_0) = \bs \alpha e^{\bs S(x_0-y_{\ell_0})}/\bs \alpha e^{\bs S(x_0-y_{\ell_0})}\bs e\). 
% For convenience, and without loss of generality, we reorder the rows and columns of \(\bs B^{(p)}\) and partition \(\bs{B}^{(p)}\) into 
% \begin{align}
% 	\bs B^{(p)} = \left[\begin{array}{ccc} \bs B_{++}^{(p)} & \bs B_{+-}^{(p)} \\ \bs B_{-+}^{(p)} & \bs B_{--}^{(p)}\end{array}\right]
% \end{align}
% where 
% \begin{align*}
% \bs{B}_{++}^{(p)} &= \left[\begin{array}{cc} \bs{C}_+\otimes \bs{S}^{(p)} + \bs{T}_{++}\otimes \bs{I} & \bs T_{+0}\otimes \bs I \\ \bs T_{0+}\otimes \bs I & \bs T_{00}\otimes \bs I\end{array}\right] ,
% %
% \bs{B}_{+-}^{(p)} = \left[\begin{array}{cc} \bs{T}_{+-}\otimes \bs{D}^{(p)} & \bs 0 \\ \bs T_{0-}\otimes \bs D^{(p)} & \bs 0 \end{array}\right],
% %
% \\ \bs{B}_{-+}^{(p)} &=  \left[\begin{array}{cc} \bs{T}_{-+}\otimes \bs{D}^{(p)} & \bs 0 \\ \bs T_{0+}\otimes \bs D^{(p)} & \bs 0 \end{array}\right] ,
% % 
% \bs{B}_{--}^{(p)} = \left[\begin{array}{cc} \bs{C}_-\otimes \bs{S}^{(p)} - \bs{T}_{--}\otimes \bs{I} & \bs T_{-0}\otimes \bs I \\ \bs T_{0-}\otimes \bs I & \bs T_{00}\otimes \bs I\end{array}\right].
% \end{align*}
% For \(m\in\{+,-,0\},\,n\in\{+,-\},\,m\neq n\), \(j\in\mathcal S_n\), denote \(\bs T_{mj} = \bs T_{mn} (\bs e_j\bs e_j')\) and for \(m,n\in\{+,-\},\,m\neq n\), \(j\in\calS_n\), denote
% \begin{align}
% \bs{B}_{mj}^{(p)} &= \bs B_{mn}^{(p)} \bs e_j\bs e_j'= \left[\begin{array}{cc} \bs{T}_{mj}\otimes \bs{D}^{(p)} & \bs 0 \\ \bs T_{0j}\otimes \bs D^{(p)} & \bs 0 \end{array}\right].
% \end{align}

I do not know of any simple expression for (\ref{eqn: sojourn}). There are expressions for the Laplace transform of (\ref{eqn: sojourn}) with respect to time. One is in terms of the of first return matrices \(\Psi(\lambda)\) and \(\Xi(\lambda)\) \citep{bean2009}. Here we opt for another expression for the Laplace transform which is obtained by partitioning as follows.

%Suppose \(i\in\calS_+,j\in\mathcal S_+\cup\mathcal S_{+0}\). The arguments for all other combinations of \(i,j\in\mathcal S\) are analogous and notable differences will be pointed out where necessary. For this analysis we suppose the QBD-RAP approximation uses ephemeral states \(\calS_0^*\) to model the fluid queue whenever the phase starts in \(k\in\calS_0\). As a result, certain expressions for the QBD-RAP which start in phase \(k\in\calS_0^*\) can be written as linear combinations of expressions for the QBD-RAP which start in phases \(i\in\calS_+\cup\calS_-\). Thus, once we show convergence for starting phases \(i\in\calS_+\cup\calS_-\) we get convergence for starting in \(\calS_0\) too. 

As before, we use the sequence of times of the up-down transitions, \(\Sigma_m,\, m\geq 1\), and the sequence of times of the down-up transitions, \(\Gamma_m, m\geq 1\). For a digramatic definition of these stopping times for the fluid queue, see Figure~\ref{fig: sample paths}. For times \(t\) such that \(\Gamma_m\leq t<\Sigma_{m+1}\), then \(\varphi(t)\in\mathcal S_+\cup\calS_{+0}\). For times \(t\) such that \(\Sigma_{m+1}\leq t< \Gamma_{m+1}\), then \(\varphi(t)\in\mathcal S_-\cup\calS_{-0}\). The events \(\{\Gamma_m\leq t< \Sigma_{m+1}\} , \mbox{ and } \{\Sigma_{m+1}\leq t< \Gamma_{m+1}\}, \) \(m\geq 0\), partition the sample paths of (\ref{eqn: sojourn}) into periods where the fluid is either non-decreasing or non-increasing, respectively; see Figure~\ref{fig: sample paths}.
\begin{figure}
    \centering\begin{tikzpicture}
    	\draw[->,thick] (0,0) -- (14,0);
    %	\draw[-,dashed] (0,4) -- (14.5,4);
    	\draw (14,-0.75) node {$x$};
    	\draw[-,thick] (0,0) -- (0,4.5);
    	\draw[-,thick] (-0.1,4) -- (0.1,4);
    	\draw (-0.75,4) node {$y_{\ell_0}+\Delta$};
    	\draw (-0.75,0) node {$y_{\ell_0}$};
	
    	\draw (0,-0.75) node {$\Gamma_0$};
    %	\foreach \i in {1,2} {
    %        		\draw[-,thick] (\i*4,-0.1) -- (\i*4,0.1);
    %%		\draw (\i*4,-0.75) node {$\i \Delta$};
    %        }
    %	\edef\mya{0}
    %	\foreach \x/\y [count=\c] in {0/2.6,2.6/2,2/2,2/2} {
    %		\draw[-,thick] (\mya,\x) -- (\mya+\y,{(-(-1)^\c)*\y+\x});
    %                	\pgfmathparse{\mya+\y}
    %                	\xdef\mya{\pgfmathresult}
    %	}
            \draw[-,thick] (0,0) -- (2.6,2.6);

			% \draw (3.2,3.0) node {$\varphi(t)\in\calS_{+0}$};
			\draw[-,thick] (2.6,2.6) -- (3.6,2.6); 

            \draw[-,thick] (3.6,-0.1) -- (3.6,0.1);
    %        \draw[-,dashed] (2.6,0) -- (2.6,5);
    	\draw (3.6,-0.75) node {$\Sigma_1$};

    	
%    	\draw[<->,thick] (3.6-1,1) -- (3.6+1,1);
%    	\draw (3.6,1) node[fill=white] {$ R_1$};
    	
%    	\draw[<->,thick] (5.6-1,1) -- (5.6+1,1);
%    	\draw (5.6,1) node[fill=white] {$ R_1$};
    	
%            \draw[-,dashed] (2.6,2.6) -- (2.6+2,2.6+2);
%            \draw[-,thick] (4.6,-0.1) -- (4.6,0.1);
    %        \draw[-,dashed] (4.6,0) -- (4.6,5);
%    	\draw (4.6,-0.75) node {$Y_1$};
    	
%            \draw[-,dashed] (4.6,4.6) -- (4.6+2,4.6-2);
%            \draw[-,thick] (6.6,-0.1) -- (6.6,0.1);
    %        \draw[-,dashed] (6.6,0) -- (6.6,5);
    %	\draw (6.6,-0.75) node {$Y_1+R_1$};
    	
    	\draw[-,thick] (3.6,2.6) -- (5.6,2.6-2);

		\draw[-,thick] (5.6,0.6) -- (7,0.6);

    %        \draw[-,dashed] (6.6+2,0) -- (6.6+2,5);
		\draw[-,thick] (7,-0.1) -- (7,0.1);
    	\draw (7,-0.75) node {$\Gamma_1$};
    	
%    	\draw[<->,thick] (9.1-0.5,1) -- (9.1+0.5,1);
%    	\draw (9.1,1) node[fill=white] {$R_2$};
%    	\draw[<->,thick] (10.1-0.5,1) -- (10.1+0.5,1);
%    	\draw (10.1,1) node[fill=white] {$R_2$};
            
%            	\draw[-,dashed] (6.6+2,0.6) -- (6.6+3,0.6-1);
%            \draw[-,thick] (6.6+3,-0.1) -- (6.6+3,0.1);
%    %        \draw[-,dashed] (6.6+3,0) -- (6.6+3,5);
%            	\draw (9.6,-0.75) node {$Y_2$};
    	
%            	\draw[-,dashed] (6.6+3,-0.4) -- (6.6+4,0.6);
%    	\draw[-,thick] (6.6+4,-0.1) -- (6.6+4,0.1);
    %        \draw[-,dashed] (6.6+4,0) -- (6.6+4,5);
            
		\draw[-,thick] (7,0.6) -- (7+0.8,1.4);

		\draw[-,thick] (7.8,1.4) -- (8.8,1.4);
		\draw[-,thick] (8.8,1.4) -- (8+3,3.6);
%    	\draw[-,thick] (4.6+3,-0.1) -- (4.6+3,0.1);
    %        \draw[-,dashed] (6.6+7,0) -- (6.6+7,5);
%            	\draw (6.6+7,-0.75) node {$Y_3$};
    
            	\draw[-,thick] (7+3+1,3.6) -- (7+4+1,2.6);
            \draw[-,thick] (7+3+1,-0.1) -- (7+3+1,0.1);
	    	\draw (7+3+1,-0.75) node {$\Sigma_2$};
    \end{tikzpicture}
	\begin{tikzpicture}
    	\draw[->,thick] (0,0) -- (14,0);
    %	\draw[-,dashed] (0,4) -- (14.5,4);
    	\draw (14,-0.75) node {$x$};
    	\draw[-,thick] (0,0) -- (0,4.5);
    	\draw[-,thick] (-0.1,4) -- (0.1,4);
    	\draw (-0.75,4) node {$y_{\ell_0}+\Delta$};
    	\draw (-0.75,0) node {$y_{\ell_0}$};
	
    	\draw (0,-0.75) node {$\Sigma_0$};
    %	\foreach \i in {1,2} {
    %        		\draw[-,thick] (\i*4,-0.1) -- (\i*4,0.1);
    %%		\draw (\i*4,-0.75) node {$\i \Delta$};
    %        }
    %	\edef\mya{0}
    %	\foreach \x/\y [count=\c] in {0/2.6,2.6/2,2/2,2/2} {
    %		\draw[-,thick] (\mya,\x) -- (\mya+\y,{(-(-1)^\c)*\y+\x});
    %                	\pgfmathparse{\mya+\y}
    %                	\xdef\mya{\pgfmathresult}
    %	}
            \draw[-,thick] (0,4) -- (2.6,4-2.6);

			% \draw (3.2,3.0) node {$\varphi(t)\in\calS_{+0}$};
			\draw[-,thick] (2.6,4-2.6) -- (3.6,4-2.6); 

            \draw[-,thick] (3.6,-0.1) -- (3.6,0.1);
    %        \draw[-,dashed] (2.6,0) -- (2.6,5);
    	\draw (3.6,-0.75) node {$\Gamma_1$};

    	
%    	\draw[<->,thick] (3.6-1,1) -- (3.6+1,1);
%    	\draw (3.6,1) node[fill=white] {$ R_1$};
    	
%    	\draw[<->,thick] (5.6-1,1) -- (5.6+1,1);
%    	\draw (5.6,1) node[fill=white] {$ R_1$};
    	
%            \draw[-,dashed] (2.6,2.6) -- (2.6+2,2.6+2);
%            \draw[-,thick] (4.6,-0.1) -- (4.6,0.1);
    %        \draw[-,dashed] (4.6,0) -- (4.6,5);
%    	\draw (4.6,-0.75) node {$Y_1$};
    	
%            \draw[-,dashed] (4.6,4.6) -- (4.6+2,4.6-2);
%            \draw[-,thick] (6.6,-0.1) -- (6.6,0.1);
    %        \draw[-,dashed] (6.6,0) -- (6.6,5);
    %	\draw (6.6,-0.75) node {$Y_1+R_1$};
    	
    	\draw[-,thick] (3.6,4-2.6) -- (5.6,4-2.6+2);

		\draw[-,thick] (5.6,4-0.6) -- (7,4-0.6);

    %        \draw[-,dashed] (6.6+2,0) -- (6.6+2,5);
		\draw[-,thick] (7,-0.1) -- (7,0.1);
    	\draw (7,-0.75) node {$\Sigma_1$};
    	
%    	\draw[<->,thick] (9.1-0.5,1) -- (9.1+0.5,1);
%    	\draw (9.1,1) node[fill=white] {$R_2$};
%    	\draw[<->,thick] (10.1-0.5,1) -- (10.1+0.5,1);
%    	\draw (10.1,1) node[fill=white] {$R_2$};
            
%            	\draw[-,dashed] (6.6+2,0.6) -- (6.6+3,0.6-1);
%            \draw[-,thick] (6.6+3,-0.1) -- (6.6+3,0.1);
%    %        \draw[-,dashed] (6.6+3,0) -- (6.6+3,5);
%            	\draw (9.6,-0.75) node {$Y_2$};
    	
%            	\draw[-,dashed] (6.6+3,-0.4) -- (6.6+4,0.6);
%    	\draw[-,thick] (6.6+4,-0.1) -- (6.6+4,0.1);
    %        \draw[-,dashed] (6.6+4,0) -- (6.6+4,5);
            
		\draw[-,thick] (7,4-0.6) -- (7+0.8,4-1.4);

		\draw[-,thick] (7.8,4-1.4) -- (8.8,4-1.4);
		\draw[-,thick] (8.8,4-1.4) -- (8+3,4-3.6);
%    	\draw[-,thick] (4.6+3,-0.1) -- (4.6+3,0.1);
    %        \draw[-,dashed] (6.6+7,0) -- (6.6+7,5);
%            	\draw (6.6+7,-0.75) node {$Y_3$};
    
            	\draw[-,thick] (7+3+1,4-3.6) -- (7+4+1,4-2.6);
            \draw[-,thick] (7+3+1,-0.1) -- (7+3+1,0.1);
	    	\draw (7+3+1,-0.75) node {$\Gamma_2$};
    \end{tikzpicture}
    \caption{\label{fig: sample paths} Sample paths and time of up-down and down-up transitions for \(\varphi(0)\in\calS_+\) (top) and \(\varphi(0)\in\calS_-\) (bottom).}
\end{figure}
Similarly, for sample paths with \(\varphi(0)\in\mathcal S_-\), let \(\Sigma_0=0\), then for \(m\geq 1\), 
\begin{align}
	\Gamma_m &:=\inf\{t > \Sigma_{m-1} \mid \varphi(t)\in\mathcal S_+\}.
	\\\Sigma_m &:=\inf\{t > \Gamma_{m} \mid \varphi(t)\in\mathcal S_-\}, 
\end{align}

% Using the law of total probability, we may write (\ref{eqn: sojourn}) with \(i\in\calS_+,\,j\in\calS_+\cup\calS_{+0}\), as 
% \begin{align}
% 	\label{eqn: sojourn partition}&\sum_{m=0}^\infty \mu_{m,+,+}^{\ell_0}(t)(\wrt x,j;x_0,i),
% 	%\\&{}+ \sum_{m=1}^\infty \mathbb P(X(t)\in \wrt x, X(s)\in \mathcal D_{\ell_0}, s\in[0,t], \varphi(t) = j, \Sigma_m\leq t< \Gamma_{m}\mid X(0) = x_0, \varphi(0) = i).
% \end{align}
% We can further partition by including the phases at times \(\Sigma_1,\Gamma_1, \Sigma_2,\Gamma_2,\dots\), i.e. 
% \begin{align}
% 	&\sum_{m=0}^\infty \sum_{j_1\in\mathcal S_-} \sum_{k_1\in\mathcal S_+}\dots \sum_{j_m\in\mathcal S_-}\sum_{k_m\in\mathcal S_+}  \mathbb P\Big(X(t)\in\wrt x,  \tau_1^X>t, \varphi(t) = j, \Gamma_m\leq t<\Sigma_{m+1}, \nonumber
% 	\varphi(\Sigma_\ell) = j_\ell ,\\&\quad{}\varphi(\Gamma_\ell) = k_\ell, \ell = 1,\dots,m\mid X(0) = x_0, \varphi(0) = i\Big).\label{eqn: sojourn partition again}
% 	%\\&{}+\sum_{m=1}^\infty \sum_{j_1\in\mathcal S_-} \sum_{k_2\in\mathcal S_+}\dots \sum_{k_{m-1}\in\mathcal S_+}\sum_{j_m\in\mathcal S_-}  \mathbb P\Big(X(t)\in \wrt x, X(s)\in \mathcal D_{\ell_0}, s\in[0,t], \varphi(t) = j, \Sigma_m\leq t<\Gamma_{m}, \nonumber
% 	%\\&\quad{} \bigcap\limits_{\ell=1}^m\varphi(\Sigma_\ell) = j_\ell , \bigcap\limits_{\ell=1}^{m-1}\varphi(\Gamma_\ell) = k_\ell \mid X(0) = x_0, \varphi(0) = i\Big).
% \end{align}
% Define \(\mu^{\ell_0}_{m,+,+}(t)(\wrt x, j_1,k_1,\dots,j_m,k_m, j; x_0,i) \) by
% \begin{align}
% 	&\mathbb P\Big(X(t)\in\wrt x, \tau_1^X>t, \varphi(t) = j, \Gamma_m\leq t<\Sigma_{m+1}, \nonumber
% 	\varphi(\Sigma_\ell) = j_\ell , \varphi(\Gamma_\ell) = k_\ell, \ell = 1,\dots,m \\&\qquad{}\mid X(0) = x_0, \varphi(0) = i\Big),\label{eqn: density part +}
% \end{align}
% \(i\in\calS_+,\) \(j\in\calS_+\cup\calS_{+0},\) \( j_1,j_2,\dots\in\calS_-,\) \(k_1,k_2,\dots\in\calS_+,\) \(x_0\in\calD_{\ell_0,i},\) \(x\in\calD_{\ell_0,i},\) \(t\geq0,\) \(\ell_0\in\{0,\dots,K\},\) \(m\geq 0\), which are the summands in the sums (\ref{eqn: sojourn partition again}). Analogously, define measures 
% \begin{align}
% 	&\mu^{\ell_0}_{m,+,-}(t)(\wrt x, j_1,k_1,\dots,j_m,k_m,j_{m+1}, j; x_0,i)  \nonumber
% 	\\&= 
% 	\mathbb P\Big(X(t)\in\wrt x, \tau_1^X>t, \varphi(t) = j, \Sigma_{m+1}\leq t<\Gamma_{m+1},\varphi(\Sigma_{m+1})=j_{m+1}, \nonumber
% 	 \varphi(\Sigma_\ell) = j_\ell , 
% 	 \\&\qquad{} \varphi(\Gamma_\ell) = k_\ell, 
% 	\ell = 1,\dots,m
% 	 \mid X(0) = x_0, \varphi(0) = i\Big),\label{eqn: gq er h}
% 	\intertext{for \(i\in\mathcal S_+ ,\,j\in\mathcal S_{-}\cup\mathcal S_{-0},\, m\geq 0\);}
% 	&\mu^{\ell_0}_{m,-,+}(t)(\wrt x, k_1,j_1,\dots,j_m,k_{m+1}, j; x_0,i)  \nonumber
% 	\\&= 
% 	\mathbb P\Big(X(t)\in\wrt x,  \tau_1^X>t, \varphi(t) = j, \Gamma_{m+1}\leq t<\Sigma_{m+1},\varphi(\Gamma_{m+1})=k_{m+1}, \nonumber
% 	\varphi(\Sigma_\ell) = j_\ell , 
% 	\\&\qquad{}\varphi(\Gamma_\ell) = k_\ell, 
% 	\ell = 1,\dots,m\mid X(0) = x_0, \varphi(0) = i\Big),
% 	\intertext{for \(i\in\mathcal S_-  ,\, j\in\calS_+\cup\mathcal S_{+0};\)}
% 	&\mu^{\ell_0}_{m,-,-}(t)(\wrt x, k_1, j_1,\dots,k_m,j_m, j; x_0,i)  \nonumber
% 	\\&= 
% 	\mathbb P\Big(X(t)\in\wrt x,  \tau_1^X>t, \varphi(t) = j, \Sigma_{m}\leq t<\Gamma_{m+1}, \nonumber
% 	 \varphi(\Sigma_\ell) = j_\ell , \varphi(\Gamma_\ell) = k_\ell, \ell = 1,\dots,m \\&\qquad{} \mid X(0) = x_0, \varphi(0) = i\Big),
% \label{eqn: 63}
% \end{align}
% for \(i\in\mathcal S_- ,\,j\in\calS_-\cup\mathcal S_{-0},\) where in each case \(j_1,j_2,\dots\in\calS_-,\,k_1,k_2,\dots\in\calS_+,\,x_0\in\calD_{\ell_0,i},\,x\in\calD_{\ell_0,j},\,t\geq0,\,\ell_0\in\{0,\dots,K\},\,m\geq 0\). 
% where we define \(\mu_{m,+,+}^{\ell_0}(t)(\cdot,j;x_0,i) \), \(m\geq 0\), to be the measures on \(\calD_{\ell_0}\)
% \begin{align}
% 	&\mathbb P(\bs X(t)\in(\cdot,j), t<\tau_1^X,  \Gamma_m\leq t<\Sigma_{m+1}\mid \bs X(0) = (x_0,  i)), \label{eqn: loop mu}
% \end{align}
% \(x_0\in\calD_{\ell_0}\). 
For \(m\geq 0\), \(i\in\calS_+,\,j\in\calS_+\cup\calS_{+0}\) define 
\begin{align}
	& \mu_{m,+,+}^{\ell_0}(t)(\cdot,j;x_0,i) = \mathbb P(\bs X(t)\in(\cdot,j), t<\tau_1^X,  \Gamma_m\leq t<\Sigma_{m+1}\mid \bs X(0) = (x_0,  i)), \label{eqn: loop mu}
	%
	\intertext{for \(i\in\calS_+,\,j\in\calS_-\cup\calS_{-0}\) define}
	&\mu^{\ell_0}_{m+1,+,-}(t)(\cdot, j; x_0,i) 
	= \mathbb P(X(t)\in(\cdot,j), t<\tau_1^X,  \Sigma_{m+1}\leq t<\Gamma_{m+1}\mid \bs X(0) = (x_0, i)),\label{eqn: fkl}%\sum_{j_1\in\mathcal S_-} \sum_{k_1\in\mathcal S_+}\dots \sum_{k_m\in\mathcal S_+}\sum_{j_{m+1}\in\mathcal S_-}\mu^{\ell_0}_{m,+,-}(t)(\wrt x, j_1,k_1,\dots,k_m,j_{m+1}, j; x_0,i)
	\intertext{for \(i\in\calS_-,\,j\in\calS_+\cup\calS_{+0}\) define}
	&\mu^{\ell_0}_{m+1,-,+}(t)(\cdot, j; x_0,i)  
	= \mathbb P(\bs X(t)\in(\cdot,j), t<\tau_1^X, \Gamma_{m+1}\leq t<\Sigma_{m+1}\mid X(0) = (x_0, i)),%\sum_{k_1\in\mathcal S_+} \sum_{j_1\in\mathcal S_-}\dots \sum_{j_m\in\mathcal S_-}\sum_{k_{m+1}\in\mathcal S_+}\mu^{\ell_0}_{m,-,+}(t)(\wrt x, k_1,j_1,\dots,j_m,k_{m+1}, j; x_0,i) 
	\intertext{and \(i\in\calS_-,\,j\in\calS_-\cup\calS_{-0}\) define}
	&\mu^{\ell_0}_{m,-,-}(t)(\cdot, j; x_0,i) = \mathbb P(\bs X(t)\in(\cdot,j), t<\tau_1^X, \Sigma_m\leq t<\Gamma_{m+1}\mid \bs X(0) = (x_0, i)). \label{eqn: many eqns mu}% \sum_{k_1\in\mathcal S_+} \sum_{j_1\in\mathcal S_-}\dots \sum_{k_m\in\mathcal S_+}\sum_{j_m\in\mathcal S_-}\mu^{\ell_0}_{m,-,-}(t)(\wrt x, k_1, j_1,\dots,k_m,j_m, j; x_0,i)  
\end{align}
%For \(i\notin \mathcal S_{+0}\cup\calS_{-0}\), the quantity (\ref{eqn: sojourn}) is 
%\begin{align}\label{eqn: trie thingy}
%	\mu^{\ell_0}(t)(\wrt x,j;x_0,i)
%	&=\begin{cases}
%		\mu^{\ell_0}_{+,+}(t)(\wrt x,j;x_0,i)  & i\in\calS_+,\,j\in\mathcal S_+\cup\calS_{+0},
%		\\ \mu^{\ell_0}_{+,-}(t)(\wrt x,j;x_0,i)  & i\in\mathcal S_+,\,j\in\mathcal S_-\cup\calS_{-0},
%		\\ \mu^{\ell_0}_{-,+}(t)(\wrt x,j;x_0,i)  & i\in\mathcal S_-,\,j\in\mathcal S_+\cup\calS_{+0},
%		\\ \mu^{\ell_0}_{-,-}(t)(\wrt x,j;x_0,i)  & i\in\mathcal S_-,\,j\in\mathcal S_-\cup\calS_{-0},
%	\end{cases}
%\end{align}
Furthermore, let 
\begin{align}
		\mu^{\ell_0}_{+,+}(t)(\cdot,j;x_0,i)  &:= \sum_{m=0}^\infty \mu^{\ell_0}_{m,+,+}(t)(\cdot,j;x_0,i)  && i\in\calS_+,\,j\in\mathcal S_+\cup\calS_{+0}, \label{eqn: ljg97skg}
		\\ \mu^{\ell_0}_{+,-}(t)(\cdot,j;x_0,i)  &:= \sum_{m=1}^\infty \mu^{\ell_0}_{m,+,-}(t)(\cdot,j;x_0,i)  && i\in\mathcal S_+,\,j\in\mathcal S_-\cup\calS_{-0},
		\\ \mu^{\ell_0}_{-,+}(t)(\cdot,j;x_0,i) &:= \sum_{m=1}^\infty \mu^{\ell_0}_{m,-,+}(t)(\cdot,j;x_0,i)  && i\in\mathcal S_-,\,j\in\mathcal S_+\cup\calS_{+0},
\end{align}
and
\begin{align}
		\mu^{\ell_0}_{-,-}(t)(\cdot,j;x_0,i)  &:= \sum_{m=0}^\infty \mu^{\ell_0}_{m,-,-}(t)(\cdot,j;x_0,i)  && i\in\calS_-,\,j\in\mathcal S_-\cup\calS_{-0}. \label{eqn: ljg97skg2}
\end{align}
Then we can write (\ref{eqn: sojourn}) as 
\begin{align}
	\mu^{\ell_0}(t)(\cdot,j; x_0,i) =\begin{cases}
		\mu^{\ell_0}_{+,+}(t)(\cdot,j;x_0,i)  & i\in\calS_+,\,j\in\calS_+\cup\calS_{+0},
	\\     \mu^{\ell_0}_{+,-}(t)(\cdot,j;x_0,i)  & i\in\mathcal S_+,\,\,j\in\mathcal S_{0}\cup\calS_{-0},
	\\ \mu^{\ell_0}_{-,+}(t)(\cdot,j;x_0,i)  & i\in\mathcal S_-,\,\,j\in\mathcal S_{+}\cup\calS_{+0},
	\\   \mu^{\ell_0}_{-,-}(t)(\cdot,j;x_0,i) & i\in\calS_-,\,j\in\mathcal S_-\cup\calS_{-0}.
	\end{cases}\label{eqn:ghghghghggg2}
\end{align}

For states \(k\in\mathcal S_{0}\), and \(p\in \{+,-\}, \, q\in\{+,-\}\), \(m\geq 0\), then
\begin{align}
	\mu_{m,0,q}^{\ell_0}(t)(x,j;x_0,k)  
	&:= \sum_{r\in\{+,-\}}\sum_{i\in\calS_r}\int_{t_0=0}^t \bs e_ke^{\bs T_{00}t_0} \bs T_{0i} \mu_{m+1(r\neq q),r,q}^{\ell_0}(t-t_0)(x,j;x_0,i)\wrt t_0 . \label{eqn: vma0}
\end{align}

\section{Laplace transforms with respect to time}\label{sec: lst on no change}
In this section we take the Laplace transform with respect to time of the densities \(f_{m,q,r}^{\ell_0,(p)}(t)(x,j;x_0,k)\), and measures \(\mu_{m,q,r}^{\ell_0}(t)(x,j;x_0,k)\), \(q\in\{+,-,+0,-0\}\), \(r\in\{+,-\}\). The Laplace transform is convenient as it allows us to manipulate the expressions for the QBD-RAP into one component related to the orbit process and one component related the phase process and the rates \(c_i,\,i\in\mathcal S\). 

Here, we only need to consider Laplace transforms with real transform parameter, \(\lambda \in \mathbb R, \lambda > 0\), as this is what the convergence results we use require. Therefore, take \(\lambda \in\mathbb R, \lambda > 0\), throughout.

The following matrices play a key roles in the analysis of fluid queues (see, for example \cite{bean2009,dasilva2005}). Here, they appear in the Laplace trasforms of the QBD-RAP and the fluid queue. Define 
\begin{align*}
	\bs Q_{+0}(\lambda) &= \bs C_+^{-1}\bs T_{+0}\left[\lambda \bs I - \bs T_{00}\right]^{-1},
	%
	\\\bs Q_{-0}(\lambda) &= \bs C_-^{-1}\bs T_{-0}\left[\lambda \bs I - \bs T_{00}\right]^{-1},
	%
	\\\bs Q_{++}(\lambda) &= \bs C_+^{-1} \left(\bs T_{++} - \lambda \bs I + \bs T_{+0}\left[\lambda \bs I - \bs T_{00}\right]^{-1}\bs T_{0+}\right),
	%
	\\\bs Q_{+-}(\lambda) &= \bs C_+^{-1} \left(\bs T_{+-} + \bs T_{+0}\left[\lambda \bs I - \bs T_{00}\right]^{-1}\bs T_{0-} \right) ,
	%
	\\\bs Q_{--}(\lambda) &= \bs C_-^{-1} \left(\bs T_{--}  - \lambda \bs I + \bs T_{-0}\left[\lambda \bs I - \bs T_{00}\right]^{-1}\bs T_{0-}\right),
	%
	\\\bs Q_{-+}(\lambda) &=\bs C_-^{-1} \left(\bs T_{-+}+ \bs T_{-0}\left[\lambda \bs I - \bs T_{00}\right]^{-1}\bs T_{0+}\right) ,
\end{align*}
and the functions,
\begin{align}
	\bs H^{++}(\lambda,x)&= \left[h_{ij}^{++}(\lambda,x)\right]_{i\in \mathcal S_+,j\in\mathcal S_{+}\cup\calS_{+0}} := e^{\bs{Q}_{++}(\lambda)x}\vligne{\bs C_+^{-1} & \bs Q_{+0}(\lambda)},  \label{eqn: lst 1}
	\\\bs H^{--}(\lambda,x) &= \left[h_{ij}^{--}(\lambda,x)\right]_{i\in\mathcal S_-,j\in\mathcal S_{-}\cup\calS_{-0}}:= e^{\bs{Q}_{--}(\lambda)x}\vligne{\bs C_-^{-1} & \bs Q_{-0}(\lambda)},
	\\\bs H^{+-}(\lambda,x)  &= \left[h_{ij}^{+-}(\lambda,x)\right]_{i\in\mathcal S_+,\,j\in\mathcal S_-}:= e^{\bs{Q}_{++}(\lambda)x}\bs{Q}_{+-}(\lambda), 
	\\\bs H^{-+}(\lambda,x)&= \left[h_{ij}^{-+}(\lambda,x)\right]_{i\in\mathcal S_-,\, j\in\mathcal S_+} := e^{\bs{Q}_{--}(\lambda)x}\bs{Q}_{-+}(\lambda) , \label{eqn: lst 4}
\end{align}
for \(x,\lambda\geq 0\). The function \(h_{ij}^{++}(\lambda,x)\) (\(h_{ij}^{--}(\lambda,x)\)) is the Laplace transform with respect to time of the time taken for the fluid level to shift by an amount \(x\) whilst remaining in phases in \(\mathcal S_+\cup\calS_{+0}\) (\(\mathcal S_-\cup\calS_{-0}\)), given the phase was initially \(i\in\mathcal S_+\) (\(i\in\mathcal S_-\)) \citep{bean2005}. The function \(h_{ij}^{+-}(\lambda,x)\) (\(h_{ij}^{-+}(\lambda,x)\)) is the Laplace transform with respect to time of the time taken for the fluid level, \(\{X(t)\}\) to shift by an amount \(x\) whilst remaining in phases in \(\mathcal S_+\cup\calS_{+0}\) (\(\mathcal S_-\cup\calS_{-0}\)), after which time the phase instantaneously changes to \(j\in\mathcal S_-\) (\(\mathcal S_+\)), given the phase was initially \(i\in\mathcal S_+\) (\(\mathcal S_-\)) \citep{bean2005}.

% Let \(\psi:[0,\Delta)\to\mathbb R\) be bounded and Lipschitz continuous and consider expectations with respect to measures (\ref{eqn: loop mu})-(\ref{eqn: many eqns mu}). For example, 
% \begin{align}
% 	\int_{x=0}^\Delta \mu^{\ell_0}_{m,+,+}(t)( y_{\ell_0} + \wrt x, j; x_0,i) \psi(x)\wrt x. \label{eqn: papPP}
% \end{align}
Consider taking the Laplace transform with respect to time of (\ref{eqn: loop mu});
\begin{align}
	\int_{t=0}^\infty e^{-\lambda t} \mu^{\ell_0}_{m,+,+}(t)(\cdot, j; x_0,i)  \wrt t \nonumber 
	&= \int_{t=0}^\infty e^{-\lambda t} \mu^{\ell_0}_{m,+,+}(t)(\cdot, j; x_0,i)  \wrt t   \nonumber 
	\\&= \widehat \mu^{\ell_0}_{m,+,+}(\lambda)( \cdot, j; x_0,i)    \label{eqn: loop mu2}
\end{align}
where we use \(\widehat \mu^{\ell_0}_{m,+,+}(\lambda)( \cdot, j; x_0,i) \) to denote the Laplace transform with respect to time of (\ref{eqn: loop mu}). We now proceed to derive an expression for the Laplace transform with respect to time, \(\widehat \mu^{\ell_0}_{m,+,+}(\lambda)( \cdot, j; x_0,i) \). Throughout, we use the hat \(\,\widehat{}\,\)  notation to denote Laplace transforms with respect to time. 

From the stochastic interpretations of the Laplace transforms (\ref{eqn: lst 1})-(\ref{eqn: lst 4}) given in \citep{bean2005} and summarised above, the Laplace transforms with respect to time, \(\widehat \mu^{\ell_0}_{m,+,+}(\lambda)( x, j; x_0,i)\), of (\ref{eqn: loop mu}) are given by 
\begin{align*}
	\widehat \mu^{\ell_0}_{0,+,+}(\lambda)( x,j;x_0,i) \wrt x&= h_{ij}^{++}(\lambda,x-x_0)1(x\geq x_0)\wrt x,
\end{align*}
for \(m=0\),  and 
\begin{align}
	\nonumber&\int_{x_1 = 0}^{\Delta-(x_0-y_{\ell_0})} \bs e_i \bs H^{+-}(\lambda,\Delta-(x_0-y_{\ell_0})-x_1)  %\int_{x_2 = 0}^{\Delta-x_1} h_{j_1k_1}^{-+}(\lambda,\Delta - x_2-x_1) \wrt x_1 
	\\&  \nonumber\left[\prod_{r=1}^{m-1} \int_{x_{2r}=0}^{\Delta-x_{2r-1}} \bs H^{-+}(\lambda,\Delta-x_{2r}-x_{2r-1})\wrt x_{2r-1}\int_{x_{2r+1}=0}^{\Delta-x_{2r}}\bs H^{+-}(\lambda,\Delta-x_{2r+1}-x_{2r})\wrt x_{2r}\right]
	\\& \int_{x_{2m}=0}^{\Delta-x_{2m-1}} \bs H^{-+}(\lambda,\Delta -x_{2m-1} - x_{2m}) \wrt x_{2m-1}\bs H^{++}(\lambda,\Delta -x_{2m}- (y_{\ell_0+1}- x))\bs e_j \nonumber	
	\\& 1(\Delta-x_{2m}-(y_{\ell_0+1}-x)\geq 0)\wrt x_{2m} \wrt x \label{eqn: lst herwe}
\end{align} 
for \(m\geq 1\). Figure~\ref{fig: sample paths lst} shows an example of the sample paths to which these Laplace transforms correspond.  Analogously, we can write down similar expressions for the Laplace transform with respect to time of (\ref{eqn: loop mu})-(\ref{eqn: many eqns mu}). 

% Observe that \(\widehat \mu^{\ell_0}_{0,+,+}(\lambda)(x,j;x_0,i)\) is a smooth function in the variable \(x> x_0\) (and \(\lambda>0\) too), even though the measure \(\mu^{\ell_0}_{0,+,+}(t)(\wrt x,j;x_0,i)\) may have point masses. For example, given \(X(0)=x_0\) and \(\varphi(0)=i\), then \(\mu^{\ell_0}_{0,+,+}(t)(\wrt x,j;x_0,i)\) has a point mass of size (at least) \(e^{T_{ii}t}\delta(x-(x_0+c_it))1(j=i)1(x\in\mathcal D_{\ell_0,i})\) which arises from the event that the phase remains in \(i\) until time \(t\). Meanwhile, the Laplace transform \(\widehat \mu^{\ell_0}_{0,+,+}(\lambda)(\wrt x,j;x_0,i)\) is given by \(\displaystyle \left[e^{\bs Q_{++}(\lambda)(x-x_0)}1(x\geq x_0)\right]_{ij}\), which is a smooth function for all \(x\) except at \(x=x_0\). 

% The Laplace transform with respect to time of (\ref{eqn: loop mu}), \(\widehat \mu^{\ell_0}_{m,+,+}(t)(\wrt x,j;x_0,i)\), is the \((i,j)\)th entry of 
% \begin{align}
% 	\nonumber&\int_{x_1 = 0}^{\Delta-(x_0-y_{\ell_0})} \bs H^{+-}(\lambda,\Delta-(x_0-y_{\ell_0})-x_1) \int_{x_2 = 0}^{\Delta-x_1} \bs H^{-+}(\lambda,\Delta - x_2-x_1) \wrt x_1 \dots  
% 	\\&\quad\times \int_{x_{2m}=0}^{\Delta-x_{2m-1}} \bs H^{-+}(\lambda,\Delta -x_{2m-1} - x_{2m}) \wrt x_{2m-1}\bs H^{++}(\lambda,\Delta -x_{2m}- (y_{\ell_0+1}- x)) \nonumber
% 	\\&\quad\times 1(\Delta-x_{2m}-(y_{\ell_0+1}-x)\geq 0)\wrt x_{2m}\wrt x. \label{eqn: lst herwe2}
% \end{align} 
% Analogous expressions can be written down for the Laplace transforms of (\ref{eqn: fkl})-(\ref{eqn: many eqns mu}).

\begin{figure}
    \centering\begin{tikzpicture}
    	\draw[->,thick] (0,-1) -- (7.5,-1);
   	\draw[-,dashed] (0,4) -- (7.5,4);
    	\draw (7.5,-1.75) node {$t$};
    	\draw[-,thick] (0,-1) -- (0,4.5);
    	\draw[-,thick] (-0.1,4) -- (0.1,4);
    	\draw[-,thick] (-0.1,0) -- (0.1,0);
	\draw (-0.75,0) node {$x_0$};
    	\draw (-0.75,4) node {$y_{\ell_0+1}$};
    	\draw (-0.75,-1) node {$y_{\ell_0}$};
    	\draw[|-|,thick] (0,-1) -- (0,0);
    	\draw (0,-0.5) node[fill=white] {$x_0-y_{\ell_0}$};
	
    	\draw (0,-1.75) node {$\sigma_0$};
    %	\foreach \i in {1,2} {
    %        		\draw[-,thick] (\i*4,-0.1) -- (\i*4,0.1);
    %%		\draw (\i*4,-0.75) node {$\i \Delta$};
    %        }
    %	\edef\mya{0}
    %	\foreach \x/\y [count=\c] in {0/2.6,2.6/2,2/2,2/2} {
    %		\draw[-,thick] (\mya,\x) -- (\mya+\y,{(-(-1)^\c)*\y+\x});
    %                	\pgfmathparse{\mya+\y}
    %                	\xdef\mya{\pgfmathresult}
    %	}
    
	
            \draw[-,thick] (0,0) -- (2.6,2.6);
            \draw[-,thick] (2.6,-1.1) -- (2.6,-0.9);
    %        \draw[-,dashed] (2.6,0) -- (2.6,5);
       	\draw[|-|] (2.6,2.6) -- (2.6,4);
	\draw (2.6,2.6+0.7) node[fill=white] {$x_1$};
    	\draw (2.6,-1.75) node {$\sigma_1$};
    	
       	\draw[|-|] (1.3,0) -- (1.3,2.6);
	\draw (1.3,1.3) node[fill=white] {$z_1$};
	
%    	\draw[|-|,thick] (3.6-1,1) -- (3.6+1,1);
%    	\draw (3.6,1) node[fill=white] {$ R_1$};
    	
%    	\draw[|-|,thick] (5.6-1,1) -- (5.6+1,1);
%    	\draw (5.6,1) node[fill=white] {$ R_1$};
    	
%            \draw[-,dashed] (2.6,2.6) -- (2.6+2,2.6+2);
%            \draw[-,thick] (4.6,-0.1) -- (4.6,0.1);
    %        \draw[-,dashed] (4.6,0) -- (4.6,5);
%    	\draw (4.6,-0.75) node {$Y_1$};
    	
%            \draw[-,dashed] (4.6,4.6) -- (4.6+2,4.6-2);
%            \draw[-,thick] (6.6,-0.1) -- (6.6,0.1);
    %        \draw[-,dashed] (6.6,0) -- (6.6,5);
    %	\draw (6.6,-0.75) node {$Y_1+R_1$};
    
    	
    	\draw[-,thick] (2.6,2.6) -- (4.6,2.6-2);
            \draw[-,thick] (4.6,-1.1) -- (4.6,-0.9);
    %        \draw[-,dashed] (6.6+2,0) -- (6.6+2,5);
        \draw[|-|] (4.6,-1) -- (4.6,0.6);
	\draw (4.6,-0.2) node[fill=white] {$x_2$};
    	\draw (4.6,-1.75) node {$\sigma_2$};
	
       	\draw[|-|] (3.6,2.6) -- (3.6,0.6);
	\draw (3.6,1.6) node[fill=white] {$z_2$};
    	
%    	\draw[<->,thick] (9.1-0.5,1) -- (9.1+0.5,1);
%    	\draw (9.1,1) node[fill=white] {$R_2$};
%    	\draw[<->,thick] (10.1-0.5,1) -- (10.1+0.5,1);
%    	\draw (10.1,1) node[fill=white] {$R_2$};
            
%            	\draw[-,dashed] (6.6+2,0.6) -- (6.6+3,0.6-1);
%            \draw[-,thick] (6.6+3,-0.1) -- (6.6+3,0.1);
%    %        \draw[-,dashed] (6.6+3,0) -- (6.6+3,5);
%            	\draw (9.6,-0.75) node {$Y_2$};
    	
%            	\draw[-,dashed] (6.6+3,-0.4) -- (6.6+4,0.6);
%    	\draw[-,thick] (6.6+4,-0.1) -- (6.6+4,0.1);
    %        \draw[-,dashed] (6.6+4,0) -- (6.6+4,5);
            
            	\draw[-,thick] (4.6,0.6) -- (4.6+3,3.6);
%    	\draw[-,thick] (4.6+3,-0.1) -- (4.6+3,0.1);
    %        \draw[-,dashed] (6.6+7,0) -- (6.6+7,5);
%            	\draw (6.6+7,-0.75) node {$Y_3$};
    
            	\draw[-,dotted] (4.6+3,3.6) -- (4.6+3.2,3.8);
    \end{tikzpicture}
    \caption{\label{fig: sample paths lst} Sample paths corresponding to the Laplace transforms (\ref{eqn: lst herwe}). \(z_1 = \Delta - x_1-(x-y_{\ell_0}),\, z_2= \Delta - x_2-x_1\).}
\end{figure}

% Now consider expectations with respect to measures (\ref{eqn: approx end conv})-(\ref{eqn: 67}). For example,
% \begin{align}
% 	\int_{x=0}^\Delta f^{\ell_0,(p)}_{m,+,+}(t)( y_{\ell_0} + \wrt x,j; x_0,i) \psi(x). \label{eqn: fpapPP}
% \end{align}
Now consider taking the Laplace transform with respect to time of (\ref{eqn: approx end conv});
\begin{align}
	\int_{t=0}^\infty e^{-\lambda t} f^{\ell_0,(p)}_{m,+,+}(t)(  x, j; x_0,i) \wrt t \nonumber 
	&= \int_{t=0}^\infty e^{-\lambda t} f^{\ell_0,(p)}_{m,+,+}(t)( x, j; x_0,i)  \wrt t  \nonumber 
	\\&= \widehat f^{\ell_0,(p)}_{m,+,+}(\lambda)( x, j; x_0,i) \label{eqn: fpapPP2}
\end{align}
where we use \(\widehat f^{\ell_0,(p)}_{m,+,+}(\lambda)(  x, j; x_0,i) \) to denote the Laplace transform with respect to time of (\ref{eqn: approx end conv}). We now proceed to derive an expression for the Laplace transform with respect to time, \(\widehat f^{\ell_0,(p)}_{m,+,+}(\lambda)( x, j; x_0,i) \).

Notice that (\ref{eqn: approx end conv}) is a convolution. Hence, the Laplace transform with respect to time of (\ref{eqn: approx end conv}) is
\begin{align}
	&\widehat f^{\ell_0,(p)}_{m,+,+}(\lambda)(x, j; x_0,i) \nonumber 
	% \\&=\int_{t=0}^\infty e^{-\lambda t}\int_{\bs a \in\mathcal A^{(p)}}\mathbb P\Big(\bs A^{(p)}(t)\in \wrt \bs a, t<\tau_1^{(p)}, \varphi(t) = j, \Gamma_m\leq t<\Sigma_{m+1}, \nonumber
	% \varphi(\Sigma_\ell) = j_\ell , 
	% \\&\qquad\varphi(\Gamma_\ell) = k_\ell, \ell = 1,\dots,m \mid \bs A^{(p)}(0) = \bs   a_{\ell_0,i}^{(p)}(x_0), \varphi(0) = i\Big)\bs a {\bs v}^{(p)}(y_{\ell_0+1}-x)\wrt t\nonumber
	%
	\\&=(\bs e_i\otimes \bs  a_{\ell_0,i}^{(p)}(x_0))  \int_{t_1=0}^\infty e^{-\lambda t_1} e^{\bs{B}^{(p)}_{++}t_1} \wrt t_1 \bs{B}^{(p)}_{+{-}} \nonumber
	\int_{t_2=0}^\infty e^{-\lambda t_2} e^{\bs{B}^{(p)}_{--}t_2} \wrt t_2\bs{B}^{(p)}_{-{+}} 
	\\&\hdots 
	\int_{t_{2m}=0}^\infty e^{-\lambda t_{2m}}e^{\bs{B}^{(p)}_{--}t_{2m}} \wrt t_{2m}\bs{B}^{(p)}_{-{+}} 
	\int_{t=0}^\infty e^{-\lambda t}e^{\bs{B}^{(p)}_{++}t} \wrt t 
	%
	\left(\bs e_j \otimes {\bs v}_{\ell_0,j}^{(p)}(x)\right). \label{eqn: approx end conv lst}
\end{align}
Analogous expressions can be computed for the Laplace transforms with respect to time of (\ref{eqn:gljagj})-(\ref{eqn: 67}). 

%The Laplace transforms are convenient formulations with which to work as the integrands in (\ref{eqn: approx end conv lst}) simplify to a product of one of the functions in (\ref{eqn: lst 1})-(\ref{eqn: lst 4}) and a matrix exponential effectively separating the behaviour. 

In Corollary~\ref{cor: mpr B} in Appendix~\ref{appendix: kronecker}, we show the following relation
\begin{align}
	&\vligne{\bs I_{p|\calS_m|} & \bs 0_{p|\calS_m|\times p|\calS_0|} }\int_{t=0}^\infty e^{-\lambda t} e^{\bs{B}^{(p)}_{mm}t} \wrt t \bs{B}^{(p)}_{m{n}} \nonumber %= \vligne{\bs I & \bs 0 }\int_{t=0}^\infty e^{-\lambda t} e^{\bs{B}_{mm}t} \wrt t  \left[\begin{array}{cc} \bs{T}_{mn}\otimes \bs{D} & \bs 0 \\ \bs T_{0n}\otimes \bs D & \bs 0 \end{array}\right]
	\\&= \int_{x=0}^\infty \bs H^{mn}(\lambda,x)  \otimes  e^{\bs S^{(p)} x}\bs D^{(p)}\wrt x \vligne{\bs I_{p|\calS_n|} & \bs 0_{p|\calS_n|\times p|\calS_0|}}, \label{eqn: akgj987adKLDJaf}
\end{align}
for \(m,n\in\{+,-\}\), \(m\neq n\). Before we can apply this result, observe that we can write the inital vector in (\ref{eqn: approx end conv lst}) as 
\[((\bs e_i)_{1\times |\calS_+\cup\calS_{+0}|}\otimes \bs  a_{\ell_0,i}^{(p)}(x_0)) = ((\bs e_i)_{1\times |\calS_+|}\otimes \bs  a_{\ell_0,i}^{(p)}(x_0))\vligne{\bs I_{p|\calS_+|} & \bs 0_{p|\calS_+|\times p|\calS_0|}}\]
by the~\ref{eqn:mpr}. 
Now, applying (\ref{eqn: akgj987adKLDJaf}) to the first integral in (\ref{eqn: approx end conv lst}) transforms the expression to 
\begin{align}
	&(\bs e_i\otimes \bs  a_{\ell_0,i}^{(p)}(x_0)) \int_{x_1=0}^\infty \left(\bs H^{+{-}}(\lambda,x_1) \otimes e^{\bs S^{(p)} x}\bs D^{(p)}\right) \wrt x_1 \vligne{\bs I_{p|\calS_-|} & \bs 0_{p|\calS_-|\times p|\calS_0|}} \nonumber
	\\&\int_{t_2=0}^\infty e^{-\lambda t_2} e^{\bs{B}^{(p)}_{--}t_2} \wrt t_2\bs{B}^{(p)}_{-{+}} 
	\hdots 
	\int_{t_{2m}=0}^\infty e^{-\lambda t_{2m}}e^{\bs{B}^{(p)}_{--}t_{2m}} \wrt t_{2m}\bs{B}^{(p)}_{-{+}} 
	\int_{t=0}^\infty e^{-\lambda t}e^{\bs{B}^{(p)}_{++}t} \wrt t \nonumber
	%
	\\&\left(\bs e_j \otimes {\bs v}_{\ell_0,j}^{(p)}( x)\right). \label{eqn: approx end conv lst2}
\end{align}
We may now apply (\ref{eqn: akgj987adKLDJaf}) to the second integral, after which we can apply (\ref{eqn: akgj987adKLDJaf}) to the third integral and so on. Ultimately, after applying (\ref{eqn: akgj987adKLDJaf}) to all of the integrals in (\ref{eqn: approx end conv lst}), we get 
%\begin{align}
%	&(\bs e_i\otimes \bs  a_{\ell_0,i}(x_0))  \int_{x_1=0}^\infty e^{\bs{Q}_{++}(\lambda)} \otimes e^{\bs{S}x_1}\wrt x_1 (\bs{Q}_{+-}(\lambda)\otimes \bs{D})\bs e_{j_1}\bs e_{j_1}'
%	\int_{x_2=0}^\infty e^{\bs{Q}_{--}(\lambda)x_2}\otimes e^{\bs{S}x_2} \wrt x_2 \nonumber
%	\\&(\bs{Q}_{-{+}}(\lambda)\otimes \bs{D})\bs e_{k_1}\bs e_{k_1}'
%	\hdots \int_{x_{2m}=0}^\infty e^{\bs{Q}_{--}(\lambda)x_{2m}}\otimes e^{\bs{S}x_{2m}} \wrt x_{2m}(\bs{Q}_{-+}(\lambda)\otimes  \bs{D}) \bs e_{k_m}\bs e_{k_m}'\nonumber
%	\\&\times\int_{x_{2m+1}=0}^\infty e^{\bs{Q}_{++} ( \lambda )x_{2m+1}}\otimes e^{\bs{S}x_{2m+1}} \wrt x_{2m+1}\left(\vligne{\bs C_+^{-1} & \bs Q_{+0}(\lambda)}\otimes  \bs I\right)(\bs e_j\bs e_j') (\bs I \otimes U_+(y_{\ell_0}-x) ) . \label{eqn: approx end conv lst 4}
%\end{align}
%By the~\ref{eqn:mpr}, we can rewrite (\ref{eqn: approx end conv lst 4}) as 
\begin{align}
	&(\bs e_i \otimes \bs   a_{\ell_0,i}^{(p)}(x_0)) \left( \int_{x_1=0}^\infty \bs H^{+-}(\lambda,x_1)\otimes e^{\bs{S}^{(p)}x_1}\bs{D}^{(p)}\wrt x_1 \right)\nonumber%
	% \int_{x_2=0}^\infty \left[\bs H^{-+}(\lambda,x_2)\right]_{j_1,k_1} e^{\bs{S}x_2} \wrt x_2 \bs{D} 	\nonumber
	\\&\Bigg[\prod_{r=1}^{m-1} \left(\int_{x_{2r}=0}^\infty \bs H^{-+}(\lambda,x_{2r}) \otimes e^{\bs{S}^{(p)}x_{2r}}\bs{D}^{(p)}\wrt x_{2r}\right) \nonumber 
	\\&\nonumber \left(\int_{x_{2r+1}=0}^\infty \bs H^{+-}(\lambda,x_{2r+1})
	\otimes e^{\bs{S}^{(p)}x_{2r+1}}\bs{D}^{(p)} \wrt x_{2r+1} \right)\Bigg] 
	\\&\left(\int_{x_{2m}=0}^\infty \bs H^{-+}(\lambda,x_{2m})
	  \otimes e^{\bs{S}^{(p)}x_{2m}}\bs{D}^{(p)} \wrt x_{2m} \right) \nonumber 
	\\&\left(\int_{x_{2m+1}=0}^\infty \bs H^{++}(\lambda,x_{2m+1})\otimes 
	e^{\bs{S}^{(p)}x_{2m+1}} \wrt x_{2m+1}\right) \left(\bs e_j \otimes {\bs v}_{\ell_0,j}^{(p)}(x)\right) \nonumber
	%
	\\&=\int_{x_1=0}^\infty \dots \int_{x_{2m+1}=0}^\infty \bs e_i\bs M^{m}_{++}(\lambda,x_1,\dots,x_{2m+1})\bs e_j \nonumber 
	\\&\times \bs a_{\ell_0,i}^{(p)}(x_0) \bs N^{2m+1,(p)}(\lambda,x_1,\dots,x_{2m+1}) {\bs v}_{\ell_0,j}^{(p)}( x)\wrt x_{2m+1}\dots\wrt x_1 ,
	%
	% \\
	% &=\int_{x_1=0}^\infty h^{+-}_{i,j_{1}}(\lambda,x_1)\bs   a_{\ell_0,i}^{(p)}(x_0)e^{\bs{S^{(p)}}x_1}\wrt x_1 \bs{D}^{(p)}\nonumber
	% %\int_{x_2=0}^\infty h^{-+}_{j_1,k_1}(\lambda,x_2) e^{\bs{S}x_2} \wrt x_2 \bs{D} 
	% %\hdots
	% \Bigg[\prod_{r=1}^{m-1} \int_{x_{2r}=0}^\infty h^{-+}_{j_r,k_r}(\lambda,x_{2r}) e^{\bs{S}^{(p)}x_{2r}}\wrt x_{2r}\bs D^{(p)}
	% \\&\nonumber \quad \int_{x_{2r+1}=0}^\infty h^{+-}_{k_rj_{r+1}}(\lambda,x_{2r+1}) e^{\bs{S}^{(p)}x_{2r+1}} \wrt x_{2r+1}\bs D^{(p)}\Bigg]
	% \int_{x_{2m}=0}^\infty h^{-+}_{j_{m},k_m}(\lambda, x_{2m}) e^{\bs{S}^{(p)}x_{2m}} \wrt x_{2m} \bs{D}^{(p)}
	% \\&\nonumber \quad \int_{x_{2m+1}=0}^\infty h^{++}_{k_m,j}(\lambda,x_{2m+1}) e^{\bs{S}^{(p)}x_{2m+1}}\wrt x_{2m+1}{\bs v}^{(p)}(y_{\ell_0+1}-x) \nonumber
	% %
	% \\&=\int_{x_1=0}^\infty h^{+-}_{i,j_{1}}(\lambda,x_1)
	% \int_{x_2=0}^\infty h^{-+}_{j_1,k_1}(\lambda,x_2)
	% \hdots \int_{x_{2m}=0}^\infty h^{-+}_{j_{m},k_m}(\lambda, x_{2m})  \int_{x_{2m+1}=0}^\infty h^{++}_{k_m,j}(\lambda,x_{2m+1})\nonumber
	% \\&
	%  \quad \bs   a_{\ell_0,i}^{(p)}(x_0)e^{\bs{S}^{(p)}x_1} \bs{D}^{(p)} e^{\bs{S}^{(p)}x_2} \bs{D}^{(p)}\dots e^{\bs{S}^{(p)}x_{2m}}  \bs{D}^{(p)}e^{\bs{S}^{(p)}x_{2m+1}} {\bs v}^{(p)}(y_{\ell_0+1}-x)  \nonumber  
	%  \\&\quad{}\wrt x_{2m+1} \wrt x_{2m}\dots\wrt x_2 \wrt x_1.
	 \label{eqn: approx final end 2}
\end{align}
where we define matrices 
\begin{align}
	&\bs M^{m}_{++}(\lambda,x_1,\dots,x_{2m+1}) \nonumber 
	\\&= \bs H^{+-}(\lambda,x_1)\prod_{r=1}^{m-1}\bs H^{-+}(\lambda,x_{2r}) \bs H^{+-}(\lambda,x_{2r+1}) \bs H^{-+}(\lambda,x_{2m}) 
	\bs H^{++}(\lambda,x_{2m+1})\nonumber 
\end{align}
and
\begin{align}
	%
	\bs N^{n,(p)}(\lambda,x_1,\dots,x_{n}) &= \prod_{r=1}^{n-1} e^{\bs{S}^{(p)}x_{r}}\bs{D}^{(p)} e^{\bs{S}^{(p)}x_{n}}. \nonumber 
\end{align}
The relation (\ref{eqn: akgj987adKLDJaf}) is key to our analysis. It allows us to factorise the integrand of the Laplace transform (\ref{eqn: approx final end 2}) into one factor solely related to the orbit process \(\{\bs A^{(p)}(t)\}\) and another factor solely related to the fluid queue. 

Further, if we define matrices
\begin{align*}
	\bs M^{m}_{-+}(\lambda,x_1,\dots,x_{2m}) &= \bs H^{-+}(\lambda,x_1)\prod_{r=1}^{m-1}\bs H^{+-}(\lambda,x_{2r}) \bs H^{-+}(\lambda,x_{2r+1}) \bs H^{++}(\lambda,x_{2m}),
	%
	\\\bs M^{m}_{+-}(\lambda,x_1,\dots,x_{2m}) &= \bs H^{+-}(\lambda,x_1)\prod_{r=1}^{m-1}\bs H^{-+}(\lambda,x_{2r}) \bs H^{+-}(\lambda,x_{2r+1}) \bs H^{--}(\lambda,x_{2m}),
\end{align*}
and
\begin{align*}
	&\bs M^{m}_{--}(\lambda,x_1,\dots,x_{2m+1}) \nonumber 
	\\&= \bs H^{-+}(\lambda,x_1)\prod_{r=1}^{m-1}\bs H^{+-}(\lambda,x_{2r}) \bs H^{-+}(\lambda,x_{2r+1}) \bs H^{+-}(\lambda,x_{2m}) 
	\bs H^{--}(\lambda,x_{2m+1}),\nonumber 
\end{align*}
then analogous expressions can be shown for (\ref{eqn:gljagj})-(\ref{eqn: 67}) in terms of these matrices; for \(m\geq 0\), 
\begin{align*}
	\widehat f^{\ell_0,(p)}_{m+1,-,+}(\lambda)(x, j; x_0,i) &= 
		\int_{x_1=0}^\infty \dots \int_{x_{2m+2}=0}^\infty \bs e_i\bs M^{m+1}_{-+}(\lambda,x_1,\dots,x_{2m+2})\bs e_j \nonumber 
		\\&\times \bs a_{\ell_0,i}^{(p)}(x_0) \bs N^{2m+2,(p)}(\lambda,x_1,\dots,x_{2m+2}) {\bs v}_{\ell_0,j}^{(p)}( x)\wrt x_{2m+2}\dots\wrt x_1,
	%
	\\ \widehat f^{\ell_0,(p)}_{m+1,+,-}(\lambda)(x, j; x_0,i) &= 
		\int_{x_1=0}^\infty \dots \int_{x_{2m+2}=0}^\infty \bs e_i\bs M^{m+1}_{+-}(\lambda,x_1,\dots,x_{2m+2})\bs e_j \nonumber 
		\\&\times \bs a_{\ell_0,i}^{(p)}(x_0) \bs N^{2m+1,(p)}(\lambda,x_1,\dots,x_{2m+1}) {\bs v}_{\ell_0,j}^{(p)}( x)\wrt x_{2m+2}\dots\wrt x_1,
	\\ \widehat f^{\ell_0,(p)}_{m,-,-}(\lambda)(x, j; x_0,i) &= 
		\int_{x_1=0}^\infty \dots \int_{x_{2m+1}=0}^\infty \bs e_i\bs M^m_{--}(\lambda,x_1,\dots,x_{2m+1})\bs e_j \nonumber 
		\\&\times \bs a_{\ell_0,i}^{(p)}(x_0) \bs N^{2m+1,(p)}(\lambda,x_1,\dots,x_{2m+1}) {\bs v}_{\ell_0,j}^{(p)}( x)\wrt x_{2m+1}\dots\wrt x_1.
\end{align*}

In general, for \(k\in\calS_0^*\), \(p\in \{+,-\}, \, q\in\{+,-\}\), \(m\geq 0\),
\begin{align}
	\widehat f_{m,0,q}^{\ell_0}(\lambda)(x,j;x_0,k)  
	&:= \sum_{r\in\{+,-\}}\sum_{i\in\calS_r}\bs e_k\vligne{\lambda \bs I - \bs T_{00}}^{-1}\bs T_{0i}\widehat f_{m+1(r\neq q),r,q}^{\ell_0}(\lambda)(x,j;x_0,i), \label{eqn: vma}
\end{align}
and
\begin{align}
	\widehat \mu_{m,0,q}^{\ell_0}(\lambda)(x,j;x_0,k) \label{eqn:kdneee}
	&:= \sum_{r\in\{+,-\}}\sum_{i\in\calS_r}\bs e_k\vligne{\lambda \bs I - \bs T_{00}}^{-1}\bs T_{0i}\widehat \mu_{m+1(r\neq q),r,q}^{\ell_0}(\lambda)(x,j;x_0,i).
\end{align}

\section{Convergence on no change of level and a fixed number of up-down/down-up transitions}\label{sec: no change convergence}
We now wish to establish that \(\widehat f_{m,q,r}^{\ell_0,(p)}(t)(x,j;x_0,k)\to\widehat \mu_{m,q,r}^{\ell_0}(t)(x,j;x_0,k)\), \(q\in\{+,-,+0,-0\}\), \(r\in\{+,-\}\).

We need the functions \(h_{ij}^{++}(\lambda,x),\,h_{ij}^{--}(\lambda,x),\) \(h_{ij}^{+-}(\lambda,x) ,\) \( h_{ij}^{-+}(\lambda,x) \) to satisfy the Assumptions~\ref{asu: g} as functions of \(x\) and for all \(i,j\in\mathcal S\), and \(\lambda>0\). To this end, we observe the following bounds, which follow from the stochastic interpretation of the functions. Let \(c_{min}=\min_{i\in\mathcal S_-\cup\calS_+} |c_i|\). For all \(\lambda \geq 0\), there is some \(0\leq G<\infty\) such that 
\begin{align*}
	0\leq h_{ij}^{++}(\lambda,x) &\leq   \max\left\{1/c_{min},1\right\}\leq G,\, i\in\mathcal S_+,j\in\calS_+\cup\mathcal S_{+0},
	%
	\\0\leq  h_{ij}^{--}(\lambda,x)  &  \leq \max\left\{1/c_{min},1\right\}\leq G,\, i\in\mathcal S_-,j\calS_-\in\mathcal S_{-0},
	%
	\\0\leq  h_{ij}^{+-}(\lambda,x)  &  \leq \max_{k,\ell}\left[\bs{Q}_{+-}(0)\right]_{k,\ell}\leq G, \, i\in\mathcal S_+,\, j\in\mathcal S_-,
	%
	\\0\leq  h_{ij}^{-+}(\lambda,x)  &   \leq \max_{k,\ell}\left[\bs{Q}_{-+}(0)\right]_{k,\ell}\leq G, \, i\in\mathcal S_-,\, j \in\mathcal S_+,
\end{align*}
Furthermore, there exists some \(0\leq \widehat G<\infty\) such that, for \( i\in\mathcal S_+,\,j\in\mathcal S_{+0},\)
\begin{align*}
	\int_{x=0}^\infty  h_{ij}^{++}(\lambda,x) \wrt x &\leq \int_{x=0}^\infty  h_{ij}^{++}(0,x) \wrt x = \vligne{-\bs{Q}_{++}(0)^{-1}\bs{C}_+ & -\bs Q_{++}(0)^{-1}\bs Q_{+0}(0)}_{ij}\leq \widehat G, 
	\intertext{for \(i\in\mathcal S_-,\,j\in\mathcal S_{-0},\)}
	\int_{x=0}^\infty  h_{ij}^{--}(\lambda,x) \wrt x &\leq \int_{x=0}^\infty  h_{ij}^{--}(0,x) \wrt x = \vligne{-\bs{Q}_{--}(0)^{-1}\bs{C}_- & -\bs Q_{--}(0)^{-1}\bs Q_{-0}(0)}_{ij}\leq \widehat G,
	\intertext{for \( i\in\mathcal S_+,\, j\in\mathcal S_-,\) }
	\int_{x=0}^\infty  h_{ij}^{+-}(\lambda,x) \wrt x &\leq \int_{x=0}^\infty  h_{ij}^{+-}(0,x) \wrt x = \left[-\bs{Q}_{++}(0)^{-1}\bs{Q}_{+-}(0)\right]_{ij}\leq \widehat G,
	\intertext{for \(i\in\mathcal S_-,\, j \in\mathcal S_+\),}
	\int_{x=0}^\infty  h_{ij}^{-+}(\lambda,x) \wrt x &\leq \int_{x=0}^\infty  h_{ij}^{-+}(0,x) \wrt x = \left[-\bs{Q}_{--}(0)^{-1}\bs{Q}_{-+}(0)\right]_{ij}\leq \widehat G. 
\end{align*}
Moreover, since \(h_{ij}^{qr}(\lambda,x)\), \(q,r\in \{+,-\}\), \(i\in\mathcal S_q,\,j\in\mathcal S_r\), are matrix exponential functions with exponent matrix which is a sub-generator matrix, then for every \(\lambda >0\), \(h_{ij}^{qr}(\lambda,x)\) is Lipschitz continuous with respect to \(x\) on \(x\in[0,\infty)\). Therefore there exists some \(0<L<\infty\) such that \(\left|h_{ij}^{qr}(\lambda,x)-h_{ij}^{qr}(\lambda,y)\right|\leq L|x-y|,\) \(q,r\in \{+,-\}\), \(i\in\mathcal S_q,\,j\in\mathcal S_r\cup\calS_{r0}\).

The main result of this Chapter is the following theorem.
\begin{thm}\label{thm: a thm!}
	Let \(\psi:\mathbb R\to\mathbb R\), be bounded. As \(p\to \infty\), for \(m\geq 1\), \(q\in\{+,-,0\},\, r\in\{+,-\}\), or \(m=0\), \(q=0\), \(r\in\{+,-\}\), or \(m=0\), \(q=r\), \(q,r\in\{+,-\},\) then
	\begin{align}\int_{x\in\calD_{\ell_0}}\widehat f_{m,q,r}^{\ell_0,(p)}(\lambda)(x,j;x_0,k)\psi(x)\wrt x \to \int_{x\in\calD_{\ell_0}}\widehat \mu_{m,q,r}^{\ell_0}(\lambda)(x,j;x_0,k)\psi(x)\wrt x.\label{eqn: thm 2}\end{align}
\end{thm}
The proof of Theorem~\ref{thm: a thm!} is at the end of this section as it is the result of numerous other sub-results, which we now proceed to show. Notice that the convergence in Theorem~\label{thm: a thm!} is a weak result in that we intagrate the spatial variable, \(x\), against test functions \(\psi\). This is necessary due to the discontinuity at \(x=x_0\) what \(m=0\).

Recall that we show results for an arbitrary parameter \(\varepsilon>0\), however, keep in mind the ultimate intention is to show convergence, for which we choose this parameter to be \(\varepsilon^{(p)}=\var\left(Z^{(p)}\right)^{1/3}\). 

%\section{One integral}\label{appendix: int one}
The results rely on the fact that integrating a function, \(g\) say, against a ME density function, or against the density function of a ME conditional on the ME-life-time surviving until some time \(u<\Delta-\varepsilon\) where \(\Delta = \mathbb E[Z]\) is the mean of the matrix exponential distribution, approximates integrating said function against a Kronecker delta situated at \(\Delta\), provided the variance of the ME is sufficiently low. 

 The next result is used in the proof of Theorem~\ref{thm: a thm!} to prove convergence on the event that there is no change of phase from \(\calS_+\) to \(\calS_-\) or \(\calS_-\) to \(\calS_+\) before the first change of level. 
 \begin{lem}\label{lem: Dcoajc}
	Let \(\psi:[0,\Delta)\to \mathbb R\) be bounded, \(\psi(x)\leq F\). Then, for \(x\in\calD_{\ell_0,j}\), \(\ell_0\in\mathcal K\setminus\{-1,K+1\}\), \(\lambda > 0\), \(q\in\{+,-\}\), 
	\begin{align}
		\left|\int_{x=0}^\Delta \widehat f^{\ell_0,(p)}_{0,q,q}(\lambda)(x,j; x_0,i)\psi(x)\wrt x - \int_{x=0}^\Delta\widehat \mu^{\ell_0}_{0,q,q}(\lambda)(x,j; x_0,i)\psi(x)\wrt x\right| \leq \left(R_{{\bs v},2}^{(p)} + \varepsilon^{(p)}\right) GF.
		\label{eqn: anue}
	\end{align} 
\end{lem}
\begin{proof} 
                Property \ref{properties: 2} states
                \begin{align}
                	\left|\int_{x=0}^\infty \cfrac{\bs \alpha e^{\bs{S}(u+x)} }{\bs \alpha e^{\bs{S}u} \bs e} {\bs v}(v)g(x)\wrt x -g(\Delta-u-v) 1(u+v\leq\Delta-\varepsilon)\right| =  |r_{\bs v}(u,v)|.
                \end{align}
                Setting \(g(x) = h_{ij}^{++}(\lambda,x)\), 
                \begin{align}
                	\left|\int_{x=0}^\infty \cfrac{\bs \alpha e^{\bs{S}(u+x)} }{\bs \alpha e^{\bs{S}u} \bs e} {\bs v}(v)h_{ij}^{++}(\lambda,x)\wrt x -h_{ij}^{++}(\lambda,\Delta-u-v) 1(u+v\leq\Delta-\varepsilon)\right| =  |r_{\bs v}(u,v)|. \label{eqn: akv}
                \end{align}
                We recognise the left-most term as 
                \[\widehat f^{\ell_0,(p)}_{0,+,+}(\lambda)(y_{\ell_0+1}-v,j; y_{\ell_0}+u,i) = \int_{x=0}^\infty \cfrac{\bs \alpha e^{\bs{S}(u+x)} }{\bs \alpha e^{\bs{S}u} \bs e} {\bs v}(v)h_{ij}^{++}(\lambda,x)\wrt x.\]
                
                Now consider 
                \begin{align}
                	&\Bigg|\int_{v=0}^\Delta \int_{x=0}^\infty \cfrac{\bs \alpha e^{\bs{S}(u+x)} }{\bs \alpha e^{\bs{S}u} \bs e} {\bs v}(v)h_{ij}^{++} (\lambda,x)\wrt x \psi(v)\wrt v  \nonumber 
		\\&\qquad{}- \int_{v=0}^\Delta h_{ij}^{++}(\lambda,\Delta-u-v) 1(\Delta-u-v\geq 0) \psi(v)\wrt v\Bigg| \nonumber 
                	%
                	\\&\leq \int_{v=0}^\Delta \left|  \int_{x=0}^\infty \cfrac{\bs \alpha e^{\bs{S}(u+x)} }{\bs \alpha e^{\bs{S}u} \bs e} {\bs v}(v)h_{ij}^{++} (\lambda,x)\wrt x  - h_{ij}^{++}(\lambda,\Delta-u-v) 1(\Delta-u-v\geq 0) \right| \left| \psi(v) \right| \wrt v\nonumber 
                	%
                	\\&\leq \int_{v=0}^\Delta \left|  \int_{x=0}^\infty \cfrac{\bs \alpha e^{\bs{S}(u+x)} }{\bs \alpha e^{\bs{S}u} \bs e} {\bs v}(v)h_{ij}^{++} (\lambda,x)\wrt x  - h_{ij}^{++}(\lambda,\Delta-u-v) 1(\Delta-u-v\geq \varepsilon) \right| \left| \psi(v) \right| \wrt v\nonumber 
                	\\&\qquad {}+ \int_{v=0}^\Delta \left| h_{ij}^{++}(\lambda,\Delta-u-v) 1(\varepsilon \geq \Delta-u-v\geq 0) \right| \left| \psi(v) \right| \wrt v. \label{eqn: ALllllLsdnn}
                \end{align}
                Recognising the first term as the left-hand side of (\ref{eqn: akv}), then (\ref{eqn: ALllllLsdnn}) is less than or equal to 
                \begin{align}
                	& \int_{v=0}^\Delta |r_{\bs v}(u,v)| \left| \psi(v) \right| \wrt v 
                	+ \int_{v=0}^\Delta \left| h_{ij}^{++}(\lambda,\Delta-u-v) 1(\varepsilon \geq \Delta-u-v\geq 0) \right| \left| \psi(v) \right| \wrt v\nonumber 
                	\\&\leq R_{{\bs v},2} GF + \varepsilon GF. \nonumber 
                \end{align}
                Finally, noting \(h_{ij}^{++}(\lambda,\Delta-u-v) 1(\Delta-u-v\geq 0)=\widehat \mu_{0,+,+}^{\ell_0}(\lambda)(y_{\ell_0+1}-v,j;y_{\ell_0}+u,i)\), then we have shown (\ref{eqn: anue}) for \(q=+\). 
                
                Using analogous arguments we can show  (\ref{eqn: anue}) for \(q=-\).
\end{proof}

Next, we proceed to show results needed to prove convergence on the event that there are one or more changes of phase from \(\calS_+\) to \(\calS_-\) or \(\calS_-\) to \(\calS_+\) before the first change of level. The expressions arising from the QBD-RAP which we wish to show converge have the form 
\begin{align}
	\Bigg| \int_{x_1=0}^\infty g_1(x_1) \bs k(x_0) e^{\bs{S}x_1}\wrt x_1\bs D 
			\left[\prod_{k=2}^{n-1}\int_{x_k=0}^\infty g_k(x_k) e^{\bs{S}x_k} \wrt x_k \bs D\right] \int_{x_n=0}^\infty g_{n}(x_n) e^{\bs{S}x_n} \wrt x_n {\bs v}(x), \label{eqn: salkdjgaf} 
%
% 	\\&{}- \int_{u_1=0}^{\Delta-x_0}g_1(\Delta - u_1 - x_0)
% %		\int_{u_2=0}^{\Delta-u_1}g_2(\Delta - u_2 - u_1)\wrt u_1  \nonumber 
% 	\left[\prod_{k=2}^{n-1} \int_{u_k=0}^{\Delta-u_{k-1}} g_k(\Delta-u_k-u_{k-1})\wrt u_{k-1}\right] \nonumber 
% 	%\\&{}\nonumber
% 			%\int_{u_{n-1}=0}^{\Delta-u_{n-2}} g_{n-1}(\Delta - u_{n-1} - u_{n-2}) \wrt u_{n-2}
% 			g_{n}(\Delta - x-u_{n-1})
% 	\\&\qquad{} 1(\Delta-x-u_{n-1}\geq0) \wrt u_{n-1} \Bigg|
\end{align}
where \(n\geq 2\), \(\bs v(x)\) is a closing operator with the Properties~\ref{properties: some props} and \(\{g_k\}\) are functions satisfying Assumptions~\ref{asu: g}. Define \(w_n(x_0,x)\) to be the expression (\ref{eqn: salkdjgaf}).

Define the column vectors 
\begin{align}
	\mathcal I_{m,k}(u_k) = \left[\prod_{\ell=m}^{k-1}\int_{x_\ell=0}^\infty g_\ell(x_\ell) e^{\bs{S}x_\ell}\wrt x_\ell \bs{D} \right]
%            	\int_{x_{m+1}=0}^\infty g_{m+1}(x_{m+1}) e^{\bs{S}x_{m+1}} \wrt x_{m+1} \bs{D} \nonumber
%            	\dots 
            	\int_{x_k=0}^\infty g_{k}(x_k) e^{\bs{S}x_k} \wrt x_k e^{\bs{S}u_k}\bs s
\end{align}
for \(m,k\in\{1,2,\dots\}\), \(m\leq k\), where a product over an empty set is equal to 1.
And define the row vectors 
\begin{align}
	\mathcal J_{k+1,k+1}(u_k,x_{k+1}) &:= g_{k+1}(x_{k+1})\cfrac{\bs \alpha e^{\bs{S}u_{k}}}{\bs \alpha e^{\bs{S}u_{k}}\bs e}e^{\bs{S}x_{k+1}} 
	\intertext{and}
	\mathcal J_{k+1,n}(u_k,x_{k+1}) &:= g_{k+1}(x_{k+1})\cfrac{\bs \alpha e^{\bs{S}u_{k}}}{\bs \alpha e^{\bs{S}u_{k}}\bs e} e^{\bs{S}x_{k+1}} \bs{D} \left[\prod_{m=k+2}^{n-1}\int_{x_{m}=0}^\infty g_{m}(x_{m}) e^{\bs{S}x_{m}} \wrt x_{m} \bs{D} \right]\nonumber
%		\\&\quad\hdots 
%		\int_{x_{n-1}=0}^\infty g_{n-1}(x_{n-1}) e^{\bs{S}x_{n-1}} \wrt x_{n-1}  
            	\\&\qquad\times\int_{x_n=0}^\infty g_{n}(x_n) e^{\bs{S}x_n} \wrt x_n
\end{align}
for \(k,n\in\{0,1,2,\dots\}\), \(k+1<n\). Also define \(\displaystyle\bs D(b) = \int_{u=0}^be^{\bs Su}\bs s \cfrac{\bs\alpha e^{\bs S u}}{\bs \alpha e^{\bs S u}\bs e} \wrt u.\)

We prove that (\ref{eqn: salkdjgaf}) converges by writing it as
\begin{align}
	&\int_{x_1=0}^\infty g_1(x_1) \bs k(x_0)e^{\bs{S}x_1}\wrt x_1\bs D(\Delta-\varepsilon)
			\left[\prod_{k=2}^{n-1}\int_{x_k=0}^\infty g_k(x_k) e^{\bs{S}x_k} \wrt x_k \bs D(\Delta-\varepsilon)\right] \nonumber 
			\\&\quad \times\int_{x_n=0}^\infty g_{n}(x_n) e^{\bs{S}x_n} \wrt x_n {\bs v}(x)  
%
+\sum_{k=1}^{n-1} \int_{x_{k+1}=0}^\infty \int_{u_k=\Delta-\varepsilon}^\infty \bs k(x_0)\mathcal I_{1,k}(u_k) \mathcal J_{k+1,n}(u_k,x_{k+1}){\bs v}(x). \label{eqn: kfvKJBawXMN0}
\end{align}
We then show that each of the terms in the last summation in (\ref{eqn: kfvKJBawXMN0}) is bounded by something which can be made arbitrarily small upon choosing the variance of the distribution \((\bs \alpha, \bs S)\) to be sufficiently small. Then we show that the difference between the first term in (\ref{eqn: kfvKJBawXMN0}) and the corresponding expression for the fluid queue is also bounded by something which can be made arbitratily small. The decomposition in (\ref{eqn: kfvKJBawXMN0}) is advantagous as in the first term, the matrices \(\bs D\) are the integrals \(\displaystyle \int_{u=0}^{\Delta-\varepsilon}e^{\bs Su}\bs s \cfrac{\bs\alpha e^{\bs S u}}{\bs \alpha e^{\bs S u}\bs e} \wrt u,\) so the variable of integration never exceeds \(\Delta-\varepsilon\). As a result, we can use Chebyshev's inequality to bound the denominator in the integrand near \(1\). 

Our next result shows a bound for the terms in the last summation in (\ref{eqn: kfvKJBawXMN0}). 

Recall the row vector function \(\bs k(x): [0,\infty)\to \mathcal A \subset \mathbb R^p\),
\[\bs k(x) = \cfrac{\bs \alpha e^{\bs Sx}}{\bs \alpha e^{\bs Sx}\bs e}.\]


\begin{cor}\label{cor: lh and rh}
	Let \(g_1, g_2, \dots,\) be functions satisfying the Assumptions \ref{asu: g} and let \(\bs v(x)\) be a closing operator with the Properties \ref{properties: some props}, then, for \(k,n \in \{1,2,\dots\}\), \(k+1\leq n\),
	\begin{align}
		&\int_{x_{k+1}=0}^\infty \int_{u_k=\Delta-\varepsilon}^\infty \bs k(x_0)\mathcal I_{1,k}(u_k) \mathcal J_{k+1,n}(u_k,x_{k+1})\bs v(x) \nonumber
		%
            	\\&\leq \cfrac{1}{\bs \alpha e^{\bs{S}x_0}\bs e}\left(\left(2\varepsilon + \cfrac{\var(Z)}{\varepsilon}\right) G^2 \widehat G^{n-2} G_{\bs v} + G\widehat G^{n}\widetilde G_{\bs v}\right)  =: |r_4(n)|. \label{eqn: the result tail}
	\end{align}
\end{cor}
The structure of the proof is as follows. First, we decompose the left-hand side of (\ref{eqn: the result tail}) into 
\begin{align}
	&\int_{x_{k+1}=0}^\infty \int_{u_k=\Delta-\varepsilon}^\infty \bs k(x_0)\mathcal I_{1,k}(u_k) \mathcal J_{k+1,n}(u_k,x_{k+1})\bs w(x) \nonumber 
	\\&{}+ \int_{x_{k+1}=0}^\infty \int_{u_k=\Delta-\varepsilon}^\infty \bs k(x_0)\mathcal I_{1,k}(u_k) \mathcal J_{k+1,n}(u_k,x_{k+1})\widetilde{\bs w}(x). \label{eqn: JABHwj2}
\end{align} 
Next, we bound \(\bs k(x_0)\mathcal I_{1,k}(u_k) \), then we bound \(\mathcal J_{k+1,n}(u_k,x_{k+1})  {\bs w}(x)\). With these two bounds we can derive a bound for the first term in (\ref{eqn: JABHwj2}). A bound on the second term of (\ref{eqn: JABHwj2}) follows from the bound on \(\bs k(x_0)\mathcal I_{1,n-1}(u_{n-1})\) along with Properties~\ref{properties: -1} and \ref{properties: 0} of \(\widetilde{\bs w}\).

\begin{proof}
	\emph{Step 1: Decompose the left-hand side of (\ref{eqn: the result tail}) as (\ref{eqn: JABHwj2}).}
		Referring to the Properties~\ref{properties: some props}, we can decompose the closing operator \(\bs v(x)=\bs w(x) + \widetilde{\bs w}(x)\), and therefore, due to the linearity of the decomposition, we can decompose (\ref{eqn: the result tail}) as (\ref{eqn: JABHwj2}).

	\emph{Step 2: Show the following bound.}
	\begin{align}
		\bs k(x_0)\mathcal I_{1,k}(u_k) 
            	&\leq \cfrac{1}{\bs \alpha e^{\bs{S}x_0}\bs e}G\widehat G^{k-1} \bs \alpha e^{\bs{S}u_k}\bs e.\label{eqn: in here}
	\end{align}
	Recall the definition of \(\bs{D}:=\displaystyle\int_{u=0}^\infty e^{\bs{S}u}\bs s \cfrac{\bs \alpha e^{\bs{S}u}}{\bs \alpha e^{\bs{S}u}\bs e}\wrt u\) and substitute it into the left-hand side of (\ref{eqn: in here}), 
	\begin{align}
		\bs k(x_0) \mathcal I_{1,k}(u_k) &=\bs k(x_0) \int_{x_1=0}^\infty g_1(x_1) e^{\bs{S}x_1} \bs{D} \mathcal I_{2,k}(u_k)\nonumber 
		\\&=\bs k(x_0)\int_{x_1=0}^\infty g_1(x_1) e^{\bs{S}x_1} \int_{u_1=0}^\infty e^{\bs{S}u_1}\bs s \cfrac{\bs \alpha e^{\bs{S}u_1}}{\bs \alpha e^{\bs{S}u_1}\bs e}\wrt u_1 \mathcal I_{2,k}(u_k). \label{eqn: vajJJ8933}
	\end{align}
	Since \(|g_1|\leq G\), then (\ref{eqn: vajJJ8933}) is less than or equal to
	\begin{align}
		&\bs k(x_0) \int_{x_1=0}^\infty G  e^{\bs{S}x_1} \int_{u_1=0}^\infty e^{\bs{S}u_1}\bs s \cfrac{\bs \alpha e^{\bs{S}u_1}}{\bs \alpha e^{\bs{S}u_1}\bs e}\wrt u_1 \mathcal I_{2,k}(u_k).\label{eqn: int this}
	\end{align}
	Computing the integral with respect to \(x_1\) in (\ref{eqn: int this}) gives 
	\begin{align}
		 &G  \bs k(x_0)(-\bs{S})^{-1} \int_{u_1=0}^\infty e^{\bs{S}u_1}\bs s \cfrac{\bs \alpha e^{\bs{S}u_1}}{\bs \alpha e^{\bs{S}u_1}\bs e}\wrt u_1 \mathcal I_{2,k}(u_k)  \nonumber
		%
		% \\&= G \bs k(x_0)\int_{u_1=0}^\infty e^{\bs{S}u_1}\bs e \cfrac{\bs \alpha e^{\bs{S}u_1}}{\bs \alpha e^{\bs{S}u_1}\bs e}\wrt u_1\mathcal I_{2,k}(u_k) \nonumber 
		% %
		\\&=  \cfrac{G}{\bs \alpha e^{\bs{S}x_0}\bs e} \int_{u_1=0}^\infty \bs \alpha e^{\bs{S}(x_0+u_1)}\bs e \cfrac{\bs \alpha e^{\bs{S}u_1}}{\bs \alpha e^{\bs{S}u_1}\bs e}\wrt u_1\mathcal I_{2,k}(u_k), \label{eqn: yet another label}
	\end{align}
	since \((-\bs{S})^{-1}\) and \(e^{\bs{S}t}\) commute, \(\bs s = - \bs{S} \bs e \) and \(e^{\bs{S}(t+u)} = e^{\bs{S}t}e^{\bs{S}u}\). 
	Since \( \bs \alpha e^{\bs{S}(x_0 +u_1)}\bs e \leq \bs \alpha e^{\bs{S}u_1}\bs e \), then (\ref{eqn: yet another label}) is less than or equal to 
	\begin{align}
		&G  \cfrac{1}{\bs \alpha e^{\bs{S}x_0}\bs e} \int_{u_1=0}^\infty \bs \alpha e^{\bs{S}u_1}\bs e \cfrac{\bs \alpha e^{\bs{S}u_1}}{\bs \alpha e^{\bs{S}u_1}\bs e}\wrt u_1 \mathcal I_{2,k}(u_k) \nonumber
		=G  \cfrac{1}{\bs \alpha e^{\bs{S} x_0 }\bs e} \int_{u_1=0}^\infty \bs \alpha e^{\bs{S}u_1}\wrt u_1 \mathcal I_{2,k}(u_k) . 
	\end{align}
	Now integrate with respect to \(u_1\) and use the facts that \((-\bs{S})^{-1}\) and \(e^{\bs{S}x}\) commute, and \(\bs s = - \bs{S} \bs e \), to get 
	\begin{align}
		& G  \cfrac{1}{\bs \alpha e^{\bs{S}x_0}\bs e} \bs \alpha (-\bs{S})^{-1}  \mathcal I_{2,k}(u_k) \label{eqn: rep from here}
		\\& = G  \cfrac{1}{\bs \alpha e^{\bs{S}x_0}\bs e}  \bs \alpha (-\bs{S})^{-1} \int_{x_2=0}^\infty g_2(x_2)  e^{\bs{S}x_2} \wrt x_2 \int_{u_2=0}^\infty e^{\bs{S}u_2}\bs s \cfrac{\bs \alpha e^{\bs{S}u_2}}{\bs \alpha e^{\bs{S}u_2}\bs e}\wrt u_2 \mathcal I_{3,k}(u_k)\nonumber 
		\\& = G  \cfrac{1}{\bs \alpha e^{\bs{S}x_0}\bs e}  \int_{x_2=0}^\infty g_2(x_2) \bs \alpha e^{\bs{S}x_2} \wrt x_2 \int_{u_2=0} ^\infty e^{\bs{S}u_2}\bs e \cfrac{\bs \alpha e^{\bs{S}u_2}}{\bs \alpha e^{\bs{S}u_2}\bs e}\wrt u_2 \mathcal I_{3,k}(u_k)\label{eqn: anoth ref here}
	\end{align}
	Since \(\bs \alpha e^{\bs{S}x_2}e^{\bs{S}u_2}\bs e \leq \bs \alpha e^{\bs{S}u_2}\bs e \), and \(\displaystyle \int_{x_2=0}^\infty g_2(x_2)\wrt x_2\leq \widehat G,\) then (\ref{eqn: anoth ref here}) is less than or equal to 
	\begin{align}
		& G  \cfrac{1}{\bs \alpha e^{\bs{S}x_0}\bs e}  \int_{x_2=0}^\infty g_2(x_2) \wrt x_2 \int_{u_2=0}^\infty \bs \alpha e^{\bs{S}u_2}\bs e \cfrac{\bs \alpha e^{\bs{S}u_2}}{\bs \alpha e^{\bs{S}u_2}\bs e}\wrt u_2 \mathcal I_{3,k}(u_k) \nonumber
		\\& \leq G  \cfrac{1}{\bs \alpha e^{\bs{S} x_0 }\bs e}  \widehat G  \int_{u_2=0}^\infty \bs \alpha e^{\bs{S}u_2} \mathcal I_{3,k}(u_k) \nonumber
		\\& =G  \cfrac{1}{\bs \alpha e^{\bs{S} x_0 }\bs e}  \widehat G \bs \alpha (-\bs S)^{-1} \mathcal I_{3,k}(u_k).  \label{eqn: rep to here}
	\end{align}
	Repeating the arguments which got us from (\ref{eqn: rep from here}) to (\ref{eqn: rep to here}) another \(k-2\) times gives the result.

\emph{Step 3: Show the bound}
	\begin{align}
            	\mathcal J_{k+1,n}(u_k,x_{k+1})  {\bs w}(x) \leq  g_{k+1}(x_{k+1})\widehat G^{n-k-2} G G_{\bs v}. \label{eqn: J bound}
	\end{align}
	Starting with the left-hand side, upon substituting \(\bs{D}\), 
	\begin{align}
		& \mathcal J_{k+1,n}(u_k,x_{k+1})  {\bs w}(x)  \nonumber 
		\\&= \mathcal J_{k+1,n-1}(u_k,x_{k+1})  \bs{D}
		\int_{x_n=0}^\infty g_{n}(x_n) e^{\bs{S}x_n} \wrt x_n{\bs w}(x) \nonumber
		\\&= \mathcal J_{k+1,n-1}(u_k,x_{k+1})  \int_{u_{n-1}=0}^\infty e^{\bs{S}u_{n-1}}\bs s \cfrac{\bs \alpha e^{\bs{S}u_{n-1}}}{\bs \alpha e^{\bs{S}u_{n-1}}\bs e}\wrt  u_{n-1}
		\int_{x_n=0}^\infty g_{n}(x_n) e^{\bs{S}x_n} \wrt x_n{\bs w}(x) \nonumber
		\\&\leq \mathcal J_{k+1,n-1}(u_k,x_{k+1})  \int_{u_{n-1}=0}^\infty e^{\bs{S}u_{n-1}}\bs s \cfrac{\bs \alpha e^{\bs{S}u_{n-1}}}{\bs \alpha e^{\bs{S}u_{n-1}}\bs e}\wrt  u_{n-1}
		\int_{x_n=0}^\infty G e^{\bs{S}x_n} \wrt x_n{\bs w}(x), \label{eqn: bnd again}
	\end{align}
	since \(|g_n|\leq G\). 
	By Property \ref{properties: 1} of \({\bs w}(x)\), \(\displaystyle\int_{x_n=0}^\infty \bs \alpha e^{\bs{S}u_{n-1}}e^{\bs{S}x_n} {\bs w}(x) \wrt x_n  \leq \bs \alpha e^{\bs{S}u_{n-1}}\bs eG_{\bs v}\). Therefore (\ref{eqn: bnd again}) is less than or equal to 
	\begin{align}
		&\mathcal J_{k+1,n-1}(u_k,x_{k+1})  \int_{u_{n-1}=0}^\infty e^{\bs{S}u_{n-1}}\bs s \cfrac{\bs \alpha e^{\bs{S}u_{n-1}}\bs e}{\bs \alpha e^{\bs{S}u_{n-1}}\bs e}\wrt  u_{n-1} G  G_{\bs v}\nonumber 
		%
		\\& = \mathcal J_{k+1,n-1}(u_k,x_{k+1})  \int_{u_{n-1}=0}^\infty e^{\bs{S}u_{n-1}}\bs s \wrt  u_{n-1} G  G_{\bs v}\nonumber
		%
		\\& = \mathcal J_{k+1,n-1}(u_k,x_{k+1})  \bs e G  G_{\bs v}\label{eqn: mid ref} 
		%
		\\& = \mathcal J_{k+1,n-2}(u_k,x_{k+1}) \int_{u_{n-2}=0}^\infty e^{\bs{S}u_{n-2}}\bs s \cfrac{\bs \alpha e^{\bs{S}u_{n-2}}}{\bs \alpha e^{\bs{S}u_{n-2}}\bs e}\wrt u_{n-2}  \int_{x_{n-1}=0}^\infty g_{n-1}(x_{n-1}) e^{\bs{S}x_{n-1}} \wrt x_{n-1} \bs e \nonumber 
		\\&\qquad {} \times G  G_{\bs v}.\label{eqn: this}
	\end{align}
	Since \(\bs\alpha e^{\bs{S}(x_{n-1}+u_{n-2})}\bs e\leq  \bs\alpha e^{\bs{S}(u_{n-2})}\bs e\) and \(\displaystyle \int_{x_{n-1}=0}^\infty g_{x_{n-1}} \wrt x_{n-1}\leq \widehat G\), then (\ref{eqn: this}) is less than or equal to
	\begin{align}
		% &\mathcal J_{k+1,n-2}(u_k,x_{k+1}) \int_{u_{n-2}=0}^\infty e^{\bs{S}u_{n-2}}\bs s \cfrac{\bs \alpha e^{\bs{S}u_{n-2}}\bs e}{\bs \alpha e^{\bs{S}u_{n-2}}\bs e}\wrt u_{n-2}  \int_{x_{n-1}=0}^\infty g_{n-1}(x_{n-1})\wrt x_{n-1} G  G_{\bs v}\nonumber
		%\\& = 
		\mathcal J_{k+1,n-2}(u_k,x_{k+1}) \int_{u_{n-2}=0}^\infty e^{\bs{S}u_{n-2}}\bs s \wrt u_{n-2} \widehat G G G_{\bs v}
		%
		&= \mathcal J_{k+1,n-2}(u_k,x_{k+1}) \bs e \widehat G G G_{\bs v}. \label{eqn: ref here too}
	\end{align} 
	This is of the same form as (\ref{eqn: mid ref}), hence repeating the same arguments which got us from (\ref{eqn: mid ref}) to (\ref{eqn: ref here too}) another \(n-k-3\) more times gives
	 \begin{align*}
		\mathcal J_{k+1,k+1}(u_k,x_{k+1}) \bs e  \widehat G^{n-k-2}G G_{\bs v}
		%
		&= g_{k+1}(x_{k+1}) \cfrac{\bs\alpha e^{\bs{S}(u_k+x_{k+1})}}{\bs \alpha e^{\bs{S}u_k}\bs e} \bs e\widehat G^{n-k-2}G G_{\bs v}
		%
		\\& \leq g_{k+1}(x_{k+1}) \widehat G^{n-k-2}G G_{\bs v}.
	\end{align*} 

\emph{Step 4: Combine the bounds on \(\bs k(x_0)\mathcal I_{1,k}(u_k) \) and \(\mathcal J_{k+1,n}(u_k,x_{k+1})  {\bs w}(x)\) to bound the first term in (\ref{eqn: JABHwj2}).}	

		With the bounds (\ref{eqn: in here}) and (\ref{eqn: J bound}), the first term of (\ref{eqn: JABHwj2}) is less than or equal to 
		\begin{align}
			&\cfrac{1}{\bs \alpha e^{\bs{S} x_0 }\bs e}G \widehat G^{k-1}
			\int_{x_{k+1}=0}^\infty \int_{u_k=\Delta-\varepsilon}^\infty \bs \alpha e^{\bs{S}u_k}\bs e g_{k+1}(x_{k+1}) \wrt u_k \wrt x_{k+1}\widehat G^{n-k-2} G G_{\bs v}\nonumber 
			%
			\\&\leq \cfrac{1}{\bs \alpha e^{\bs{S} x_0}\bs e}G \widehat G^{k-1}  
			\int_{u_k=\Delta-\varepsilon}^\infty \bs \alpha e^{\bs{S}u_k}\bs e \wrt u_k \widehat G \widehat G^{n-k-2} G G_{\bs v}. \label{eqnL afejhm789}
		\end{align}
		Now, observe that 
		\begin{align}
			\int_{u_k=\Delta-\varepsilon}^\infty \bs \alpha e^{\bs{S}u_k}\bs e \wrt u_k &= \int_{u_k=\Delta-\varepsilon}^{\Delta+\varepsilon} \mathbb P(Z> u_k) \wrt u_k + \int_{u_k=\Delta+\varepsilon}^\infty \mathbb P(Z> u_k) \wrt u_k\nonumber
			%
			\\&\leq \int_{u_k=\Delta-\varepsilon}^{\Delta+\varepsilon} \wrt u_k + \int_{u_k=\Delta+\varepsilon}^\infty \cfrac{\var(Z)}{(u_k-\Delta)^2} \wrt u_k\nonumber
			% 
			\\&= 2\varepsilon + \cfrac{\var(Z)}{\varepsilon},\label{eqn:kdjf55}
		\end{align}
		where we have used Chebyshev's inequality to bound the tail probability, 
		\[\mathbb P(Z> u_k) \leq \mathbb P(|Z-\Delta|> |u_k-\Delta|) \leq \cfrac{\var(Z)}{(u_k-\Delta)^2},\]
		for \(u_k \geq \Delta +\varepsilon\). Hence (\ref{eqnL afejhm789}) is less than or equal to 
		\[\cfrac{1}{\bs \alpha e^{\bs{S} x_0}\bs e}G \widehat G^{k-1}  
			\left(2\varepsilon + \cfrac{\var(Z)}{\varepsilon}\right) \widehat G^{n-k-1} G G_{\bs v}.\]
		
		Now consider \(k+1=n\). By the bound (\ref{eqn: in here}), the first term of (\ref{eqn: JABHwj2}) is less than or equal to 
		\begin{align}
			&\cfrac{1}{\bs \alpha e^{\bs{S} x_0 }\bs e}G \widehat G^{k-1}
			\int_{x_{k+1}=0}^\infty \int_{u_k=\Delta-\varepsilon}^\infty \bs \alpha e^{\bs{S}u_k}\bs e g_{k+1}(x_{k+1}) \cfrac{\bs\alpha e^{\bs{S}(u_k+x_{k+1})}}{\bs \alpha e^{\bs{S}u_k}\bs e}\bs v(x)\wrt u_k \wrt x_{k+1}. \label{eqn: yet yet another label 2}
			%
			\end{align}
			{Since \(g_{k+1}\leq G\), and upon integrating over \(x_{k+1}\), then (\ref{eqn: yet yet another label 2}) is less than or equal to }
			\begin{align}
			 \cfrac{1}{\bs \alpha e^{\bs{S} x_0 }\bs e}G^2\widehat G^{k-1}  
			\int_{u_k=\Delta-\varepsilon}^\infty \bs\alpha e^{\bs{S}u_k}(-\bs S)^{-1}{\bs v}(x) \wrt u_k 
			%
			\leq \cfrac{1}{\bs \alpha e^{\bs{S} x_0 }\bs e}G^2 \widehat G^{k-1}  
			\int_{u_k=\Delta-\varepsilon}^\infty  \bs \alpha e^{\bs S u_k} \bs e G_{\bs v} \wrt u_k , \label{eqn: aksgm}
		\end{align}
		where we have used Property \ref{properties: 1} to get the upper bound on the right-hand side of (\ref{eqn: aksgm}). Using (\ref{eqn:kdjf55}) again, then (\ref{eqn: aksgm}) is less than or equal to
		\begin{align}
			\cfrac{1}{\bs \alpha e^{\bs{S} x_0 }\bs e}G\widehat G^{n-2}   G G_{\bs v}\left(2\varepsilon + \cfrac{\var\left(Z\right)}{\varepsilon}\right).
		\end{align}
		Thus, we have show the desired bound. 





		% Consider first \(k+1<n\). The second term of (\ref{eqn: JABHwj}) is similar to that appearing in Corollary \ref{cor: ksjkd}, except that here the integral over \(u_k\) is over \((\Delta-\varepsilon,\infty)\), where as in Corollary \ref{cor: ksjkd} the corresponding integral os over \((0,\infty)\). Hence, since all factors are non-negative, the second term is bounded by \(G\widehat G^{n}O(\var(Z))/\bs \alpha e^{\bs{S}x_0}\bs e.\)

\emph{Step 5: Bound the second term in (\ref{eqn: JABHwj2}).}

To bound the second term in (\ref{eqn: JABHwj2}) we instead bound 
	\begin{align}
		&\int_{x_1=0}^\infty g_1(x_1) \bs k(x_0) e^{\bs{S}x_1}\wrt x_1\bs D 
            	\left[\prod_{k=2}^{n-1}\int_{x_k=0}^\infty g_k(x_k) e^{\bs{S}x_k} \wrt x_k \bs D\right] \int_{x_n=0}^\infty g_{n}(x_n) e^{\bs{S}x_n} \wrt x_n \widetilde{\bs w}(x), 
		% \\&= O(\var(Z)) 
		\label{eqn :mmmm2}
	\end{align}
	which is the same as the second term in (\ref{eqn: JABHwj2}) except that in (\ref{eqn :mmmm2}) the integral is over a larger interval. Replacing the last \(\bs D\) matrix in (\ref{eqn :mmmm2}) by its integral definition, gives 
	\begin{align*}
		&\int_{x_1=0}^\infty g_1(x_1) \bs k(x_0) e^{\bs{S}x_1}\wrt x_1
            \left[\prod_{k=2}^{n-1}\bs D \int_{x_k=0}^\infty g_k(x_k) e^{\bs{S}x_k} \wrt x_k \right] \int_{u=0}^\infty e^{\bs S u}\bs s \cfrac{\bs \alpha e^{\bs S u}}{\bs \alpha e^{\bs S u} \bs e}\wrt u 
			\\&\qquad{}\int_{x_n=0}^\infty g_{n}(x_n) e^{\bs{S}x_n} \wrt x_n \widetilde{\bs w}(x) 
		\\&=\bs k(x_0) \int_{u=0}^\infty \mathcal I_{1,n}(u) \cfrac{\bs \alpha e^{\bs S u}}{\bs \alpha e^{\bs S u} \bs e}\wrt u \int_{x_n=0}^\infty g_{n}(x_n) e^{\bs{S}x_n} \wrt x_n \widetilde{\bs w}(x) 
		\\&\leq \cfrac{1}{\bs \alpha e^{\bs{S}x_0}\bs e}G\widehat G^{n-1} \int_{u=0}^\infty \bs \alpha e^{\bs{S}u}\bs e \cfrac{\bs \alpha e^{\bs S u}}{\bs \alpha e^{\bs S u} \bs e}\wrt u \int_{x_n=0}^\infty g_{n}(x_n) e^{\bs{S}x_n} \wrt x_n \widetilde{\bs w}(x) 
	\end{align*}
	by the bound in (\ref{eqn: in here}). Integrating over \(u\) gives 
	\begin{align*}
		&\cfrac{1}{\bs \alpha e^{\bs{S}x_0}\bs e}G\widehat G^{n-1} \bs \alpha (-\bs S)^{-1} \int_{x_n=0}^\infty g_{n}(x_n) e^{\bs{S}x_n} \wrt x_n \widetilde{\bs w}(x) 
		\\&\leq \cfrac{1}{\bs \alpha e^{\bs{S}x_0}\bs e}G\widehat G^{n-1} \bs \alpha (-\bs S)^{-1} \int_{x_n=0}^\infty g_{n}(x_n) \wrt x_n \widetilde{\bs w}(x),
	\end{align*}
	by Property \ref{properties: -1}. Integrating over \(x_n\), gives
	\begin{align}
		\cfrac{1}{\bs \alpha e^{\bs{S}x_0}\bs e}G\widehat G^{n} \bs \alpha (-\bs S)^{-1} \widetilde{\bs w}(x) 
		&=\cfrac{1}{\bs \alpha e^{\bs{S}x_0}\bs e}G\widehat G^{n}\widetilde G_{\bs v},\label{eqn:FGHJSjjs sj}
	\end{align}
	by Property \ref{properties: 0}.

	Combining all the bounds proves the result. 
\end{proof}	
% \begin{cor}\label{cor: ksjkd}
% 	Let \(g_1, g_2, \dots,\) be functions satisfying the Assumptions \ref{asu: g} and let \(\bs v(x)\) be a closing operator with the Properties \ref{properties: some props}, then,
% 	\begin{align}
% 		&\int_{x_1=0}^\infty g_1(x_1) \bs k(x_0) e^{\bs{S}x_1}\wrt x_1\bs D 
%             	\left[\prod_{k=2}^{n-1}\int_{x_k=0}^\infty g_k(x_k) e^{\bs{S}x_k} \wrt x_k \bs D\right] \int_{x_n=0}^\infty g_{n}(x_n) e^{\bs{S}x_n} \wrt x_n \bs v(x) \nonumber 
% 		\\&\leq \cfrac{1}{\bs \alpha e^{\bs{S}x_0}\bs e} \widehat{G}^{n-1}G(G_{\bs v}+\widehat G O(\var(Z)) \label{eqn :mmmm}
% 	\end{align}
% \end{cor}
% \begin{proof}
% 	By Corollary \ref{cor: amammme} 
% 	\begin{align}
% 		&\int_{x_1=0}^\infty g_1(x_1) \bs k(x_0) e^{\bs{S}x_1}\wrt x_1\bs D 
%             	\left[\prod_{k=2}^{n-1}\int_{x_k=0}^\infty g_k(x_k) e^{\bs{S}x_k} \wrt x_k \bs D\right] \int_{x_n=0}^\infty g_{n}(x_n) e^{\bs{S}x_n} \wrt x_n \bs v(x)\nonumber
% 		\\&\leq \int_{x_1=0}^\infty g_1(x_1) \bs k(x_0) e^{\bs{S}x_1}\wrt x_1\bs D 
% 		\left[\prod_{k=2}^{n-1}\int_{x_k=0}^\infty g_k(x_k) e^{\bs{S}x_k} \wrt x_k \bs D\right] \int_{x_n=0}^\infty g_{n}(x_n) e^{\bs{S}x_n} \wrt x_n\bs w(x) \nonumber
% 		\\&\quad{} + O(\var(Z)). \label{eqn: LLaaNNab}
% 	\end{align}
% 	The first term of (\ref{eqn: LLaaNNab}) can be seen to be equivalent to \(\mathcal J_{1,n+1}(x_0,x_1),\) with \(g_1(x_1)=1\), and the integrability condition on \(g_1\) is not required to prove the bound. 
% \end{proof}

Next we wish to prove a bound on the difference between the first term in (\ref{eqn: kfvKJBawXMN0}) and \(g^*_{1,n}(x_0,x)\), where we define 
	\begin{align}
		g^*_{2,n}(u_1,x) &:= \int_{u_2=0}^{\Delta-u_1}g_2(\Delta - u_2 - u_1)\wrt u_1 \dots \nonumber 
            	\int_{u_{n-1}=0}^{\Delta-u_{n-2}} g_{n-1}(\Delta - u_{n-1} - u_{n-2}) \wrt u_{n-2}
            	\\&\qquad{}g_{n}(\Delta - x-u_{n-1})1(\Delta-x-u_{n-1}\geq0)\wrt u_{n-1},
		\intertext{and}%
		g^*_{1,n}(x_0,x) &:= \int_{u_1=0}^{\Delta-x_0}g_1(\Delta - u_1 - x_0)g^*_{2,n}(u_1,x)\wrt u_1.
	\end{align}
The idea of the proof is to first show a bound for the difference between the first term in (\ref{eqn: kfvKJBawXMN0}) and the expression \(g^{*,\varepsilon}_{1,n}(x_0,x)\) given by
	\begin{align}
		% g^{*,\varepsilon}_{1,n}(x_0,x) &:= 
		&\int_{u_1=0}^{\Delta-\varepsilon-x_0}g_1(\Delta - u_1 - x_0)
		\int_{u_2=0}^{\Delta-\varepsilon-u_1}g_2(\Delta - u_2 - u_1)\wrt u_1  \nonumber 
		\\&\quad\hdots 
            	\int_{u_{n-1}=0}^{\Delta-\varepsilon-u_{n-2}} g_{n-1}(\Delta - u_{n-1} - u_{n-2}) \wrt u_{n-2}
            	g_{n}(\Delta-x-u_{n-1}) 
		1(\Delta-x-u_{n-1}\geq\varepsilon).
	\end{align}
	We then establish a bound on the difference between \(g^{*,\varepsilon}_{1,n}(x_0,x)\) and \(g^{*}_{1,n}(x_0,x)\) which can also be made arbitrarily small. 

	Recall that the first term in (\ref{eqn: kfvKJBawXMN0}) looks like 
	\begin{align}
		&\int_{x_1=0}^\infty g_1(x_1) \bs k(x_0)e^{\bs{S}x_1}\wrt x_1\bs D(\Delta-\varepsilon)
				\left[\prod_{k=2}^{n-1}\int_{x_k=0}^\infty g_k(x_k) e^{\bs{S}x_k} \wrt x_k \bs D(\Delta-\varepsilon)\right] \nonumber 
				\\&\quad \times\int_{x_n=0}^\infty g_{n}(x_n) e^{\bs{S}x_n} \wrt x_n {\bs v}(x)  
		% \\&=\int_{x_1=0}^\infty g_1(x_1) \bs k(x_0)e^{\bs{S}x_1}\wrt x_1\bs D(\Delta-\varepsilon)
		% \left[\prod_{k=2}^{n-1}\int_{x_k=0}^\infty g_k(x_k) e^{\bs{S}x_k} \wrt x_k \bs D(\Delta-\varepsilon)\right] \nonumber 
	\end{align}
	which, upon substituting the definition of \(\bs D(\Delta-\varepsilon)\), can be written as 
	\begin{align}
		& \int_{u_1=0}^{\Delta-\varepsilon} \int_{x_1=0}^\infty \cfrac{\bs \alpha e^{\bs{S}(x_{0}+x_1+u_1)}\bs s}{\bs \alpha e^{\bs{S}x_0}\bs e}g_1(x_1) \wrt x_1 \nonumber 
		\left[\prod_{\ell=2}^{n-1}\int_{u_\ell=0}^{\Delta-\varepsilon} \int_{x_\ell=0}^\infty \cfrac{\bs \alpha e^{\bs{S}(u_{\ell-1}+x_\ell+u_\ell)}\bs s}{\bs \alpha e^{\bs{S}u_{\ell-1}}\bs e}g_\ell(x_\ell) \wrt x_\ell \wrt u_{\ell-1} \right]
            	\\&{}\quad\times\int_{x_n=0}^\infty \cfrac{\bs \alpha e^{\bs{S}(u_{n-1}+x_n )}}{\bs \alpha e^{\bs{S}u_{n-1}}\bs e} {\bs v}(x)g_{n}(x_n)\wrt x_n \wrt u_{n-1} .\label{eqnL akfhcka}
	\end{align}
	Appearing in (\ref{eqnL akfhcka}) are intergals of the form
	\begin{align}
		\int_{x_\ell=0}^\infty \cfrac{\bs \alpha e^{\bs{S}(u_{\ell-1}+x_\ell+u_\ell)}\bs s}{\bs \alpha e^{\bs{S}u_{\ell-1}}\bs e}g_\ell(x_\ell) \wrt x_\ell. \label{eqn: skhgintegral}
	\end{align}
	Intuitively, if the variance of \(Z\) is sufficiently small, and \(u_{\ell-1}\leq \Delta +\varepsilon\) where \(\Delta\) is the expected value of \(Z\), then the integral in (\ref{eqn: skhgintegral}) should be approximately equal to \(g_{\ell}(\Delta - u_{\ell}-u_{\ell-1})\). Our first step towards showing a bound for the difference between the first term in (\ref{eqn: kfvKJBawXMN0}) and the expression \(g^{*,\varepsilon}_{1,n}(x_0,x)\) is to prove this intuition. We start results about with a simpler integral than that in (\ref{eqn: skhgintegral}), from which the result we require follows as a Corollary. 

	% For later, observe that 
	% \begin{align}
	% 	g^*_{2,n}(u_1,x) &= \int_{u_2=0}^{\Delta-u_1}g_2(\Delta - u_2 - u_1)\wrt u_1 \dots \nonumber 
    %         	\int_{u_{n-1}=0}^{\Delta-u_{n-2}} g_{n-1}(\Delta - u_{n-1} - u_{n-2}) \wrt u_{n-2}
    %         	\\&\qquad{}g_{n}(\Delta - x-u_{n-1})1(\Delta-x-u_{n-1}\geq0)\wrt u_{n-1} \nonumber
	% %
	% 	\\&\leq G^{n-1}\int_{u_2=0}^{\Delta-u_1}\wrt u_1 \dots\nonumber
    %         	\int_{u_{n-1}=0}^{\Delta-u_{n-2}}  \wrt u_{n-1}
	% %
	% 	\\&\leq G^{n-1}\Delta^{n-2}:=G^*_n.
	% \end{align}

\begin{lem}\label{lemma:bound}
	Let \(g\) be a function satisfying Assumptions \ref{asu: g}, then, for \(u \leq \Delta - \varepsilon\), 
	\begin{align*}
		\int_{x=0}^\infty g\left(x\right)\bs \alpha e^{\bs{S}\left(x+u\right)} \bs s \wrt x = g\left(\Delta-u\right) + r_1,
	\end{align*}
	where 
	\[\left|r_1\right|\leq 2G\cfrac{\var \left(Z\right)}{\varepsilon^2} + 2L\varepsilon.\]
\end{lem}
The proof follows closely that of \cite[Appendix A, Theorem 4]{hht2020}.
\begin{proof}
	By a change of variables, 
	\begin{align*}
		\\&\left|\int_{x=0}^\infty g\left(x\right)\bs \alpha  e^{\bs{S} \left(x+u\right)} \bs s \wrt x - g\left(\Delta-u\right)\right| 
		%
		\\&= \left|\int_{x=u}^\infty g\left(x-u\right)\bs \alpha  e^{\bs{S} x} \bs s \wrt x - g\left(\Delta-u\right)\right| 
		%
		\\&= \Bigg|\int_{x=u}^\infty g\left(x-u\right)\bs \alpha  e^{\bs{S} x} \bs s \wrt x - \int_{x=u}^\infty g\left(\Delta-u\right)\bs\alpha  e^{\bs{S} x}\bs s\wrt x - g\left(\Delta-u\right)\left(1-\bs\alpha  e^{\bs{S} u}\bs e \right)\Bigg|.
		%
	\end{align*}
	{By the triangle inequality this is less than or equal to}
	\begin{align*}
		&\left|\int_{x=u}^\infty \left(g\left(x-u\right)- g\left(\Delta-u\right)\right)\bs \alpha  e^{\bs{S} x} \bs s \wrt x \right| 
		{}+ \left|g\left(\Delta-u\right)\left(1-\bs\alpha  e^{\bs{S} u}\bs e \right)\right|
		%
		\\&= \left|\int_{x=u}^\infty \left(g\left(x-u\right)- g\left(\Delta-u\right)\right)\bs \alpha  e^{\bs{S} x} \bs s \wrt x\right| 
		%
		+ \left|\int_{x=0}^u g\left(\Delta-u\right)\bs \alpha  e^{\bs{S} x} \bs s \wrt x \right| 
		%
		\\&\leq d_1 +{d_2} 
	\end{align*}
	where 
	\begin{align*}
		d_1 &= \left|\int_{x=0}^u g\left(\Delta-u\right)\bs \alpha  e^{\bs{S} x} \bs s \wrt x\right| + \left|\int_{x=u}^{\Delta-\varepsilon } \left(g\left(x-u\right)- g\left(\Delta-u\right)\right)\bs \alpha  e^{\bs{S} x} \bs s \wrt x \right| 
		\\&\quad{}+ \left|\int_{x=\Delta+\varepsilon }^{\infty} \left(g\left(x-u\right)- g\left(\Delta-u\right)\right)\bs \alpha  e^{\bs{S} t} \bs s \wrt x \right|,
	\\{d_2} &= \left|\int_{x=\Delta-\varepsilon }^{\Delta+\varepsilon } \left(g\left(t-u\right)- g\left(\Delta-u\right)\right)\bs \alpha  e^{\bs{S} x} \bs s \wrt x\right| .
	\end{align*}
	
 	Applying the triangle inequality to \(d_1\),
	\begin{align}
		d_1  &\leq \int_{x=u}^{\Delta-\varepsilon } \left|g\left(x-u\right)- g\left(\Delta-u\right)\right|\bs \alpha  e^{\bs{S} x} \bs s \wrt x
		+ \int_{x=\Delta+\varepsilon }^{\infty} \left|g\left(x-u\right)- g\left(\Delta-u\right)\right|\bs \alpha  e^{\bs{S} x} \bs s \wrt x \nonumber
		\\&\quad{}+ \left|\int_{x=0}^u g\left(\Delta-u\right)\bs \alpha  e^{\bs{S} x} \bs s \wrt x \right| .\label{eqn: kkkka}
		%
		\end{align}
		{Since \(|g\left(x\right)|\leq G\), then (\ref{eqn: kkkka}) is less than or equal to}
		\begin{align}
		& 2G\Bigg( \int_{x=u}^{\Delta-\varepsilon }\bs \alpha  e^{\bs{S} x} \bs s \wrt x
		+ \int_{x=\Delta+\varepsilon }^{\infty}\bs \alpha  e^{\bs{S} x} \bs s \wrt x
		+ \int_{x=0}^u \bs \alpha  e^{\bs{S} x} \bs s \wrt x \Bigg)
		%
		=2G\mathbb P\left(|Z -\Delta|>\varepsilon \right).
		%
		\intertext{By Chebyshev's inequality,}
		&2G\mathbb P\left(|Z -\Delta|>\varepsilon \right)\leq 2G\cfrac{\var \left(Z \right)}{\varepsilon ^2}.
		%
%		\\&= \left(2GL^2\var \left(Z_1 \right)\right)^{1/3}\Delta^2
	\end{align}
	For the term \({d_2} \) we have 
	\begin{align*}
		{d_2}  &= \left|\int_{x=\Delta-\varepsilon }^{\Delta+\varepsilon } \left(g\left(x-u\right)- g\left(\Delta-u\right)\right)\bs \alpha  e^{\bs{S} x} \bs s \wrt x\right| 
		\\&\leq \int_{x=\Delta-\varepsilon }^{\Delta+\varepsilon } \left|g\left(x-u\right)- g\left(\Delta-u\right)\right|\bs \alpha  e^{\bs{S} x} \bs s \wrt x
		\\&\leq \int_{x=\Delta-\varepsilon }^{\Delta+\varepsilon } 2L\varepsilon \bs \alpha  e^{\bs{S} x} \bs s \wrt x
		%
		\\&=2L\varepsilon \mathbb P(Z\in(\Delta-\varepsilon, \Delta+\varepsilon))
		%
		\\&\leq 2L\varepsilon ,
	\end{align*}
	where the first inequality is the triangle inequality and the second inequality is from the Lipschitz property of \(g\) in Assumption \ref{asu: lipschitz}. 
	Hence there is some \(r_1\) such that 
	\[\left|\int_{x=0}^\infty g\left(x\right)\bs \alpha  e^{\bs{S} \left(x+u\right)} \bs s \wrt x - g\left(\Delta-u\right)\right| = |r_1| \leq 2G\cfrac{\var(Z)}{\varepsilon^2} + 2 L \varepsilon,\]
	and this completes the proof. 
\end{proof}
\begin{cor}\label{cor: cond bnd 2}
	Let \(g\) be a function satisfying the Assumptions \ref{asu: g}. For \(u\leq \Delta-\varepsilon \), \(v\geq 0\), 
	\[\int_{x=0}^\infty \cfrac{\bs \alpha  e^{\bs{S} (x+u+v)} \bs s}{\bs \alpha  e^{\bs{S} u} \bs e} g(x)\wrt x = g(\Delta-u-v) 1(u+v\leq\Delta-\varepsilon) + r_3 (u+v),\]
	where 
	\[\left|r_3 (u+v)\right|\leq \begin{cases} 
		r_2  & u+v\leq \Delta-\varepsilon,\\
		G & u+v\in(\Delta-\varepsilon,\Delta+\varepsilon), \\
		G\cfrac{\var(Z)/\varepsilon^2}{1-\var(Z)/\varepsilon^2} & u+v \geq \Delta + \varepsilon.
		\end{cases}\]
\end{cor}
\begin{proof}
	First consider \(u+v \leq \Delta - \varepsilon\). Observe that Chebyshev's inequality gives
	\begin{align*}
		\bs \alpha e^{\bs Su}\bs e&=\mathbb P\left(Z >u\right) 
		\\&\geq \mathbb P\left(|Z -\Delta|\leq \varepsilon \right) 
		%
		\\&\geq 1 - \cfrac{\var\left(Z \right)}{\varepsilon ^2} 
%		%
%		\\&= 1-\Delta^2\left(\cfrac{L^2\var\left(Z_1 \right)}{4G^2}\right)^{1/3}
		%
		\\&=: 1-\delta .
	\end{align*}
	
	Now, since \(1-\delta\leq\bs \alpha e^{\bs Su}\bs e\leq 1\), then
	\begin{align*}
		\int_{x=0}^\infty \bs \alpha  e^{\bs{S} (x+u+v)} \bs s g(x)\wrt x
		\leq \int_{x=0}^\infty \cfrac{\bs \alpha  e^{\bs{S} (x+u+v)} \bs s}{\bs \alpha  e^{\bs{S} u} \bs e} g(x)\wrt x
		%
		&\leq \frac{1}{1-\delta }\int_{x=0}^\infty \bs \alpha  e^{\bs{S} (x+u+v)} \bs s g(x)\wrt x.
	\end{align*}
	By Lemma~\ref{lemma:bound}  
	\begin{align*}
		g(\Delta-u-v)+r _1
		&\leq \int_{x=0}^\infty \cfrac{\bs \alpha  e^{\bs{S} (x+u+v)} \bs s}{\bs \alpha  e^{\bs{S} u} \bs e} g(x)\wrt x
		%
		\leq \frac{g(\Delta-u-v)+r _1}{1-\delta }. 
	\end{align*}
	Multiplying by \(1-\delta \), then subtracting \(g(\Delta-u-v)\) and adding \(\displaystyle\int_{x=0}^\infty \cfrac{\bs \alpha  e^{\bs{S} (x+u+v)} \bs s}{\bs \alpha  e^{\bs{S} u} \bs e} g(x)\wrt x\delta \) gives
	\begin{align*}
		&r _1(1-\delta ) - g(\Delta-u-v)\delta +\int_{x=0}^\infty \cfrac{\bs \alpha  e^{\bs{S} (x+u+v)} \bs s}{\bs \alpha  e^{\bs{S} u} \bs e} g(x)\wrt x\delta 
		\\&\leq \int_{x=0}^\infty \cfrac{\bs \alpha  e^{\bs{S} (x+u+v)} \bs s}{\bs \alpha  e^{\bs{S} u} \bs e} g(x)\wrt x -g(\Delta-u-v)
		%
		\\&\leq r _1+\int_{x=0}^\infty \cfrac{\bs \alpha  e^{\bs{S} (x+u+v)} \bs s}{\bs \alpha  e^{\bs{S} u} \bs e} g(x)\wrt x\delta .
	\end{align*}
	The right-hand side is bounded above by 
	\begin{align*}
		r _1+\int_{x=0}^\infty \cfrac{\bs \alpha  e^{\bs{S} (x+u+v)} \bs s}{\bs \alpha  e^{\bs{S} u} \bs e} g(x)\wrt x\delta 
		%
		&\leq r _1 + G \delta .
	\end{align*}
	The left-hand side is bounded below by 
	\begin{align*}
		r_1 (1-\delta ) - g(\Delta-u-v)\delta +\int_{x=0}^\infty \cfrac{\bs \alpha  e^{\bs{S} (x+u+v)} \bs s}{\bs \alpha  e^{\bs{S} u} \bs e} g(x)\wrt x\delta 
		%
		&\geq r _1(1-\delta ) - g(\Delta-u-v)\delta .
	\end{align*}
	Therefore 
	\begin{align}
		\int_{x=0}^\infty \cfrac{\bs \alpha  e^{\bs{S} (x+u+v)} \bs s}{\bs \alpha  e^{\bs{S} u} \bs e} g(x)\wrt x  = g(\Delta-u-v) + r_2 ,
	\end{align}
	where 
	\begin{align}
		\nonumber\left|r_2 \right| 
		&\leq \max\left(r _1(1-\delta ) + g(\Delta-u-v)\delta , r _1 + G \delta \right) 
		%
		\\\nonumber&\leq  r_1 + G\delta
		%
		\\&=3G\cfrac{\var \left(Z \right)}{\varepsilon^2} + 2L\varepsilon .
	\end{align}
	as required. 
	
	For \(u+v\in (\Delta-\varepsilon, \Delta + \varepsilon)\),
	\begin{align}
		\int_{x=0}^\infty \cfrac{\bs \alpha  e^{\bs{S} (x+u+v)} \bs s}{\bs \alpha  e^{\bs{S} u} \bs e} g(x)\wrt x & \leq G \mathbb P(Z>u+v\mid Z>u) \leq G
	\end{align}
	
	For \(u+v \geq \Delta + \varepsilon\),
	\begin{align}
		\int_{x=0}^\infty \cfrac{\bs \alpha  e^{\bs{S} (x+u+v)} \bs s}{\bs \alpha  e^{\bs{S} u} \bs e} g(x)\wrt x & \leq G \cfrac{\mathbb P(Z>u+v)}{\mathbb P( Z>u)} 
		%
		 \leq G\cfrac{\var(Z)/\varepsilon^2}{1-\var(Z)/\varepsilon^2} .
	\end{align}
\end{proof}

% The error term \(r_1^{(p)}\) depends on \(p\), as it is defined by \(Z^{(p)}\) and \(\varepsilon^{(p)}\), but we have omitted the superscript \(p\) here. Note that, upon choosing \(\varepsilon=\var(Z^{(p)})^{1/3}\), the error term \(|r_1^{(p)}|\) is at most \(O\left(\var\left(Z^{(p)}\right)^{1/3}\right)\), which tends to 0 as \(p\to\infty\).

% \begin{cor}\label{cor: cond bnd}
% 	Let \(g\) be a function satisfying the Assumptions \ref{asu: g}. For \(u\leq \Delta-\varepsilon \), 
% 	\[\int_{x=0}^\infty \cfrac{\bs \alpha  e^{\bs{S} (x+u)} \bs s}{\bs \alpha  e^{\bs{S} u} \bs e} g(x)\wrt x = g(\Delta-u) + r_2 ,\]
% 	where 
% 	\[\left|r_2 \right|\leq 3G\cfrac{\var \left(Z \right)}{\varepsilon ^2} + 2L\varepsilon .\]
% \end{cor}
% \begin{proof}
% 	Observe that Chebyshev's inequality gives
% 	\begin{align*}
% 		\bs \alpha e^{\bs Su}\bs e&=\mathbb P\left(Z >u\right) 
% 		\\&\geq \mathbb P\left(|Z -\Delta|\leq \varepsilon \right) 
% 		%
% 		\\&\geq 1 - \cfrac{\var\left(Z \right)}{\varepsilon ^2} 
% %		%
% %		\\&= 1-\Delta^2\left(\cfrac{L^2\var\left(Z_1 \right)}{4G^2}\right)^{1/3}
% 		%
% 		\\&=: 1-\delta .
% 	\end{align*}
	
% 	Now, since \(1-\delta\leq\bs \alpha e^{\bs Su}\bs e\leq 1\), then
% 	\begin{align*}
% 		\int_{x=0}^\infty \bs \alpha  e^{\bs{S} (x+u)} \bs s g(x)\wrt x
% 		\leq \int_{x=0}^\infty \cfrac{\bs \alpha  e^{\bs{S} (x+u)} \bs s}{\bs \alpha  e^{\bs{S} u} \bs e} g(x)\wrt x
% 		%
% 		&\leq \frac{1}{1-\delta }\int_{x=0}^\infty \bs \alpha  e^{\bs{S} (x+u)} \bs s g(x)\wrt x.
% 	\end{align*}
% 	By Lemma~\ref{lemma:bound} we have 
% 	\begin{align*}
% 		g(\Delta-u)+r _1
% 		&\leq \int_{x=0}^\infty \cfrac{\bs \alpha  e^{\bs{S} (x+u)} \bs s}{\bs \alpha  e^{\bs{S} u} \bs e} g(x)\wrt x
% 		%
% 		\leq \frac{g(\Delta-u)+r _1}{1-\delta }. 
% 	\end{align*}
% 	Multiplying by \(1-\delta \), then subtracting \(g(\Delta-u)\) and adding \(\displaystyle\int_{x=0}^\infty \cfrac{\bs \alpha  e^{\bs{S} (x+u)} \bs s}{\bs \alpha  e^{\bs{S} u} \bs e} g(x)\wrt x\delta \) gives 
% 	\begin{align*}
% 		&r _1(1-\delta ) - g(\Delta-u)\delta +\int_{x=0}^\infty \cfrac{\bs \alpha  e^{\bs{S} (x+u)} \bs s}{\bs \alpha  e^{\bs{S} u} \bs e} g(x)\wrt x\delta 
% 		\\&\leq \int_{x=0}^\infty \cfrac{\bs \alpha  e^{\bs{S} (x+u)} \bs s}{\bs \alpha  e^{\bs{S} u} \bs e} g(x)\wrt x -g(\Delta-u)
% 		%
% 		\\&\leq r _1+\int_{x=0}^\infty \cfrac{\bs \alpha  e^{\bs{S} (x+u)} \bs s}{\bs \alpha  e^{\bs{S} u} \bs e} g(x)\wrt x\delta .
% 	\end{align*}
% 	The right-hand side is bounded above as 
% 	\begin{align*}
% 		r _1+\int_{x=0}^\infty \cfrac{\bs \alpha  e^{\bs{S} (x+u)} \bs s}{\bs \alpha  e^{\bs{S} u} \bs e} g(x)\wrt x\delta 
% 		%
% 		&\leq r _1 + G \delta .
% 	\end{align*}
% 	The left-hand side is bounded below as 
% 	\begin{align*}
% 		r_1 (1-\delta ) - g(\Delta-u)\delta +\int_{x=0}^\infty \cfrac{\bs \alpha  e^{\bs{S} (x+u)} \bs s}{\bs \alpha  e^{\bs{S} u} \bs e} g(x)\wrt x\delta 
% 		%
% 		&\geq r _1(1-\delta ) - g(\Delta-u)\delta .
% 	\end{align*}
% 	Hence, 
% 	\begin{align}
% 		\left|\int_{x=0}^\infty \cfrac{\bs \alpha  e^{\bs{S} (x+u)} \bs s}{\bs \alpha  e^{\bs{S} u} \bs e} g(x)\wrt x -g(\Delta-u)\right| \leq \max\left(r _1(1-\delta )+g(\Delta-u)\delta , r _1 + G \delta \right)
% 	\end{align}
% 	and therefore 
% 	\begin{align}
% 		\int_{x=0}^\infty \cfrac{\bs \alpha  e^{\bs{S} (x+u)} \bs s}{\bs \alpha  e^{\bs{S} u} \bs e} g(x)\wrt x  = g(\Delta-u) + r_2 ,
% 	\end{align}
% 	where 
% 	\begin{align}
% 		\nonumber\left|r_2 \right| 
% 		&\leq \max\left(r _1(1-\delta ) + g(\Delta-u)\delta , r _1 + G \delta \right) 
% 		%
% 		\\\nonumber&\leq  r_1 + G\delta%\max\left(G\cfrac{\var \left(Z \right)}{\varepsilon^2}, 3G\cfrac{\var \left(Z \right)}{\varepsilon^2} + 2L\varepsilon  \right) 
% 		%
% 		\\&=3G\cfrac{\var \left(Z \right)}{\varepsilon^2} + 2L\varepsilon .
% 	\end{align}
% 	This completes the proof. 
% \end{proof}

% The error term \(r_2^{(p)}\) also depends on \(p\), as it is defined by \(Z^{(p)}\) and \(\varepsilon^{(p)}\), but we have omitted the superscript \(p\) here. Choosing \(\varepsilon=\var(Z^{(p)})\), the error term \(|r_2^{(p)}|\) is at most \(O\left(\var\left(Z^{(p)}\right)^{1/3}\right)\), which tends to 0 as \(p\to\infty\).


 
The error term \(r_3^{(p)}\) depends on \(p\), as it is defined by \(Z^{(p)}\) and \(\varepsilon^{(p)}\), but we have omitted the superscript \(p\) here. Choosing \(\varepsilon=\var(Z^{(p)})^{1/3}\) then, outside of the vanishingly small interval \(u\in(\Delta-\varepsilon^{(p)},\Delta+\varepsilon^{(p)})\), the error term \(|r_3^{(p)}(u)|\) is bounded by \(O\left(\var\left(Z^{(p)}\right)^{1/3}\right)\), which tends to 0 as \(p\to\infty\). On \(u\in(\Delta-\varepsilon^{(p)},\Delta+\varepsilon^{(p)})\) the error term \(|r_3^{(p)}(u)|\) is bounded by a constant which does not tend to \(0\) as \(p \to \infty\). However, when we integrate a bounded function against \(r_3^{(p)}(u)\), then the resulting integral tends to \(0\), i.e.~for \(|\psi(x)|\leq F, \, M<\infty\), \(\displaystyle \int_{0}^M \psi(u) r_3^{(p)}(u)\wrt u\leq F\Delta |r_2^{(p)}| + 2GF\varepsilon^{(p)} + (M-\Delta)GF\cfrac{\var(Z^{(p)})/\left(\varepsilon^{(p)}\right)^2}{1-\var\left(Z^{(p)}\right)/\left(\varepsilon^{(p)}\right)^2}=O\left(\var\left(Z^{(p)}\right)^{1/3}\right)\to 0 \) as \(p\to\infty\). This is the context in which we we apply Corollary~\ref{cor: cond bnd 2} and thus the error bound is sufficient. See, for example, Corollary~\ref{cor: cond bnd 2 V}. 

We are now in a position to prove the desired bound on the difference between the first term in (\ref{eqn: kfvKJBawXMN0}) and \(g^*_{1,n}(x_0,x)\).
\begin{lem}\label{lem: lst convergence}
	Let \(g_1,g_2,\dots,\) be functions satisfying the Assumptions \ref{asu: g} and let \(\bs v(x)\) be a closing operator with the Properties \ref{properties: some props}. Then, for \(n\geq 2\),  
	\begin{align}
		&\int_{x_1=0}^\infty g_1(x_1) \bs k(x_0) e^{\bs{S}x_1}\wrt x_1 \bs D(\Delta-\varepsilon)
            	\left[\prod_{k=2}^{n-1}\int_{x_k=0}^\infty g_k(x_k) e^{\bs{S}x_k} \wrt x_k \right]
		\bs D(\Delta-\varepsilon) \nonumber 
		\\&\qquad\times\int_{x_n=0}^\infty g_{n}(x_n) e^{\bs{S}x_n} \wrt x_n {\bs v}(x) \nonumber 
	%
		\\& =g^{*}_{1,n}(x_0,x) + r_5(n) + r_6(n), \label{eqn: rhs g 2}
	\end{align}
	where  
	\begin{align*}
		|r_5(n)|&= O\left(\max\left\{G^{n-1}\Delta^{n-2}\left(\frac{1}{2}\Delta|r_2 |+ 2\varepsilon G 
		%
		+ \cfrac{1}{2}\Delta G\cfrac{\var(Z)/\varepsilon^2}{1-\var(Z)/\varepsilon^2}\right),
		G^{n-1}\Delta^{n-2}R_{{\bs v},1}\right\}\right),
		%
		\\|r_6(n)| &\leq \varepsilon^{n-2}G^{n-1}
	\end{align*}
\end{lem}
\begin{proof}
	Rewriting the left-hand side of (\ref{eqn: rhs g 2}) as in (\ref{eqnL akfhcka}), then we see that we can apply Corollary~\ref{cor: cond bnd 2} to all of the integrals over \(x_k,\, k=1,\dots,n-1\) and use Property \ref{properties: 2} of \({\bs v}(x)\) for the integral over \(x_n\), to get  
	\begin{align*}
		& \int_{u_1=0}^{\Delta-\varepsilon}\left[g_1(\Delta - u_1 - x_0)1(u_1 + x_0\leq \Delta - \varepsilon) + r_3 (u_1 + x_0)\right]
		\\&\quad\times\int_{u_2=0}^{\Delta-\varepsilon}\left[g_2(\Delta - u_2 - u_1)1(u_2 + u_1\leq \Delta - \varepsilon) + r_3 (u_2 + u_1)\right]\wrt u_1
		\\&\quad\hdots 
            	 \int_{u_{n-1}=0}^{\Delta-\varepsilon}  \left[g_{n-1}(\Delta - u_{n-1} - u_{n-2}) 1(u_{n-1} + u_{n-2}\leq \Delta - \varepsilon) +   r_3 (u_{n-1} + u_{n-2})\right] \wrt u_{n-2}
            	\\&\quad\times\left[g_{n}(\Delta-u_{n-1}-x)1(u_{n-1}+x\leq \Delta - \varepsilon) + r_{\bs v} (u_{n-1},x)\right]\wrt u_{n-1}
		%
		\\&=g^{*,\varepsilon}_{1,n}(x_0,x) + r_5(n)
	\end{align*}
	where \(r_5(n)\) is an error term. The leading terms of \(r_5(n)\) are of the form 
	\begin{align}
		&\int_{u_1=0}^{\Delta-\varepsilon-x_0}g_1(\Delta - u_1 - x_0)
		\int_{u_2=0}^{\Delta-\varepsilon-u_1}g_2(\Delta - u_2 - u_1)\wrt u_1 \nonumber 
		\\&\quad\hdots\int_{u_{k-1}=0}^{\Delta-\varepsilon-u_{k-2}}g_{k-1}(\Delta - u_{k-1} - u_{k-2}) \wrt u_{k-2}
		\int_{u_k=0}^{\Delta-\varepsilon}r_3(u_{k}+u_{k-1}) \wrt u_{k-1} \nonumber 
		\\&\quad\times\int_{u_{k+1}=0}^{\Delta-\varepsilon-u_{k}}g_{k+1}(\Delta - u_{k+1} - u_{k}) \wrt u_{k}
		\hdots
            	\int_{u_{n-1}=0}^{\Delta-\varepsilon-u_{n-2}} g_{n-1}(\Delta - u_{n-1} - u_{n-2}) \wrt u_{n-2} \nonumber 
            	\\&\quad\times g_{n}(\Delta-u_{n-1}-x)1(u_{n-1}+x\leq \Delta -\varepsilon)\wrt u_{n-1} \label{eqn: akfj111112}
		\intertext{or}
		&\int_{u_1=0}^{\Delta-\varepsilon-x_0}g_1(\Delta - u_1 - x_0)
		\int_{u_2=0}^{\Delta-\varepsilon-u_1}g_2(\Delta - u_2 - u_1)\wrt u_1 \nonumber 
		\\&{}\hdots
            	\int_{u_{n-1}=0}^{\Delta-\varepsilon-u_{n-2}} g_{n-1}(\Delta - u_{n-1} - u_{n-2}) \wrt u_{n-2}
            	 r_{\bs v}(u_{n-1},x)\wrt u_{n-1},\label{eqn: akfj11111}
	\end{align} 
	whichever is larger. Since \(|g_k|\leq G\), then (\ref{eqn: akfj11111}) and (\ref{eqn: akfj111112}) are less than or equal to 
	\begin{align*}
		& G^{k-1} \Delta^{k-2} \int_{u_{k-1}=0}^{\Delta-\varepsilon}
		\int_{u_k=0}^{\Delta-\varepsilon }r_3(u_{k}+u_{k-1}) \wrt u_k \wrt u_{k-1} G^{n-k}\Delta^{n-k-1},
		%
		\intertext{and}
		&G^{n-1} \Delta^{n-2} \int_{u_{n-1}=0}^{\Delta-\varepsilon}
		 r_{\bs v}(u_{n-1},x) \wrt u_{n-1},
	\end{align*} 
	respectively. 

	Recall that we have a bound on \(|r_3|\) which is piecewise constant. Breaking up the integral of \(r_3\) above into three interval over which the bound is constant and using the triangle inequality, then
	\begin{align}
		 \nonumber &\left|\int_{u_{k-1}=0}^{\Delta-\varepsilon}\int_{u_k=0}^{\Delta-\varepsilon}r_3(u_{k}+u_{k-1}) \wrt u_k \wrt u_{k-1} \right|
		\\\nonumber & \leq \int_{u_{k-1}=0}^{\Delta-\varepsilon}\Bigg[ \int_{u_k=u_{k-1}}^{\Delta-\varepsilon} | r_3(u_k) |\wrt u_k + \int_{u_k=\Delta-\varepsilon}^{\min(\Delta+\varepsilon,\Delta-\varepsilon+u_{k-1})} |r_3(u_k)|\wrt u_k 
		%
		\\&\qquad{}+ \int_{u_k = \Delta+\varepsilon}^{\Delta-\varepsilon+u_{k-1}} |r_3(u_{k})|\wrt u_k1(u_{k-1}>2\varepsilon)\Bigg] \wrt u_{k-1}.\label{eqn: dkj5678G5F}
	\end{align}
	The second integral is less than or equal to 
	\[\int_{u_k=\Delta-\varepsilon}^{\Delta+\varepsilon} |r_3(u_k)|\wrt u_k \leq 2\varepsilon G. \]
	With this and substituting the upper bound for \(|r_3|\), (\ref{eqn: dkj5678G5F}) is less than or equal to
	\begin{align*}
		&\Bigg[\int_{u_{k-1}=0}^{\Delta-\varepsilon} (\Delta-\varepsilon-u_{k-1})|r_2 |+ 2\varepsilon G 
		%
		+ G\cfrac{\var(Z)/\varepsilon^2}{1-\var(Z)/\varepsilon^2}(u_{k-1}-2\varepsilon)1(u_{k-1}>2\varepsilon) \Bigg]\wrt u_{k-1}
		% 
		\\& \leq \frac{1}{2}\Delta^2|r_2 |+ 2\Delta\varepsilon G 
		%
		+ \cfrac{1}{2}\Delta^2G\cfrac{\var(Z)/\varepsilon^2}{1-\var(Z)/\varepsilon^2},
	\end{align*}
	
	By Property \ref{properties: 2}, 
	\begin{align*}
		&\int_{u_{n-1}=0}^{\Delta-\varepsilon}| r_{\bs v}(u_{n-1},x)|\wrt u_{n-1} 
		\leq R_{{\bs v},1}.
	\end{align*}

	Therefore, the error term \(|r_5(n)|\) is 
	\begin{align*}
		|r_5(n)|= O\left(\max\left\{G^{n-1}\Delta^{n-2}\left(\frac{1}{2}\Delta|r_2 |+ 2\varepsilon G 
		%
		+ \cfrac{1}{2}\Delta G\cfrac{\var(Z)/\varepsilon^2}{1-\var(Z)/\varepsilon^2}\right),
		G^{n-1}\Delta^{n-2}R_{{\bs v},1}\right\}\right).
	\end{align*}
	
	Now, 
	\begin{align*}
		&\Bigg|g_{1,n}^{*,\varepsilon}(x_0,x) - g_{1,n}^{*}(x_0,x)
		%
		\Bigg|\nonumber
		%
		\\&= \int_{u_1=\Delta-\varepsilon-x_0}^{\Delta-x_0}g_1(\Delta - u_1 - x_0)
		\int_{u_2=\Delta-\varepsilon-u_1}^{\Delta-u_1}g_2(\Delta - u_2 - u_1)\wrt u_1  \nonumber 
		\\&\quad\hdots 
            	\int_{u_{n-1}=\Delta-\varepsilon-u_{n-2}}^{\Delta-u_{n-2}} g_{n-1}(\Delta - u_{n-1} - u_{n-2}) \wrt u_{n-2}
            	g_{n}(\Delta - x-u_{n-1})\nonumber 
		\\&\quad\times 1(\Delta - x-u_{n-1}\geq 0)\wrt u_{n-1}\nonumber
		\\&\leq \int_{u_1=\Delta-\varepsilon-x_0}^{\Delta-x_0}G 
		\int_{u_2=\Delta-\varepsilon-u_1}^{\Delta-u_1}G \wrt u_1  \hdots 
            	\int_{u_{n-1}=\Delta-\varepsilon-u_{n-2}}^{\Delta-u_{n-2}} G \wrt u_{n-2}\nonumber
            	G\wrt u_{n-1} 
		\\& = \varepsilon^{n-1}G ^n 
	\end{align*}
	Therefore, the left-hand side of (\ref{eqn: rhs g 2}) is equal to 
	\begin{align*}
		g^{*,\varepsilon}_{1,n}(x_0,x) + r_5(n) + r_6(n),
	\end{align*}
	where \(r_6(n) = g_{1,n}^{*}(x_0,x) - g_{1,n}^{*,\varepsilon}(x_0,x) \), and \(|r_6(n)|\leq \varepsilon^{n-1}G ^n\).
\end{proof}

%\begin{lem}\label{lem: Dcoajc}
%	Let \(\psi:[0,\Delta)\to \mathbb R\) be bounded, \(\psi(x)\leq F\), and Lipschitz. Then, for \(x\in\calD_{\ell_0,j}\), \(\ell_0\in\mathcal K\setminus\{-1,K+1\}\), \(\lambda > 0\),
%	\begin{align}
%            	\left|\int_{x=0}^\Delta \widehat f^{\ell_0,(p)}_{0,+,+}(\lambda)(x,j; x_0,i)\psi(x)\wrt x - \int_{x=0}^\Delta\widehat \mu^{\ell_0}_{0,+,+}(\lambda)(x,j; x_0,i)\psi(x)\wrt x\right| \leq R_{{\bs v},2}^{(p)} GF + \varepsilon^{(p)} GF, \label{eqn: anue}
%            \end{align}
%            and 
%            \begin{align}
%            	\left|\int_{x=0}^\Delta \widehat f^{\ell_0,(p)}_{0,-,-}(\lambda)(x,j; x_0,i)\psi(x)\wrt x - \int_{x=0}^\Delta\widehat \mu^{\ell_0}_{0,-,-}(\lambda)(x,j; x_0,i)\psi(x)\wrt x\right| \leq R_{{\bs v},2}^{(p)} GF + \varepsilon^{(p)} GF. \label{eqn: anue2}
%            \end{align} 
%\end{lem}


% \subsection{Many integrals.}\label{appendix: many}
%In this section we first show bounds for tails of the integrals in expressions of the form (\ref{eqn: approx final end 2}) (Lemmas \ref{lem: lh bnd}, \ref{lem: rh bnd} and Corollary~\ref{cor: lh and rh}); these bounds are in terms of the variance of the ME. We then combine these results with the Properties \ref{properties: 1}-\ref{properties: 2} to prove Corollary~\ref{cor: a cor}. 





% \begin{cor}\label{cor: amammme}
% 	Let \(g_1, g_2, \dots,\) be functions satisfying the Assumptions \ref{asu: g} and let \({\bs v}(x)\) be a closing operator with the Properties \ref{properties: some props}, then,
% 	\begin{align}
% 		&\int_{x_1=0}^\infty g_1(x_1) \bs k(x_0) e^{\bs{S}x_1}\wrt x_1\bs D 
%             	\left[\prod_{k=2}^{n-1}\int_{x_k=0}^\infty g_k(x_k) e^{\bs{S}x_k} \wrt x_k \bs D\right] \int_{x_n=0}^\infty g_{n}(x_n) e^{\bs{S}x_n} \wrt x_n \widetilde{\bs w}(x) \nonumber 
% 		% \\&= \int_{x_1=0}^\infty g_1(x_1) \bs k(x_0) e^{\bs{S}x_1}\wrt x_1\bs D 
% 		% \left[\prod_{k=2}^{n-1}\int_{x_k=0}^\infty g_k(x_k) e^{\bs{S}x_k} \wrt x_k \bs D\right] \int_{x_n=0}^\infty g_{n}(x_n) e^{\bs{S}x_n} \wrt x_n {\bs w}(x) \nonumber
% 		\\&= O(\var(Z)) \label{eqn :mmmm2}
% 	\end{align}
% \end{cor}
% \begin{proof}
% 	% First observe that 
% 	% \begin{align*} 
% 	% 	\int_{x_n=0}^\infty \bs\alpha e^{\bs S(u+x_n)}{\bs v}(x)\wrt x_n 
% 	% 	&=\int_{x_n=0}^\infty \bs\alpha e^{\bs S(u+x_n)} (\bs w(x)+\widetilde{\bs w}(x))\wrt x_n
% 	% 	\\&=\int_{x_n=0}^\infty \bs\alpha e^{\bs S(u+x_n)} \bs w(x)\wrt x_n + \bs\alpha e^{\bs S(u+x_n)} \widetilde{\bs w}(x))\wrt x_n
% 	% \end{align*}
% 	% and recall that Property \ref{properties: 0} states 
% 	% \begin{align}
% 	% 	\int_{x_n=0}^\infty \bs\alpha e^{\bs S(u+x_n)} \widetilde{\bs w}(x)\wrt x_n = \bs\alpha e^{\bs Su}(-\bs S)^{-1} \widetilde{\bs w}(x) = O(\var(Z)).
% 	% \end{align}

% 	Consider the left-hand side of (\ref{eqn :mmmm2}). Replacing the last \(\bs D\) matrix in (\ref{eqn :mmmm2}) by its integral definition, gives 
% 	\begin{align*}
% 		&\int_{x_1=0}^\infty g_1(x_1) \bs k(x_0) e^{\bs{S}x_1}\wrt x_1
%             \left[\prod_{k=2}^{n-1}\bs D \int_{x_k=0}^\infty g_k(x_k) e^{\bs{S}x_k} \wrt x_k \right] \int_{u=0}^\infty e^{\bs S u}\bs s \cfrac{\bs \alpha e^{\bs S u}}{\bs \alpha e^{\bs S u} \bs e}\wrt u 
% 			\\&\qquad{}\int_{x_n=0}^\infty g_{n}(x_n) e^{\bs{S}x_n} \wrt x_n \widetilde{\bs w}(x) 
% 		\\&=\bs k(x_0) \int_{u=0}^\infty \mathcal I_{1,n}(u) \cfrac{\bs \alpha e^{\bs S u}}{\bs \alpha e^{\bs S u} \bs e}\wrt u \int_{x_n=0}^\infty g_{n}(x_n) e^{\bs{S}x_n} \wrt x_n \widetilde{\bs w}(x) 
% 		\\&\leq \cfrac{1}{\bs \alpha e^{\bs{S}x_0}\bs e}G\widehat G^{n-1} \int_{u=0}^\infty \bs \alpha e^{\bs{S}u}\bs e \cfrac{\bs \alpha e^{\bs S u}}{\bs \alpha e^{\bs S u} \bs e}\wrt u \int_{x_n=0}^\infty g_{n}(x_n) e^{\bs{S}x_n} \wrt x_n \widetilde{\bs w}(x) 
% 	\end{align*}
% 	by Lemma \ref{lem: lh bnd}. Integrating over \(u\) gives 
% 	\begin{align*}
% 		&\cfrac{1}{\bs \alpha e^{\bs{S}x_0}\bs e}G\widehat G^{n-1} \bs \alpha (-\bs S)^{-1} \int_{x_n=0}^\infty g_{n}(x_n) e^{\bs{S}x_n} \wrt x_n \widetilde{\bs w}(x) 
% 		\\&\leq \cfrac{1}{\bs \alpha e^{\bs{S}x_0}\bs e}G\widehat G^{n-1} \bs \alpha (-\bs S)^{-1} \int_{x_n=0}^\infty g_{n}(x_n) \wrt x_n \widetilde{\bs w}(x),
% 	\end{align*}
% 	by Property \ref{properties: -1}. Integrating over \(x_n\), gives
% 	\begin{align*}
% 		\cfrac{1}{\bs \alpha e^{\bs{S}x_0}\bs e}G\widehat G^{n} \bs \alpha (-\bs S)^{-1} \widetilde{\bs w}(x) 
% 		&=\cfrac{1}{\bs \alpha e^{\bs{S}x_0}\bs e}G\widehat G^{n}O(\var(Z)),
% 	\end{align*}
% 	by Property \ref{properties: 0}.
% \end{proof}



% The error term \(r_4(n)\) depends on \(p\) and we write \(r_4^{(p)}(n)\) when we need to make this dependence explicit, otherwise it is omitted from the notation. Upon choosing \(\varepsilon = \var(Z^{(p)})^{1/3}\), then for fixed \(n<\infty\) the error term \(|r_4^{(p)}(n)|= O\left(\var\left(Z^{(p)}\right)^{1/3}\right)\to 0\) as \(p\to\infty\). 




% The error terms \(r_5(n)\) depend on \(p\) as they are functions of \(r_2^{(p)},\, \varepsilon^{(p)},\, \var\left(Z^{(p)}\right)\) and \(R_{{\bs v},1}^{(p)}\). We write \(r_5^{(p)}(n)\) when this dependence is explicitly needed, otherwise this dependence it omitted from the notation. Choosing \(\varepsilon = \var(Z^{(p)})^{1/3}\), the error term \(|r_5^{(p)}(n)|= O\left(\var\left(Z^{(p)}\right)^{1/3}\right)\to 0\) as \(p\to\infty\). Similarly, \(r_6(n)\) depends on \(p\) as it is a function of \(\varepsilon^{(p)}\) and we write \(r_6^{(p)}(n)\) when we need to denote this explicitly. 

%\begin{cor}\label{cor: yet another}
%	Let \(g_1,g_2,\dots,\) be functions satisfying the Assumptions \ref{asu: g} and let \({\bs v}(x)\) be a closing operator with the Properties \ref{properties: some props}. Then, for \(n\geq 2\), 
%	\begin{align}
%		&\Bigg| \int_{x_1=0}^\infty g_1(x_1) \bs k(x_0) e^{\bs{S}x_1}\wrt x_1\bs D(\Delta-\varepsilon)
%            	\left[\prod_{k=2}^{n-1}\int_{x_k=0}^\infty g_k(x_k) e^{\bs{S}x_k} \wrt x_k \nonumber 
%		\bs D(\Delta-\varepsilon)\right]
%%		\hdots
%%            	\int_{x_{n-1}=0}^\infty g_{n-1}(x_{n-1}) e^{\bs{S}x_{n-1}} \wrt x_{n-1} \int_{u_{n-1}=0}^{\Delta-\varepsilon} e^{\bs{S}u_{n-1}}\bs s \cfrac{\bs \alpha e^{\bs{S}u_{n-1}}}{\bs \alpha e^{\bs{S}u_{n-1}}\bs e}\wrt u_{n-1} \nonumber 
%            	\int_{x_n=0}^\infty g_{n}(x_n) e^{\bs{S}x_n} \wrt x_n {\bs v}(x) \nonumber 
%	%
%		\\&{}- g_{1,n}^*(x_0,x)\Bigg|%\int_{u_1=0}^{\Delta-x_0}g_1(\Delta - u_1 - x_0)
%%		\int_{u_2=0}^{\Delta-u_1}g_2(\Delta - u_2 - u_1)\wrt u_1  \nonumber 
%%		\\&\quad\hdots 
%%            	\int_{u_{n-1}=0}^{\Delta-u_{n-2}} g_{n-1}(\Delta - u_{n-1} - u_{n-2}) \wrt u_{n-2}
%%            	g_{n}(\Delta - x-u_{n-1})1(\Delta - x-u_{n-1}\geq 0)\wrt u_{n-1} \Bigg| \nonumber
%		\\&\leq |r_5(n)| + |r_6(n)|, \label{eqn: rhs g 3}
%	\end{align}
%	where \(|r_6(n)| \leq\varepsilon^{n-1}G^n \).
%\end{cor}
%\begin{proof}
%	First observe that the difference 
%	\begin{align}
%		&\Bigg|\int_{u_1=0}^{\Delta-\varepsilon-x_0}g_1(\Delta - u_1 - x_0)
%		\int_{u_2=0}^{\Delta-\varepsilon-u_1}g_2(\Delta - u_2 - u_1)\wrt u_1  \nonumber 
%		\\&\quad\hdots 
%            	\int_{u_{n-1}=0}^{\Delta-\varepsilon-u_{n-2}} g_{n-1}(\Delta - u_{n-1} - u_{n-2}) \wrt u_{n-2}
%            	g_{n}(\Delta - x-u_{n-1})1(\Delta - x-u_{n-1}\geq\varepsilon)\wrt u_{n-1}\nonumber
%		%
%		\\&\quad - \int_{u_1=0}^{\Delta-x_0}g_1(\Delta - u_1 - x_0)
%		\int_{u_2=0}^{\Delta-u_1}g_2(\Delta - u_2 - u_1)\wrt u_1  \nonumber 
%		\\&\quad\hdots 
%            	\int_{u_{n-1}=0}^{\Delta-u_{n-2}} g_{n-1}(\Delta - u_{n-1} - u_{n-2}) \wrt u_{n-2}
%            	g_{n}(\Delta-x-u_{n-1})1(\Delta - x-u_{n-1}\geq 0)\wrt u_{n-1} \Bigg|\nonumber
%		%
%		\\&= \int_{u_1=\Delta-\varepsilon-x_0}^{\Delta-x_0}g_1(\Delta - u_1 - x_0)
%		\int_{u_2=0}^{\Delta-\varepsilon-u_1}g_2(\Delta - u_2 - u_1)\wrt u_1  \nonumber 
%		\\&\quad\hdots 
%            	\int_{u_{n-1}=\Delta-\varepsilon-u_{n-2}}^{\Delta-u_{n-2}} g_{n-1}(\Delta - u_{n-1} - u_{n-2}) \wrt u_{n-2}
%            	g_{n}(\Delta - x-u_{n-1})1(\Delta - x-u_{n-1}\geq 0)\wrt u_{n-1}\nonumber
%		\\&\leq \int_{u_1=\Delta-\varepsilon-x_0}^{\Delta-x_0}G 
%		\int_{u_2=0}^{\Delta-\varepsilon-u_1}G \wrt u_1  \hdots 
%            	\int_{u_{n-1}=\Delta-\varepsilon-u_{n-2}}^{\Delta-u_{n-2}} G_{n-1} \wrt u_{n-2}\nonumber
%            	G_{n}\wrt u_{n-1} 
%		\\& = \varepsilon^{n-1}G \dots G \label{eqn: this pnn}
%	\end{align}
%	
%	Adding and subtracting the integral on the right-hand side of (\ref{eqn: rhs g 2}) to the left-hand side of (\ref{eqn: rhs g 3}) (within the absolute value), applying the triangle inequality, Lemma~\ref{lem: lst convergence} gives the required bound. 
%\end{proof}
%
%The error term \(r_6(n)\) depends on \(p\). We write \(r_6^{(p)}(n)\) when this dependence needs to be made explicit. Also note that \(|r_6^{(p)}(n)|\to 0 \) as \(p\to\infty\). 

We are now able to prove the result we need. 
\begin{cor}\label{cor: a cor}
	Let \(g_1,g_2,\dots,\) be functions satisfying Assumptions \ref{asu: g} and let \({\bs v}(x)\), \(x\in[0,\Delta)\), be a closing operator with Properties \ref{properties: some props}. Then, for \(n\geq 2\), \(x_0\in[0,\Delta)\), 
	\begin{align}
		&\left|w_n(x_0,x)- g_{1,n}^{*}(x_0,x) \right|
		\leq |r_5(n)| + |r_6(n)| + (n-1)|r_4(n)|, \label{eqn: rhs g 4}
	\end{align}
	where 
	\begin{align*}
		|r_4(n)| &= \left(2\varepsilon + \cfrac{\var(Z)}{\varepsilon}\right) \cfrac{1}{1-\var(Z)/(\Delta-x_0)} G \widehat G^{n-2} G ,
		\\|r_5(n)|&= O\left(\max\left\{G^{n-1}\Delta^{n-2}\left(\frac{1}{2}\Delta|r_2 |+ 2\varepsilon G 
		%
		+ \cfrac{1}{2}\Delta G\cfrac{\var(Z)/\varepsilon^2}{1-\var(Z)/\varepsilon^2}\right),
		G^{n-1}\Delta^{n-2}R_{{\bs v},1}\right\}\right)
		\\|r_6(n)| &\leq\varepsilon^{n-1}G^n.
	\end{align*}
\end{cor}
\begin{proof}
	The left-most term on the left-hand side of (\ref{eqn: rhs g 4}), \(w_n(x_0,x)\), can be written as (\ref{eqn: kfvKJBawXMN0}). So substitute (\ref{eqn: kfvKJBawXMN0}) into the left-hand side of (\ref{eqn: rhs g 4}), apply the triangle inequality and Lemmas \ref{cor: lh and rh} and \ref{lem: lst convergence} to get the result.  
\end{proof}

A direct Corollary is the following.
\begin{cor}\label{cor: asjdajaaaaa}
	 Let \(g_1,g_2,\dots,\) be functions satisfying Assumptions \ref{asu: g} and let \({\bs v}(x)\), \(x\in[0,\Delta)\), be a closing operator with Properties \ref{properties: some props}. Then, for \(n\geq 2\), \(x_0\in[0,\Delta)\), 
	\begin{align}
		&\left| \int_{x=0}^\Delta w_n(x_0,x) \psi(x)-g_{1,n}^*(x_0,x)\psi(x)\wrt x\right|  
		\leq (|r_5(n)| + |r_6(n)| + (n-1)|r_4(n)|)\Delta F. \label{eqn: rhs g 4dvfklsmv}
	\end{align}
\end{cor}
\begin{proof}
	The left-hand side of (\ref{eqn: rhs g 4dvfklsmv}) is less than or equal to 
	\begin{align}
		&\int_{x=0}^\Delta \left| w_n(x_0,x) 
	%
		{}- g_{1,n}^*(x_0,x)\right| \left|\psi(x)\right| \wrt x. \label{eqn: rhs g 4dvfklsmsssv}
	\end{align}
	Apply Corollary~\ref{cor: a cor} to bound the first absolute value so that (\ref{eqn: rhs g 4dvfklsmsssv}) is less than or equal to 
	\begin{align}
		&\int_{x=0}^\Delta (|r_5(n)| + |r_6(n)| + (n-1)|r_4(n)|) \left|\psi(x)\right| \wrt x \nonumber
		\\&\leq \int_{x=0}^\Delta(|r_5(n)| + |r_6(n)| + (n-1)|r_4(n)|) F \wrt x \nonumber 
		\\&= (|r_5(n)| + |r_6(n)| + (n-1)|r_4(n)|)\Delta F 
	\end{align}
\end{proof}

We have assumed throughout the appendix that the functions \(g\) and \(\{g_k\}\) are scalar functions, however, we are ulitmately interested in expressions of the form (\ref{eqn: approx final end 2}), which contain matrix functions. 
\begin{lem}\label{lem: boobies}
	Let \(\bs G_k(x)\), \(k\in\{1,2,...\}\), be matrix functions with dimensions \(N_k \times N_{k+1}\). Further, suppose \([\bs G_k(x)]_{ij}\), \(i\in\{1,...,N_{k}\}\), \(j\in\{1,...,N_{k+1}\}\), \(k\in\{1,2,...\}\) satisfy Assumptions \ref{asu: g}. Then, 
	\begin{align}
		&\Bigg| \int_{x=0}^\Delta \int_{x_1=0}^\infty \bs G_1(x_1) \otimes \bs k(x_0) e^{\bs{S}x_1} \bs D (x_1) \wrt x_1
		\left[\prod_{k=2}^{n-1}\int_{x_k=0}^\infty \bs G_{k }(x_k) \otimes e^{\bs{S}x_k} \wrt x_k \bs D\right] \nonumber
\\&\qquad{}\int_{x_n=0}^\infty \bs G_{n }(x_n)\otimes e^{\bs{S}x_n} \wrt x_n {\bs v}(x) \psi(x) \wrt x \nonumber 
	%
		\\&{}- \int_{x=0}^\Delta \int_{u_1=0}^{\Delta-x_0}\bs G_1(\Delta - u_1 - x_0)
	%		\int_{u_2=0}^{\Delta-u_1}g_2(\Delta - u_2 - u_1)\wrt u_1  \nonumber 
		\left[\prod_{k=2}^{n-1} \int_{u_k=0}^{\Delta-u_{k-1}} \bs G_{k}(\Delta-u_k-u_{k-1})\wrt u_{k-1}\right] \nonumber 
		%\\&{}\nonumber
				%\int_{u_{n-1}=0}^{\Delta-u_{n-2}} g_{n-1}(\Delta - u_{n-1} - u_{n-2}) \wrt u_{n-2}
				\\&\qquad{} \bs G_{n }(\Delta - x-u_{n-1})
			1(\Delta-x-u_{n-1}\geq0) \wrt u_{n-1}\psi(x) \wrt x \Bigg| \nonumber
		\\&\leq (|r_5(n)| + |r_6(n)| + (n-1)|r_4(n)|)\Delta F \prod_{k=2}^{n}N_{k}, \label{eqn: rhs g 4dvfklsmv2G}
	\end{align}
	where the inequality is an element-wise inequality. Moreover, choosing \(\varepsilon=\var(Z)\), then, for fixed \(n\), the bound (\ref{eqn: rhs g 4dvfklsmv2G}) is \(\mathcal O(\var(Z)^{1/3})\). 
\end{lem}
\begin{proof}
	By the \ref{eqn:mpr} the \((i,j)\)th element of the first term on the left-hand side of (\ref{eqn: rhs g 4dvfklsmv2G}) is 
	\begin{align}
		&\int_{x=0}^\Delta \int_{x_1=0}^\infty \dots \int_{x_n=0}^\infty \left[\bs G_1(x_1)\dots \bs G_n(x_n)\right]_{i,j} \bs k(x_0) e^{\bs{S}x_1} \bs D (x_1) 
		e^{\bs{S}x_k} \bs D \nonumber
		\\&\qquad{} e^{\bs{S}x_n} \wrt x_n \dots \wrt x_1 {\bs v}(x) \psi(x) \wrt x \nonumber 
		% 
		\\&= \int_{x=0}^\Delta \int_{x_1=0}^\infty \dots \int_{x_n=0}^\infty \sum_{j_1=1}^{N_2}[\bs G_1(x_1)]_{i,j_1}\sum_{j_2=1}^{N_3}[\bs G_2(x_2)]_{j_1,j_2}\dots \sum_{j_{n-1}=1}^{N_n} [\bs G_n(x_n)]_{j_{n-1},j} \nonumber
		\\&\qquad{} \bs k(x_0) e^{\bs{S}x_1} \bs D (x_1) 
		e^{\bs{S}x_k} \bs D 
		e^{\bs{S}x_n} \wrt x_n \dots \wrt x_1 {\bs v}(x) \psi(x) \wrt x, \label{eqn: s}
	\end{align}
	from which we see that (\ref{eqn: s}) is a linear combination of the scalar function case. Applying the bound for the scalar case to each term in the linear combination, then summing the bounds gives the bounds. 

	The fact that the error bound is \(\mathcal O(\var(Z)^{1/3})\) follows by substituting \(\varepsilon=\var(Z)\) into each term and observing that each term is at most \(\mathcal O(\var(Z)^{1/3})\). 
\end{proof}
Lemma \ref{lem: boobies} effectively shows that, as \(p \to \infty\), then 
\begin{align}
	& \int_{x=0}^\Delta \int_{x_1=0}^\infty \bs G_1(x_1) \otimes \bs k^{(p)} (x_0) e^{\bs{S}^{(p)}x_1} \bs D^{(p)} (x_1) \wrt x_1
	\left[\prod_{k=2}^{n-1}\int_{x_k=0}^\infty \bs G_{k }(x_k) \otimes e^{\bs{S}^{(p)}x_k} \wrt x_k \bs D^{(p)}\right] \nonumber
\\&\qquad{}\int_{x_n=0}^\infty \bs G_{n }(x_n)\otimes e^{\bs{S}^{(p)}x_n} \wrt x_n {\bs v}^{(p)}(x) \psi(x) \wrt x \nonumber 
%
	\\&{}\to \int_{x=0}^\Delta \int_{u_1=0}^{\Delta-x_0}\bs G_1(\Delta - u_1 - x_0)
%		\int_{u_2=0}^{\Delta-u_1}g_2(\Delta - u_2 - u_1)\wrt u_1  \nonumber 
	\left[\prod_{k=2}^{n-1} \int_{u_k=0}^{\Delta-u_{k-1}} \bs G_{k}(\Delta-u_k-u_{k-1})\wrt u_{k-1}\right] \nonumber 
	%\\&{}\nonumber
			%\int_{u_{n-1}=0}^{\Delta-u_{n-2}} g_{n-1}(\Delta - u_{n-1} - u_{n-2}) \wrt u_{n-2}
			\\&\qquad{} \bs G_{n }(\Delta - x-u_{n-1})
		1(\Delta-x-u_{n-1}\geq0) \wrt u_{n-1}\psi(x) \wrt x \nonumber.
\end{align}



\begin{proof}[Proof of Theorem~\ref{thm: a thm!}]
	\textit{Cases \(q=r \in \{+,-\}\) and \(m=0\).} Lemma~\ref{lem: Dcoajc} bounds the absolute difference 
	\[\left|\int_{x\in\calD_{\ell_0}}\widehat f_{0,r,r}^{\ell_0,(p)}(\lambda)(x,j;x_0,k)\psi(x)\wrt x-\int_{x\in\calD_{\ell_0}}\widehat \mu_{0,r,r}^{\ell_0}(\lambda)(x,j;x_0,k)\psi(x)\wrt x\right|.\]
	Since the bounds from Lemma~\ref{lem: Dcoajc} are \(\mathcal O(\var(Z^{(p)})^{1/3})\) then, as we take \(p \to \infty\), the bounds becomes arbitrarily small which gives us the required convergence. 

	\textit{Cases \(q,r\in \{+,-\},\) and \(m\geq 1\).} Given the properties of the functions \(\bs h_{ij}^{u,v}\), \(u,v\in\{+,-\}\), then \(\int_{x\in\calD_{\ell_0}}\widehat f_{0,q,r}^{\ell_0,(p)}(\lambda)(x,j;x_0,k)\psi(x)\wrt x\) satisfies the assumptions of Lemma~\ref{lem: boobies}. To see this, let \(q'\) be the opposite sign to \(q\), i.e.~\(q'\in\{+,-\},\, q\neq q'\). Then, in Equation (\ref{eqn: rhs g 4dvfklsmv2G}), take \(n=2m+1(q=r)\), \(\bs G_1(x_1) = \bs e_i\bs H^{qq'}(\lambda, x_1)\), \(\bs G_{2k}(x_{2k}) = \bs H^{q'q}(\lambda, x_{2k})\), \(\bs G_{2k+1}(x_{2k}) = \bs H^{qq'}(\lambda, x_{2k+1})\), \(k=1,\dots,m-1\); if \(q\neq r\) then take \(\bs G_{2m}(x_{2m}) = \bs H^{rr}(x_{2m})\bs e_j\), otherwise, take \(\bs G_{2m}(x_{2m}) = \bs H^{q'r}(x_{2m})\) and \(\bs G_{2m+1} = \bs H^{rr}(\lambda,x_{2m+1})\bs e_j\). By the remarks following Lemma~\ref{lem: boobies}, this gives the required convergence for this case. 

	\textit{Cases \(q=0,\, r\in\{+,-\}\) and \(m\geq 0\).} 
	Since
	\begin{align}
		\widehat f_{m,0,r}^{\ell_0}(\lambda)(x,j;x_0,k)  
		&= \sum_{q\in\{+,-\}}\sum_{i\in\calS_q}\bs e_k\vligne{\lambda \bs I - \bs T_{00}}^{-1}\bs T_{0i}\widehat f_{m+1(r\neq q),q,r}^{\ell_0}(\lambda)(x,j;x_0,i), \label{eqkadv}
	\end{align}
	is a linear combination of terms which are treated in the two cases above, then (\ref{eqkadv}) converges to 
	\begin{align}
		\widehat \mu_{m,0,r}^{\ell_0}(\lambda)(x,j;x_0,k) 
		&= \sum_{q\in\{+,-\}}\sum_{i\in\calS_q}\bs e_k\vligne{\lambda \bs I - \bs T_{00}}^{-1}\bs T_{0i}\widehat \mu_{m+1(r\neq q),q,r}^{\ell_0}(\lambda)(x,j;x_0,i),
	\end{align}
	as required. 
\end{proof}

% We now present a series of results which provide provide error bounds and/or show convergence of the Laplace transforms presented in the previous section. 
% \begin{lem}\label{lem: Dcoajc}
% 	Let \(\psi:[0,\Delta)\to \mathbb R\) be bounded, \(\psi(x)\leq F\), and Lipschitz. Then, for \(x\in\calD_{\ell_0,j}\), \(\ell_0\in\mathcal K\setminus\{-1,K+1\}\), \(\lambda > 0\),
% 	\begin{align}
%             	\left|\int_{x=0}^\Delta \widehat f^{\ell_0,(p)}_{0,+,+}(\lambda)(x,j; x_0,i)\psi(x)\wrt x - \int_{x=0}^\Delta\widehat \mu^{\ell_0}_{0,+,+}(\lambda)(x,j; x_0,i)\psi(x)\wrt x\right| \leq R_{{\bs v},2}^{(p)} GF + \varepsilon^{(p)} GF, \label{eqn: anue}
%             \end{align}
%             and 
%             \begin{align}
%             	\left|\int_{x=0}^\Delta \widehat f^{\ell_0,(p)}_{0,-,-}(\lambda)(x,j; x_0,i)\psi(x)\wrt x - \int_{x=0}^\Delta\widehat \mu^{\ell_0}_{0,-,-}(\lambda)(x,j; x_0,i)\psi(x)\wrt x\right| \leq R_{{\bs v},2}^{(p)} GF + \varepsilon^{(p)} GF. \label{eqn: anue2}
%             \end{align} 
% \end{lem}

% \begin{cor}\label{cor: Dcoajc}
% 	Let \(\psi:\calD_{\ell_0}\to \mathbb R\) be bounded, \(|\psi(x)|\leq F\), and Lipschitz continuous. Then, for \(x_0\in(y_{\ell_0},y_{\ell_0+1}\), \(k\in\calS_{0+}\), \(i\in\calS_-\), \(j\in\calS_-\cup\calS_{-0}\), there exists \(r_{10}^{(p)}\to 0\) as \(p \to \infty\), 
% 	\begin{align}
% 		\left|\int_{x\in\calD_{\ell_0}} \widehat f_{0,-0,-}^{\ell_0,(p)}(x,i,j;j,x_0)\psi(x)\wrt x  \to \int_{x\in\calD_{\ell_0}} \widehat \mu_{0,-0,-}^{\ell_0}(x,i,j;k,x_0)\psi(x)\wrt x\right|\leq r_{10}^{(p)}.\label{eqn:xs}
% 	\end{align}
% 	Similarly, for \(k\in\calS_{0-}\), \(i\in\calS_+\), \(j\in\calS_+\cup\calS_{+0}\)
% 	\begin{align}
% 		\left|\int_{x\in\calD_{\ell_0}} \widehat f_{0,+0,+}^{\ell_0,(p)}(x,i,j;j,x_0)\psi(x)\wrt x  - \int_{x\in\calD_{\ell_0}} \widehat \mu_{0,+0,+}^{\ell_0}(x,i,j;k,x_0)\psi(x)\wrt x\right|\leq r_{10}^{(p)},\label{eqn:!124mcds}
% 	\end{align}
% 	where \(r_{10}^{(p)}\to 0\) as \(p\to \infty\). 
% \end{cor}
% \begin{proof}
% 	The result follows from (\ref{lem: ppp}) in Appendix~\ref{app:tand}.
% \end{proof}

% Define 
% \begin{align}
% 		w_n^{(p)}(x_0,x) &= \int_{x_1=0}^\infty g_1(x_1) \bs a ^{(p)}(x_0) e^{\bs{S}^{(p)}x_1}\wrt x_1\bs D^{(p)} 
%             	\left[\prod_{k=2}^{n-1}\int_{x_k=0}^\infty g_k(x_k) e^{\bs{S}^{(p)}x_k} \wrt x_k \bs D^{(p)}\right] \nonumber 
% 		\\&\qquad{} \int_{x_n=0}^\infty g_{n}(x_n) e^{\bs{S}^{(p)}x_n} \wrt x_n {\bs v}^{(p)}(x), \label{eqn: kagorevAJ}
% \end{align}
% where \(g_1,g_2,\dots,\) are functions satisfying the Assumptions~\ref{asu: g} and \({\bs v}^{(p)}(x)\) is a closing operator with the Properties~\ref{properties: some props}. Expressions such as (\ref{eqn: approx end conv lst}) are specific forms of (\ref{eqn: kagorevAJ}). In Appendix~\ref{appendix: bounds}, we prove the following results. %By Lemma~\ref{cor: ksjkd}, for any \(x_0,x\in [0,\Delta)\) 
% %\begin{align}
% %		w_n(x_0,x) &\leq \widehat G^{n-2}GG_{\bs v}.
% %\end{align}

% \begin{cor}\label{eqn: lafkjebjcbbalbvvbrb}
% 	 Let \(g_1,g_2,\dots,\) be functions satisfying Assumptions~\ref{asu: g} and let \({\bs v}(x)\), \(x\in[0,\Delta)\), be a closing operator with Properties~\ref{properties: some props}. Then, for \(n\geq 2\), \(x_0\in[0,\Delta)\), 
% 	\begin{align}
% 		&\Bigg| \int_{x=0}^\Delta w_n^{(p)}(x_0,x) \psi(x) \wrt x \nonumber 
% 	%
% 		- \int_{x=0}^\Delta \int_{u_1=0}^{\Delta-x_0}g_1(\Delta - u_1 - x_0)
% 		\\&\qquad{} \left[\prod_{k=2}^{n-1} \int_{u_k=0}^{\Delta-u_{k-1}} g_k(\Delta-u_k-u_{k-1})\wrt u_{k-1}\right] \nonumber 
% 	g_{n}(\Delta - x-u_{n-1}) 
% 		\\&\qquad{} 1(\Delta-x-u_{n-1}\geq0) \wrt u_{n-1}\psi(x) \wrt x \Bigg| \nonumber
% 		\\&\leq (|r_5^{(p)}(n)| + |r_6^{(p)}(n)| + (n-1)|r_4^{(p)}(n)|)\Delta F, \label{eqn: rhs gs 4dvfklsmv}
% 	\end{align}
% 	where 
% 	\begin{align*}
% 		|r_4^{(p)}(n)| &= \left(2\varepsilon^{(p)} + \cfrac{\var(Z^{(p)})}{\varepsilon^{(p)}}\right) \cfrac{1}{1-\var(Z^{(p)})/(\Delta-x_0)} G \widehat G^{n-2} G ,
% 		\\|r_5(n)|&= O\Bigg(\max\Bigg\{G^{n-1}\Delta^{n-2}\left(\frac{1}{2}\Delta|r_2 |+ 2\varepsilon^{(p)} G 
% 		%
% 		+ \cfrac{1}{2}\Delta G\cfrac{\var(Z^{(p)})/\left(\varepsilon^{(p)}\right)^2}{1-\var(Z^{(p)})/\left(\varepsilon^{(p)}\right)^2}\right),
% 		\\&\qquad{} G^{n-1}\Delta^{n-2}R_{{\bs v},1}^{(p)}\Bigg\}\Bigg)
% 		\\|r_6^{(p)}(n)| &\leq\left(\varepsilon^{(p)}\right)^{n-1}G^n.
% 	\end{align*}
% \end{cor}
% \begin{cor}\label{cor: fljm7778}
% 	Let \(g_1,g_2,\dots,\) be functions satisfying Assumptions~\ref{asu: g} and let \({\bs v}^{(p)}(x)\), \(x\in(0,\Delta)\), be a closing operator with Properties~\ref{properties: some props}. For \(x_0,x\in(0,\Delta)\), \(n\geq 2\)
% 	\begin{align}
% 		&\Bigg| \int_{x\in[0,\Delta)} \int_{x_1=0}^\infty g_1(x_1) \bs k^{(p)} (x_0) \bs D^{(p)} e^{\bs{S}^{(p)}x_1}\wrt x_1\bs D^{(p)} 
%             	\left[\prod_{n=2}^{k-1}\int_{x_n=0}^\infty g_n(x_n) e^{\bs{S}^{(p)}x_n} \wrt x_n
% 		\bs D^{(p)}\right]\nonumber 
%             	\\&\qquad{}\int_{x_n=0}^\infty g_{n}(x_n) e^{\bs{S}^{(p)}x_n} \wrt x_n {\bs v}^{(p)}(x) \psi(x)\wrt x - \int_{x\in[0,\Delta)} \int_{u_1=0}^{x_0}g_1(x_0 - u_1)
% 		\nonumber 
%             	\\&\qquad{} \left[\prod_{k=2}^{n-1} \int_{u_k=0}^{\Delta-u_{k-1}} g_k(\Delta-u_k-u_{k-1})\wrt u_{k-1}\right]g_{n}(\Delta - x-u_{n-1}) 
% 		\\&\qquad{} \times 1(\Delta-x-u_{n-1}\geq0)\wrt u_{n-1}\psi(x)\wrt x \Bigg| \nonumber
% 		\\&\leq \left(|r_8^{(p)}(n)|+|r_5^{(p)}(n)|+|r_6^{(p)}(n)| + (n-1)|r_4^{(p)}(n)|\right)F\Delta.\label{eqn: KAFnnmna}
% 	\end{align}
% 	where 
% 	\begin{align*}
% 		|r_8^{(p)}(n)|&\leq  \left( 2|r_5^{(p)}(n)| + 2|r_6^{(p)}(n)| + 2(n-1)|r_4^{(p)}(n)| + \varepsilon^{(p)} G^{n-1}\Delta^{n-2}(G+L\Delta) \right) \\&\qquad{}+ 2\widehat G^{n-2}GG_{\bs v}\cfrac{\var(Z^{(p)})/\left(\varepsilon^{(p)}\right)^2}{1-\var(Z^{(p)})/\left(\varepsilon^{(p)}\right)^2}.
% 	\end{align*}
% \end{cor}
% Upon choosing \(\varepsilon^{(p)}=\var(Z^{(p)})^{1/3}\), for fixed \(n\geq 2\), \( |r_5^{(p)}(n)| + |r_6^{(p)}(n)| + (n-1)|r_4^{(p)}(n)|\to 0\) and \( |r_8^{(p)}|+|r_5^{(p)}(n)| + |r_6^{(p)}(n)| + (n-1)|r_4^{(p)}(n)|\to 0\) as \(p\to \infty\). 

% Our first main result towards proving the convergence of the QBD-RAP scheme to the fluid queue is the following bound. 
% %Using Corollary~\ref{cor: fljm7778}, we can have the following result. 
% \begin{cor}\label{cor: lst diff}Let \(\psi:\calD_{\ell_0}\to\mathbb R\) be bounded, \(|\psi(x)|\leq F\), and Lipschitz continuous. For \(x\in\calD_{\ell_0,j}\), \(x_0\in\calD_{\ell_0,i}\), \(j_1,j_2\dots\in\calS_-\), \(k_1,k_2,\dots\in\calS_+\), \(\ell_0\in\mathcal K\setminus\{-1,K+1\}\), \(m\geq 0\), \(\lambda > 0\), then 
% \begin{enumerate}
% 	\item if \(i\in\calS_{+}\), \(j\in\calS_+\cup\calS_{+0}\),
% 	\begin{align}
%                 	&\Bigg|\int_{x\in\calD_{\ell_0}}\widehat f^{\ell_0,(p)}_{m,+,+}(\lambda)(x,j_1,k_1,\dots,j_m,k_m,j; x_0,i)\psi(x) \wrt x \nonumber 
% 	\\&\qquad{} - \int_{x\in\calD_{\ell_0}}\widehat \mu^{\ell_0}_{m,+,+}(\lambda)(x,j_1,k_1,\dots,j_m,k_m,j; x_0,i)\psi(x)\wrt x\Bigg| \nonumber
%                 	%
%                 	\\&\leq \begin{cases}
% 			R_{{\bs v},2}^{(p)} GF + \varepsilon^{(p)} GF, & m = 0, \\
% 			(|r_5^{(p)}(2m)| + |r_6^{(p)}(2m)| + (2m-1)|r_4^{(p)}(2m)|)\Delta F,  &  m\geq 1,
% 			\end{cases}\label{eqn: ++}
% 	\end{align}
% 	\item if \(i\in\calS_{-}\), \(j\in\calS_-\cup\calS_{-0}\),
% 	\begin{align}
%                 	&\Bigg|\int_{x\in\calD_{\ell_0}}\widehat f^{\ell_0,(p)}_{m,-,-}(\lambda)(x,k_1,j_1,\dots,k_m,j_m,j; x_0,i)\psi(x)\wrt x \nonumber
% 	\\&\qquad{} - \int_{x\in\calD_{\ell_0}} \widehat \mu^{\ell_0}_{m,-,-}(\lambda)(x,k_1,j_1,\dots,k_m,j_m,j; x_0,i)\psi(x)\wrt x\Bigg| \nonumber
%                 	%
%                 	\\&\leq \begin{cases}
% 			R_{{\bs v},2}^{(p)} GF + \varepsilon^{(p)} GF, & m = 0, \\
% 			(|r_5^{(p)}(2m)| + |r_6^{(p)}(2m)| + (2m-1)|r_4^{(p)}(2m)|)\Delta F,  &  m\geq 1,
% 			\end{cases}\label{eqn: --}
% 	\end{align}
% 	\item if \(i\in\calS_{+}\), \(j\in\calS_-\cup\calS_{-0}\),
% 	\begin{align}
%                 	&\Bigg|\int_{x\in\calD_{\ell_0}}\widehat f^{\ell_0,(p)}_{m,+,-}(\lambda)(x,j_1,k_1,\dots,j_m,k_m,j_{m+1},j; x_0,i)\psi(x)\wrt x \nonumber
% 	\\&\qquad{}- \int_{x\in\calD_{\ell_0}}\widehat \mu^{\ell_0}_{m,+,-}(\lambda)(x,j_1,k_1,\dots,j_m,k_m,j_{m+1},j; x_0,i)\psi(x)\wrt x\Bigg| \nonumber
%                 	%
%                 	\\&\leq 
% 			(|r_5^{(p)}(2m+1)| + |r_6^{(p)}(2m+1)| + 2m|r_4^{(p)}(2m+1)|)\Delta F ,\,   m\geq 0,\label{eqn: +-}
% 	\end{align}
% 	\item and if \(i\in\calS_{-}\), \(j\in\calS_+\cup\calS_{+0}\),
% 	\begin{align}
%                 	&\Bigg|\int_{x\in\calD_{\ell_0}}\widehat f^{\ell_0,(p)}_{m,-,+}(\lambda)(x,k_1,j_1,\dots,k_m,j_m,k_{m+1},j; x_0,i)\psi(x)\wrt x \nonumber
% 	\\&\qquad{} - \int_{x\in\calD_{\ell_0}}\widehat \mu^{\ell_0}_{m,-,+}(\lambda)(x,k_1,j_1,\dots,k_m,j_m,k_{m+1},j; x_0,i)\psi(x)\wrt x\Bigg| \nonumber
%                 	%
%                 	\\&\leq 
% 			(|r_5^{(p)}(2m+1)| + |r_6^{(p)}(2m+1)| + 2m|r_4^{(p)}(2m+1)|)\Delta F,\,    m\geq 0.\label{eqn: -+}
% 	\end{align}
% %\end{enumerate}
% %
% %	Furthermore, for \(x\in\calD_{\ell_0,j}\), \(x_0\in\calD_{\ell_0,i}\), \(j_1,j_2,\dots\in\calS_-\), \(k_1,k_2,\dots\in\calS_+\), \(\ell_0\in\mathcal K, \ell_0\notin\{-1,K+1\}\), \(m\geq 0\), \(\lambda > 0\), then 
% %	\begin{enumerate}
% 	\item  if \(k\in\calS_{-0}\), \(i\in\mathcal S_+\), \(j\in\calS_+\cup\calS_{+0}\), 
% 	\begin{align}
%                 	\Big|&\int_{x=0}^\Delta \widehat{f}_{m,-0,+}^{\ell_0}(\lambda)(x,i,j_1,k_1,\dots,j_m,k_m,j;x_0,k)\psi(x) \wrt x\nonumber 
% 		\\&\qquad{} -\int_{x=0}^\Delta \widehat \mu_{m,-0,+}^{\ell_0}(\lambda)(x,i,j_1,k_1,\dots,j_m,k_m,j;x_0,k)\psi(x)\wrt x \Big| \nonumber
% 	\\&\leq \begin{cases} |r_{10}^{(p)}|, & m=0 \\ (|r_8^{(p)}(2m)|+|r_5^{(p)}(2m)|+|r_6^{(p)}(2m)| + (2m-1)|r_4^{(p)}(2m)|)\Delta F & m \geq 1 \end{cases}  \label{eqn: MMv}
% 	\end{align}
% 	\item {if \(k\in\calS_{-0}\), \(i\in\mathcal S_+\), \(j\in\calS_-\cup\calS_{-0}\), }
% 	\begin{align}
%                 	&\Big|\int_{x=0}^\Delta \widehat{f}_{m,-0,-}^{\ell_0,(p)}(\lambda)(x,i,j_1,k_1,\dots,k_m,j_{m+1},j;x_0,k)\psi(x)\wrt x\nonumber 
%                 	\\&\qquad{} -\int_{x=0}^\Delta \widehat \mu_{m,-0,-}^{\ell_0}(\lambda)(x,i,j_1,k_1,\dots,k_m,j_{m+1},j;x_0,k)\psi(x)\wrt x\Big|\nonumber 
% 	\\&\leq   (|r_8^{(p)}(2m+1)|+|r_5^{(p)}(2m+1)|+|r_6^{(p)}(2m+1)| + 2m|r_4^{(p)}(2m+1)|)\Delta F,\,  m \geq 0   \label{eqn: MMv22}
% 	\end{align}
% 	\item {if \(k\in\calS_{+0}\), \(i\in\mathcal S_-\), \(j\in\calS_+\cup\calS_{+0}\),}
% 	\begin{align}
%                 	&\Big|\int_{x=0}^\Delta \widehat{f}_{m,+0,+}^{\ell_0,(p)}(\lambda)(x,i,k_1,j_1,\dots,j_m,k_{m+1},j;x_0,k)\psi(x) \wrt x \nonumber 
%                 	\\&\qquad{} - \int_{x=0}^\Delta \widehat \mu_{m,+0,+}^{\ell_0}(\lambda)(x,i,k_1,j_1,\dots,j_m,k_{m+1},j;x_0,k)\psi(x) \wrt x\Big| \nonumber 
% 	\\&\leq   (|r_8^{(p)}(2m+1)|+|r_5^{(p)}(2m+1)|+|r_6^{(p)}(2m+1)| + 2m|r_4^{(p)}(2m+1)|)\Delta F,\,  m \geq 0   \label{eqn: MMv222}
% 	\end{align}
% 	\item {and if \(k\in\calS_{+0}\), \(i\in\mathcal S_-\), \(j\in\calS_-\cup\calS_{-0}\),}
% 	\begin{align}
%                 	&\Big|\int_{x=0}^\Delta \widehat{f}_{m,+0,-}^{\ell_0,(p)}(\lambda)(x,i,k_1,j_1,\dots,k_{m},j_{m},j;x_0,k)\psi(x) \wrt x\nonumber 
%                 	\\&\qquad{}-\int_{x=0}^\Delta \widehat{\mu}_{m,+0,-}^{\ell_0}(\lambda)(x,i,k_1,j_1,\dots,k_{m},j_{m},j;x_0,k)\psi(x)\wrt x\Big| \nonumber 
% 	\\&\leq \begin{cases}  |r_{10}^{(p)}|, & m=0 \\( |r_8^{(p)}(2m)|+|r_5^{(p)}(2m)|+|r_6^{(p)}(2m)| + (2m-1)|r_4^{(p)}(2m)|)\Delta F & m \geq 1. \end{cases}\label{eqn: MMv2}
% 	\end{align}
% 	\end{enumerate}
% \end{cor}
% %For \(r_2\) we have suppressed the dependence of \(r_2\) on \(p\) for simplicity here. When this dependence needs to be made explicit we write \(r_2^{(p)}\). 
% \begin{proof} 
% 	For (\ref{eqn: ++}) and (\ref{eqn: --}) and \(m= 0\) apply Lemma~\ref{lem: Dcoajc} to the differences on the left-hand sides.
	
% 	For (\ref{eqn: ++}) and (\ref{eqn: --}) and \(m\geq 1\) apply Corollary~\ref{eqn: lafkjebjcbbalbvvbrb} to the differences on the left-hand sides.
	
% 	For (\ref{eqn: +-}) and (\ref{eqn: -+}) apply Corollary~\ref{eqn: lafkjebjcbbalbvvbrb} to the differences on the left-hand sides.
	
% 	For (\ref{eqn: MMv}) and (\ref{eqn: MMv2}) and \(m= 0\) apply Corollary~\ref{cor: Dcoajc} to the differences on the left-hand sides.
	
% 	For (\ref{eqn: MMv}) and (\ref{eqn: MMv2}) and \(m\geq 1\) apply Corollary~\ref{cor: fljm7778} to the differences on the left-hand sides.
	
% 	For (\ref{eqn: MMv22}) and (\ref{eqn: MMv222}) apply Corollary~\ref{cor: fljm7778} to the differences on the left-hand sides.
% \end{proof}
% These bounds are enough to show the weak convergence (in space and time) of the QBD-RAP scheme to the fluid queue on the event on the event that there is no change of level, and on a given partition as described in (\ref{eqn: 63})-(\ref{eqn: density part +}). Since the state space of the phases, \(\mathcal S\), is finite, then the same bounds show convergence when we sum over all possible phases at times \(\{\Sigma_m\}\) and \(\{\Gamma_m\}\). We formalise this with Corollary~\ref{cor: k}, below.

% \begin{cor}\label{cor: k}
% 	For \(q \in \{+,-,+0,-0\}\), \(r\in\{+,-\}\), \(i\in\calS_q\), \(j\in\calS_r\), \(x\in\calD_{\ell_0,j}\), \(x_0\in\calD_{\ell_0,i}\), \(\ell_0\in\mathcal K, \ell_0\notin\{-1,K+1\}\), \(m\geq 0\), \(\lambda > 0\), then 
% 	\begin{align}
% 		\left| \int_{x\in\calD_{\ell_0}}\widehat f^{\ell_0,(p)}_{m,q,r}(\lambda)(x,j;x_0,i)\psi(x) \wrt x - \int_{x\in\calD_{\ell_0}}\widehat \mu_{m,q,r}^{\ell_0}(\lambda)( x,j;x_0,i)\psi(x)\wrt x \right| \to 0 \label{eqnALKJ{\bs v}}
% 	\end{align}
% 	as \(p\to\infty\). 
% \end{cor}
% \begin{proof}
% 	We prove the result for \(q=r=+\) only, with the proofs for the other cases being analogous. 
	
% 	By (\ref{eqn: mu advh}) and (\ref{eqn: f advh}) and the triangle inequality, 
%             \begin{align*}
%             	&\left| \int_{x\in\calD_{\ell_0}} \widehat f^{\ell_0,(p)}_{m,+,+}(\lambda)(x,j;x_0,i)\psi(x)\wrt x -  \int_{x\in\calD_{\ell_0}} \widehat \mu^{\ell_0}_{m,+,+}(\lambda)(x,j;x_0,i)\psi(x)\wrt x\right|
% 		\\&\leq \sum_{j_1\in\mathcal S_-} \sum_{k_1\in\mathcal S_+}\dots \sum_{j_m\in\mathcal S_-}\sum_{k_m\in\mathcal S_+} \Big|  \int_{x\in\calD_{\ell_0}} \widehat f^{\ell_0,(p)}_{m,+,+}(\lambda)( x, j_1,k_1,\dots,j_m,k_m, j; x_0,i) \psi(x)\wrt x
% 		\\&\quad{}- \int_{x\in\calD_{\ell_0}} \widehat \mu^{\ell_0}_{m,+,+}(\lambda)( x, j_1,k_1,\dots,j_m,k_m, j; x_0,i)\psi(x)\wrt x \Big|
%            %
% 		\\&\leq  \begin{cases}
% 			R_{{\bs v},2}^{(p)} GF + \varepsilon^{(p)} GF & m = 0 \\
% 			(|r_5^{(p)}(2m)| + |r_6^{(p)}(2m)| + 2m|r_4^{(p)}(2m)|)\Delta F |\calS_+|^m|\calS_-|^m  &  m\geq 1.
% 			\end{cases}
%         \end{align*}
%          since, by Corollary~\ref{cor: lst diff}, each term in the sum is bounded by either \(R_{{\bs v},2}^{(p)} GF + \varepsilon^{(p)} GF\) for \(m=0\) or by \((|r_5^{(p)}(2m)| + |r_6^{(p)}(2m)| + 2m|r_4^{(p)}(2m)|)|\Delta F \) for \(m\geq 0\). 
% \end{proof}



\section{Convergence on no change of level via the Dominated Convergence Theorem}\label{sec: dom 1}
In this chapter, we ultimately want to show the convergence of 
\begin{align*}
	\widehat f_{q,r}^{\ell_0,(p)}(\lambda)(x,j;x_0,i) 
\to
	\widehat \mu_{q,r}^{\ell_0}(\lambda)(x,j;x_0,i),
\end{align*} 
where 
\[\widehat f^{\ell_0,(p)}(\lambda)(x,j;x_0,i)= \int_{t=0}^\infty \sum\limits_{m=0}^\infty e^{-\lambda t} f_{m+1(p\neq q),q,r}^{\ell_0,(p)}(t)(x,j;x_0,k)\wrt t.\] 
Now, since \(f_{m+1(p\neq q),q,r}^{\ell_0,(p)}\) are positive, as is \(e^{-\lambda t}\), then we can use the Fubini-Tonelli Theorem to justify a swap of the integral and infinite sum to get 
\begin{align}
	\widehat f^{\ell_0,(p)}(\lambda)(x,j;x_0,i)=  \sum\limits_{m=0}^\infty \widehat f_{m+1(p\neq q),q,r}^{\ell_0,(p)}(t)(x,j;x_0,k).\label{eqn:LLL}
\end{align}
Similarly, we can write 
\[\widehat \mu^{\ell_0}(\lambda)(x,j;x_0,i)=  \sum\limits_{m=0}^\infty \widehat \mu_{m+1(p\neq q),q,r}^{\ell_0}(t)(x,j;x_0,k).\] 
The previous section proved that the Laplace transforms 
\[\widehat f_{m,q,r}^{\ell_0,(p)}(t)(x,j;x_0,k)\to\widehat \mu_{m,q,r}^{\ell_0}(t)(x,j;x_0,k),\] 
\(q\in\{+,-,+0,-0\}\), \(r\in\{+,-\}\). Thus, all we need to show is that, upon taking the limit of (\ref{eqn:LLL}), we can swap the limit and the summation. Here we apply the Dominated Convergence Theorem to justify the swap. To this end, we show a domination condition in (\ref{lem: gkjljklgagjklagsjlk}) below.

Let \(c_{min} = \min\limits_{i\in\mathcal S_{+}\cup\calS_-} |c_i|\) and let \(E^\lambda\) be an independent exponential random variable with rate \(\lambda\). In the following we use the stochastic interpretation of the Laplace transform of a probability distribution with non-negative support: for a random variable \(W\) with distribution function \(F_W(w)= \mathbb P(W<w)\), then \(\displaystyle\int_{w=0}^\infty e^{-\lambda w} \wrt F_W(w) = \mathbb P(W < E^{\lambda})\). That is, the Laplace transform with parameter \(\lambda >0\) is the probability that \(W\) occurs before an exponential time with rate \(\lambda\) occurs. 
\begin{lem}\label{lem: gkjljklgagjklagsjlk}For all \(M\geq 0\), \(x\in\calD_{\ell_0,j}\), \(x_0\in\calD_{\ell_0,i}\), \(\ell_0\in\mathcal K\), \(\lambda > 0\), \(q\in\{+,-,0\}\), \(r\in\{+,-\}\), \(i\in\calS_q\), \(j\in\calS_r\cup\calS_{r0}\),
	\begin{align}
		\sum_{m=M+1}^\infty \left| \int_{x\in\calD_{\ell_0}} \widehat f^{\ell_0,(p)}_{m,q,r}(\lambda)(x,j;x_0,i)\psi(x)\wrt x
		-
		\int_{x\in\calD_{\ell_0}} \widehat \mu^{\ell_0}_{m,q,r}(\lambda)(x,j;x_0,i)\psi(x) \wrt x\right| \leq r_7^M
	\end{align} 
	where 
	\[r_7^M =  F(\Delta G + \widehat G)\left(\frac{q}{q+\lambda}\right)^{2M+2} \left(1-\left(\frac{q}{q+\lambda}\right)^2\right)^{-1} .\]
\end{lem}
Note that the bound \(r_7^M\) is independent of \(p\). 

We prove the result for \(q=r=+\) only, with the proof for the other cases following analogously. Essentially, this result follows from noting the probabilistic interpretation of the Laplace transforms \(\widehat f^{\ell_0}_{m,+,+}(t)(x,j;x_0,i)\), as the probability that, 
\begin{itemize}
	\item there are \(m\) changes from \(\mathcal S_+\) to \(\mathcal S_-\) and \(\calS_-\) to \(\calS_+\), 
	\item the orbit process \(\{\bs A(t)\}\) evolves accordingly, 
	\item and an independent exponential random variable with rate \(\lambda\), \(E^\lambda\), has not yet occurred.
\end{itemize}
We obtain an upper bound by ignoring the behaviour of the orbit process \(\{\bs A(t)\}\), then, by a uniformisation argument, we bound the probability that there are \(m\) changes from \(\mathcal S_+\) to \(\mathcal S_-\) and \(\calS_-\) to \(\calS_+\) before an independent exponential random variable with rate \(\lambda\) occurs, by the event that there are \(m\) independent exponential events before an exponential random variable with rate \(\lambda\) occurs.

Similarly, the probabilistic interpretation of the Laplace transforms \(\widehat \mu^{\ell_0}_{m,+,+}(\lambda)(x,j;x_0,i)\), is the probability that, 
\begin{itemize}
	\item there are \(m\) changes from \(\mathcal S_+\) to \(\mathcal S_-\) and \(\calS_-\) to \(\calS_+\), 
	\item the fluid level \(X(t)\) remains in \(\mathcal D_{\ell_0}\), 
	\item and an independent exponential random variable with rate \(\lambda\), \(E^\lambda\), has not yet occurred.
\end{itemize}
We obtain an upper bound by removing the requirement that the fluid level \(X(t)\) remain in \(\mathcal D_{\ell_0}\), then applying the same uniformisation argument as we do for \(\widehat f^{\ell_0}_{m,+,+}(t)(x,j;x_0,i)\).

\begin{proof}
	The same arguments and results apply for all \(p\), so let us drop the dependence on \(p\). 
	
	Consider \(i\in\calS_+,j\in\mathcal S_+\cup\calS_{+0}\). By the triangle inequality, 
	\begin{align*}
		&\sum_{m=M+1}^\infty \left| \int_{x\in\calD_{\ell_0}} \widehat f^{\ell_0}_{m,+,+}(\lambda)(x,j;x_0,i)\psi(x) \wrt x
		-
		 \int_{x\in\calD_{\ell_0}} \widehat \mu^{\ell_0}_{m,+,+}(\lambda)(x,j;x_0,i) \psi(x) \wrt x \right|
		\\&\leq\sum_{m=M+1}^\infty \int_{x\in\calD_{\ell_0}} \widehat f^{\ell_0}_{m,+,+}(\lambda)(x,j;x_0,i) |\psi(x)| \wrt x
		\\&\qquad{} +\sum_{m=M+1}^\infty \int_{x\in\calD_{\ell_0}} \widehat  \mu^{\ell_0}_{m,+,+}(\lambda)(x,j;x_0,i) |\psi(x)| \wrt x,
	\end{align*}
	since all terms are non-negative. 
	
	Consider \(\int_{x\in\calD_{\ell_0}}\widehat f^{\ell_0}_{m,+,+}(\lambda)(x,j;x_0,i)|\psi(x)|\wrt x\), which is given by the \((i,j)\)th entry of
        \begin{align}
        	&\int_{x\in\calD_{\ell_0}}\int_{x_1=0}^\infty \bs H^{+-}(\lambda,x_1)\nonumber
	\int_{x_2=0}^\infty \bs H^{-+}(\lambda,x_2) 
	\hdots \int_{x_m=0}^\infty \bs H^{-+}(\lambda, x_m) 
	\int_{x_{m+1}=0}^\infty \bs H^{++}(\lambda,x_{m+1}) 
	\\&\quad\bs   a_{\ell_0,i}(x_0)e^{\bs{S}x_1}\wrt x_1 \bs{D}e^{\bs{S}x_2} \wrt x_2 \bs{D}\dots e^{\bs{S}x_m} \wrt x_m \bs{D}e^{\bs{S}x_{m+1}}{\bs v}(y_{\ell_0+1}-x) \wrt x_{m+1}\psi(x)\wrt x\nonumber
			\\&\leq \int_{x\in\calD_{\ell_0}}\int_{x_1=0}^\infty \bs H^{+-}(\lambda,x_1)\nonumber
			\int_{x_2=0}^\infty \bs H^{-+}(\lambda,x_2) 
			\hdots \int_{x_m=0}^\infty \bs H^{-+}(\lambda, x_m) 
			\int_{x_{m+1}=0}^\infty \bs H^{++}(\lambda,x_{m+1}) 
			\\&\quad\bs   a_{\ell_0,i}(x_0)e^{\bs{S}x_1}\wrt x_1 \bs{D}e^{\bs{S}x_2} \wrt x_2 \bs{D}\dots e^{\bs{S}x_m} \wrt x_m \bs{D}e^{\bs{S}x_{m+1}}{\bs v}(y_{\ell_0+1}-x) \wrt x_{m+1}\wrt x F 
			% \\&=\int_{x\in\calD_{\ell_0}}\int_{x_1=0}^\infty \bs H^{+-}(\lambda,x_1)\nonumber
			% \int_{x_2=0}^\infty \bs H^{-+}(\lambda,x_2) 
			% \hdots \int_{x_m=0}^\infty \bs H^{-+}(\lambda, x_m) 
			% \int_{x_{m+1}=0}^\infty \bs H^{++}(\lambda,x_{m+1}) 
			% \\&\quad\bs   a_{\ell_0,i}(x_0)e^{\bs{S}x_1}\wrt x_1 \bs{D}e^{\bs{S}x_2} \wrt x_2 \bs{D}\dots e^{\bs{S}x_m} \wrt x_m \bs{D}e^{\bs{S}x_{m+1}}{\bs w}(y_{\ell_0+1}-x) \wrt x_{m+1} \psi(x)\wrt x \nonumber
			% \\&{}+\int_{x\in\calD_{\ell_0}}\int_{x_1=0}^\infty \bs H^{+-}(\lambda,x_1)\nonumber
			% \int_{x_2=0}^\infty \bs H^{-+}(\lambda,x_2) 
			% \hdots \int_{x_m=0}^\infty \bs H^{-+}(\lambda, x_m) 
			% \int_{x_{m+1}=0}^\infty \bs H^{++}(\lambda,x_{m+1}) 
			% \\&\quad\bs   a_{\ell_0,i}(x_0)e^{\bs{S}x_1}\wrt x_1 \bs{D}e^{\bs{S}x_2} \wrt x_2 \bs{D}\dots e^{\bs{S}x_m} \wrt x_m \bs{D}e^{\bs{S}x_{m+1}}\widetilde{\bs w}(y_{\ell_0+1}-x) \wrt x_{m+1} \psi(x)\wrt x. 
			\label{eqn:llkjhslkj}
	\end{align}
	By Property~\ref{properties: -2}, for \(\bs a \in \mathcal A\), 
	\begin{align*}
		\bs a\int_{x\in\calD_{\ell_0}}\bs De^{\bs Sx_{m+1}}{\bs v}(x) 
		&= \bs a\int_{x\in\calD_{\ell_0}}\int_{u=0}^\infty e^{\bs Su}\bs s\cfrac{\bs \alpha e^{\bs S u}}{\bs \alpha e^{\bs S u}\bs e}e^{\bs Sx_{m+1}}{\bs v}(x)\wrt u\wrt x
		\\&\leq \bs a\int_{u=0}^\infty e^{\bs Su}\bs s\cfrac{\bs \alpha e^{\bs S u}}{\bs \alpha e^{\bs S u}\bs e}e^{\bs Sx_{m+1}}{\bs e}\wrt u
		\\&= \bs a\bs D e^{\bs Sx_{m+1}}{\bs e}\wrt u.
	\end{align*}
	
    Therefore (\ref{eqn:llkjhslkj}) is less than or equal to 
	\begin{align}
		&\int_{x_1=0}^\infty \bs H^{+-}(\lambda,x_1)\nonumber
		\int_{x_2=0}^\infty \bs H^{-+}(\lambda,x_2) 
		\hdots \int_{x_m=0}^\infty \bs H^{-+}(\lambda, x_m) 
		\int_{x_{m+1}=0}^\infty \bs H^{++}(\lambda,x_{m+1}) 
		\\&\quad\bs   a_{\ell_0,i}(x_0)e^{\bs{S}x_1}\wrt x_1 \bs{D}e^{\bs{S}x_2} \wrt x_2 \bs{D}\dots e^{\bs{S}x_m} \wrt x_m \bs{D}e^{\bs{S}x_{m+1}}{\bs e} \wrt x_{m+1} F\nonumber 
		%
		\\&\leq\int_{x_1=0}^\infty \bs H^{+-}(\lambda,x_1)\nonumber
		\int_{x_2=0}^\infty \bs H^{-+}(\lambda,x_2) 
		\hdots \int_{x_m=0}^\infty \bs H^{-+}(\lambda, x_m) 
		\int_{x_{m+1}=0}^\infty \bs H^{++}(\lambda,x_{m+1}) 
		\\&\quad\wrt x_1  \wrt x_2 \dots \wrt x_{m+1}  F \label{eqn :NNeeaefjn}
	\end{align}
	where the inequality follow by noting that \(\bs a_{\ell_0,i}(x_0)e^{\bs{S}x_1} \bs{D}e^{\bs{S}x_2} \bs{D}\dots e^{\bs{S}x_m} \bs{D}e^{\bs{S}x_{m+1}}{\bs e}\leq 1\). To see this, consider 
	\begin{align*}
		&\bs a_{\ell_0,i}(x_0)e^{\bs{S}x_1} \bs{D}e^{\bs{S}x_2} \bs{D}\dots e^{\bs{S}x_m} \bs{D}e^{\bs{S}x_{m+1}}{\bs e}
		\\& = \bs a_{\ell_0,i}(x_0)e^{\bs{S}x_1} \bs{D}e^{\bs{S}x_2} \bs{D}\dots e^{\bs{S}x_m} \int_{u=0}^\infty e^{\bs Su}\cfrac{\bs \alpha e^{\bs S}u}{\bs \alpha e^{\bs Su}\bs e}\wrt u e^{\bs{S}x_{m+1}}{\bs e}
		\\&\leq \bs a_{\ell_0,i}(x_0)e^{\bs{S}x_1} \bs{D}e^{\bs{S}x_2} \bs{D}\dots e^{\bs{S}x_m} \int_{u=0}^\infty e^{\bs Su}\cfrac{\bs \alpha e^{\bs S}u}{\bs \alpha e^{\bs Su}\bs e}\wrt u {\bs e}
		\\&=\bs a_{\ell_0,i}(x_0)e^{\bs{S}x_1} \bs{D}e^{\bs{S}x_2} \bs{D}\dots e^{\bs{S}x_m} \bs e
	\end{align*}
	where the inequality holds since \(\bs a_{\ell_0,i}(x_0)e^{\bs{S}x_1} \bs{D}e^{\bs{S}x_2} \bs{D}\dots e^{\bs{S}x_m}e^{\bs Su}\bs s\geq 0\) and \(\bs \alpha e^{\bs Su}e^{\bs{S}x_{m+1}}{\bs e}\geq 0\). Repeating \(m\) more times gives \(\bs a_{\ell_0,i}(x_0)\bs e=1\). 
	        
	Since all of the elements of \(\bs H^{++}(\lambda,x_{m+1}) \bs e\) are non-negative, then, for any length \(|\calS_+|+|\calS_{+0}|\) row-vector \(\bs b\), 
	\[\bs b\int_{x_{m+1}} \bs H^{++}(\lambda,x_{m+1}) \bs e \wrt x_{m+1} \leq \bs b \bs e \widehat{\bs G}.\]

	Therefore (\ref{eqn :NNeeaefjn}) is less than or equal to 
	\begin{align}
		&\int_{x_1=0}^\infty \bs H^{+-}(\lambda,x_1)
		\int_{x_2=0}^\infty \bs H^{-+}(\lambda,x_2) 
		\hdots \int_{x_m=0}^\infty \bs H^{-+}(\lambda, x_m) 
		\wrt x_1  \wrt x_2 \dots \wrt x_{m} \widehat{G} F \label{eqn :NNeeaefjn12}
	\end{align}

	The stochastic interpretation of the \(i\)th element of the vector \(\bs H^{+-}(\lambda,x)\bs e\) is that it is the probability density that the phase of the fluid queue changes from a positive to a negative phase at the time when the in-out fluid has increased by \(\wrt x\) and before an exponential random variable with rate \(\lambda\) occurs, given the phase is initially \(i\). There may be multiple changes of phase within \(\mathcal S_+\cup\calS_{+0}\) before the first change from \(\mathcal \mathcal S_+\cup\calS_{+0}\) to \(\mathcal S_-\). The first change of phase occurs at rate (with respect to the in-out level) \(-T_{ii}/|c_i|\) and this is the lowest in-out fluid level at which it may be possible for us to see a transition from \(\mathcal S_+\) to \(\mathcal S_-\). Consider a uniformised version of the in-out fluid process with uniformisation parameter \(q = \max\limits_{i\in\mathcal S\setminus \calS_0}-T_{ii}/|c_i|\). Then the first event of the phase process of the uniformised version of the in-out fluid process occurs at rate \(q\) and occurs at, or before, the first change of phase of the uniformised process. Therefore, the first uniformisation event occurs at, or before, the first change from \(\mathcal S_+\cup\calS_{+0}\) to \(\mathcal S_-\) of the uniformised version of the in-out process. Hence, the first uniformisation event occurs at, or before, the first change of phase from \(\mathcal S_+\cup\calS_{+0}\) to \(\mathcal S_-\) of the original process (since they are versions of each other). This gives the bound \(\bs b\bs H^{+-}(\lambda,x)\bs e\leq qe^{-(\lambda + q)x}\bs e\), for any length \(|\calS_+|+|\calS_{+0}|\) row-vector of non-negative number \(\bs b\).
	
	Similarly for \(\bs b\bs H^{-+}(\lambda,x)\bs e\leq qe^{-(\lambda + q)x}\bs e\), for any length \(|\calS_-|+|\calS_{-0}|\) row-vector of non-negative number \(\bs b\)
	
	From this stochastic interpretation above, (\ref{eqn :NNeeaefjn12}) is less than or equal to 
	\begin{align}
	%
	&\bs H^{+-}(\lambda,x_1) \wrt x_1 \int_{x_2=0}^\infty \bs H^{-+}(\lambda,x_2)  \wrt x_2  
				\hdots \int_{x_m=0}^\infty qe^{(-q-\lambda)x_m}\wrt x_{2m}\widehat GF\nonumber
	%
	\\&\leq \bs e\int_{x_1=0}^\infty qe^{(-q-\lambda)x_1}  \wrt x_1 \int_{x_2=0}^\infty qe^{(-q-\lambda)x_2}  \wrt x_2  
				\hdots \int_{x_{2m}=0}^\infty qe^{(-q-\lambda)x_{2m}}\wrt x_{2m}\widehat GF \nonumber
	%
	\\&= \bs e\left(\cfrac{q}{q+\lambda}\right)^{2m}\widehat GF.\label{eqn: bound ggggaaaa}
	\end{align}
	Hence,  
	\begin{align}
			&\sum_{m=M+1}^\infty \int_{x\in\calD_{\ell_0}} \widehat f^{\ell_0}_{m,+,+}(\lambda)(x,j;x_0,i)|\psi(x)|\wrt x \nonumber
		\\&\leq  \widehat GF  \sum_{m=M+1}^\infty \left(\cfrac{q}{q+\lambda}\right)^{2m}\int_{x\in\calD_{\ell_0}} |\psi(x)|\wrt x \nonumber
		\\&\leq \widehat GF \left(\cfrac{q}{q+\lambda}\right)^{2M+2} \left(1-\left(\cfrac{q}{q+\lambda}\right)^2\right)^{-1} \Delta F.
	\end{align}
	
	Now consider \(\widehat\mu_{m,+,+}^{\ell_0}(\lambda)( x,j;x_0,i) \) which is given by the \((i,j)\)th entry of 
	\begin{align}
	\nonumber& \int_{x_1 = 0}^{\Delta-(x_0-y_{\ell_0})} \bs H^{+-}(\lambda,\Delta-(x_0-y_{\ell_0})-x_1) \int_{x_2 = 0}^{\Delta-x_1} \bs H^{-+}(\lambda,\Delta - x_2-x_1) \wrt x_1 
	\\\nonumber &\quad\dots  
	\int_{x_{2m}=0}^{\Delta-x_{m-1}} \bs H^{-+}(\lambda,\Delta -x_{2m-1} - x_{2m}) \wrt x_{2m-1}
		\bs H^{++}(\lambda,\Delta -x_{2m}- (y_{\ell_0+1}- x))\wrt x_{2m}
	\\\nonumber&= \int_{x_1 = (x_0-y_{\ell_0})}^{\Delta} \bs H^{+-}(\lambda,\Delta-x_1) \int_{x_2 = x_1}^{\Delta} \bs H^{-+}(\lambda,\Delta -x_2) \wrt x_1 
	\dots  
	\\&\int_{x_{2m}=x_{2m-1}}^{\Delta} \bs H^{-+}(\lambda,\Delta - x_{2m}) 
	\bs H^{++}(\lambda,\Delta -x_{2m}-x_{2m-1}- (y_{\ell_0+1}- x)) 
	\wrt x_{2m-1}\wrt x_{2m}.\label{eqn:m789J}
	\end{align}
	Using the bound \(h^{++}_{j_m,j}(\lambda,x_{m+1})\leq G\), then (\ref{eqn:m789J}) is less than or equal to 
	\begin{align}
			&\int_{x_1 = (x_0-y_{\ell_0})}^{\Delta} \bs H^{+-}(\lambda,\Delta-x_1) \int_{x_2 = x_1}^{\Delta} \bs H^{-+}(\lambda,\Delta -x_2) \wrt x_1 \nonumber
	\\&\quad\dots  
	\int_{x_{2m}=x_{2m-1}}^{\Delta} \bs H^{-+}(\lambda,\Delta - x_{2m})  
	\wrt x_{2m-1}\wrt x_{2m}G.\label{eqn:m789J2}
	\end{align}
	The expression (\ref{eqn:m789J2}) differs from (\ref{eqn :NNeeaefjn12}) only by a constant factor and that the integrals in the former are not finite, hence we may bound it in the same way. Therefore, 
	\begin{align}
			& \sum_{m=M+1}^\infty \int_{x\in\calD_{\ell_0}} |\psi(x)|\wrt x \widehat \mu^{\ell_0}_{m,+,+}(\lambda)(x,j;x_0,i) |\psi(x)|\wrt x \nonumber
		\\&\leq G\left(\cfrac{q}{q+\lambda}\right)^{2M+2} \left(1-\left(\cfrac{q}{q+\lambda}\right)^2\right)^{-1}\Delta F.
	\end{align}
        
	Analogous arguments show the same bounds for any \(i,j\in\mathcal S\). 
\end{proof}

\begin{lem} \label{lem:vn4}
	For all \(x\in\calD_{\ell_0,j}\), \(x_0\in\calD_{\ell_0,i}\), \(i,j\in\calS\), \(\ell_0\in\mathcal K\), \(\lambda > 0\),  
	\begin{align}
		&\left|\int_{x\in\calD_{\ell_0}}\widehat f^{\ell_0,(p)}(\lambda)(x,j;x_0,i)\psi(x) \wrt x - \int_{x\in\calD_{\ell_0}}\widehat \mu^{\ell_0}(\lambda)( x,j; x_0,i)\psi(x) \wrt x\right|\to 0  \label{eqn: akhv}
%		\leq R(i,j,M).
	\end{align}
	as \(p\to\infty\). 
\end{lem}
\begin{proof}
	 
	
	Consider \(i\in\calS_+\) and \(j\in\calS_+\cup\calS_{+0}\). By partitioning on the number of changes from \(\calS_-\to\calS_+\), (\ref{eqn: akhv}) can be written as
	\begin{align}
		&\left|\sum_{m=0}^\infty \int_{x\in\calD_{\ell_0}}\widehat f^{\ell_0,(p)}_{m,+,+}(\lambda)(x,j;x_0,i)\psi(x)\wrt x
		-
		\sum_{m=0}^\infty \int_{x\in\calD_{\ell_0}} \widehat \mu^{\ell_0}_{m,+,+}(\lambda)(x,j;x_0,i)\psi(x)\wrt x\right| \nonumber
		\\&\leq \sum_{m=0}^\infty \left| \int_{x\in\calD_{\ell_0}}\widehat f^{\ell_0,(p)}_{m,+,+}(\lambda)(x,j;x_0,i)\psi(x)\wrt x
		-
		\int_{x\in\calD_{\ell_0}} \widehat \mu^{\ell_0}_{m,+,+}(\lambda)(x,j;x_0,i)\psi(x)\wrt x\right|. \label{eqn:ASLKF}
        \end{align}
        By Theorem~\ref{thm: a thm!} each term in the sum (\ref{eqn:ASLKF}) converges to \(0\) as \(p\to\infty\). Lemma~\ref{lem: gkjljklgagjklagsjlk} gives a domination condition, so we can apply the Dominated Convergence Theorem which proves the stated convergence in (\ref{eqn: akhv}).
        
        Analogous arguments can be applied for any \(i,j\in\mathcal S\). 
\end{proof}


\begin{rem}\label{rem: point wies}
	For a fixed \(\lambda > 0\), convergence of 
	\begin{align}
		\left|\widehat f^{\ell_0,(p)}(\lambda)(x,j;x_0,i) - \widehat \mu^{\ell_0}(\lambda)( x,j; x_0,i) \right|
	\end{align}
	actually holds point-wise for each \(\ell_0\in\mathcal K\setminus\{-1,K+1\}\), and each \(i,j\in\mathcal S,\) \(x_0\in\mathcal D_{\ell_0,i}\), \(x\in\calD_{\ell_0,j}\) except at the set of points where \(x=x_0\). Specifically, the lack of point-wise convergence at this point occurs due to terms with the index \(m=0\), that is, terms where there are no changes of phase from \(\calS_+\to\calS_-\) or \(\calS_-\to \calS_+\). On these sample paths the relevant Laplace transforms of the fluid queue are discontinuous at this point. For example, 
	\begin{align*}
		\widehat \mu^{\ell_0}_{0,+,+}(\lambda)( x,j;x_0,i) \wrt x&= h_{ij}^{++}(\lambda,x-x_0)1(x\geq x_0)\wrt x,
	\end{align*}
	is discontinuous at \(x=x_0\). %If we were to insist on point-wise convergence, then we would need to enforce the error term \(r_{\bs v}(u,v)^{(p)}\to 0\) point-wise for each \(u,v\). In the cases presented here \(r_{\bs v}^{(p)}(u,v)\) converges point-wise to 0 everywhere except \(u+v=\Delta\). 
\end{rem}

