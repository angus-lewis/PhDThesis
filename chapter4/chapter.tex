%!TEX root = ../thesis.tex
\chapter{Convergence of the QBD-RAP before the first orbit restart epoch, \(\tau_1\)\label{sec: conv}}
This chapter details a convergence of the approximation scheme constructed in Chapter~\ref{sec: construction and modelling} on the event the first \emph{orbit restart epoch} is yet to occur. Unless the QBD-RAP hits a boundary and is immediately reflected, an orbit restart epoch corresponds to a change of level. If the process hits a boundary and is immediately reflected, then there is no change of level, but the orbit process does `restart' at this time. We will define orbit restart epochs more precisely, later. From the stochastic interpretation in Chapter~\ref{sec: construction and modelling}, the orbit restart epochs approximate the hitting times of the fluid queue on the points \(\{y_\ell\}\) when the fluid level is not at a boundary, or the exit times of the boundaries when sticky boundaries are present and the fluid queue is at the boundary. Thus, this chapter proves a convergence of the QBD-RAP scheme to the fluid queue in each of the sets \(\{0\}, \mathcal D_{0}, \mathcal D_1, \dots, \mathcal D_{K}, \{y_{K+1}\}\).

In Chapter~\ref{ch: global results}, we use the main results of this chapter to prove further convergence results for the QBD-RAP scheme, and ultimately provide a global result. Conceptually Chapter~\ref{ch: global results} stitches together the convergence on each of the sets \(\{0\}, \mathcal D_{0}, \dots, \mathcal D_{K}, \{y_{K+1}\}\) proved in this chapter to claim the global convergence. Chapters~\ref{sec: conv} and~\ref{ch: global results} differ somewhat in their proof techniques: Chapter~\ref{sec: conv} relies more heavily on concentrated-matrix-exponential-specific arguments whereas Chapter~\ref{ch: global results} uses more traditional arguments such as the Markov property, time-homogeneity and the law of total probability. We now detail the QBD-RAP scheme with which we will work throughout this chapter, and detail the structure of the chapter.

Recall that the QBD-RAP is constructed using \emph{matrix exponential distributions} to model, approximately, the sojourn time of the fluid queue in a given interval. This chapter shows a type of convergence under the assumption that the variance of the matrix exponential distribution(s) used in the construction tends to 0. The result applies to any sequence of matrix exponential distributions such that the variance tends to zero. The generality of the result is necessitated by the fact that, in practice, we use the class of \emph{concentrated matrix exponential distributions} found numerically in \citep{hht2020}, for which there are relatively few known properties.

In this chapter we work exclusively with the \emph{augmented state space model} to model phases with rates \(c_i=0\) as described in Section~\ref{sec: zero states}. 

Further, to model an initial condition, \(\varphi(0)=k\in\calS_0\), we use the ephemeral set of phases \(\calS_0^{*,k}\) as described in Section~\ref{sec: initial conditions}. Approximating the initial condition \(\varphi(0)=k\in\calS_0\) in this way greatly simplifies some results in this chapter. In Appendix~\ref{app:extend conv} we provide results which prove convergence of the approximation without the need to model the initial condition \(\varphi(0)=k\in\calS_0\) by an ephemeral set of phases. Appendix~\ref{app:extend conv} relies on the fact that \(\bs a_{\ell_0,i}(\Delta - x_0)\) and \(\bs a_{\ell_0,i}(x_0)\bs D\) are `close', in some sense, then leverages the results of this chapter to prove various bounds which ultimately show convergence. Beyond that, Appendix~\ref{app:extend conv} provides limited further insight into the QBD-RAP process, it is also long and somewhat tedious, hence why it is in an appendix. 

In this chapter we analyse the distribution of the QBD-RAP scheme up to the first orbit restart epoch. The structure of the analysis is to first partition the distribution of the QBD-RAP at time \(t\) (where \(t\) is before the first orbit restart epoch) on the number of changes of phase from \(\calS_+\cup\calS_{+0}\) to \(\calS_-\) or \(\calS_-\cup\calS_{-0}\) to \(\calS_+\). We will refer to changes of phase from \(\calS_+\cup\calS_{+0}\) to \(\calS_-\) as \emph{up-down} transitions and changes of phase from \(\calS_-\cup\calS_{-0}\) to \(\calS_+\) as \emph{down-up} transitions. Next, for each term in the partition we take the Laplace transform with respect to time. This is convenient as it enables algebraic manipulations such that we can separate the Laplace transforms into one factor solely about the orbit process of the QBD-RAP and one expression about the phase process and associated rates. Once we have established a convenient algebraic form, we then turn our attention to bounds and convergence, establishing bounds for the difference between the Laplace transforms of the QBD-RAP just described and corresponding Laplace transforms of the fluid queue. Thus, we establish a convergence result for the Laplace transforms with respect to time for each of the distributions in the partition. We then wish to `undo' the partitioning on the event of a given number of up-down and down-up transitions before the first orbit restart epoch to establish a convergence result for the Laplace transform on the event that the first orbit restart epoch is yet to occur. To `undo' the partitioning, we use the Dominated Convergence Theorem. 

The main steps of the convergence result of this chapter are listed below.
\begin{enumerate}
	\item\label{step 1} Define the partition on the number of up-down and down-up transitions and the collections sample paths of the QBD-RAP and fluid queue with which we will work. Describe the distributions of the processes on theses sample paths and compute their Laplace transforms with respect to time. (Sections~\ref{sec: qbd dists},~\ref{sec: no change} and~\ref{sec: lst on no change}).
	\item\label{step 2} Show error bounds for the difference between the Laplace transforms of the QBD-RAP and the fluid queue for each term in the partition. (Section~\ref{sec: no change convergence}).
	\item\label{step 3} Show a geometric domination condition so that we may apply the Dominated Convergence Theorem to `undo' the partitioning and hence prove convergence of the Laplace transforms. (Section~\ref{sec: before the first}).
\end{enumerate}

First we give some preliminaries and some technical assumptions we will use in the proofs. 

\section*{Preliminaries}
Suppose we have a sequence, \(\{Z^{(p)}\}_{p\geq 1}\), of matrix exponential random variables with, \(Z^{(p)}\sim ME(\bs\alpha^{(p)},\bs S^{(p)}, \bs s^{(p)})\), such that \(\mathbb E[Z^{(p)}] = \Delta\) and \(\var\left(Z^{(p)}\right)\to 0 \) as \(p\to \infty\). For notational convenience we suppose that \(p\) is the order of the representation \((\bs\alpha^{(p)},\bs S^{(p)}, \bs s^{(p)})\), but strictly speaking, this is not necessary. We use the superscript \((p)\) to denote dependence on the underlying choice of matrix exponential distribution that is used in the construction of the QBD-RAP scheme. To simplify notation, we may omit the superscript \((p)\) when it is not necessary. 

In the following we show error bounds for an arbitrary parameter \(\varepsilon>0\). However, keep in mind the ultimate intention is to show convergence, for which we choose this parameter to be \(\varepsilon^{(p)}=\var\left(Z^{(p)}\right)^{1/3}\). Other notations that have been defined, which are functions of \(Z^{(p)}\) and therefore also implicitly depend on \(p\), are \(\bs\alpha^{(p)},\,\bs S^{(p)},\,\bs s^{(p)},\, \bs S_i^{(p)},\, \bs s_i^{(p)},\, \bs D^{(p)},\,\mathcal A^{(p)},\) \(\bs Y^{(p)}(t) = (L^{(p)}(t),\bs A^{(p)}(t), \phi^{(p)}(t)),\, \Ydp(t),\, \bs y_0^{(p)}.\)

In the following we show various results which involve integrating a function \(g\), or a sequence of functions \(g_1,g_2,\dots\). We make the following assumptions about such functions, 
\begin{asu}\label{asu: g}
	Let \(g\) be a function \(g:[0,\infty)\to [0,\infty)\) which is \\
	\subasu \label{asu: g non-neg} non-negative, 
	\[g(x) \geq 0 \mbox{ for all } x \geq 0,\]
	\subasu bounded, 
	\[g(x) \leq G < \infty \mbox{ for all } x \geq 0,\]
	\subasu integrable, 
	\[\int_{x=0}^\infty g(x)\wrt x \leq \widehat G < \infty,\]
	\subasu \label{asu: lipschitz} and Lipschitz continuous 
	\begin{align}
		|g(x) - g(u)|&\leq L|x - u| \mbox{ for all } x,\, u \geq 0,\, 0<L<\infty.
	\end{align}
\end{asu}

We also need a sequence of closing operators which we denote by \({\bs v}^{(p)}\). For the convergence results, we require the following properties of the closing operators \({\bs v}^{(p)}(x),\, x \in[0,\Delta)\).
\begin{property}\label{properties: some props}
	Let \(\{\bs v^{(p)}(x)\}_{p\geq 1}\) be a sequence of closing operators such that they may be decomposed into \(\bs v^{(p)}(x)=\bs w^{(p)}(x) + \widetilde{\bs w}^{(p)}(x)\), where; \\
	\subproperty \label{properties: -1} for \(x\in[0,\Delta),u,v\geq 0\),  
        \begin{align*}
        		\bs \alpha^{(p)} e^{\bs S^{(p)}(u+v)}(-\bs S^{(p)})^{-1} \widetilde{\bs w}^{(p)}(x) &\leq \bs \alpha^{(p)} e^{\bs S^{(p)}u}(-\bs S^{(p)})^{-1} \widetilde{\bs w}^{(p)}(x).
		\end{align*}
	\subproperty \label{properties: 0} for \(x\in[0,\Delta),u\geq 0\),
		\begin{align*}
			\bs \alpha^{(p)} e^{\bs S^{(p)}u}(-\bs S^{(p)})^{-1} \widetilde{\bs w}^{(p)}(x) &=\widetilde G_{\bs v}^{(p)} \to 0,\, \mbox{ as }p \to \infty.  
		\end{align*}
	\subproperty \label{properties: 1} for \(x\in[0,\Delta),u\geq 0\),  
        \begin{align*}
        		\bs \alpha^{(p)} e^{\bs S^{(p)}u}(-\bs S^{(p)})^{-1} \bs w^{(p)}(x) &\leq \bs \alpha^{(p)} e^{\bs S^{(p)}u} \bs e G_{\bs v},
	\end{align*}
	for some \(0\leq G_{\bs v}<\infty\) independent of \(p\) for \(p>p_0\) where \(p_0<\infty\). \\
	\subproperty \label{properties: -2} for \(\bs a \in\mathcal A,\,u\geq 0\),  
	\begin{align*}
			\int_{x\in[0,\Delta)}\bs a^{(p)} e^{\bs S^{(p)}u} {\bs v}^{(p)}(x) \wrt x&\leq \bs a^{(p)} e^{\bs S^{(p)}u} \bs e.
	\end{align*}
	\subproperty \label{properties: 2} Let \(g\) be a function satisfying the Assumptions~\ref{asu: g}. For \(u\leq \Delta-\varepsilon^{(p)}\), \(v\in[0,\Delta)\), then
	\[\left|\int_{x=0}^\infty \cfrac{\bs \alpha^{(p)} e^{\bs{S}^{(p)}(u+x)} }{\bs \alpha^{(p)} e^{\bs{S}^{(p)}u} \bs e} {\bs v}^{(p)}(v)g(x)\wrt x -g(\Delta-u-v) 1(u+v\leq\Delta-\varepsilon^{(p)})\right| =  |r_{\bs v}^{(p)}(u,v)|,\]
	where 
	\[ \int_{u=0}^{\Delta}\left| r_{\bs v}^{(p)}(u,v)\right| \wrt u  \leq R_{{\bs v},1}^{(p)} \to 0\]
	and 
	\[ \int_{v=0}^{\Delta}\left| r_{\bs v}^{(p)}(u,v)\right| \wrt v  \leq R_{{\bs v},2}^{(p)} \to 0\]
	as \(\var(Z^{(p)})\to 0\). 
\end{property}

In Appendix~\ref{appendix: sec: 2} we provide results which show that the closing operators (\ref{eqn: alal}) - (\ref{eqn:5462}) satisfy Properties~\ref{properties: some props}. 

Though it is a slight abuse of notation, for convenience, let us write 
\begin{align*}
	& \mathbb P({\bs Y}(t) \in (\ell,\wrt x, j)\mid \bs Y(0)=\bs y_0 )
\end{align*}
in place of 
\begin{align}
	\int_{\bs a \in\mathcal A}\mathbb P({\bs Y}(t) \in (\ell,\wrt \bs a, j)\mid \bs Y(0)=\bs y_0)\bs a\bs v_{\ell,j}(x)\wrt x \label{eqn:GSW}
\end{align}
where \(\bs v_{\ell,j}(x)\) is a closing operator. Expression~(\ref{eqn:GSW}) is an approximation to 
\begin{align}
	\mathbb P(\bs X(t)\in(\wrt x, j)\mid \bs X(0)=(x_0, i)),
\end{align}
\(x\in \calD_{\ell,j}\), \(x_0\in\calD_{\ell_0,i}\).
Further, let us write 
\begin{align*}
	& \mathbb P({\bs Y} (t) \in (\ell,E, j)\mid \bs Y (0)=\bs y_0 )
\end{align*}
in place of 
\begin{align}
	\int_{x\in E}\int_{\bs a \in\mathcal A}\mathbb P({\bs Y} (t) \in (\ell,\wrt \bs a, j)\mid \bs Y (0)=\bs y_0  )\bs a\bs v_{\ell,j}(x)\wrt x \label{eqn:GSW2}
\end{align}
for some measurable set \(E\subseteq \mathcal D_{\ell,j}\). 

Ultimately, in Chapter~\ref{ch: global results}, we will apply the Extended Continuity Theorem for Laplace transforms \cite[Chapter XIII, Theorem 2a]{feller1957} to claim convergence. The Extended Continuity Theorem for Laplace transforms requires us to show convergence of the Laplace transform pointwise with respect to the transform parameter, \(\lambda\), on the set \(\lambda\in\mathbb R,\, \lambda>0\). Therefore, we can fix \(\lambda\in\mathbb R,\, \lambda>0\) in the following. 

\section{Describing the distribution of the QBD-RAP before \(\tau_1\)}\label{sec: qbd dists}
In this chapter we are interested in the QBD-RAP up to the first orbit restart epoch, which we denote by \(\tau_1^{(p)}\), which is the random (stopping) time at which the QBD-RAP changes level, or hits the boundary, or exits a boundary, for the first time. More precisely, 
\[\tau_1^{(p)} = \inf\left\{t>0\mid L^{(p)}(t)\neq L^{(p)}(0),\mbox{ or }(L^{(p)}(t),\bs A^{(p)}(t),\varphi(t))\mbox{ hits a boundary}\right\}.\]
At this time, the orbit process is restarted at the initial value \(\bs A^{(p)}(\tau_1^{(p)}) = \bs \alpha^{(p)}\), unless the QBD-RAP hits a sticky boundary and is temporarily absorbed, in which case the orbit process is restarted at the value \(\bs A^{(p)}(\tau_1^{(p)}) = 1\).\footnote{Recall from the discussion in Section~\ref{sec: boundary conditions} in the paragraph above Remark~\ref{rem: 111}, that the orbit process is not actually required to model the behaviour at the boundary. We set it to be \(\bs A(t) =1\) for all times \(t\) when the QBD-RAP is at the boundary for notational convenience.}

Consider the initial condition \(\bs y_0^{(p)} = (L^{(p)}(0),\bs A^{(p)}(0),\varphi(0))=(\ell_0,\bs a^{(p)}_{\ell_0,i}(x_0),i)\), where \(\ell_0\in\mathcal K^\circ\), and \(i\in\calS\). The value \(\bs y_0\) is the approximation to the initial condition \(\bs X(0)=(x_0,i)\). We are interested in the quantity, \(f^{\ell_0,(p)}(t)(x,j;x_0,i)\wrt x\) given by
\begin{align}
	&\int_{\bs a \in \mathcal A^{(p)}}\mathbb P\Big(\bs A^{(p)}(t)\in \wrt \bs a, \varphi(t) = j , t<\tau_1^{(p)}\nonumber
	\mid \bs Y^{(p)}(0) = \bs y_0^{(p)}\Big)
	\bs a{\bs v}_{\ell_0,j}^{(p)}(x)\wrt x \nonumber 
	\\&= (\bs e_i\otimes \bs  a_{\ell_0,i}^{(p)}(x_0)) e^{\bs{B}^{(p)}t}(\bs e_j\tr{} \otimes {\bs v}_{\ell_0,j}^{(p)}( x))\wrt x, \label{eqn: ldll}
\end{align}
which is the QBD-RAP approximation to the distribution 
\[\mathbb P(\bs X(t)\in (\wrt x,j), t<\tau_1^X \mid \bs X(0) = (x_0, i)).\]

For now consider \(\varphi(0)=i\in\calS_+\cup\calS_-\). As we shall see, certain Laplace transform expressions for phases \(i^*\in\calS_0^{*,k}\) with rate \(c_{i^*}=0\) can be written as a linear combination of certain Laplace transforms of phases in \(\calS_+\cup\calS_-\). 

Now, introduce a partition on the number of up-down and down-up transitions of the sample paths. Denote by \(\{\Sigma_m\}_{m\geq 1}\) the sequence of (stopping) times at which \(\{\varphi(t)\}\) has an up-down transition (i.e.~jumps from \(\mathcal S_+\cup\calS_{+0}\) to \(\mathcal S_- \)) for the \(m\)th time. Denote by \(\{\Gamma_m\}_{m\geq 1}\) the sequence of (stopping) times at which \(\{\varphi(t)\}\) has a down-up transition (i.e.~jumps from \(\mathcal S_-\cup\calS_{-0}\) to \(\mathcal S_+\)) for the \(m\)th time. More precisely, for sample paths with \(\varphi(0)\in\mathcal S_+\), let \(\Gamma_0=0\), then for \(m\geq 1\), 
\begin{align}
	\Sigma_m &:=\inf\{t > \Gamma_{m-1} \mid \varphi(t)\in\mathcal S_-\}, 
	\\ \Gamma_m &:=\inf\{t > \Sigma_{m} \mid \varphi(t)\in\mathcal S_+\}.
\end{align}
Similarly, for sample paths with \(\varphi(0)\in\mathcal S_-\), let \(\Sigma_0=0\), then for \(m\geq 1\), 
\begin{align}
	\Gamma_m &:=\inf\{t > \Sigma_{m-1} \mid \varphi(t)\in\mathcal S_+\},
	\\\Sigma_m &:=\inf\{t > \Gamma_{m} \mid \varphi(t)\in\mathcal S_-\}.
\end{align}
For times \(t\) such that \(\Gamma_m\leq t<\Sigma_{m+1}\), then \(\varphi(t)\in\mathcal S_+\cup\calS_{+0}\). For times \(t\) such that \(\Sigma_{m+1}\leq t< \Gamma_{m+1}\), then \(\varphi(t)\in\mathcal S_-\cup\calS_{-0}\).


With these stopping times, partition the sample paths of the QBD-RAP by the number of up-down and down-up transition as follows. For \(x_0\in\calD_{\ell_0,i},\) \(x\in\calD_{\ell_0,j},\) \(t\geq0,\) \(\ell_0\in\mathcal K^\circ,\) \(m\geq 0\), and for \(i\in\calS_+,j\in\mathcal S_+\cup\mathcal S_{+0}\), let \(f^{\ell_0,(p)}_{m,+,+}(t)( x,j; x_0,i) ,\) be 
\begin{align}
	&\int_{\bs a \in\mathcal A^{(p)}}\mathbb P\Big(\bs A^{(p)}(t)\in \wrt \bs a, \varphi(t) = j, t<\tau_1^{(p)}, \Gamma_m\leq t<\Sigma_{m+1} \nonumber
	\mid \bs Y^{(p)}(0)=\bs y_0^{(p)}\Big)
	\bs a{\bs v}_{\ell_0,j}^{(p)}(x) \nonumber
	%
	\\
	&=\int_{\sigma_1=0}^t (\bs e_i\otimes \bs  a_{\ell_0,i}^{(p)}(x_0)) e^{\bs{B}^{(p)}_{++}\sigma_1}\bs{B}^{(p)}_{+-}	\nonumber
	\int_{\gamma_1=\sigma_1}^{t} e^{\bs{B}^{(p)}_{--}(\gamma_1-\sigma_1)}\bs{B}^{(p)}_{-+}
	\hdots 
	 \int_{\gamma_m=\sigma_{m}}^{t} e^{\bs{B}^{(p)}_{--}(\gamma_m-\sigma_{m})}\bs{B}^{(p)}_{-+}\\&\quad\times
	e^{\bs{B}^{(p)}_{++}(t-\gamma_m)}\left(\bs e_j\tr{}  \otimes {\bs v}_{\ell_0,j}^{(p)}(x)\right) 
	\wrt \gamma_m\wrt \sigma_m\dots \wrt \gamma_1\wrt \sigma_1 , \label{eqn: approx end conv}
\end{align}
where \(\bs e_k\) is a row-vector of zeros except the \(k\)th position which is a 1. 
Analogously, for \(i\in\mathcal S_+ ,\,j\in\calS_{-}\cup\mathcal S_{-0}\), let 
\begin{align}
	&f^{\ell_0,(p)}_{m+1,+,-}(t)(  x, j; x_0,i) \nonumber
	\\&= \int_{\bs a \in\mathcal A^{(p)}}\mathbb P\Big(\bs A^{(p)}(t)\in \wrt \bs a, \varphi(t) = j, t<\tau_1^{(p)}, \Sigma_{m+1}\leq t<\Gamma_{m+1}\mid 
	%
	\bs Y^{(p)}(0)=\bs y_0^{(p)}\Big)
	\bs a{\bs v}_{\ell_0,j}^{(p)}( x)  \nonumber 
	%
	\\&=\int_{\sigma_1=0}^t (\bs e_i\otimes \bs  a_{\ell_0,i}^{(p)}(x_0)) e^{\bs{B}^{(p)}_{++}\sigma_1}\bs{B}^{(p)}_{+-}	\nonumber
	\int_{\gamma_1=\sigma_1}^{t} e^{\bs{B}^{(p)}_{--}(\gamma_1-\sigma_1)}\bs{B}^{(p)}_{-+}
	\hdots 
	\int_{\sigma_{m+1}=\gamma_m}^t e^{\bs{B}^{(p)}_{++}(\sigma_{m+1}-\gamma_m)}
	\\&\quad\times \bs B^{(p)}_{+-}e^{\bs{B}^{(p)}_{--}(t-\sigma_{m+1})}\left(\bs e_j\tr{}  \otimes {\bs v}_{\ell_0,j}^{(p)}(x)\right) 
	\wrt \sigma_1\wrt \gamma_1\dots \wrt \sigma_m\wrt \gamma_m\wrt \sigma_{m+1}\label{eqn:gljagj}
	 %
	\intertext{for \( i\in\mathcal S_-  ,\, j\in\mathcal S_{+}\cup\calS_{+0}\) let}
	%
	&f^{\ell_0,(p)}_{m+1,-,+}(t)( x, j; x_0,i) \nonumber
	\\&= \int_{\bs a \in\mathcal A^{(p)}}\mathbb P\Big(\bs A^{(p)}(t)\in \wrt \bs a, \varphi(t) = j, t<\tau_1^{(p)},  \Gamma_{m+1}\leq t<\Sigma_{m+1} \mid 
	%
	\bs Y^{(p)}(0)=\bs y_0^{(p)}\Big)\nonumber
	  \bs a{\bs v}_{\ell_0,j}^{(p)}(x) 
	%
	\\&=\int_{\gamma_1=0}^t (\bs e_i\otimes \bs  a_{\ell_0,i}^{(p)}(x_0)) e^{\bs{B}^{(p)}_{--}\gamma_1}\bs{B}^{(p)}_{-+}	\nonumber
	\int_{\sigma_1=\gamma_1}^{t} e^{\bs{B}^{(p)}_{++}(\sigma_1-\gamma_1)}\bs{B}^{(p)}_{+-}
	\hdots 
	\int_{\gamma_{m+1}=\sigma_m}^t e^{\bs{B}^{(p)}_{--}(\gamma_{m+1}-\sigma_m)}
	\\&\quad\times \bs B^{(p)}_{-+}e^{\bs{B}^{(p)}_{++}(t-\gamma_{m+1})}\left(\bs e_j\tr{}  \otimes {\bs v}_{\ell_0,j}^{(p)}(x)\right) 
	\wrt \gamma_1\wrt \sigma_1\dots \wrt \gamma_m\wrt \sigma_m\wrt \gamma_{m+1}
	 %
	\intertext{and for \( i\in\calS_-,j\in\mathcal S_-\cup\calS_{-0}\) let}
	%
	&f^{\ell_0,(p)}_{m,-,-}(t)( x, j; x_0, i) \nonumber
	\\&= \int_{\bs a \in\mathcal A^{(p)}}\mathbb P\Big(\bs A^{(p)}(t)\in \wrt \bs a, \varphi(t) = j,  t<\tau_1^{(p)}, \Sigma_{m}\leq t<\Gamma_{m+1} \mid \bs Y^{(p)}(0)=\bs y_0^{(p)} \Big)\bs a{\bs v}_{\ell_0,j}^{(p)}(x) \nonumber 
	%
	\\&=\int_{\gamma_1=0}^t (\bs e_i\otimes \bs  a_{\ell_0,i}^{(p)}(x_0)) e^{\bs{B}^{(p)}_{--}\gamma_1}\bs{B}^{(p)}_{-+}	\nonumber
	\int_{\sigma_1=\gamma_1}^{t} e^{\bs{B}^{(p)}_{++}(\sigma_1-\gamma_1)}\bs{B}^{(p)}_{+-}
	\hdots 
	 \int_{\sigma_m=\gamma_{m}}^{t} e^{\bs{B}^{(p)}_{++}(\sigma_m-\gamma_{m})}\bs{B}^{(p)}_{+-}\\&\quad\times
	e^{\bs{B}^{(p)}_{--}(t-\sigma_m)}\left(\bs e_j\tr{}  \otimes {\bs v}_{\ell_0,j}^{(p)}(x)\right) 
	\wrt \sigma_m\wrt \gamma_m\dots \wrt \sigma_1 \wrt \gamma_1.\label{eqn: 67}
\end{align}
Now, for \(q,r\in\{+,-\},\, q\neq r\), define  
\begin{align*}
	f^{\ell_0,(p)}_{q,q}(t)(x,j;x_0,i)  &:= \sum_{m=0}^\infty f^{\ell_0,(p)}_{m,q,q}(t)(x,j;x_0,i)  & i\in\calS_q,\,j\in\calS_q\cup\calS_{q0},
	\\ f^{\ell_0,(p)}_{q,r}(t)(x,j;x_0,i)  &:= \sum_{m=1}^\infty f^{\ell_0,(p)}_{m,q,r}(t)(x,j;x_0,i)  & i\in\mathcal S_q,\,\,j\in\mathcal S_{0}\cup\calS_{r0},
\end{align*}
so that 
\begin{align}
	f^{\ell_0,(p)}(t)(x,j;x_0,i)  = \begin{cases}
		 f^{\ell_0,(p)}_{q,q}(t)(x,j;x_0,i)  & i\in\calS_q,\,j\in\calS_q\cup\calS_{q0},
	\\    f^{\ell_0,(p)}_{q,r}(t)(x,j;x_0,i)  & i\in\mathcal S_q,\,\,j\in\mathcal S_{r}\cup\calS_{r0}.
	\end{cases}\label{eqn:ghghghghggg}
\end{align}

Recall, in this chapter we suppose the QBD-RAP approximation uses ephemeral states \(\calS_0^{*,k}\) to model the fluid queue whenever the phase starts in \(k\in\calS_0\). In general, for \(r\in\{+,-\}\), \(m\geq 0\), we define
\begin{align}
	f_{m,0,r}^{\ell_0,(p)}(t)(x,j;x_0,k)  
	&:= \sum_{q\in\{+,-\}}\sum_{i\in\calS_q}\int_{t_0=0}^t \bs e_ke^{\bs T_{00}t_0} \bs T_{0i} f_{m+1(q\neq r),q,r}^{\ell_0,(p)}(t-t_0)(x,j;x_0,i)\wrt t_0 . \label{eqn: vma0}
\end{align}
Upon taking the Laplace transform of (\ref{eqn: vma0}) the convolutions become products, so the Laplace transform of \(f_{m,0,r}^{\ell_0}(t)(x,j;x_0,k)\) is a linear combination of the Laplace transforms of (\ref{eqn: approx end conv})-(\ref{eqn: 67}). Thus, once we show convergence for the Laplace transforms of (\ref{eqn: approx end conv})-(\ref{eqn: 67}) we get convergence of the Laplace transform for starting in \(\calS_0^{*,k}\) too. 


\section{Describing the distribution of the fluid queue before \(\tau_1\)}\label{sec: no change}
Let \(\tau_1^X\) be the minimum of the time at which \(\{X(t)\}\) hits a boundary, or exits a boundary, or exits \(\mathcal D_{\ell_0}\), where \(X(0)=x_0\in\mathcal D_{\ell_0}\). More precisely, 
\[\tau_1^X = \min\left\{\begin{array}{c}\inf\left\{t>0\mid X(t)=y_{\ell}, \ell\in\mathcal K\right\}, \\ \inf\left\{t>0 \mid X(t) \neq 0, X(0)=0\right\}, \\ \inf\left\{t>0 \mid X(t) \neq y_{K+1}, X(0)=y_{K+1}\right\} \end{array} \right\}.\]
Consider the measures on the Borel sets of \(\calD_{\ell_0,j}\) given by
\begin{align}\label{eqn: sojourn}
	\mu^{\ell_0}(t)(\cdot,j; x_0,i) := \mathbb P(\bs X(t)\in (\cdot ,j), t<\tau_1^X \mid \bs X(0) = (x_0, i)),
\end{align}
\(\ell_0\in\mathcal K^\circ\), \(x_0 \in\mathcal D_{\ell_0,i}, i,j\in\mathcal S, t \geq 0. \)
In words, this is the distribution of the fluid queue at time \(t\) on the event that the fluid level remains within \(\mathcal D_{\ell_0}\) up to and including time \(t\) and is in phase \(j\) at time \(t\), given that is started at \(X(0)=x_0\in\mathcal D_{\ell_0,i}\) in phase \(i\). 

I do not know of any simple expression for (\ref{eqn: sojourn}). There are expressions for the Laplace transform of (\ref{eqn: sojourn}) with respect to time. One is in terms of the of first return matrices \(\Psi(\lambda)\) and \(\Xi(\lambda)\) \citep{bean2009}. Here we opt for another expression for the Laplace transform which is obtained by partitioning as follows.

As before, we use the sequence of up-down transition times, \(\{\Sigma_m\}_{m\geq 1}\), and the sequence of down-up transition times, \(\{\Gamma_m\}_{m\geq 1},\) to partition sample paths. The events \(\{\Gamma_m\leq t< \Sigma_{m+1}\} , \mbox{ and } \{\Sigma_{m+1}\leq t< \Gamma_{m+1}\}, \) \(m\geq 0\), partition the sample paths of (\ref{eqn: sojourn}) into periods where the fluid is either non-decreasing or non-increasing, respectively; see Figure~\ref{fig: sample paths}.
\begin{figure}
    \centering\begin{tikzpicture}
    	\draw[->,thick] (0,0) -- (14,0);
    	\draw (14,-0.75) node {$x$};
    	\draw[-,thick] (0,0) -- (0,4.5);
    	\draw[-,thick] (-0.1,4) -- (0.1,4);
    	\draw (-0.75,4) node {$y_{\ell_0}+\Delta$};
    	\draw (-0.75,0) node {$y_{\ell_0}$};
    	\draw (0,-0.75) node {$\Gamma_0$};
            \draw[-,thick] (0,0) -- (2.6,2.6);
			\draw[-,thick] (2.6,2.6) -- (3.6,2.6); 
            \draw[-,thick] (3.6,-0.1) -- (3.6,0.1);
    	\draw (3.6,-0.75) node {$\Sigma_1$};
    	\draw[-,thick] (3.6,2.6) -- (5.6,2.6-2);
		\draw[-,thick] (5.6,0.6) -- (7,0.6);
		\draw[-,thick] (7,-0.1) -- (7,0.1);
    	\draw (7,-0.75) node {$\Gamma_1$};
		\draw[-,thick] (7,0.6) -- (7+0.8,1.4);
		\draw[-,thick] (7.8,1.4) -- (8.8,1.4);
		\draw[-,thick] (8.8,1.4) -- (8+3,3.6);
            	\draw[-,thick] (7+3+1,3.6) -- (7+4+1,2.6);
            \draw[-,thick] (7+3+1,-0.1) -- (7+3+1,0.1);
	    	\draw (7+3+1,-0.75) node {$\Sigma_2$};
    \end{tikzpicture}
	\begin{tikzpicture}
    	\draw[->,thick] (0,0) -- (14,0);
    	\draw (14,-0.75) node {$x$};
    	\draw[-,thick] (0,0) -- (0,4.5);
    	\draw[-,thick] (-0.1,4) -- (0.1,4);
    	\draw (-0.75,4) node {$y_{\ell_0}+\Delta$};
    	\draw (-0.75,0) node {$y_{\ell_0}$};
    	\draw (0,-0.75) node {$\Sigma_0$};
            \draw[-,thick] (0,4) -- (2.6,4-2.6);
			\draw[-,thick] (2.6,4-2.6) -- (3.6,4-2.6); 
            \draw[-,thick] (3.6,-0.1) -- (3.6,0.1);
    	\draw (3.6,-0.75) node {$\Gamma_1$};
    	\draw[-,thick] (3.6,4-2.6) -- (5.6,4-2.6+2);
		\draw[-,thick] (5.6,4-0.6) -- (7,4-0.6);
		\draw[-,thick] (7,-0.1) -- (7,0.1);
    	\draw (7,-0.75) node {$\Sigma_1$};
		\draw[-,thick] (7,4-0.6) -- (7+0.8,4-1.4);
		\draw[-,thick] (7.8,4-1.4) -- (8.8,4-1.4);
		\draw[-,thick] (8.8,4-1.4) -- (8+3,4-3.6);
            	\draw[-,thick] (7+3+1,4-3.6) -- (7+4+1,4-2.6);
            \draw[-,thick] (7+3+1,-0.1) -- (7+3+1,0.1);
	    	\draw (7+3+1,-0.75) node {$\Gamma_2$};
    \end{tikzpicture}
    \caption{\label{fig: sample paths} Sample paths and times of up-down and down-up transitions for \(\varphi(0)\in\calS_+\) (top) and \(\varphi(0)\in\calS_-\) (bottom).}
\end{figure}

For \(m\geq 0\), \(i\in\calS_+,\,j\in\calS_+\cup\calS_{+0}\) define 
\begin{align}
	& \mu_{m,+,+}^{\ell_0}(t)(\cdot,j;x_0,i) = \mathbb P(\bs X(t)\in(\cdot,j), t<\tau_1^X,  \Gamma_m\leq t<\Sigma_{m+1}\mid \bs X(0) = (x_0,  i)), \label{eqn: loop mu}
	%
	\intertext{for \(i\in\calS_+,\,j\in\calS_-\cup\calS_{-0}\) define}
	&\mu^{\ell_0}_{m+1,+,-}(t)(\cdot, j; x_0,i) 
	= \mathbb P(\bs X(t)\in(\cdot,j), t<\tau_1^X,  \Sigma_{m+1}\leq t<\Gamma_{m+1}\mid \bs X(0) = (x_0, i)),\label{eqn: fkl}
	\intertext{for \(i\in\calS_-,\,j\in\calS_+\cup\calS_{+0}\) define}
	&\mu^{\ell_0}_{m+1,-,+}(t)(\cdot, j; x_0,i)  
	= \mathbb P(\bs X(t)\in(\cdot,j), t<\tau_1^X, \Gamma_{m+1}\leq t<\Sigma_{m+1}\mid X(0) = (x_0, i)),
	\intertext{and for \(i\in\calS_-,\,j\in\calS_-\cup\calS_{-0}\) define}
	&\mu^{\ell_0}_{m,-,-}(t)(\cdot, j; x_0,i) = \mathbb P(\bs X(t)\in(\cdot,j), t<\tau_1^X, \Sigma_m\leq t<\Gamma_{m+1}\mid \bs X(0) = (x_0, i)). \label{eqn: many eqns mu} 
\end{align}
Furthermore, for \(q,r\in\{+,-\},\, q\neq r\), let 
\begin{align}
		\mu^{\ell_0}_{q,q}(t)(\cdot,j;x_0,i)  &:= \sum_{m=0}^\infty \mu^{\ell_0}_{m,q,q}(t)(\cdot,j;x_0,i)  && i\in\calS_q,\,j\in\mathcal S_q\cup\calS_{q0}, \label{eqn: ljg97skg}
		\\ \mu^{\ell_0}_{q,r}(t)(\cdot,j;x_0,i)  &:= \sum_{m=1}^\infty \mu^{\ell_0}_{m,q,r}(t)(\cdot,j;x_0,i)  && i\in\mathcal S_q,\,j\in\mathcal S_r\cup\calS_{r0}.
\end{align}
Then we can write (\ref{eqn: sojourn}) as 
\begin{align}
	\mu^{\ell_0}(t)(\cdot,j; x_0,i) =\begin{cases}
		\mu^{\ell_0}_{q,q}(t)(\cdot,j;x_0,i)  & i\in\calS_q,\,j\in\calS_q\cup\calS_{q0},
	\\     \mu^{\ell_0}_{q,r}(t)(\cdot,j;x_0,i)  & i\in\mathcal S_q,\,\,j\in\mathcal S_{r}\cup\calS_{r0},
	\end{cases}\label{eqn:ghghghghggg2}
\end{align}

For states \(k\in\mathcal S_{0}\), and \(q\in \{+,-\}, \, r\in\{+,-\}\), \(m\geq 0\), then
\begin{align}
	\mu_{m,0,r}^{\ell_0}(t)(x,j;x_0,k)  
	&:= \sum_{q\in\{+,-\}}\sum_{i\in\calS_q}\int_{t_0=0}^t \bs e_ke^{\bs T_{00}t_0} \bs T_{0i} \mu_{m+1(q\neq r),q,r}^{\ell_0}(t-t_0)(x,j;x_0,i)\wrt t_0 . \label{eqn: vma02}
\end{align}

\section{Laplace transforms with respect to time of the distributions before \(\tau_1\)}\label{sec: lst on no change}
In this section we take the Laplace transform with respect to time of the densities \(f_{m,q,r}^{\ell_0,(p)}(t)(x,j;x_0,k)\), and measures \(\mu_{m,q,r}^{\ell_0}(t)(x,j;x_0,k)\), \(q\in\{+,-,0\}\), \(r\in\{+,-\}\). The Laplace transform is convenient as it allows us to manipulate the expressions for the QBD-RAP into one component related to the orbit process and one component related the phase process and the rates \(c_i,\,i\in\mathcal S\). 

The following matrices play a key role in the analysis of fluid queues (see, for example, \cite{bean2009,dasilva2005}). Here, they appear in the Laplace transforms of the QBD-RAP and the fluid queue. Define matrices
\begin{align*}
	\bs Q_{+0}(\lambda) &= \bs C_+^{-1}\bs T_{+0}\left[\lambda \bs I - \bs T_{00}\right]^{-1},
	%
	\\\bs Q_{-0}(\lambda) &= \bs C_-^{-1}\bs T_{-0}\left[\lambda \bs I - \bs T_{00}\right]^{-1},
	%
	\\\bs Q_{++}(\lambda) &= \bs C_+^{-1} \left(\bs T_{++} - \lambda \bs I + \bs T_{+0}\left[\lambda \bs I - \bs T_{00}\right]^{-1}\bs T_{0+}\right),
	%
	\\\bs Q_{+-}(\lambda) &= \bs C_+^{-1} \left(\bs T_{+-} + \bs T_{+0}\left[\lambda \bs I - \bs T_{00}\right]^{-1}\bs T_{0-} \right) ,
	%
	\\\bs Q_{--}(\lambda) &= \bs C_-^{-1} \left(\bs T_{--}  - \lambda \bs I + \bs T_{-0}\left[\lambda \bs I - \bs T_{00}\right]^{-1}\bs T_{0-}\right),
	%
	\\\bs Q_{-+}(\lambda) &=\bs C_-^{-1} \left(\bs T_{-+}+ \bs T_{-0}\left[\lambda \bs I - \bs T_{00}\right]^{-1}\bs T_{0+}\right) ,
\end{align*}
and matrix functions,
\begin{align}
	\bs H^{++}(\lambda,x)&= \left[h_{ij}^{++}(\lambda,x)\right]_{i\in \mathcal S_+,j\in\mathcal S_{+}\cup\calS_{+0}} := e^{\bs{Q}_{++}(\lambda)x}\vligne{\bs C_+^{-1} & \bs Q_{+0}(\lambda)},  \label{eqn: lst 1}
	\\\bs H^{--}(\lambda,x) &= \left[h_{ij}^{--}(\lambda,x)\right]_{i\in\mathcal S_-,j\in\mathcal S_{-}\cup\calS_{-0}}:= e^{\bs{Q}_{--}(\lambda)x}\vligne{\bs C_-^{-1} & \bs Q_{-0}(\lambda)},
	\\\bs H^{+-}(\lambda,x)  &= \left[h_{ij}^{+-}(\lambda,x)\right]_{i\in\mathcal S_+,\,j\in\mathcal S_-}:= e^{\bs{Q}_{++}(\lambda)x}\bs{Q}_{+-}(\lambda), 
	\\\bs H^{-+}(\lambda,x)&= \left[h_{ij}^{-+}(\lambda,x)\right]_{i\in\mathcal S_-,\, j\in\mathcal S_+} := e^{\bs{Q}_{--}(\lambda)x}\bs{Q}_{-+}(\lambda) , \label{eqn: lst 4}
\end{align}
for \(x,\lambda\geq 0\). The function \(h_{ij}^{++}(\lambda,x)\) (\(h_{ij}^{--}(\lambda,x)\)) is the Laplace transform with respect to time of the time taken for the fluid level to shift by an amount \(x\) whilst remaining in phases in \(\mathcal S_+\cup\calS_{+0}\) (\(\mathcal S_-\cup\calS_{-0}\)), given the phase was initially \(i\in\mathcal S_+\) (\(i\in\mathcal S_-\)) \citep{bean2005}. The function \(h_{ij}^{+-}(\lambda,x)\) (\(h_{ij}^{-+}(\lambda,x)\)) is the Laplace transform with respect to time of the time taken for the fluid level, \(\{X(t)\}\) to shift by an amount \(x\) whilst remaining in phases in \(\mathcal S_+\cup\calS_{+0}\) (\(\mathcal S_-\cup\calS_{-0}\)), after which time the phase instantaneously changes to \(j\in\mathcal S_-\) (\(\mathcal S_+\)), given the phase was initially \(i\in\mathcal S_+\) (\(\mathcal S_-\)) \citep{bean2005}.

Consider taking the Laplace transform with respect to time of (\ref{eqn: loop mu});
\begin{align}
	\int_{t=0}^\infty e^{-\lambda t} \mu^{\ell_0}_{m,+,+}(t)(\cdot, j; x_0,i)  \wrt t  
	% &= \int_{t=0}^\infty e^{-\lambda t} \mu^{\ell_0}_{m,+,+}(t)(\cdot, j; x_0,i)  \wrt t   \nonumber 
	&= \widehat \mu^{\ell_0}_{m,+,+}(\lambda)( \cdot, j; x_0,i)    \label{eqn: loop mu2}
\end{align}
where we use \(\widehat \mu^{\ell_0}_{m,+,+}(\lambda)( \cdot, j; x_0,i) \) to denote the Laplace transform with respect to time of (\ref{eqn: loop mu}). Throughout, we use the hat \(\,\widehat{}\,\)  notation to denote Laplace transforms with respect to time. 

From the stochastic interpretations of the Laplace transforms (\ref{eqn: lst 1})-(\ref{eqn: lst 4}) the Laplace transforms with respect to time, \(\widehat \mu^{\ell_0}_{m,+,+}(\lambda)( \wrt x, j; x_0,i)\), of (\ref{eqn: loop mu}) are given by 
\begin{align*}
	\widehat \mu^{\ell_0}_{0,+,+}(\lambda)( \wrt x,j;x_0,i) \wrt x&= h_{ij}^{++}(\lambda,x-x_0)1(x\geq x_0)\wrt x,
\end{align*}
for \(m=0\),  and 
\begin{align}
	\nonumber&\int_{x_1 = 0}^{\Delta-(x_0-y_{\ell_0})} \bs e_i \bs H^{+-}(\lambda,\Delta-(x_0-y_{\ell_0})-x_1)  
	\\&  \nonumber\left[\prod_{r=1}^{m-1} \int_{x_{2r}=0}^{\Delta-x_{2r-1}} \bs H^{-+}(\lambda,\Delta-x_{2r}-x_{2r-1})\wrt x_{2r-1}\int_{x_{2r+1}=0}^{\Delta-x_{2r}}\bs H^{+-}(\lambda,\Delta-x_{2r+1}-x_{2r})\wrt x_{2r}\right]
	\\& \int_{x_{2m}=0}^{\Delta-x_{2m-1}} \bs H^{-+}(\lambda,\Delta -x_{2m-1} - x_{2m}) \wrt x_{2m-1}\bs H^{++}(\lambda,\Delta -x_{2m}- (y_{\ell_0+1}- x))\bs e_j\tr{} \nonumber	
	\\& 1(\Delta-x_{2m}-(y_{\ell_0+1}-x)\geq 0)\wrt x_{2m} \wrt x \label{eqn: lst herwe}
\end{align} 
for \(m\geq 1\). Figure~\ref{fig: sample paths lst} shows an example of the sample paths to which these Laplace transforms correspond.  Analogously, we can write down similar expressions for the Laplace transforms with respect to time of (\ref{eqn: loop mu})-(\ref{eqn: many eqns mu}) (omitted).

\begin{figure}
    \centering\begin{tikzpicture}
    	\draw[->,thick] (0,-1) -- (7.5,-1);
   	\draw[-,dashed] (0,4) -- (7.5,4);
    	\draw (7.5,-1.75) node {$t$};
    	\draw[-,thick] (0,-1) -- (0,4.5);
    	\draw[-,thick] (-0.1,4) -- (0.1,4);
    	\draw[-,thick] (-0.1,0) -- (0.1,0);
	\draw (-0.75,0) node {$x_0$};
    	\draw (-0.75,4) node {$y_{\ell_0+1}$};
    	\draw (-0.75,-1) node {$y_{\ell_0}$};
    	\draw[|-|,thick] (0,-1) -- (0,0);
    	\draw (0,-0.5) node[fill=white] {$x_0-y_{\ell_0}$};
    	\draw (0,-1.75) node {$\sigma_0$};
            \draw[-,thick] (0,0) -- (2.6,2.6);
            \draw[-,thick] (2.6,-1.1) -- (2.6,-0.9);
       	\draw[|-|] (2.6,2.6) -- (2.6,4);
	\draw (2.6,2.6+0.7) node[fill=white] {$x_1$};
    	\draw (2.6,-1.75) node {$\sigma_1$};
       	\draw[|-|] (1.3,0) -- (1.3,2.6);
	\draw (1.3,1.3) node[fill=white] {$z_1$};
    	\draw[-,thick] (2.6,2.6) -- (4.6,2.6-2);
            \draw[-,thick] (4.6,-1.1) -- (4.6,-0.9);
        \draw[|-|] (4.6,-1) -- (4.6,0.6);
	\draw (4.6,-0.2) node[fill=white] {$x_2$};
    	\draw (4.6,-1.75) node {$\sigma_2$};
       	\draw[|-|] (3.6,2.6) -- (3.6,0.6);
	\draw (3.6,1.6) node[fill=white] {$z_2$};
            	\draw[-,thick] (4.6,0.6) -- (4.6+3,3.6);
            	\draw[-,dotted] (4.6+3,3.6) -- (4.6+3.2,3.8);
    \end{tikzpicture}
    \caption{\label{fig: sample paths lst} Sample paths corresponding to the Laplace transforms (\ref{eqn: lst herwe}). \(z_1 = \Delta - x_1-(x-y_{\ell_0}),\, z_2= \Delta - x_2-x_1\).}
\end{figure}

Now consider taking the Laplace transform with respect to time of (\ref{eqn: approx end conv});
\begin{align}
	\int_{t=0}^\infty e^{-\lambda t} f^{\ell_0,(p)}_{m,+,+}(t)(  x, j; x_0,i) \wrt t  
	% &= \int_{t=0}^\infty e^{-\lambda t} f^{\ell_0,(p)}_{m,+,+}(t)( x, j; x_0,i)  \wrt t  \nonumber 
	&= \widehat f^{\ell_0,(p)}_{m,+,+}(\lambda)( x, j; x_0,i) \label{eqn: fpapPP2}
\end{align}
where we use \(\widehat f^{\ell_0,(p)}_{m,+,+}(\lambda)(  x, j; x_0,i) \) to denote the Laplace transform with respect to time of (\ref{eqn: approx end conv}). Notice that (\ref{eqn: approx end conv}) is a convolution. Hence, the Laplace transform with respect to time of (\ref{eqn: approx end conv}) is
\begin{align}
	&\widehat f^{\ell_0,(p)}_{m,+,+}(\lambda)(x, j; x_0,i) \nonumber 
	%
	\\&=(\bs e_i\otimes \bs  a_{\ell_0,i}^{(p)}(x_0))  \int_{t_1=0}^\infty e^{-\lambda t_1} e^{\bs{B}^{(p)}_{++}t_1} \wrt t_1 \bs{B}^{(p)}_{+{-}} \nonumber
	\int_{t_2=0}^\infty e^{-\lambda t_2} e^{\bs{B}^{(p)}_{--}t_2} \wrt t_2\bs{B}^{(p)}_{-{+}} 
	\\&\hdots 
	\int_{t_{2m}=0}^\infty e^{-\lambda t_{2m}}e^{\bs{B}^{(p)}_{--}t_{2m}} \wrt t_{2m}\bs{B}^{(p)}_{-{+}} 
	\int_{t=0}^\infty e^{-\lambda t}e^{\bs{B}^{(p)}_{++}t} \wrt t 
	%
	\left(\bs e_j\tr{} \otimes {\bs v}_{\ell_0,j}^{(p)}(x)\right). \label{eqn: approx end conv lst}
\end{align}
Analogous expressions can be computed for the Laplace transforms with respect to time of (\ref{eqn:gljagj})-(\ref{eqn: 67}). 

In Corollary~\ref{cor: mpr B} in Appendix~\ref{appendix: kronecker}, we show the following relation
\begin{align}
	&\vligne{\bs I_{p|\calS_m|} & \bs 0_{p|\calS_m|\times p|\calS_0|} }\int_{t=0}^\infty e^{-\lambda t} e^{\bs{B}^{(p)}_{mm}t} \wrt t \bs{B}^{(p)}_{m{n}} \nonumber 
	%
	\\&= \int_{x=0}^\infty \bs H^{mn}(\lambda,x)  \otimes  e^{\bs S^{(p)} x}\bs D^{(p)}\wrt x \vligne{\bs I_{p|\calS_n|} & \bs 0_{p|\calS_n|\times p|\calS_0|}}, \label{eqn: akgj987adKLDJaf}
\end{align}
for \(m,n\in\{+,-\}\), \(m\neq n\). Before we can apply this result, observe that, since \(i\in\calS_+\), we can write the initial vector in (\ref{eqn: approx end conv lst}) as 
\begin{align}
	(\bs e_i)_{1\times |\calS_+\cup\calS_{+0}|}\otimes \bs  a_{\ell_0,i}^{(p)}(x_0) &= 
	\vligne{(\bs e_i)_{1\times |\calS_+|} & \bs 0_{1\times |\calS_{+0}|}} \otimes \bs  a_{\ell_0,i}^{(p)}(x_0) \nonumber 
	\\&= \vligne{(\bs e_i)_{1\times |\calS_+|}\otimes \bs  a_{\ell_0,i}^{(p)}(x_0)  & \bs 0_{1\times p|\calS_{+0}|}} \nonumber 
	\\&= ((\bs e_i)_{1\times |\calS_+|}\otimes \bs  a_{\ell_0,i}^{(p)}(x_0))\vligne{\bs I_{p|\calS_+|} & \bs 0_{p|\calS_+|\times p|\calS_{+0}|}},
\end{align}
where explicitly indicate the size of the vectors \((\bs e_i)_{1\times |\calS_+\cup\calS_{+0}|}\) via the subscript outside the brackets. 
With this observation, applying (\ref{eqn: akgj987adKLDJaf}) to the first integral in (\ref{eqn: approx end conv lst}) transforms the expression to 
\begin{align}
	&(\bs e_i\otimes \bs  a_{\ell_0,i}^{(p)}(x_0)) \int_{x_1=0}^\infty \left(\bs H^{+{-}}(\lambda,x_1) \otimes e^{\bs S^{(p)} x}\bs D^{(p)}\right) \wrt x_1 \vligne{\bs I_{p|\calS_-|} & \bs 0_{p|\calS_-|\times p|\calS_0|}} \nonumber
	\\&\int_{t_2=0}^\infty e^{-\lambda t_2} e^{\bs{B}^{(p)}_{--}t_2} \wrt t_2\bs{B}^{(p)}_{-{+}} 
	\hdots 
	\int_{t_{2m}=0}^\infty e^{-\lambda t_{2m}}e^{\bs{B}^{(p)}_{--}t_{2m}} \wrt t_{2m}\bs{B}^{(p)}_{-{+}} 
	\int_{t=0}^\infty e^{-\lambda t}e^{\bs{B}^{(p)}_{++}t} \wrt t \nonumber
	%
	\\&\left(\bs e_j\tr{} \otimes {\bs v}_{\ell_0,j}^{(p)}( x)\right). \label{eqn: approx end conv lst2}
\end{align}
We may now apply (\ref{eqn: akgj987adKLDJaf}) to the second integral, after which we can apply (\ref{eqn: akgj987adKLDJaf}) to the third integral and so on. Ultimately, after applying (\ref{eqn: akgj987adKLDJaf}) to all the integrals in (\ref{eqn: approx end conv lst}), we get 
\begin{align}
	&(\bs e_i \otimes \bs   a_{\ell_0,i}^{(p)}(x_0)) \left( \int_{x_1=0}^\infty \bs H^{+-}(\lambda,x_1)\otimes e^{\bs{S}^{(p)}x_1}\bs{D}^{(p)}\wrt x_1 \right)\nonumber%
	\\&\Bigg[\prod_{r=1}^{m-1} \left(\int_{x_{2r}=0}^\infty \bs H^{-+}(\lambda,x_{2r}) \otimes e^{\bs{S}^{(p)}x_{2r}}\bs{D}^{(p)}\wrt x_{2r}\right) \nonumber 
	\\&\nonumber \left(\int_{x_{2r+1}=0}^\infty \bs H^{+-}(\lambda,x_{2r+1})
	\otimes e^{\bs{S}^{(p)}x_{2r+1}}\bs{D}^{(p)} \wrt x_{2r+1} \right)\Bigg] 
	\\&\left(\int_{x_{2m}=0}^\infty \bs H^{-+}(\lambda,x_{2m})
	  \otimes e^{\bs{S}^{(p)}x_{2m}}\bs{D}^{(p)} \wrt x_{2m} \right) \nonumber 
	\\&\left(\int_{x_{2m+1}=0}^\infty \bs H^{++}(\lambda,x_{2m+1})\otimes 
	e^{\bs{S}^{(p)}x_{2m+1}} \wrt x_{2m+1}\right) \left(\bs e_j\tr{} \otimes {\bs v}_{\ell_0,j}^{(p)}(x)\right) \nonumber
	%
	\\&=\int_{x_1=0}^\infty \dots \int_{x_{2m+1}=0}^\infty \bs e_i\bs M^{m}_{++}(\lambda,x_1,\dots,x_{2m+1})\bs e_j\tr{} \nonumber 
	\\&\times \bs a_{\ell_0,i}^{(p)}(x_0) \bs N^{2m+1,(p)}(\lambda,x_1,\dots,x_{2m+1}) {\bs v}_{\ell_0,j}^{(p)}( x)\wrt x_{2m+1}\dots\wrt x_1 ,
	%
	 \label{eqn: approx final end 2}
\end{align}
by the~\ref{eqn:mpr} (see~\ref{appendix: kronecker}), where we define matrices 
\begin{align}
	&\bs M^{m}_{++}(\lambda,x_1,\dots,x_{2m+1}) \nonumber 
	= \prod_{r=1}^{m}\bs H^{+-}(\lambda,x_{2r-1})\bs H^{-+}(\lambda,x_{2r})  
	\bs H^{++}(\lambda,x_{2m+1}),\nonumber 
\end{align}
for \(m\geq 0\), and
\begin{align}
	%
	\bs N^{n,(p)}(\lambda,x_1,\dots,x_{n}) &= \prod_{r=1}^{n-1} e^{\bs{S}^{(p)}x_{r}}\bs{D}^{(p)} e^{\bs{S}^{(p)}x_{n}}, \nonumber 
\end{align}
for \(n\geq 1\). By convention, we take a product over an empty set to be \(1\). The relation~(\ref{eqn: akgj987adKLDJaf}) is key to our analysis. It allows us to factorise the integrand of the Laplace transform (\ref{eqn: approx final end 2}) into one factor solely related to the orbit process \(\{\bs A^{(p)}(t)\}\), 
\[\bs a_{\ell_0,i}^{(p)}(x_0) \bs N^{2m+1,(p)}(\lambda,x_1,\dots,x_{2m+1}) {\bs v}_{\ell_0,j}^{(p)}( x)\] 
and another factor solely related to the fluid queue,
\[\bs e_i\bs M^{m}_{++}(\lambda,x_1,\dots,x_{2m+1})\bs e_j\tr{}.\] 

Further, if we define matrices
\begin{align*}
	\bs M^{m}_{-+}(\lambda,x_1,\dots,x_{2m}) &= \prod_{r=1}^{m-1}\bs H^{-+}(\lambda,x_{2r-1})\bs H^{+-}(\lambda,x_{2r}) \bs H^{-+}(\lambda,x_{2m-1}) \bs H^{++}(\lambda,x_{2m}),
	%
	\\\bs M^{m}_{+-}(\lambda,x_1,\dots,x_{2m}) &= \prod_{r=1}^{m-1}\bs H^{+-}(\lambda,x_{2r-1})\bs H^{-+}(\lambda,x_{2r}) \bs H^{+-}(\lambda,x_{2m-1}) \bs H^{--}(\lambda,x_{2m}),
\end{align*}
for \(m\geq 1\), and
\begin{align*}
	&\bs M^{m}_{--}(\lambda,x_1,\dots,x_{2m+1}) \nonumber 
	= \prod_{r=1}^{m}\bs H^{-+}(\lambda,x_{2r-1})\bs H^{+-}(\lambda,x_{2r}) 
	\bs H^{--}(\lambda,x_{2m+1}),\nonumber 
\end{align*}
for \(m\geq 1\), then analogous expressions can be shown for the Laplace transforms of (\ref{eqn:gljagj})-(\ref{eqn: 67}) in terms of these matrices. For \(m\geq 0\), 
\begin{align*}
	\widehat f^{\ell_0,(p)}_{m+1,-,+}(\lambda)(x, j; x_0,i) &= 
		\int_{x_1=0}^\infty \dots \int_{x_{2m+2}=0}^\infty \bs e_i\bs M^{m+1}_{-+}(\lambda,x_1,\dots,x_{2m+2})\bs e_j\tr{} \nonumber 
		\\&\times \bs a_{\ell_0,i}^{(p)}(x_0) \bs N^{2m+2,(p)}(\lambda,x_1,\dots,x_{2m+2}) {\bs v}_{\ell_0,j}^{(p)}( x)\wrt x_{2m+2}\dots\wrt x_1,
	%
	\\ \widehat f^{\ell_0,(p)}_{m+1,+,-}(\lambda)(x, j; x_0,i) &= 
		\int_{x_1=0}^\infty \dots \int_{x_{2m+2}=0}^\infty \bs e_i\bs M^{m+1}_{+-}(\lambda,x_1,\dots,x_{2m+2})\bs e_j\tr{} \nonumber 
		\\&\times \bs a_{\ell_0,i}^{(p)}(x_0) \bs N^{2m+1,(p)}(\lambda,x_1,\dots,x_{2m+1}) {\bs v}_{\ell_0,j}^{(p)}( x)\wrt x_{2m+2}\dots\wrt x_1,
	\\ \widehat f^{\ell_0,(p)}_{m,-,-}(\lambda)(x, j; x_0,i) &= 
		\int_{x_1=0}^\infty \dots \int_{x_{2m+1}=0}^\infty \bs e_i\bs M^m_{--}(\lambda,x_1,\dots,x_{2m+1})\bs e_j\tr{} \nonumber 
		\\&\times \bs a_{\ell_0,i}^{(p)}(x_0) \bs N^{2m+1,(p)}(\lambda,x_1,\dots,x_{2m+1}) {\bs v}_{\ell_0,j}^{(p)}( x)\wrt x_{2m+1}\dots\wrt x_1.
\end{align*}

In general, for \(k\in\calS_0\), \(q\in \{+,-\}, \, r\in\{+,-\}\), \(m\geq 0\),
\begin{align}
	\widehat f_{m,0,r}^{\ell_0}(\lambda)(x,j;x_0,k)  
	&:= \sum_{q\in\{+,-\}}\sum_{i\in\calS_q}\bs e_k\vligne{\lambda \bs I - \bs T_{00}}^{-1}\bs T_{0i}\widehat f_{m+1(q\neq r),q,r}^{\ell_0}(\lambda)(x,j;x_0,i), \label{eqn: vma}
\end{align}
and
\begin{align}
	\widehat \mu_{m,0,r}^{\ell_0}(\lambda)(\wrt x,j;x_0,k) \label{eqn:kdneee}
	&:= \sum_{q\in\{+,-\}}\sum_{i\in\calS_q}\bs e_k\vligne{\lambda \bs I - \bs T_{00}}^{-1}\bs T_{0i}\widehat \mu_{m+1(q\neq r),q,r}^{\ell_0}(\lambda)(\wrt x,j;x_0,i).
\end{align}

In Section~\ref{sec: no change convergence} we establish that \(\widehat f_{m,q,r}^{\ell_0,(p)}(\lambda)(x,j;x_0,k)\wrt x\to\widehat \mu_{m,q,r}^{\ell_0}(\lambda)(\wrt x,j;x_0,k)\), \(q\in\{+,-,0\}\), \(r\in\{+,-\}\). To do so we use the fact that the functions \(h_{ij}^{qq}(\lambda,x),\,h_{ij}^{qr}(\lambda,x),\) \(q,r\in\{+,-\},\) \(i\in\calS_q,j\in\mathcal S_r\cup\calS_{r0}\) and \(\lambda>0\) satisfy the Assumptions~\ref{asu: g} as functions of \(x\). To this end, we observe the following bounds, which follow from the stochastic interpretation of the functions. Let \(c_{min}=\min_{i\in\mathcal S_-\cup\calS_+} |c_i|\) and recall that we fix \(\lambda\in\mathbb R, \lambda >0\). For all \(\lambda > 0\), there is some \(0\leq G<\infty\) such that, for \(q,r\in\{+,-\},\, q\neq r\),  
\begin{align*}
	0\leq h_{ij}^{qq}(\lambda,x) &\leq   \max\left\{1/c_{min},1\right\}\leq G,\, i\in\mathcal S_q,j\in\calS_q\cup\mathcal S_{q0},
	%
	\\0\leq  h_{ij}^{qr}(\lambda,x)  &  \leq \max_{k,\ell}\left[\bs{Q}_{qr}(0)\right]_{k,\ell}\leq G, \, i\in\mathcal S_q,\, j\in\mathcal S_r.
	%
\end{align*}
Furthermore, there exists some \(0\leq \widehat G<\infty\) such that, 
\begin{align*}
	\int_{x=0}^\infty  h_{ij}^{qq}(\lambda,x) \wrt x &\leq \int_{x=0}^\infty  h_{ij}^{qq}(0,x) \wrt x = \vligne{-\bs{Q}_{qq}(0)^{-1}\bs{C}_q & -\bs Q_{qq}(0)^{-1}\bs Q_{q0}(0)}_{ij}\leq \widehat G, 
	\\\int_{x=0}^\infty  h_{ij}^{qr}(\lambda,x) \wrt x &\leq \int_{x=0}^\infty  h_{ij}^{qr}(0,x) \wrt x = \left[-\bs{Q}_{qq}(0)^{-1}\bs{Q}_{qr}(0)\right]_{ij}\leq \widehat G.
\end{align*}
Moreover, since \(h_{ij}^{qq}(\lambda,x)\) and \(h_{ij}^{qr}(\lambda,x)\), are matrix exponential functions with exponent which is a sub-generator matrix, then for every \(\lambda >0\), \(h_{ij}^{qq}(\lambda,x)\) and \(h_{ij}^{qr}(\lambda,x)\) is Lipschitz continuous with respect to \(x\) on \(x\in[0,\infty)\). Therefore, there exists some \(0<L<\infty\) such that \(\left|h_{ij}^{qq}(\lambda,x)-h_{ij}^{qq}(\lambda,y)\right|\leq L|x-y|\) and \(\left|h_{ij}^{qr}(\lambda,x)-h_{ij}^{qr}(\lambda,y)\right|\leq L|x-y|.\)

\section{Convergence on fixed number of up-down/down-up transitions before \(\tau_1\)}\label{sec: no change convergence}
The main result of this chapter is the following theorem.
\begin{thm}\label{thm: a thm!}
	Let \(\psi:\mathbb R\to\mathbb R\), be bounded, \(|\psi|\leq F\). As \(p\to \infty\), for \(m\geq 1\), \(q\in\{+,-,0\},\, r\in\{+,-\}\), and for \(m=0\), \(q=0\), \(r\in\{+,-\}\), and for \(m=0\), \(q=r\), \(q,r\in\{+,-\},\) then
	\begin{align}\int_{x\in\calD_{\ell_0}}\widehat f_{m,q,r}^{\ell_0,(p)}(\lambda)(x,j;x_0,k)\psi(x)\wrt x \to \int_{x\in\calD_{\ell_0}}\widehat \mu_{m,q,r}^{\ell_0}(\lambda)(\wrt x,j;x_0,k)\psi(x).\label{eqn: thm 2}\end{align}
\end{thm}
The proof of Theorem~\ref{thm: a thm!} is at the end of this section as it is the result of numerous other sub-results, which we now proceed to show. Notice that the convergence in Theorem~\ref{thm: a thm!} is a weak result as we integrate the spatial variable, \(x\), against test functions \(\psi\). This is necessary due to the discontinuity at \(x=x_0\) in terms with \(m=0\).

%\section{One integral}\label{appendix: int one}
Let \(\Delta = \mathbb E[Z]\) be the mean of the matrix exponential random variable, \(Z\). The convergence results rely on the fact that integrating a function, \(g\) say, against the density function of a matrix exponential random variable conditional on the ME-life-time surviving until some time \(u<\Delta-\varepsilon\), approximates integrating said function against a Kronecker delta situated at \(\Delta-u\), provided the variance of the ME is sufficiently low. 

 The next result is used in the proof of Theorem~\ref{thm: a thm!} to prove convergence on the event that there are no up-down or down-up transitions before, \(\tau_1\), the first orbit restart epoch.
 \begin{lem}\label{lem: Dcoajc}
	Let \(\psi:[0,\Delta)\to \mathbb R\) be bounded, \(\psi(x)\leq F\). Then, for \(x_0\in\calD_{\ell_0,i}\), \(x\in\calD_{\ell_0,j}\), \(\ell_0\in\mathcal K^\circ\), \(\lambda > 0\), \(q\in\{+,-\}\), 
	\begin{align}
		&\left|\int_{x\in\mathcal D_{\ell_0,j}} \widehat f^{\ell_0,(p)}_{0,q,q}(\lambda)(x,j; x_0,i)\psi(x-y_{\ell_0})\wrt x - \int_{x\in\mathcal D_{\ell_0,j}} \mu^{\ell_0}_{0,q,q}(\lambda)(\wrt x,j; x_0,i)\psi(x-y_{\ell_0})\right| \nonumber 
		\\&\leq \left(R_{{\bs v},2}^{(p)} + \varepsilon^{(p)}G\right) F.
		\label{eqn: anue}
	\end{align} 
\end{lem}
\begin{proof} 
				Let us write \(x_0=y_{\ell_0}+u\), for \(u\in[0,\Delta)\). Recalling
                \[\widehat f^{\ell_0,(p)}_{0,+,+}(\lambda)(v,j; y_{\ell_0}+u,i) = \int_{x_1=0}^\infty \cfrac{\bs \alpha^{(p)} e^{\bs{S}^{(p)}(u+x_1)} }{\bs \alpha^{(p)} e^{\bs{S}^{(p)}u} \bs e} {\bs v}_{\ell_0,j}^{(p)}(x)h_{ij}^{++}(\lambda,x_1)\wrt x_1,\]
				and 
				\[\widehat \mu^{\ell_0}_{0,+,+}(\lambda)(\wrt v,j; y_{\ell_0}+u,i) = h_{ij}^{++}(\lambda,\Delta-u-x)1(u+x<\Delta)\wrt v,\]
				then (\ref{eqn: anue}) is equal to 
                \begin{align}
                	&\Bigg|\int_{x\in\calD_{\ell_0,j}} \int_{x_1=0}^\infty \cfrac{\bs \alpha^{(p)} e^{\bs{S}^{(p)}(u+x_1)} }{\bs \alpha^{(p)} e^{\bs{S}^{(p)}u} \bs e} {\bs v}^{(p)}_{\ell_0,j}(x)h_{ij}^{++} (\lambda,x_1)\wrt x_1 \psi(x-y_{\ell_0})\wrt x  \nonumber 
		\\&\quad{}- \int_{x\in\calD_{\ell_0,j}} h_{ij}^{++}(\lambda,\Delta-u-x) 1(\Delta-u-x\geq 0) \psi(x-y_{\ell_0})\wrt x\Bigg| \nonumber 
                	%
                	\\&\leq \int_{x\in\calD_{\ell_0,j}} \Bigg|  \int_{x_1=0}^\infty \cfrac{\bs \alpha^{(p)} e^{\bs{S}^{(p)}(u+x_1)} }{\bs \alpha e^{\bs{S}^{(p)}u} \bs e} {\bs v}_{\ell_0,j}^{(p)}(x)h_{ij}^{++} (\lambda,x_1)\wrt x_1  \nonumber 
					\\&\quad{}- h_{ij}^{++}(\lambda,\Delta-u-x) 1(u+x<\Delta) \Bigg| F \wrt x\nonumber 
                	%
                	\\&\leq \int_{x\in\calD_{\ell_0,j}} \Bigg|  \int_{x_1=0}^\infty \cfrac{\bs \alpha^{(p)} e^{\bs{S}^{(p)}(u+x_1)} }{\bs \alpha^{(p)} e^{\bs{S}^{(p)}u} \bs e} {\bs v}^{(p)}_{\ell_0,j}(x)h_{ij}^{++} (\lambda,x_1)\wrt x_1 \nonumber
					\\&\quad{} - h_{ij}^{++}(\lambda,\Delta-u-x) 1(u+x< \Delta - \varepsilon) \Bigg| F \wrt x\nonumber 
                	\\&\quad {}+ \int_{x\in\calD_{\ell_0,j}} \left| h_{ij}^{++}(\lambda,\Delta-u-x) 1(\Delta - \varepsilon \leq u+x<\Delta) \right| F \wrt x, \label{eqn: ALllllLsdnn}
                \end{align}
				by the triangle inequality and since \(\psi\) is bounded. 
                By Property~\ref{properties: 2}, then 
				\begin{align}
					& \Bigg|  \int_{x_1=0}^\infty \cfrac{\bs \alpha^{(p)} e^{\bs{S}^{(p)}(u+x_1)} }{\bs \alpha^{(p)} e^{\bs{S}^{(p)}u} \bs e} {\bs v}^{(p)}_{\ell_0,j}(x)h_{ij}^{++} (\lambda,x_1)\wrt x_1 
					- h_{ij}^{++}(\lambda,\Delta-u-x) 1(u+x< \Delta - \varepsilon) \Bigg| \wrt x\nonumber 
					\\&\leq \left| r_{\bs v}(u,x) \right|.
				\end{align}
				Hence, (\ref{eqn: ALllllLsdnn}) is less than or equal to 
                \begin{align}
                	& \int_{x\in\calD_{\ell_0,j}} |r_{\bs v}^{(p)}(u,x)| F \wrt x
                	+ \int_{x\in\calD_{\ell_0,j}} \left| h_{ij}^{++}(\lambda,\Delta-u-x) 1(\Delta-\varepsilon \leq u+x < \Delta) \right| F \wrt x\nonumber 
                	\\&\leq R^{(p)}_{{\bs v},2} F + \varepsilon GF, \nonumber 
                \end{align}
				since \(|h_{ij}^{++}|\leq G\). Thus, we have shown (\ref{eqn: anue}) for \(q=+\). 

                Using analogous arguments we can show  (\ref{eqn: anue}) for \(q=-\).
\end{proof}
Upon choosing \(\varepsilon^{(p)}=\var(Z^{(p)})^{1/3}\), then the bounds in (\ref{eqn: anue}) tend to 0. 

Next, we proceed to show results needed to prove convergence on the event that there are one or more up-down or down-up transitions before the first orbit restart epoch. The expressions arising from the QBD-RAP which we wish to show converge have the form 
\begin{align}
	\Bigg| \int_{x_1=0}^\infty g_1(x_1) \bs k(x_0) e^{\bs{S}x_1}\wrt x_1\bs D 
			\left[\prod_{k=2}^{n-1}\int_{x_k=0}^\infty g_k(x_k) e^{\bs{S}x_k} \wrt x_k \bs D\right] \int_{x_n=0}^\infty g_{n}(x_n) e^{\bs{S}x_n} \wrt x_n {\bs v}(x), \label{eqn: salkdjgaf} 
\end{align}
where \(n\geq 2\), \(\bs v(x)\) is a closing operator with the Properties~\ref{properties: some props}, \(\{g_k\}\) are functions satisfying Assumptions~\ref{asu: g} and \(\bs k(x_0)=\bs\alpha e^{\bs Sx_0}/\bs\alpha e^{\bs Sx_0}\bs e\). 

We now introduce some notation we will use in the sequel. Define \(w_n(x_0,x)\) to be the expression~(\ref{eqn: salkdjgaf}). Define the column vectors 
\begin{align}
	\mathcal I_{m,k}(u_k) = \left[\prod_{\ell=m}^{k-1}\int_{x_\ell=0}^\infty g_\ell(x_\ell) e^{\bs{S}x_\ell}\wrt x_\ell \bs{D} \right]
            	\int_{x_k=0}^\infty g_{k}(x_k) e^{\bs{S}x_k} \wrt x_k e^{\bs{S}u_k}\bs s
\end{align}
for \(m,k\in\{1,2,\dots\}\), \(m\leq k\), where a product over an empty set is equal to 1. Notice that \(\mathcal I_{m,k}(u_k)\) can be written as 
\begin{align}
	\mathcal I_{m,k}(u_k) = \int_{x_m=0}^\infty g_m(x_m)e^{\bs Sx_m} \wrt x_m \bs D \mathcal I_{m+1,k}(u_k). \label{eqn: I recursive} 
\end{align}
Define the row vectors 
\begin{align}
	\mathcal J_{k+1,k+1}(u_k,x_{k+1}) &:= g_{k+1}(x_{k+1})\cfrac{\bs \alpha e^{\bs{S}u_{k}}}{\bs \alpha e^{\bs{S}u_{k}}\bs e}e^{\bs{S}x_{k+1}},
\end{align}
and
\begin{align}
	\mathcal J_{k+1,n}(u_k,x_{k+1}) &:= g_{k+1}(x_{k+1})\cfrac{\bs \alpha e^{\bs{S}u_{k}}}{\bs \alpha e^{\bs{S}u_{k}}\bs e} e^{\bs{S}x_{k+1}} \bs{D} \left[\prod_{m=k+2}^{n-1}\int_{x_{m}=0}^\infty g_{m}(x_{m}) e^{\bs{S}x_{m}} \wrt x_{m} \bs{D} \right]\nonumber
            	\\&\qquad\times\int_{x_n=0}^\infty g_{n}(x_n) e^{\bs{S}x_n} \wrt x_n
\end{align}
for \(k,n\in\{0,1,2,\dots\}\), \(k+1<n\). The vectors \(\mathcal J_{k+1,n}(u_k,x_{k+1})\) can also be written recursively, 
\begin{align}
	\mathcal J_{k+1,n}(u_k,x_{k+1}) = \mathcal J_{k+1,n-1}(u_k,x_{k+1})\bs D \int_{x_n=0}^\infty g_n(x_n)e^{\bs Sx_n}\wrt x_n.\label{eqn: J recursive} 
\end{align}
Also define \(\displaystyle\bs D(b) = \int_{u=0}^be^{\bs Su}\bs s \cfrac{\bs\alpha e^{\bs S u}}{\bs \alpha e^{\bs S u}\bs e} \wrt u.\)

We prove that (\ref{eqn: salkdjgaf}) converges by writing it as
\begin{align}
	&\int_{x_1=0}^\infty g_1(x_1) \bs k(x_0)e^{\bs{S}x_1}\wrt x_1\bs D(\Delta-\varepsilon)
			\left[\prod_{k=2}^{n-1}\int_{x_k=0}^\infty g_k(x_k) e^{\bs{S}x_k} \wrt x_k \bs D(\Delta-\varepsilon)\right] \nonumber 
			\\&\quad \times\int_{x_n=0}^\infty g_{n}(x_n) e^{\bs{S}x_n} \wrt x_n {\bs v}(x)  
%
+\sum_{k=1}^{n-1} \int_{x_{k+1}=0}^\infty \int_{u_k=\Delta-\varepsilon}^\infty \bs k(x_0)\mathcal I_{1,k}(u_k) \mathcal J_{k+1,n}(u_k,x_{k+1}){\bs v}(x). \label{eqn: kfvKJBawXMN0}
\end{align}
We show that each of the terms in the last summation in (\ref{eqn: kfvKJBawXMN0}) are bounded by something which can be made arbitrarily small upon choosing the variance of the distribution \((\bs \alpha, \bs S)\) to be sufficiently small. Then we show that the difference between the first term in (\ref{eqn: kfvKJBawXMN0}) and the corresponding expression for the fluid queue is also bounded by something which can be made arbitrarily small. The decomposition in (\ref{eqn: kfvKJBawXMN0}) is advantageous since in the first term, the matrices \(\bs D(\Delta-\varepsilon)\) are the integrals \(\displaystyle \int_{u=0}^{\Delta-\varepsilon}e^{\bs Su}\bs s \cfrac{\bs\alpha e^{\bs S u}}{\bs \alpha e^{\bs S u}\bs e} \wrt u,\) so the variable of integration never exceeds \(\Delta-\varepsilon\). As a result, we can use Chebyshev's inequality to bound the denominator in the integrand of \(\bs D(\Delta-\varepsilon)\) near \(1\). 

Our next result shows a bound for the terms in the last summation in (\ref{eqn: kfvKJBawXMN0}). 

Recall the row vector function \(\bs k(x): [0,\infty)\to \mathcal A \subset \mathbb R^p\),
\[\bs k(x) = \cfrac{\bs \alpha e^{\bs Sx}}{\bs \alpha e^{\bs Sx}\bs e}.\]

\begin{cor}\label{cor: lh and rh}
	Let \(g_1, g_2, \dots,\) be functions satisfying the Assumptions~\ref{asu: g} and let \(\bs v(x)\) be a closing operator with the Properties~\ref{properties: some props}, then, for \(k,n \in \{1,2,\dots\}\), \(k+1\leq n\),
	\begin{align}
		&\int_{x_{k+1}=0}^\infty \int_{u_k=\Delta-\varepsilon}^\infty \bs k(x_0)\mathcal I_{1,k}(u_k) \mathcal J_{k+1,n}(u_k,x_{k+1})\bs v(x) \nonumber
		%
            	\\&\leq \cfrac{1}{\bs \alpha e^{\bs{S}x_0}\bs e}\left(\left(2\varepsilon + \cfrac{\var(Z)}{\varepsilon}\right) G^2 \widehat G^{n-2} G_{\bs v} + G\widehat G^{n}\widetilde G_{\bs v}\right)  =: |r_1(n)|. \label{eqn: the result tail}
	\end{align}
\end{cor}
The structure of the proof is as follows. First, recall that we can decompose \(\bs v(x)=\bs w(x)+\widetilde{\bs{w}}(x)\), by Properties~\ref{properties: some props}, hence we can decompose the left-hand side of (\ref{eqn: the result tail}) into 
\begin{align}
	&\int_{x_{k+1}=0}^\infty \int_{u_k=\Delta-\varepsilon}^\infty \bs k(x_0)\mathcal I_{1,k}(u_k) \mathcal J_{k+1,n}(u_k,x_{k+1})\bs w(x) \nonumber 
	\\&{}+ \int_{x_{k+1}=0}^\infty \int_{u_k=\Delta-\varepsilon}^\infty \bs k(x_0)\mathcal I_{1,k}(u_k) \mathcal J_{k+1,n}(u_k,x_{k+1})\widetilde{\bs w}(x). \label{eqn: JABHwj2}
\end{align} 
Next, we bound \(\bs k(x_0)\mathcal I_{1,k}(u_k) \) and \(\mathcal J_{k+1,n}(u_k,x_{k+1})  {\bs w}(x)\). With these two bounds we can derive a bound for the first term in (\ref{eqn: JABHwj2}). A bound on the second term of (\ref{eqn: JABHwj2}) follows from the bound on \(\bs k(x_0)\mathcal I_{1,n-1}(u_{n-1})\) along with Properties~\ref{properties: -1} and~\ref{properties: 0} of \(\widetilde{\bs w}\).

\begin{proof}
	\emph{Step 1: Decompose the left-hand side of (\ref{eqn: the result tail}) as (\ref{eqn: JABHwj2}).}
		Referring to the Properties~\ref{properties: some props}, we can decompose the closing operator \(\bs v(x)=\bs w(x) + \widetilde{\bs w}(x)\), and therefore, due to the linearity of the decomposition, we can decompose the left-hand side of (\ref{eqn: the result tail}) as (\ref{eqn: JABHwj2}).

	\emph{Step 2: Show the following bound.}
	\begin{align}
		\bs k(x_0)\mathcal I_{1,k}(u_k) 
            	&\leq \cfrac{1}{\bs \alpha e^{\bs{S}x_0}\bs e}G\widehat G^{k-1} \bs \alpha e^{\bs{S}u_k}\bs e.\label{eqn: in here}
	\end{align}
	Recall the definition of \(\bs{D}:=\displaystyle\int_{u=0}^\infty e^{\bs{S}u}\bs s \cfrac{\bs \alpha e^{\bs{S}u}}{\bs \alpha e^{\bs{S}u}\bs e}\wrt u\) and substitute it into the left-hand side of (\ref{eqn: in here}), 
	\begin{align}
		\bs k(x_0) \mathcal I_{1,k}(u_k) &=\bs k(x_0) \int_{x_1=0}^\infty g_1(x_1) e^{\bs{S}x_1} \wrt x_1 \bs{D} \mathcal I_{2,k}(u_k) \nonumber 
		\\&=\bs k(x_0)\int_{x_1=0}^\infty g_1(x_1) e^{\bs{S}x_1} \wrt x_1 \int_{u_1=0}^\infty e^{\bs{S}u_1}\bs s \cfrac{\bs \alpha e^{\bs{S}u_1}}{\bs \alpha e^{\bs{S}u_1}\bs e}\wrt u_1 \mathcal I_{2,k}(u_k). \label{eqn: vajJJ8933}
	\end{align}
	Since \(|g_1|\leq G\), then (\ref{eqn: vajJJ8933}) is less than or equal to
	\begin{align}
		&\bs k(x_0) \int_{x_1=0}^\infty G  e^{\bs{S}x_1} \wrt x_1 \int_{u_1=0}^\infty e^{\bs{S}u_1}\bs s \cfrac{\bs \alpha e^{\bs{S}u_1}}{\bs \alpha e^{\bs{S}u_1}\bs e}\wrt u_1 \mathcal I_{2,k}(u_k).\label{eqn: int this}
	\end{align}
	Computing the integral with respect to \(x_1\) in (\ref{eqn: int this}) gives 
	\begin{align}
		 &G  \bs k(x_0)(-\bs{S})^{-1} \int_{u_1=0}^\infty e^{\bs{S}u_1}\bs s \cfrac{\bs \alpha e^{\bs{S}u_1}}{\bs \alpha e^{\bs{S}u_1}\bs e}\wrt u_1 \mathcal I_{2,k}(u_k)  \nonumber
		\\&=\cfrac{G}{\bs \alpha e^{\bs{S}x_0}\bs e} \int_{u_1=0}^\infty \bs \alpha e^{\bs{S}(x_0+u_1)}\bs e \cfrac{\bs \alpha e^{\bs{S}u_1}}{\bs \alpha e^{\bs{S}u_1}\bs e}\wrt u_1\mathcal I_{2,k}(u_k), \label{eqn: yet another label}
	\end{align}
	since \((-\bs{S})^{-1}\) and \(e^{\bs{S}t}\) commute, \(\bs s = - \bs{S} \bs e \) and \(e^{\bs{S}(t+u)} = e^{\bs{S}t}e^{\bs{S}u}\). 
	Since \( \bs \alpha e^{\bs{S}(x_0 +u_1)}\bs e \leq \bs \alpha e^{\bs{S}u_1}\bs e \), then (\ref{eqn: yet another label}) is less than or equal to 
	\begin{align}
		&G  \cfrac{1}{\bs \alpha e^{\bs{S}x_0}\bs e} \int_{u_1=0}^\infty \bs \alpha e^{\bs{S}u_1}\bs e \cfrac{\bs \alpha e^{\bs{S}u_1}}{\bs \alpha e^{\bs{S}u_1}\bs e}\wrt u_1 \mathcal I_{2,k}(u_k) \nonumber
		=G  \cfrac{1}{\bs \alpha e^{\bs{S} x_0 }\bs e} \int_{u_1=0}^\infty \bs \alpha e^{\bs{S}u_1}\wrt u_1 \mathcal I_{2,k}(u_k), 
	\end{align}
	where we have cancelled the terms \(\bs \alpha e^{\bs{S}u_1}\bs e\) on the numerator and denominator.\footnote{The cancellation of terms is important as, for \(u_1>\Delta\), then \(\bs \alpha^{(p)} e^{\bs{S}^{(p)}u_1}\bs e\) becomes small as \(p\to\infty\).}

	Now integrate with respect to \(u_1\) and use the facts that \((-\bs{S})^{-1}\) and \(e^{\bs{S}x}\) commute, and \(\bs s = - \bs{S} \bs e \), to get 
	\begin{align}
		& G  \cfrac{1}{\bs \alpha e^{\bs{S}x_0}\bs e} \bs \alpha (-\bs{S})^{-1}  \mathcal I_{2,k}(u_k) \label{eqn: rep from here}
		\\& = G  \cfrac{1}{\bs \alpha e^{\bs{S}x_0}\bs e}  \bs \alpha (-\bs{S})^{-1} \int_{x_2=0}^\infty g_2(x_2)  e^{\bs{S}x_2} \wrt x_2 \int_{u_2=0}^\infty e^{\bs{S}u_2}\bs s \cfrac{\bs \alpha e^{\bs{S}u_2}}{\bs \alpha e^{\bs{S}u_2}\bs e}\wrt u_2 \mathcal I_{3,k}(u_k)\nonumber 
		\\& = G  \cfrac{1}{\bs \alpha e^{\bs{S}x_0}\bs e}  \int_{x_2=0}^\infty g_2(x_2) \bs \alpha e^{\bs{S}x_2} \wrt x_2 \int_{u_2=0} ^\infty e^{\bs{S}u_2}\bs e \cfrac{\bs \alpha e^{\bs{S}u_2}}{\bs \alpha e^{\bs{S}u_2}\bs e}\wrt u_2 \mathcal I_{3,k}(u_k)\label{eqn: anoth ref here}
	\end{align}
	Since \(\bs \alpha e^{\bs{S}x_2}e^{\bs{S}u_2}\bs e \leq \bs \alpha e^{\bs{S}u_2}\bs e \), then (\ref{eqn: anoth ref here}) is less than or equal to 
	\begin{align}
		& G  \cfrac{1}{\bs \alpha e^{\bs{S}x_0}\bs e}  \int_{x_2=0}^\infty g_2(x_2) \wrt x_2 \int_{u_2=0}^\infty \bs \alpha e^{\bs{S}u_2}\bs e \cfrac{\bs \alpha e^{\bs{S}u_2}}{\bs \alpha e^{\bs{S}u_2}\bs e}\wrt u_2 \mathcal I_{3,k}(u_k) \nonumber
		\\&= G  \cfrac{1}{\bs \alpha e^{\bs{S}x_0}\bs e}  \int_{x_2=0}^\infty g_2(x_2) \wrt x_2 \int_{u_2=0}^\infty \bs \alpha e^{\bs{S}u_2}\bs e\wrt u_2 \mathcal I_{3,k}(u_k), \label{eqn: allalalalalalal}
	\end{align}
	where we have cancelled the terms \(\bs \alpha e^{\bs{S}u_2}\bs e\) on the numerator and denominator.\footnote{As I mentioned in the previous footnote, this is important as, for \(u_k>\Delta\), then \(\bs \alpha^{(p)} e^{\bs{S}^{(p)}u_k}\bs e\) becomes small as \(p\to\infty\). Deriving a bound in such a way that this cancellation occurs was one of the main challenges I encountered with this proof -- in retrospect it is somewhat obvious once we accept that \(g_1\) is bounded and \(g_k,\,k>1,\) are integrable.}

	Now, since \(\displaystyle \int_{x_2=0}^\infty g_2(x_2)\wrt x_2\leq \widehat G,\) then (\ref{eqn: allalalalalalal}) is less than or equal to 
	\begin{align}
		G  \cfrac{1}{\bs \alpha e^{\bs{S} x_0 }\bs e}  \widehat G  \int_{u_2=0}^\infty \bs \alpha e^{\bs{S}u_2} \wrt u_2 \mathcal I_{3,k}(u_k) 
		=G  \cfrac{1}{\bs \alpha e^{\bs{S} x_0 }\bs e}  \widehat G \bs \alpha (-\bs S)^{-1} \mathcal I_{3,k}(u_k).  \label{eqn: rep to here}
	\end{align}
	Repeating the arguments which got us from (\ref{eqn: rep from here}) to (\ref{eqn: rep to here}) another \(k-2\) times gives the result.

\emph{Step 3: Show the bound}
	\begin{align}
            	\mathcal J_{k+1,n}(u_k,x_{k+1})  {\bs w}(x) \leq  g_{k+1}(x_{k+1})\widehat G^{n-k-2} G G_{\bs v}. \label{eqn: J bound}
	\end{align}
	The argument is much the same as that we used to bound~(\ref{eqn: in here}).

	Starting with the left-hand side, upon substituting \(\bs{D}\), 
	\begin{align}
		& \mathcal J_{k+1,n}(u_k,x_{k+1})  {\bs w}(x)  \nonumber 
		\\&= \mathcal J_{k+1,n-1}(u_k,x_{k+1})  \bs{D}
		\int_{x_n=0}^\infty g_{n}(x_n) e^{\bs{S}x_n} \wrt x_n{\bs w}(x) \nonumber
		\\&= \mathcal J_{k+1,n-1}(u_k,x_{k+1})  \int_{u_{n-1}=0}^\infty e^{\bs{S}u_{n-1}}\bs s \cfrac{\bs \alpha e^{\bs{S}u_{n-1}}}{\bs \alpha e^{\bs{S}u_{n-1}}\bs e}\wrt  u_{n-1}
		\int_{x_n=0}^\infty g_{n}(x_n) e^{\bs{S}x_n} \wrt x_n{\bs w}(x) \nonumber
		\\&\leq \mathcal J_{k+1,n-1}(u_k,x_{k+1})  \int_{u_{n-1}=0}^\infty e^{\bs{S}u_{n-1}}\bs s \cfrac{\bs \alpha e^{\bs{S}u_{n-1}}}{\bs \alpha e^{\bs{S}u_{n-1}}\bs e}\wrt  u_{n-1}
		\int_{x_n=0}^\infty G e^{\bs{S}x_n} \wrt x_n{\bs w}(x), \label{eqn: bnd again}
	\end{align}
	since \(|g_n|\leq G\). 
	By Property~\ref{properties: 1} of \({\bs w}(x)\), \(\bs \alpha e^{\bs{S}u_{n-1}} \displaystyle\int_{x_n=0}^\infty e^{\bs{S}x_n} {\bs w}(x) \wrt x_n  \leq \bs \alpha e^{\bs{S}u_{n-1}}\bs eG_{\bs v}\), hence (\ref{eqn: bnd again}) is less than or equal to 
	\begin{align}
		&\mathcal J_{k+1,n-1}(u_k,x_{k+1})  \int_{u_{n-1}=0}^\infty e^{\bs{S}u_{n-1}}\bs s \cfrac{\bs \alpha e^{\bs{S}u_{n-1}}\bs e}{\bs \alpha e^{\bs{S}u_{n-1}}\bs e}\wrt  u_{n-1} G  G_{\bs v}\nonumber 
		%
		\\& = \mathcal J_{k+1,n-1}(u_k,x_{k+1})  \int_{u_{n-1}=0}^\infty e^{\bs{S}u_{n-1}}\bs s \wrt  u_{n-1} G  G_{\bs v}\label{eqn: jjjjjjjj}
	\end{align}
	where the terms \(\bs \alpha e^{\bs{S}u_{n-1}}\bs e\) cancel from the numerator and denominator.\footnote{Once again, this cancellation is important. In this case Property~\ref{properties: 1} of \(\bs w(x)\) and that \(g_n\) is bounded are key the deriving an expression where this term cancels.}

	Computing the integral with respect to \(u_{n-1}\) in (\ref{eqn: jjjjjjjj}), gives
	\begin{align}
		\mathcal J_{k+1,n-1}(u_k,x_{k+1})  \bs e G  G_{\bs v} \nonumber 
		%
		&= \mathcal J_{k+1,n-2}(u_k,x_{k+1}) \int_{u_{n-2}=0}^\infty e^{\bs{S}u_{n-2}}\bs s \cfrac{\bs \alpha e^{\bs{S}u_{n-2}}}{\bs \alpha e^{\bs{S}u_{n-2}}\bs e}\wrt u_{n-2} 
		\\&\qquad {} \times \int_{x_{n-1}=0}^\infty g_{n-1}(x_{n-1}) e^{\bs{S}x_{n-1}} \wrt x_{n-1} \bs e  
		GG_{\bs v}.\label{eqn: this}
	\end{align}
	Since \(\bs\alpha e^{\bs{S}(x_{n-1}+u_{n-2})}\bs e\leq  \bs\alpha e^{\bs{S}(u_{n-2})}\bs e\), then (\ref{eqn: this}) is less than or equal to 
	\begin{align}
		&\mathcal J_{k+1,n-2}(u_k,x_{k+1}) \int_{u_{n-2}=0}^\infty e^{\bs{S}u_{n-2}}\bs s \cfrac{\bs \alpha e^{\bs{S}u_{n-2}}\bs e}{\bs \alpha e^{\bs{S}u_{n-2}}\bs e}\wrt u_{n-2}  \int_{x_{n-1}=0}^\infty g_{n-1}(x_{n-1}) \wrt x_{n-1}  G  G_{\bs v} \nonumber 
		%
		\\&= \mathcal J_{k+1,n-2}(u_k,x_{k+1}) \int_{u_{n-2}=0}^\infty e^{\bs{S}u_{n-2}}\bs s \wrt u_{n-2}  \int_{x_{n-1}=0}^\infty g_{n-1}(x_{n-1}) \wrt x_{n-1}  G  G_{\bs v} \label{eqn: ref here tootoo}
	\end{align} 
	where \(\bs \alpha e^{\bs{S}u_{n-2}}\bs e\) cancels in the numerator and denominator.\footnote{In this case the fact that the \(g_{k}\) are integrable helps us cancel these terms. } 
	Since \(\displaystyle \int_{x_{n-1}=0}^\infty g_{x_{n-1}} \wrt x_{n-1}\leq \widehat G\), then (\ref{eqn: ref here tootoo}) is less than or equal to
	\begin{align}
		\mathcal J_{k+1,n-2}(u_k,x_{k+1}) \int_{u_{n-2}=0}^\infty e^{\bs{S}u_{n-2}}\bs s \wrt u_{n-2} \widehat G G G_{\bs v}
		%
		&= \mathcal J_{k+1,n-2}(u_k,x_{k+1}) \bs e \widehat G G G_{\bs v}. \label{eqn: ref here too}
	\end{align} 
	This is of the same form as the left-hand side of (\ref{eqn: this}), hence repeating the same arguments which took us from (\ref{eqn: this}) to (\ref{eqn: ref here too}) another \(n-k-3\) more times gives
	 \begin{align*}
		\mathcal J_{k+1,n}(u_k,x_{k+1}) \bs w(x)
		&\leq \mathcal J_{k+1,k+1}(u_k,x_{k+1}) \bs e  \widehat G^{n-k-2}G G_{\bs v}
		%
		\\&= g_{k+1}(x_{k+1}) \cfrac{\bs\alpha e^{\bs{S}(u_k+x_{k+1})}}{\bs \alpha e^{\bs{S}u_k}\bs e} \bs e\widehat G^{n-k-2}G G_{\bs v}
		%
		\\& \leq g_{k+1}(x_{k+1}) \widehat G^{n-k-2}G G_{\bs v}.
	\end{align*} 

\emph{Step 4: Combine the bounds on \(\bs k(x_0)\mathcal I_{1,k}(u_k) \) and \(\mathcal J_{k+1,n}(u_k,x_{k+1})  {\bs w}(x)\) to bound the first term in (\ref{eqn: JABHwj2}).}	

		First consider \(k+1<n\). With the bounds (\ref{eqn: in here}) and (\ref{eqn: J bound}), the first term of (\ref{eqn: JABHwj2}) is less than or equal to 
		\begin{align}
			&\cfrac{1}{\bs \alpha e^{\bs{S} x_0 }\bs e}G \widehat G^{k-1}
			\int_{x_{k+1}=0}^\infty \int_{u_k=\Delta-\varepsilon}^\infty \bs \alpha e^{\bs{S}u_k}\bs e g_{k+1}(x_{k+1}) \wrt u_k \wrt x_{k+1}\widehat G^{n-k-2} G G_{\bs v}\nonumber 
			%
			\\&\leq \cfrac{1}{\bs \alpha e^{\bs{S} x_0}\bs e}G \widehat G^{k-1}  
			\int_{u_k=\Delta-\varepsilon}^\infty \bs \alpha e^{\bs{S}u_k}\bs e \wrt u_k \widehat G \widehat G^{n-k-2} G G_{\bs v}. \label{eqnL afejhm789}
		\end{align}
		Now, observe that 
		\begin{align}
			\int_{u_k=\Delta-\varepsilon}^\infty \bs \alpha e^{\bs{S}u_k}\bs e \wrt u_k &= \int_{u_k=\Delta-\varepsilon}^{\Delta+\varepsilon} \mathbb P(Z> u_k) \wrt u_k + \int_{u_k=\Delta+\varepsilon}^\infty \mathbb P(Z> u_k) \wrt u_k\nonumber
			%
			\\&\leq \int_{u_k=\Delta-\varepsilon}^{\Delta+\varepsilon} \wrt u_k + \int_{u_k=\Delta+\varepsilon}^\infty \cfrac{\var(Z)}{(u_k-\Delta)^2} \wrt u_k\nonumber
			% 
			\\&= 2\varepsilon + \cfrac{\var(Z)}{\varepsilon},\label{eqn:kdjf55}
		\end{align}
		where in the second integrand we have used Chebyshev's inequality to bound the tail probability, 
		\[\mathbb P(Z> u_k) \leq \mathbb P(|Z-\Delta|> |u_k-\Delta|) \leq \cfrac{\var(Z)}{(u_k-\Delta)^2},\]
		for \(u_k \geq \Delta +\varepsilon\). Hence, (\ref{eqnL afejhm789}) is less than or equal to 
		\[\cfrac{1}{\bs \alpha e^{\bs{S} x_0}\bs e}G \widehat G^{k-1}  
			\left(2\varepsilon + \cfrac{\var(Z)}{\varepsilon}\right) \widehat G^{n-k-1} G G_{\bs v}.\]
		
		Now consider \(k+1=n\). By the bound (\ref{eqn: in here}), the first term of (\ref{eqn: JABHwj2}), 
		\begin{align}
			&\int_{x_{k+1}=0}^\infty \int_{u_k=\Delta-\varepsilon}^\infty \bs k(x_0)\mathcal I_{1,n-1}(u_{n-1}) g_{n}(x_{n})\cfrac{\bs \alpha e^{\bs{S}u_{n-1}}}{\bs \alpha e^{\bs{S}u_{n-1}}\bs e}e^{\bs{S}x_{n}}\bs w(x) % \mathcal J_{n,n}(u_{n-1},x_{n+1})\bs w(x) 
			\\&\leq \cfrac{1}{\bs \alpha e^{\bs{S} x_0 }\bs e}G \widehat G^{k-1}
			\int_{x_{k+1}=0}^\infty \int_{u_k=\Delta-\varepsilon}^\infty \bs \alpha e^{\bs{S}u_k}\bs e g_{k+1}(x_{k+1}) \cfrac{\bs\alpha e^{\bs{S}(u_k+x_{k+1})}}{\bs \alpha e^{\bs{S}u_k}\bs e}\bs w(x)\wrt u_k \wrt x_{k+1}. \label{eqn: yet yet another label 2}
			%
			\end{align}
			{Since \(g_{k+1}\leq G\), and upon integrating over \(x_{k+1}\), then (\ref{eqn: yet yet another label 2}) is less than or equal to }
			\begin{align}
			 \cfrac{1}{\bs \alpha e^{\bs{S} x_0 }\bs e}G^2\widehat G^{k-1}  
			\int_{u_k=\Delta-\varepsilon}^\infty \bs\alpha e^{\bs{S}u_k}(-\bs S)^{-1}{\bs w}(x) \wrt u_k 
			%
			\leq \cfrac{1}{\bs \alpha e^{\bs{S} x_0 }\bs e}G^2 \widehat G^{k-1}  
			\int_{u_k=\Delta-\varepsilon}^\infty  \bs \alpha e^{\bs S u_k} \bs e G_{\bs v} \wrt u_k , \label{eqn: aksgm}
		\end{align}
		where we have used Property~\ref{properties: 1} to get the upper bound on the right-hand side of (\ref{eqn: aksgm}). Using (\ref{eqn:kdjf55}) again, then (\ref{eqn: aksgm}) is less than or equal to
		\begin{align}
			\cfrac{1}{\bs \alpha e^{\bs{S} x_0 }\bs e}G\widehat G^{n-2}   G G_{\bs v}\left(2\varepsilon + \cfrac{\var\left(Z\right)}{\varepsilon}\right).
		\end{align}
		Thus, we have shown the desired bound. 

\emph{Step 5: Bound the second term in (\ref{eqn: JABHwj2}).}

To bound the second term in (\ref{eqn: JABHwj2}) we instead bound 
	\begin{align}
		&\int_{x_{k+1}=0}^\infty \int_{u_k=0}^\infty \bs k(x_0)\mathcal I_{1,k}(u_k) \mathcal J_{k+1,n}(u_k,x_{k+1})\widetilde{\bs w}(x) \nonumber
		\\= &\int_{x_{k+1}=0}^\infty \int_{u=0}^\infty \mathcal I_{1,n}(u) \cfrac{\bs\alpha e^{\bs Su}}{\bs\alpha e^{\bs Su}\bs e}\int_{x_n=0}^\infty g_n(x_n)e^{\bs Sx_n}\wrt x_n\widetilde{\bs w}(x)
		\label{eqn :mmmm2}
	\end{align}
	which is the same as the second term in (\ref{eqn: JABHwj2}) except that in (\ref{eqn :mmmm2}) the integral with respect to \(u_k\) is over a larger interval. Using the bound in (\ref{eqn: in here}), then (\ref{eqn :mmmm2}) is less than or equal to 
	\begin{align*}
		\cfrac{1}{\bs \alpha e^{\bs{S}x_0}\bs e}G\widehat G^{n-1} \int_{u=0}^\infty \bs \alpha e^{\bs{S}u}\bs e \cfrac{\bs \alpha e^{\bs S u}}{\bs \alpha e^{\bs S u} \bs e}\wrt u \int_{x_n=0}^\infty g_{n}(x_n) e^{\bs{S}x_n} \wrt x_n \widetilde{\bs w}(x).
	\end{align*}
	Integrating over \(u\) gives 
	\begin{align*}
		&\cfrac{1}{\bs \alpha e^{\bs{S}x_0}\bs e}G\widehat G^{n-1} \bs \alpha (-\bs S)^{-1} \int_{x_n=0}^\infty g_{n}(x_n) e^{\bs{S}x_n} \wrt x_n \widetilde{\bs w}(x) 
		\\&\leq \cfrac{1}{\bs \alpha e^{\bs{S}x_0}\bs e}G\widehat G^{n-1} \bs \alpha (-\bs S)^{-1} \int_{x_n=0}^\infty g_{n}(x_n) \wrt x_n \widetilde{\bs w}(x),
	\end{align*}
	where the inequality holds by Property~\ref{properties: -1}. Integrating over \(x_n\), gives
	\begin{align}
		\cfrac{1}{\bs \alpha e^{\bs{S}x_0}\bs e}G\widehat G^{n} \bs \alpha (-\bs S)^{-1} \widetilde{\bs w}(x) 
		&=\cfrac{1}{\bs \alpha e^{\bs{S}x_0}\bs e}G\widehat G^{n}\widetilde G_{\bs v},\label{eqn:FGHJSjjs sj}
	\end{align}
	by Property~\ref{properties: 0}.

	Combining all the bounds proves the result. 
\end{proof}	

Next we wish to prove a bound on the difference between the first term in (\ref{eqn: kfvKJBawXMN0}) and \(g^*_{1,n}(x_0,x)\), where we define 
	\begin{align}
		g^*_{2,n}(u_1,x) &:= \int_{u_2=0}^{\Delta-u_1}g_2(\Delta - u_2 - u_1)\wrt u_1 \dots \nonumber 
            	\int_{u_{n-1}=0}^{\Delta-u_{n-2}} g_{n-1}(\Delta - u_{n-1} - u_{n-2}) \wrt u_{n-2}
            	\\&\qquad{}g_{n}(\Delta - x-u_{n-1})1(\Delta-x-u_{n-1}\geq0)\wrt u_{n-1},
	\end{align}
	and
	\begin{align}
		g^*_{1,n}(x_0,x) &:= \int_{u_1=0}^{\Delta-x_0}g_1(\Delta - u_1 - x_0)g^*_{2,n}(u_1,x)\wrt u_1.
	\end{align}
	The expression \(g^*_{1,n}\) is of a similar form as \(\widehat \mu^{\ell_0}_{n,q,r}(\lambda)\) except that the functions in \(\widehat \mu^{\ell_0}_{n,q,r}(\lambda)\) are matrices. Since the functions in \(\widehat \mu^{\ell_0}_{n,q,r}(\lambda)\) are matrices, then \(\widehat \mu^{\ell_0}_{n,q,r}(\lambda)\) can be written as a linear combination of terms with the form \(g^*_{1,n}\). 

The idea of the proof is to first show a bound for the difference between the first term in (\ref{eqn: kfvKJBawXMN0}) and the expression \(g^{*,\varepsilon}_{1,n}(x_0,x)\) given by
	\begin{align}
		% g^{*,\varepsilon}_{1,n}(x_0,x) &:= 
		&\int_{u_1=0}^{\Delta-\varepsilon-x_0}g_1(\Delta - u_1 - x_0)
		\int_{u_2=0}^{\Delta-\varepsilon-u_1}g_2(\Delta - u_2 - u_1)\wrt u_1  \nonumber 
		\\&\quad\hdots 
            	\int_{u_{n-1}=0}^{\Delta-\varepsilon-u_{n-2}} g_{n-1}(\Delta - u_{n-1} - u_{n-2}) \wrt u_{n-2}
            	g_{n}(\Delta-x-u_{n-1}) 
		1(\Delta-x-u_{n-1}\geq\varepsilon).
	\end{align}
	We then establish a bound on the difference between \(g^{*,\varepsilon}_{1,n}(x_0,x)\) and \(g^{*}_{1,n}(x_0,x)\) which can be made arbitrarily small by choosing \(\varepsilon\) sufficiently small. 

	Recall that the first term in (\ref{eqn: kfvKJBawXMN0}) looks like 
	\begin{align}
		&\int_{x_1=0}^\infty g_1(x_1) \bs k(x_0)e^{\bs{S}x_1}\wrt x_1\bs D(\Delta-\varepsilon)
				\left[\prod_{k=2}^{n-1}\int_{x_k=0}^\infty g_k(x_k) e^{\bs{S}x_k} \wrt x_k \bs D(\Delta-\varepsilon)\right] \nonumber 
				\\&\quad \times\int_{x_n=0}^\infty g_{n}(x_n) e^{\bs{S}x_n} \wrt x_n {\bs v}(x)  
	\end{align}
	which, upon substituting \(\bs D(\Delta-\varepsilon)=\displaystyle \int_{u=0}^{\Delta-\varepsilon}e^{\bs Su}\bs s\cfrac{\bs \alpha e^{\bs Su}}{\bs \alpha e^{\bs Su}\bs e}\wrt u \), can be written as 
	\begin{align}
		& \int_{u_1=0}^{\Delta-\varepsilon} \int_{x_1=0}^\infty \cfrac{\bs \alpha e^{\bs{S}(x_{0}+x_1+u_1)}\bs s}{\bs \alpha e^{\bs{S}x_0}\bs e}g_1(x_1) \wrt x_1 \nonumber 
		\left[\prod_{k=2}^{n-1}\int_{u_k=0}^{\Delta-\varepsilon} \int_{x_k=0}^\infty \cfrac{\bs \alpha e^{\bs{S}(u_{k-1}+x_k+u_k)}\bs s}{\bs \alpha e^{\bs{S}u_{k-1}}\bs e}g_k(x_k) \wrt x_k \wrt u_{k-1} \right]
            	\\&{}\quad\times\int_{x_n=0}^\infty \cfrac{\bs \alpha e^{\bs{S}(u_{n-1}+x_n )}}{\bs \alpha e^{\bs{S}u_{n-1}}\bs e} {\bs v}(x)g_{n}(x_n)\wrt x_n \wrt u_{n-1} .\label{eqnL akfhcka}
	\end{align}
	The last integral in (\ref{eqnL akfhcka}) (with respect to \(x_n\)) is close to \(g_n(\Delta - x-u_{n-1})\) by Property~\ref{properties: 2}.

	Also, appearing in (\ref{eqnL akfhcka}) are integrals of the form
	\begin{align}
		\int_{x_\ell=0}^\infty \cfrac{\bs \alpha e^{\bs{S}(u_{\ell-1}+x_\ell+u_\ell)}\bs s}{\bs \alpha e^{\bs{S}u_{\ell-1}}\bs e}g_\ell(x_\ell) \wrt x_\ell. \label{eqn: skhgintegral}
	\end{align}
	Intuitively, if the variance of \(Z\) is sufficiently small and \(\Delta\) is the expected value of \(Z\), then the distribution of \(Z\) will be concentrated around \(\Delta\) and the integral in (\ref{eqn: skhgintegral}) should be approximately equal to \(g_{\ell}(\Delta - u_{\ell}-u_{\ell-1})\), provided \(u_{\ell-1}\leq \Delta-\varepsilon\). Our first step towards showing a bound for the difference between the first term in (\ref{eqn: kfvKJBawXMN0}) and the expression \(g^{*,\varepsilon}_{1,n}(x_0,x)\) is to prove this intuition. We start with a result about with a simpler integral than that in (\ref{eqn: skhgintegral}), from which the result we require follows as a corollary. 

\begin{lem}\label{lemma:bound}
	Let \(g\) be a function satisfying Assumptions~\ref{asu: g}, then, for \(u \leq \Delta - \varepsilon\), 
	\begin{align}
		\int_{x=0}^\infty g\left(x\right)\bs \alpha e^{\bs{S}\left(x+u\right)} \bs s \wrt x = g\left(\Delta-u\right) + r_2,\label{eqn: ksjfks}
	\end{align}
	where 
	\[\left|r_2\right|\leq 2G\cfrac{\var \left(Z\right)}{\varepsilon^2} + 2L\varepsilon.\]
\end{lem}
The proof follows closely that of \cite[Appendix A, Theorem 4]{hht2020}. The idea of the proof is to recognise that (assuming the variance of \(Z\) is small) the largest contribution to the integral on the left-hand side of (\ref{eqn: ksjfks}) will come from integrating over the interval \(x\in(\Delta-u-\varepsilon,\Delta-u+\varepsilon)\). Since \(g\) is non-negative and bounded, then the rest of the integral is bounded by 
\[\int_{\substack{x\in[0,\infty)\\x\notin(\Delta-u-\varepsilon,\Delta-u+\varepsilon)}} G\bs \alpha e^{\bs{S}\left(x+u\right)} \bs s \wrt x,\]
which can be shown to be small by Chebyshev's inequality provided the variance of \(Z\) is small. 
\begin{proof}
	With a change of variables, 
	\begin{align*}
		\\&\left|\int_{x=0}^\infty g\left(x\right)\bs \alpha  e^{\bs{S} \left(x+u\right)} \bs s \wrt x - g\left(\Delta-u\right)\right| 
		%
		\\&= \left|\int_{x=u}^\infty g\left(x-u\right)\bs \alpha  e^{\bs{S} x} \bs s \wrt x - g\left(\Delta-u\right)\right| 
		%
		\\&= \Bigg|\int_{x=u}^\infty g\left(x-u\right)\bs \alpha  e^{\bs{S} x} \bs s \wrt x - \int_{x=u}^\infty g\left(\Delta-u\right)\bs\alpha  e^{\bs{S} x}\bs s\wrt x - g\left(\Delta-u\right)\left(1-\bs\alpha  e^{\bs{S} u}\bs e \right)\Bigg|.
		%
	\end{align*}
	{By the triangle inequality this is less than or equal to}
	\begin{align*}
		&\left|\int_{x=u}^\infty \left(g\left(x-u\right)- g\left(\Delta-u\right)\right)\bs \alpha  e^{\bs{S} x} \bs s \wrt x \right| 
		{}+ \left|g\left(\Delta-u\right)\left(1-\bs\alpha  e^{\bs{S} u}\bs e \right)\right|
		%
		\\&= \left|\int_{x=u}^\infty \left(g\left(x-u\right)- g\left(\Delta-u\right)\right)\bs \alpha  e^{\bs{S} x} \bs s \wrt x\right| 
		%
		+ \left|\int_{x=0}^u g\left(\Delta-u\right)\bs \alpha  e^{\bs{S} x} \bs s \wrt x \right| 
		%
		\\&\leq d_1 +{d_2} 
	\end{align*}
	where 
	\begin{align*}
		d_1 &= \left|\int_{x=0}^u g\left(\Delta-u\right)\bs \alpha  e^{\bs{S} x} \bs s \wrt x\right| + \left|\int_{x=u}^{\Delta-\varepsilon } \left(g\left(x-u\right)- g\left(\Delta-u\right)\right)\bs \alpha  e^{\bs{S} x} \bs s \wrt x \right| 
		\\&\quad{}+ \left|\int_{x=\Delta+\varepsilon }^{\infty} \left(g\left(x-u\right)- g\left(\Delta-u\right)\right)\bs \alpha  e^{\bs{S} x} \bs s \wrt x \right|,
	\\{d_2} &= \left|\int_{x=\Delta-\varepsilon }^{\Delta+\varepsilon } \left(g\left(x-u\right)- g\left(\Delta-u\right)\right)\bs \alpha  e^{\bs{S} x} \bs s \wrt x\right| .
	\end{align*}

	By the triangle inequality for integrals, \({d_2} \) is less than or equal to
	\begin{align*}
		\int_{x=\Delta-\varepsilon }^{\Delta+\varepsilon } \left|g\left(x-u\right)- g\left(\Delta-u\right)\right|\bs \alpha  e^{\bs{S} x} \bs s \wrt x
		&\leq \int_{x=\Delta-\varepsilon }^{\Delta+\varepsilon } 2L\varepsilon \bs \alpha  e^{\bs{S} x} \bs s \wrt x
		%
		\\&=2L\varepsilon \mathbb P(Z\in(\Delta-\varepsilon, \Delta+\varepsilon))
		%
		\\&\leq 2L\varepsilon ,
	\end{align*}
	where we have used the Lipschitz property of \(g\) from Assumption~\ref{asu: lipschitz} in the first line. 
	
 	Applying the triangle inequality to \(d_1\),
	\begin{align*}
		d_1  &\leq \int_{x=u}^{\Delta-\varepsilon } \left|g\left(x-u\right)- g\left(\Delta-u\right)\right|\bs \alpha  e^{\bs{S} x} \bs s \wrt x
		+ \int_{x=\Delta+\varepsilon }^{\infty} \left|g\left(x-u\right)- g\left(\Delta-u\right)\right|\bs \alpha  e^{\bs{S} x} \bs s \wrt x \nonumber
		\\&\quad{}+ \left|\int_{x=0}^u g\left(\Delta-u\right)\bs \alpha  e^{\bs{S} x} \bs s \wrt x \right| 
		\\& \leq 2G\Bigg( \int_{x=u}^{\Delta-\varepsilon }\bs \alpha  e^{\bs{S} x} \bs s \wrt x
		+ \int_{x=\Delta+\varepsilon }^{\infty}\bs \alpha  e^{\bs{S} x} \bs s \wrt x
		+ \int_{x=0}^u \bs \alpha  e^{\bs{S} x} \bs s \wrt x \Bigg)
		%
		=2G\mathbb P\left(|Z -\Delta|>\varepsilon \right),
		%
	\end{align*}
	where the second inequality holds since \(|g\left(x\right)|\leq G\).
	By Chebyshev's inequality, 
	\begin{align}
		&2G\mathbb P\left(|Z -\Delta|>\varepsilon \right)\leq 2G\cfrac{\var \left(Z \right)}{\varepsilon ^2}.
	\end{align}
	
	Hence, there is some \(r_2\) such that 
	\[\left|\int_{x=0}^\infty g\left(x\right)\bs \alpha  e^{\bs{S} \left(x+u\right)} \bs s \wrt x - g\left(\Delta-u\right)\right| = |r_2| \leq 2G\cfrac{\var(Z)}{\varepsilon^2} + 2 L \varepsilon,\]
	and this completes the proof. 
\end{proof}
\begin{cor}\label{cor: cond bnd 2}
	Let \(g\) be a function satisfying the Assumptions~\ref{asu: g}. For \(u\leq \Delta-\varepsilon \), \(v\geq 0\), 
	\begin{align}
		\int_{x=0}^\infty \cfrac{\bs \alpha  e^{\bs{S} (x+u+v)} \bs s}{\bs \alpha  e^{\bs{S} u} \bs e} g(x)\wrt x = g(\Delta-u-v) 1(u+v\leq\Delta-\varepsilon) + r_3 (u+v),\label{eqn: ssss}
	\end{align}
	where 
	\[\left|r_3 (u+v)\right|\leq \begin{cases} 
		3G\delta + 2L\varepsilon & u+v\leq \Delta-\varepsilon,\\
		G & u+v\in(\Delta-\varepsilon,\Delta+\varepsilon), \\
		G\cfrac{\delta}{1-\delta} & u+v \geq \Delta + \varepsilon.
		\end{cases},\]
	and \(\delta = \cfrac{\var(Z)}{\varepsilon^2}\). 
\end{cor}
\begin{proof}
	First consider \(u+v \leq \Delta - \varepsilon\). Observe that, since \(u\leq \Delta-\varepsilon\), Chebyshev's inequality gives
	\begin{align*}
		\bs \alpha e^{\bs Su}\bs e&=\mathbb P\left(Z >u\right) 
		\\&\geq \mathbb P\left(|Z -\Delta|\leq \varepsilon \right) 
		%
		\\&\geq 1 - \cfrac{\var\left(Z \right)}{\varepsilon ^2} 
		\\&=: 1-\delta ,
	\end{align*}
	thus we have a bound for the denominator in the integrand on the left-hand side of (\ref{eqn: ssss}).
	
	Now, since \(1-\delta\leq\bs \alpha e^{\bs Su}\bs e\leq 1\), then
	\begin{align*}
		\int_{x=0}^\infty \bs \alpha  e^{\bs{S} (x+u+v)} \bs s g(x)\wrt x
		\leq \int_{x=0}^\infty \cfrac{\bs \alpha  e^{\bs{S} (x+u+v)} \bs s}{\bs \alpha  e^{\bs{S} u} \bs e} g(x)\wrt x
		%
		&\leq \frac{1}{1-\delta }\int_{x=0}^\infty \bs \alpha  e^{\bs{S} (x+u+v)} \bs s g(x)\wrt x.
	\end{align*}
	By Lemma~\ref{lemma:bound}  
	\begin{align*}
		g(\Delta-u-v)+r_2
		&\leq \int_{x=0}^\infty \cfrac{\bs \alpha  e^{\bs{S} (x+u+v)} \bs s}{\bs \alpha  e^{\bs{S} u} \bs e} g(x)\wrt x
		%
		\leq \frac{g(\Delta-u-v)+r_2}{1-\delta }. 
	\end{align*}
	Multiplying by \(1-\delta \), then subtracting \(g(\Delta-u-v)\) and adding \(\displaystyle\int_{x=0}^\infty \cfrac{\bs \alpha  e^{\bs{S} (x+u+v)} \bs s}{\bs \alpha  e^{\bs{S} u} \bs e} g(x)\wrt x\delta \) gives
	\begin{align*}
		&r_2(1-\delta ) - g(\Delta-u-v)\delta +\int_{x=0}^\infty \cfrac{\bs \alpha  e^{\bs{S} (x+u+v)} \bs s}{\bs \alpha  e^{\bs{S} u} \bs e} g(x)\wrt x\delta 
		\\&\leq \int_{x=0}^\infty \cfrac{\bs \alpha  e^{\bs{S} (x+u+v)} \bs s}{\bs \alpha  e^{\bs{S} u} \bs e} g(x)\wrt x -g(\Delta-u-v)
		%
		\\&\leq r_2+\int_{x=0}^\infty \cfrac{\bs \alpha  e^{\bs{S} (x+u+v)} \bs s}{\bs \alpha  e^{\bs{S} u} \bs e} g(x)\wrt x\delta .
	\end{align*}
	The last line is bounded above by 
	\begin{align*}
		r_2+\int_{x=0}^\infty \cfrac{\bs \alpha  e^{\bs{S} (x+u+v)} \bs s}{\bs \alpha  e^{\bs{S} u} \bs e} g(x)\wrt x\delta 
		%
		&\leq r_2 + G \delta .
	\end{align*}
	The first line is bounded below by 
	\begin{align*}
		r_2 (1-\delta ) - g(\Delta-u-v)\delta +\int_{x=0}^\infty \cfrac{\bs \alpha  e^{\bs{S} (x+u+v)} \bs s}{\bs \alpha  e^{\bs{S} u} \bs e} g(x)\wrt x\delta 
		%
		&\geq r_2(1-\delta ) - g(\Delta-u-v)\delta .
	\end{align*}
	Therefore, 
	\begin{align}
		\int_{x=0}^\infty \cfrac{\bs \alpha  e^{\bs{S} (x+u+v)} \bs s}{\bs \alpha  e^{\bs{S} u} \bs e} g(x)\wrt x  = g(\Delta-u-v) + r_3 ,
	\end{align}
	where 
	\begin{align}
		\nonumber\left|r_3 \right| 
		&\leq \max\left(|r_2|(1-\delta ) + g(\Delta-u-v)\delta , |r_2| + G \delta \right) 
		%
		\\\nonumber&\leq  |r_2| + G\delta
		%
		\\&\leq 3G\delta + 2L\varepsilon,
	\end{align}
	as required. 
	
	For \(u+v\in (\Delta-\varepsilon, \Delta + \varepsilon)\),
	\begin{align}
		\int_{x=0}^\infty \cfrac{\bs \alpha  e^{\bs{S} (x+u+v)} \bs s}{\bs \alpha  e^{\bs{S} u} \bs e} g(x)\wrt x & \leq G \mathbb P(Z>u+v\mid Z>u) \leq G.
	\end{align}
	
	For \(u+v \geq \Delta + \varepsilon\),
	\begin{align}
		\int_{x=0}^\infty \cfrac{\bs \alpha  e^{\bs{S} (x+u+v)} \bs s}{\bs \alpha  e^{\bs{S} u} \bs e} g(x)\wrt x & \leq G \cfrac{\mathbb P(Z>u+v)}{\mathbb P( Z>u)} 
		%
		 \leq G\cfrac{\var(Z)/\varepsilon^2}{1-\var(Z)/\varepsilon^2} .
	\end{align}
\end{proof}

The error term \(r_3^{(p)}\) depends on \(p\), as it is defined by \(Z^{(p)}\) and \(\varepsilon^{(p)}\), but we have omitted the superscript \(p\) here. Choosing \(\varepsilon=\var(Z^{(p)})^{1/3}\) then, outside the vanishingly small interval \(u\in(\Delta-\varepsilon^{(p)},\Delta+\varepsilon^{(p)})\), the error term \(|r_3^{(p)}(u)|\) is bounded by \(O\left(\var\left(Z^{(p)}\right)^{1/3}\right)\), which tends to 0 as \(p\to\infty\). On \(u\in(\Delta-\varepsilon^{(p)},\Delta+\varepsilon^{(p)})\) the error term \(|r_3^{(p)}(u)|\) is bounded by a constant which does not tend to \(0\) as \(p \to \infty\). However, when we integrate a bounded function against \(r_3^{(p)}(u)\), then the resulting integral tends to \(0\), i.e.~for \(|\psi(x)|\leq F, \, M<\infty\), \(\displaystyle \int_{0}^M \psi(u) |r_3^{(p)}(u)|\wrt u\leq F\Delta (3G\delta^{(p)}+2L\varepsilon) + 2GF\varepsilon^{(p)} + (M-\Delta)GF{\delta^{(p)}}/({1-\delta^{(p)}})=O\left(\var\left(Z^{(p)}\right)^{1/3}\right)\to 0 \) as \(p\to\infty\). This is the context in which we apply Corollary~\ref{cor: cond bnd 2} and thus the error bound is sufficient. 

We are now in a position to prove the desired bound on the difference between the first term in (\ref{eqn: kfvKJBawXMN0}) and \(g^*_{1,n}(x_0,x)\).
\begin{lem}\label{lem: lst convergence}
	Let \(g_1,g_2,\dots,\) be functions satisfying the Assumptions~\ref{asu: g} and let \(\bs v(x)\) be a closing operator with the Properties~\ref{properties: some props}. Then, for \(n\geq 2\),  
	\begin{align}
		&\int_{x_1=0}^\infty g_1(x_1) \bs k(x_0) e^{\bs{S}x_1}\wrt x_1 \bs D(\Delta-\varepsilon)
            	\left[\prod_{k=2}^{n-1}\int_{x_k=0}^\infty g_k(x_k) e^{\bs{S}x_k} \wrt x_k \right]
		\bs D(\Delta-\varepsilon) \nonumber 
		\\&\qquad\times\int_{x_n=0}^\infty g_{n}(x_n) e^{\bs{S}x_n} \wrt x_n {\bs v}(x) \nonumber 
	%
		\\& =g^{*}_{1,n}(x_0,x) + r_4(n) + r_5(n), \label{eqn: rhs g 2}
	\end{align}
	where  
	\begin{align*}
		|r_4(n)|&= O\left(\max\left\{\delta,\varepsilon,\cfrac{\delta}{1-\delta},R_{{\bs v},1}\right\}\right),
		%
		\\|r_5(n)| &\leq \varepsilon^{n-1}G^{n-1}
	\end{align*}
\end{lem}
\begin{proof}
	Rewriting the left-hand side of (\ref{eqn: rhs g 2}) as in (\ref{eqnL akfhcka}), then we see that we can apply Corollary~\ref{cor: cond bnd 2} to all the integrals over \(x_k,\, k=1,\dots,n-1\) and use Property~\ref{properties: 2} of \({\bs v}(x)\) for the integral over \(x_n\), to get  
	\begin{align*}
		& \int_{u_1=0}^{\Delta-\varepsilon}\left[g_1(\Delta - u_1 - x_0)1(u_1 + x_0\leq \Delta - \varepsilon) + r_3 (u_1 + x_0)\right]
		\\&\quad\times\int_{u_2=0}^{\Delta-\varepsilon}\left[g_2(\Delta - u_2 - u_1)1(u_2 + u_1\leq \Delta - \varepsilon) + r_3 (u_2 + u_1)\right]\wrt u_1
		\\&\quad\hdots 
            	 \int_{u_{n-1}=0}^{\Delta-\varepsilon}  \left[g_{n-1}(\Delta - u_{n-1} - u_{n-2}) 1(u_{n-1} + u_{n-2}\leq \Delta - \varepsilon) +   r_3 (u_{n-1} + u_{n-2})\right] \wrt u_{n-2}
            	\\&\quad\times\left[g_{n}(\Delta-u_{n-1}-x)1(u_{n-1}+x\leq \Delta - \varepsilon) + r_{\bs v} (u_{n-1},x)\right]\wrt u_{n-1}
		%
		\\&=g^{*,\varepsilon}_{1,n}(x_0,x) + r_4(n)
	\end{align*}
	where \(r_4(n)\) is an error term. The leading terms of \(r_4(n)\) are of the form 
	\begin{align}
		&\int_{u_1=0}^{\Delta-\varepsilon-x_0}g_1(\Delta - u_1 - x_0)
		\int_{u_2=0}^{\Delta-\varepsilon-u_1}g_2(\Delta - u_2 - u_1)\wrt u_1 \nonumber 
		\\&\quad\hdots\int_{u_{k-1}=0}^{\Delta-\varepsilon-u_{k-2}}g_{k-1}(\Delta - u_{k-1} - u_{k-2}) \wrt u_{k-2}
		\int_{u_k=0}^{\Delta-\varepsilon}r_3(u_{k}+u_{k-1}) \wrt u_{k-1} \nonumber 
		\\&\quad\times\int_{u_{k+1}=0}^{\Delta-\varepsilon-u_{k}}g_{k+1}(\Delta - u_{k+1} - u_{k}) \wrt u_{k}
		\hdots
            	\int_{u_{n-1}=0}^{\Delta-\varepsilon-u_{n-2}} g_{n-1}(\Delta - u_{n-1} - u_{n-2}) \wrt u_{n-2} \nonumber 
            	\\&\quad\times g_{n}(\Delta-u_{n-1}-x)1(u_{n-1}+x\leq \Delta -\varepsilon)\wrt u_{n-1} \label{eqn: akfj111112}
	\end{align}
	or
	\begin{align}
		&\int_{u_1=0}^{\Delta-\varepsilon-x_0}g_1(\Delta - u_1 - x_0)
		\int_{u_2=0}^{\Delta-\varepsilon-u_1}g_2(\Delta - u_2 - u_1)\wrt u_1 \nonumber 
		\\&{}\hdots
            	\int_{u_{n-1}=0}^{\Delta-\varepsilon-u_{n-2}} g_{n-1}(\Delta - u_{n-1} - u_{n-2}) \wrt u_{n-2}
            	 r_{\bs v}(u_{n-1},x)\wrt u_{n-1},\label{eqn: akfj11111}
	\end{align} 
	whichever is larger. Since \(|g_\ell|\leq G\), \(\ell \geq 1\), then (\ref{eqn: akfj11111}) and (\ref{eqn: akfj111112}) are less than or equal to 
	\begin{align*}
		& G^{k-1} \Delta^{k-2} \int_{u_{k-1}=0}^{\Delta-\varepsilon}
		\int_{u_k=0}^{\Delta-\varepsilon }r_3(u_{k}+u_{k-1}) \wrt u_k \wrt u_{k-1} G^{n-k}\Delta^{n-k-1},
		%
	\end{align*}
	and 
	\begin{align*}
		&G^{n-1} \Delta^{n-2} \int_{u_{n-1}=0}^{\Delta-\varepsilon}
		 r_{\bs v}(u_{n-1},x) \wrt u_{n-1},
	\end{align*} 
	respectively. 

	Recall that we have a bound on \(|r_3|\) which is piecewise constant. Breaking up the integral of \(|r_3|\) above into three intervals over which the bound is constant and using the triangle inequality, then
	\begin{align}
		 \nonumber &\left|\int_{u_{k-1}=0}^{\Delta-\varepsilon}\int_{u_k=0}^{\Delta-\varepsilon}r_3(u_{k}+u_{k-1}) \wrt u_k \wrt u_{k-1} \right|
		\\\nonumber & \leq \int_{u_{k-1}=0}^{\Delta-\varepsilon}\Bigg[ \int_{u_k=u_{k-1}}^{\Delta-\varepsilon} | r_3(u_k) |\wrt u_k + \int_{u_k=\Delta-\varepsilon}^{\min(\Delta+\varepsilon,\Delta-\varepsilon+u_{k-1})} |r_3(u_k)|\wrt u_k 
		%
		\\&\qquad{}+ \int_{u_k = \Delta+\varepsilon}^{\Delta-\varepsilon+u_{k-1}} |r_3(u_{k})|\wrt u_k1(u_{k-1}>2\varepsilon)\Bigg] \wrt u_{k-1}.\label{eqn: dkj5678G5F}
	\end{align}
	Using the piecewise upper bounds on \(|r_3|\) then, in the square brackets in (\ref{eqn: dkj5678G5F}), the first integral is less than or equal to 
	\[(\Delta-\varepsilon-u_{k-1})(3G\Delta +2L\varepsilon),\]
	the second integral is less than or equal to 
	\[\int_{u_k=\Delta-\varepsilon}^{\Delta+\varepsilon} |r_3(u_k)|\wrt u_k \leq 2\varepsilon G \]
	and the third integral is less than or equal to 
	\[G\cfrac{\delta}{1-\delta}(u_{k-1}-2\varepsilon)1(u_{k-1}>2\varepsilon).\]
	With these bounds (\ref{eqn: dkj5678G5F}) is less than or equal to
	\begin{align*}
		&\Bigg[\int_{u_{k-1}=0}^{\Delta-\varepsilon} (\Delta-\varepsilon-u_{k-1})(3G\delta+2L\varepsilon)+ 2\varepsilon G 
		%
		+ G\cfrac{\delta}{1-\delta}(u_{k-1}-2\varepsilon)1(u_{k-1}>2\varepsilon) \Bigg]\wrt u_{k-1}
		% 
		\\& \leq \frac{1}{2}\Delta^2(3G\delta+2L\varepsilon)+ 2\Delta\varepsilon G 
		%
		+ \cfrac{1}{2}\Delta^2G\cfrac{\delta}{1-\delta} 
		\\&=O\left(\delta,\varepsilon,\cfrac{\delta}{1-\delta}\right).
	\end{align*}
	Thus, (\ref{eqn: akfj111112}) is, at worst, \(O\left(\delta,\varepsilon,\cfrac{\delta}{1-\delta}\right)\). 
	
	As for (\ref{eqn: akfj11111}), by Property~\ref{properties: 2}, 
	\begin{align*}
		&\int_{u_{n-1}=0}^{\Delta-\varepsilon}| r_{\bs v}(u_{n-1},x)|\wrt u_{n-1} 
		\leq R_{{\bs v},1},
	\end{align*}
	hence (\ref{eqn: akfj11111}) is \(O(R_{{\bs v},1})\).

	Therefore, the error term \(|r_4(n)| = O\left(\max\left\{\delta,\varepsilon,\cfrac{\delta}{1-\delta},R_{{\bs v},1}\right\}\right).\)
	
	Now, 
	\begin{align*}
		&\Bigg|g_{1,n}^{*,\varepsilon}(x_0,x) - g_{1,n}^{*}(x_0,x)
		%
		\Bigg|\nonumber
		%
		\\&= \int_{u_1=\Delta-\varepsilon-x_0}^{\Delta-x_0}g_1(\Delta - u_1 - x_0)
		\int_{u_2=\Delta-\varepsilon-u_1}^{\Delta-u_1}g_2(\Delta - u_2 - u_1)\wrt u_1  \nonumber 
		\\&\quad\hdots 
            	\int_{u_{n-1}=\Delta-\varepsilon-u_{n-2}}^{\Delta-u_{n-2}} g_{n-1}(\Delta - u_{n-1} - u_{n-2}) \wrt u_{n-2}
            	g_{n}(\Delta - x-u_{n-1})\nonumber 
		\\&\quad\times 1(\Delta - x-u_{n-1}\geq 0)\wrt u_{n-1}\nonumber
		\\&\leq \int_{u_1=\Delta-\varepsilon-x_0}^{\Delta-x_0}G 
		\int_{u_2=\Delta-\varepsilon-u_1}^{\Delta-u_1}G \wrt u_1  \hdots 
            	\int_{u_{n-1}=\Delta-\varepsilon-u_{n-2}}^{\Delta-u_{n-2}} G \wrt u_{n-2}\nonumber
            	G\wrt u_{n-1} 
		\\& = \varepsilon^{n-1}G ^n,
	\end{align*}
	where the inequality holds since \(|g_\ell|\) are bounded.
	Therefore, the left-hand side of (\ref{eqn: rhs g 2}) is equal to 
	\begin{align*}
		g^{*,\varepsilon}_{1,n}(x_0,x) + r_4(n) + r_5(n),
	\end{align*}
	where \(r_5(n) = g_{1,n}^{*}(x_0,x) - g_{1,n}^{*,\varepsilon}(x_0,x)\leq \varepsilon^{n-1}G ^n\).
\end{proof}

Combining the results obtained so far in this chapter we have a bound on the difference between \(w_n(x_0,x)\) and \(g_{1,n}^*(x_0,x)\) which we state formally as the following corollary.
\begin{cor}\label{cor: a cor}
	Let \(g_1,g_2,\dots,\) be functions satisfying Assumptions~\ref{asu: g} and let \({\bs v}(x)\), \(x\in[0,\Delta)\), be a closing operator with Properties~\ref{properties: some props}. Then, for \(n\geq 2\), \(x_0\in[0,\Delta)\), 
	\begin{align}
		&\left|w_n(x_0,x)- g_{1,n}^{*}(x_0,x) \right|
		\leq (n-1)|r_1(n)| + |r_4(n)| + |r_5(n)|, \label{eqn: rhs g 4}
	\end{align}
\end{cor}
\begin{proof}
	Substitute the expression for \(w_n(x_0,x)\) in (\ref{eqn: kfvKJBawXMN0}) into the left-hand side of (\ref{eqn: rhs g 4}), apply the triangle inequality and Corollary~\ref{cor: lh and rh} and Lemma~\ref{lem: lst convergence} to get the result.  
\end{proof}

A direct corollary is the following.
\begin{cor}\label{cor: asjdajaaaaa}
	 Let \(g_1,g_2,\dots,\) be functions satisfying Assumptions~\ref{asu: g}, \(\psi:[0,\Delta) \to \mathbb R\) be bounded, \(|\psi|\leq F\), and let \({\bs v}(x)\), \(x\in[0,\Delta)\), be a closing operator with Properties~\ref{properties: some props}. Then, for \(n\geq 2\), \(x_0\in[0,\Delta)\), 
	\begin{align}
		&\left| \int_{x=0}^\Delta w_n(x_0,x) \psi(x)-g_{1,n}^*(x_0,x)\psi(x)\wrt x\right|  
		\leq ((n-1)|r_1(n)| + |r_4(n)| + |r_5(n)|)\Delta F. \label{eqn: rhs g 4dvfklsmv}
	\end{align}
\end{cor}
\begin{proof}
	The left-hand side of (\ref{eqn: rhs g 4dvfklsmv}) is less than or equal to 
	\begin{align}
		&\int_{x=0}^\Delta \left| w_n(x_0,x) 
	%
		{}- g_{1,n}^*(x_0,x)\right| \left|\psi(x)\right| \wrt x. \label{eqn: rhs g 4dvfklsmsssv}
	\end{align}
	Applying Corollary~\ref{cor: a cor}, and since \(|\psi|\leq F\), then (\ref{eqn: rhs g 4dvfklsmsssv}) is less than or equal to 
	\begin{align}
		\int_{x=0}^\Delta((n-1)|r_1(n)| + |r_4(n)| + |r_5(n)|) F \wrt x \nonumber 
		= ((n-1)|r_1(n)| + |r_4(n)| + |r_5(n)|)\Delta F \nonumber
	\end{align}
	which is the desired result.
\end{proof}

We have assumed throughout this section that the functions \(g\) and \(\{g_k\}\) are scalar functions, however, we are ultimately interested in expressions of the form (\ref{eqn: approx final end 2}), which contain matrix functions. Conveniently, the matrix-function expression~(\ref{eqn: approx final end 2}) can be written as a linear combination of the scalar case. Hence, we can obtain a bound for the matrix-function case from the scalar case, which we state in the following result.
\begin{lem}\label{lem: boobies}
	Let \(\bs G_k(x)\), \(k\in\{1,2,...\}\), be matrix functions with dimensions \(N_k \times N_{k+1}\), and let \(\psi:[0,\Delta)\to\mathbb R\) be bounded, \(|\psi|\leq F\). Further, suppose the scalar function \([\bs G_k(x)]_{ij}\), \(i\in\{1,...,N_{k}\}\), \(j\in\{1,...,N_{k+1}\}\), \(k\in\{1,2,...\}\) satisfy Assumptions~\ref{asu: g}. Then, 
	\begin{align}
		&\Bigg| \int_{x=0}^\Delta \int_{x_1=0}^\infty \bs G_1(x_1) \otimes \bs k(x_0) e^{\bs{S}x_1} \bs D (x_1) \wrt x_1
		\left[\prod_{k=2}^{n-1}\int_{x_k=0}^\infty \bs G_{k }(x_k) \otimes e^{\bs{S}x_k} \wrt x_k \bs D\right] \nonumber
\\&\qquad{}\int_{x_n=0}^\infty \bs G_{n }(x_n)\otimes e^{\bs{S}x_n} \wrt x_n {\bs v}(x) \psi(x) \wrt x \nonumber 
	%
		\\&{}- \int_{x=0}^\Delta \int_{u_1=0}^{\Delta-x_0}\bs G_1(\Delta - u_1 - x_0)
		\left[\prod_{k=2}^{n-1} \int_{u_k=0}^{\Delta-u_{k-1}} \bs G_{k}(\Delta-u_k-u_{k-1})\wrt u_{k-1}\right] \nonumber 
				\\&\qquad{} \bs G_{n }(\Delta - x-u_{n-1})
			1(\Delta-x-u_{n-1}\geq0) \wrt u_{n-1}\psi(x) \wrt x \Bigg| \nonumber
		\\&\leq ((n-1)|r_1(n)| + |r_4(n)| + |r_5(n)|)\Delta F \prod_{k=2}^{n}N_{k}, \label{eqn: rhs g 4dvfklsmv2G}
	\end{align}
	where the inequality is an element-wise inequality. Moreover, choosing \(\varepsilon=\var(Z)\), then, for each \(n\), the bound (\ref{eqn: rhs g 4dvfklsmv2G}) is \(\mathcal O(\var(Z)^{1/3})\). 
\end{lem}
\begin{proof}
	By the~\ref{eqn:mpr} the \((i,j)\)th element of the first term on the left-hand side of (\ref{eqn: rhs g 4dvfklsmv2G}) is 
	\begin{align}
		&\int_{x=0}^\Delta \int_{x_1=0}^\infty \dots \int_{x_n=0}^\infty \left[\bs G_1(x_1)\dots \bs G_n(x_n)\right]_{i,j} \bs k(x_0) e^{\bs{S}x_1} \bs D \dots 
		e^{\bs{S}x_{n-1}} \bs D \nonumber
		\\&\qquad{} e^{\bs{S}x_n} \wrt x_n \dots \wrt x_1 {\bs v}(x) \psi(x) \wrt x \nonumber 
		% 
		\\&= \int_{x=0}^\Delta \int_{x_1=0}^\infty \dots \int_{x_n=0}^\infty \sum_{j_1=1}^{N_2}[\bs G_1(x_1)]_{i,j_1}\sum_{j_2=1}^{N_3}[\bs G_2(x_2)]_{j_1,j_2}\dots \sum_{j_{n-1}=1}^{N_n} [\bs G_n(x_n)]_{j_{n-1},j} \nonumber
		\\&\qquad{} \bs k(x_0) e^{\bs{S}x_1} \bs D  \dots
		e^{\bs{S}x_{n-1}} \bs D 
		e^{\bs{S}x_n} \wrt x_n \dots \wrt x_1 {\bs v}(x) \psi(x) \wrt x, \label{eqn: s}
	\end{align}
	from which we see that (\ref{eqn: s}) is a linear combination of the scalar function case in Corollary~\ref{cor: asjdajaaaaa}. Applying the bound for the scalar case, Corollary~\ref{cor: asjdajaaaaa}, to each term in the linear combination then summing the bounds obtained gives the bound (\ref{eqn: rhs g 4dvfklsmv2G}). 

	The fact that the error bound is \(\mathcal O(\var(Z)^{1/3})\) follows by substituting \(\varepsilon=\var(Z)^{1/3}\) into each term and observing that each term is at most \(\mathcal O(\var(Z)^{1/3})\). 
\end{proof}
% Lemma~\ref{lem: boobies} effectively shows that, as \(p \to \infty\), then 
% \begin{align}
% 	& \int_{x=0}^\Delta \int_{x_1=0}^\infty \bs G_1(x_1) \otimes \bs k^{(p)} (x_0) e^{\bs{S}^{(p)}x_1} \bs D^{(p)} (x_1) \wrt x_1
% 	\left[\prod_{k=2}^{n-1}\int_{x_k=0}^\infty \bs G_{k }(x_k) \otimes e^{\bs{S}^{(p)}x_k} \wrt x_k \bs D^{(p)}\right] \nonumber
% \\&\qquad{}\int_{x_n=0}^\infty \bs G_{n }(x_n)\otimes e^{\bs{S}^{(p)}x_n} \wrt x_n {\bs v}^{(p)}(x) \psi(x) \wrt x \nonumber 
% %
% 	\\&{}\to \int_{x=0}^\Delta \int_{u_1=0}^{\Delta-x_0}\bs G_1(\Delta - u_1 - x_0)
% 	\left[\prod_{k=2}^{n-1} \int_{u_k=0}^{\Delta-u_{k-1}} \bs G_{k}(\Delta-u_k-u_{k-1})\wrt u_{k-1}\right] \nonumber 
% 			\\&\qquad{} \bs G_{n }(\Delta - x-u_{n-1})
% 		1(\Delta-x-u_{n-1}\geq0) \wrt u_{n-1}\psi(x) \wrt x \nonumber.
% \end{align}


Finally, we are in a position to prove the main result of this section. 
\begin{proof}[Proof of Theorem~\ref{thm: a thm!}]
	\textit{Cases \(q=r \in \{+,-\}\) and \(m=0\).} Lemma~\ref{lem: Dcoajc} bounds the absolute difference 
	\[\left|\int_{x\in\calD_{\ell_0}}\widehat f_{0,r,r}^{\ell_0,(p)}(\lambda)(x,j;x_0,k)\psi(x)\wrt x-\int_{x\in\calD_{\ell_0}}\widehat \mu_{0,r,r}^{\ell_0}(\lambda)(\wrt x,j;x_0,k)\psi(x)\right|.\]
	Since the bounds from Lemma~\ref{lem: Dcoajc} are \(\mathcal O(\var(Z^{(p)})^{1/3})\) then, as we take \(p \to \infty\), the bounds becomes arbitrarily small which gives the required convergence. 

	\textit{Cases \(q,r\in \{+,-\},\) and \(m\geq 1\).} Given the properties of the functions \( h_{ij}^{u,v}\), \(u,v\in\{+,-\}\), then \(\int_{x\in\calD_{\ell_0}}\widehat f_{0,q,r}^{\ell_0,(p)}(\lambda)(x,j;x_0,k)\psi(x)\wrt x\) satisfies the assumptions of Lemma~\ref{lem: boobies}. To see this, let \(q'\) be the opposite sign to \(q\), i.e.~\(q'\in\{+,-\},\) \(q\neq q'\). Then, in Equation~(\ref{eqn: rhs g 4dvfklsmv2G}), take \(n=2m+1(q=r)\), \(\bs G_1(x_1) = \bs e_i\bs H^{qq'}(\lambda, x_1)\), \(\bs G_{2k}(x_{2k}) = \bs H^{q'q}(\lambda, x_{2k})\), \(\bs G_{2k+1}(x_{2k}) = \bs H^{qq'}(\lambda, x_{2k+1})\), \(k=1,\dots,m-1\); if \(q\neq r\) then take \(\bs G_{2m}(x_{2m}) = \bs H^{rr}(x_{2m})\bs e_j\tr{}\), otherwise, take \(\bs G_{2m}(x_{2m}) = \bs H^{q'r}(x_{2m})\) and \(\bs G_{2m+1} = \bs H^{rr}(\lambda,x_{2m+1})\bs e_j\tr{}\). Thus, Lemma~\ref{lem: boobies}, establishes a bound on~(\ref{eqn: thm 2}) which is \(\mathcal O(\var(Z^{(p)})^{1/3})\), thus taking \(p\to \infty\) gives the stated convergence. 

	\textit{Cases \(q=0,\, r\in\{+,-\}\) and \(m\geq 0\).} 
	Since
	\begin{align}
		\widehat f_{m,0,r}^{\ell_0}(\lambda)(x,j;x_0,k)  \wrt x
		&= \sum_{q\in\{+,-\}}\sum_{i\in\calS_q}\bs e_k\vligne{\lambda \bs I - \bs T_{00}}^{-1}\bs T_{0i}\widehat f_{m+1(r\neq q),q,r}^{\ell_0}(\lambda)(x,j;x_0,i)\wrt x, \label{eqkadv}
	\end{align}
	is a linear combination of terms which are treated in the two cases above, then (\ref{eqkadv}) converges to 
	\begin{align}
		\widehat \mu_{m,0,r}^{\ell_0}(\lambda)(\wrt x,j;x_0,k) 
		&= \sum_{q\in\{+,-\}}\sum_{i\in\calS_q}\bs e_k\vligne{\lambda \bs I - \bs T_{00}}^{-1}\bs T_{0i}\widehat \mu_{m+1(r\neq q),q,r}^{\ell_0}(\lambda)(\wrt x,j;x_0,i),
	\end{align}
	as required. 
\end{proof}


\section{Convergence before the first orbit restart epoch, \(\tau_1\)}\label{sec: before the first}
%via the Dominated Convergence Theorem}\label{sec: dom 1}
  
 Recall that the goal in this chapter is to show a convergence of 
\begin{align*}
	\widehat f_{q,r}^{\ell_0,(p)}(\lambda)(x,j;x_0,i) \wrt x
\to
	\widehat \mu_{q,r}^{\ell_0}(\lambda)(\wrt x,j;x_0,i),
\end{align*} 
where 
\[\widehat f^{\ell_0,(p)}(\lambda)(x,j;x_0,i)= \int_{t=0}^\infty \sum\limits_{m=0}^\infty e^{-\lambda t} f_{m+1(p\neq q),q,r}^{\ell_0,(p)}(t)(x,j;x_0,k)\wrt t.\] 
Since \(f_{m+1(p\neq q),q,r}^{\ell_0,(p)}\) are positive, as is \(e^{-\lambda t}\), then we can use the Fubini-Tonelli Theorem to justify a swap of the integral and infinite sum to get 
\begin{align}
	\widehat f^{\ell_0,(p)}(\lambda)(x,j;x_0,i)=  \sum\limits_{m=0}^\infty \widehat f_{m+1(p\neq q),q,r}^{\ell_0,(p)}(\lambda)(x,j;x_0,k).\label{eqn:LLL}
\end{align}
Similarly, we can write 
\[\widehat \mu^{\ell_0}(\lambda)(\wrt x,j;x_0,i)=  \sum\limits_{m=0}^\infty \widehat \mu_{m+1(p\neq q),q,r}^{\ell_0}(\lambda)(\wrt x,j;x_0,k).\] 
The previous section proved that the Laplace transforms
\[\widehat f_{m,q,r}^{\ell_0,(p)}(\lambda)(x,j;x_0,k)\wrt x\to\widehat \mu_{m,q,r}^{\ell_0}(\lambda)(\wrt x,j;x_0,k),\] 
for \(q\in\{+,-,+0,-0\}\), \(r\in\{+,-\}\). Thus, all we need to show is that, upon taking the limit of (\ref{eqn:LLL}), we can swap the limit and the summation. Here we apply the Dominated Convergence Theorem to justify the swap. To this end, we show a domination condition in Lemma~\ref{lem: gkjljklgagjklagsjlk} below.

Recall \(c_{min} = \min\limits_{i\in\mathcal S_{+}\cup\calS_-} |c_i|\), and let \(E^\lambda\) be an independent exponential random variable with rate \(\lambda\) and \(\gamma = \max_{i\in\calS_+\cup\calS_-}-T_{ii}/|c_i|\). In the following we use the stochastic interpretation of the Laplace transform of a probability distribution with non-negative support and real, non-negative parameter \(\lambda\). For a random variable \(W\) with distribution function \(F_W(w)= \mathbb P(W<w)\), then \(\displaystyle\int_{w=0}^\infty e^{-\lambda w} \wrt F_W(w) = \mathbb P(W < E^{\lambda})\). That is, the Laplace transform with parameter \(\lambda >0\) is the probability that \(W\) occurs before \(E^\lambda\), an independent random exponential time with rate \(\lambda\), occurs. 
\begin{lem}\label{lem: gkjljklgagjklagsjlk}For all \(M\geq 0\), \(x\in\calD_{\ell_0,j}\), \(x_0\in\calD_{\ell_0,i}\), \(\ell_0\in\mathcal K\), \(\lambda > 0\), \(r\in\{+,-\}\), \(j\in\calS_r\cup\calS_{r0}\), and either \(q\in\{+,-\}\), \(i\in\calS_q\), or \(q=0\), \(i\in\calS_0^*\), for any bounded function \(\psi\), \(|\psi|<F\), 
	\begin{align}
		\sum_{m=M+1}^\infty \left| \int_{x\in\calD_{\ell_0}} \widehat f^{\ell_0,(p)}_{m,q,r}(\lambda)(x,j;x_0,i)\psi(x)\wrt x
		-
		\int_{x\in\calD_{\ell_0}} \widehat \mu^{\ell_0}_{m,q,r}(\lambda)(\wrt x,j;x_0,i)\psi(x) \right| \leq r_6^M
	\end{align} 
	where 
	\[r_6^M =  F(\Delta G + \widehat G)\left(\frac{\gamma}{\gamma+\lambda}\right)^{2M+1+1(q=r)} \left(1-\left(\frac{\gamma}{\gamma+\lambda}\right)^2\right)^{-1} .\]
	% Moreover, 
	% \begin{align}
	% 	\sum_{m=M+1}^\infty \left| \int_{x\in\calD_{\ell_0}} \widehat f^{\ell_0,(p)}_{m,0,r}(\lambda)(x,j;x_0,i)\psi(x)\wrt x
	% 	-
	% 	\int_{x\in\calD_{\ell_0}} \widehat \mu^{\ell_0}_{m,0,r}(\lambda)(\wrt x,j;x_0,i)\psi(x) \right| \leq \overline r_6^M
	% \end{align} 
	% where 
	% \[\overline r_6^M =  FW(\Delta G + \widehat G)\left(\frac{\gamma}{\gamma+\lambda}\right)^{2M+2} \left(1-\left(\frac{\gamma}{\gamma+\lambda}\right)^2\right)^{-1} .\]
\end{lem}
Note that the bound \(r_6^M\) is independent of \(p\). 

We prove the result for \(q=r=+\) first, with the proof for the other cases \(q,r\in\{+,-\}\) following analogously. The proof for \(q=0\), \(i\in\calS_0^*\), follows after noting that the Laplace transforms \(\widehat f^{\ell_0,(p)}_{m,0,r}(\lambda)(x,j;x_0,i)\) and \(\widehat \mu^{\ell_0}_{m,0,r}(\lambda)(\wrt x,j;x_0,i)\) are linear combinations of \(\widehat f^{\ell_0,(p)}_{m,q,r}(\lambda)(x,j;x_0,i)\) and \(\widehat \mu^{\ell_0}_{m,q,r}(\lambda)(\wrt x,j;x_0,i)\) for \(q\in\{+,-\}\), respectively. 

Essentially, the result for \(q=r=+\) follows from noting the probabilistic interpretation of the Laplace transforms \(\widehat f^{\ell_0}_{m,+,+}(\lambda)(x,j;x_0,i)\), as the probability that, 
\begin{itemize}
	\item there are \(m\) up-down and down-up transitions, 
	\item the orbit process \(\{\bs A(t)\}\) evolves accordingly, 
	\item and an independent exponential random variable with rate \(\lambda\), \(E^\lambda\), has not yet occurred,
	\item before the first orbit restart epoch.
\end{itemize}
We obtain an upper bound by ignoring the behaviour of the orbit process \(\{\bs A(t)\}\), then, by a uniformisation argument, we bound the probability that there are \(m\) up-down and down-up transitions before \(E^\lambda\) occurs, by the event that there are \(m\) independent exponential events before an \(E^\lambda\) occurs.

Similarly, the stochastic interpretation of the Laplace transforms \(\widehat \mu^{\ell_0}_{m,+,+}(\lambda)(\wrt x,j;x_0,i)\), is the probability that, 
\begin{itemize}
	\item there are \(m\) up-down and down-up transitions, 
	\item the fluid level \(X(t)\) remains in \(\mathcal D_{\ell_0}\), 
	\item and an independent exponential random variable with rate \(\lambda\), \(E^\lambda\), has not yet occurred,
	\item before the first orbit restart epoch.
\end{itemize}
We obtain an upper bound by removing the requirement that the fluid level \(X(t)\) remain in \(\mathcal D_{\ell_0}\), then applying the same uniformisation argument as we do for \(\widehat f^{\ell_0}_{m,+,+}(\lambda)(x,j;x_0,i)\).

\begin{proof}
	The same arguments and results apply for all \(p\), so let us drop the dependence on \(p\). 
	
	Consider \(i\in\calS_+,j\in\mathcal S_+\cup\calS_{+0}\). By the triangle inequality, 
	\begin{align*}
		&\sum_{m=M+1}^\infty \left| \int_{x\in\calD_{\ell_0}} \widehat f^{\ell_0}_{m,+,+}(\lambda)(x,j;x_0,i)\psi(x) \wrt x
		-
		 \int_{x\in\calD_{\ell_0}} \widehat \mu^{\ell_0}_{m,+,+}(\lambda)(\wrt x,j;x_0,i) \psi(x) \right|
		\\&\leq\sum_{m=M+1}^\infty \int_{x\in\calD_{\ell_0}} \widehat f^{\ell_0}_{m,+,+}(\lambda)(x,j;x_0,i) |\psi(x)| \wrt x
		\\&\qquad{} +\sum_{m=M+1}^\infty \int_{x\in\calD_{\ell_0}} \widehat  \mu^{\ell_0}_{m,+,+}(\lambda)(\wrt x,j;x_0,i) |\psi(x)|,
	\end{align*}
	since all terms are non-negative. 
	
	Consider \(\displaystyle\int_{x\in\calD_{\ell_0}}\widehat f^{\ell_0}_{m,+,+}(\lambda)(x,j;x_0,i)|\psi(x)|\wrt x\), which is given by 
        \begin{align}
        	&\int_{x\in\calD_{\ell_0}}\bs e_i \left[\prod_{r=1}^m\int_{x_{2r-1}=0}^\infty \bs H^{+-}(\lambda,x_{2r-1})\nonumber
			\int_{x_{2r}=0}^\infty \bs H^{-+}(\lambda,x_{2r}) \right]
			\int_{x_{2m+1}=0}^\infty \bs H^{++}(\lambda,x_{2m+1}) \bs e_j\tr{}
			\\&\quad\bs a_{\ell_0,i}(x_0) \bs N^{2m+1}(\lambda,x_1,\dots,x_{2m+1}) {\bs v}_{\ell_0,j}( x)\wrt x_{2m+1}\dots\wrt x_1 \psi(x)\wrt x\nonumber
			%
			\\&\leq \int_{x\in\calD_{\ell_0}}\bs e_i \left[\prod_{r=1}^m\int_{x_{2r-1}=0}^\infty \bs H^{+-}(\lambda,x_{2r-1})\nonumber
			\int_{x_{2r}=0}^\infty \bs H^{-+}(\lambda,x_{2r}) \right]
			\int_{x_{2m+1}=0}^\infty \bs H^{++}(\lambda,x_{2m+1}) \bs e_j\tr{}
			\\&\quad\bs a_{\ell_0,i}(x_0) \bs N^{2m+1}(\lambda,x_1,\dots,x_{2m+1}) {\bs v}_{\ell_0,j}( x)\wrt x_{2m+1}\dots\wrt x_1 \wrt x F 
			\label{eqn:llkjhslkj}
	\end{align}
	since \(|\psi|\leq F\). To bound the last-line of (\ref{eqn:llkjhslkj}) we first observe that for \(\bs a \in \mathcal A\), 
	\begin{align}
		\bs a\int_{x\in\calD_{\ell_0}}\bs De^{\bs Sx_{2m+1}}{\bs v}_{\ell_0,j}(x) 
		&= \bs a\int_{x\in\calD_{\ell_0}}\int_{u=0}^\infty e^{\bs Su}\bs s\cfrac{\bs \alpha e^{\bs S u}}{\bs \alpha e^{\bs S u}\bs e}e^{\bs Sx_{2m+1}}{\bs v}_{\ell_0,j}(x)\wrt u\wrt x\nonumber
		\\&\leq \bs a\int_{u=0}^\infty e^{\bs Su}\bs s\cfrac{\bs \alpha e^{\bs S u}}{\bs \alpha e^{\bs S u}\bs e}e^{\bs Sx_{2m+1}}{\bs e}\wrt u\nonumber
		\\&= \bs a\bs D e^{\bs Sx_{2m+1}}{\bs e}\wrt u, \label{eqn: kkkaa}
	\end{align}
	where the inequality holds from Property~\ref{properties: -2}. By definition, the last-line of (\ref{eqn:llkjhslkj}) is 
	\begin{align}
		&\int_{x\in\calD_{\ell_0}} \bs a_{\ell_0,i}(x_0) \bs N^{2m+1}(\lambda,x_1,\dots,x_{2m+1}) {\bs v}_{\ell_0,j}( x)\wrt x_{2m+1}\dots\wrt x_1\nonumber
		\\&=\int_{x\in\calD_{\ell_0}}\bs a_{\ell_0,i}(x_0)e^{\bs{S}x_1} \bs{D}e^{\bs{S}x_2} \bs{D}\dots e^{\bs{S}x_{2m}} \bs{D}e^{\bs{S}x_{2m+1}}{\bs v}_{\ell_0,j}(x).\label{eqn:llaaaaaa}
	\end{align}
	Now, using (\ref{eqn: kkkaa}), then (\ref{eqn:llaaaaaa}) is less than or equal to 
	\begin{align*}
		&\bs a_{\ell_0,i}(x_0)e^{\bs{S}x_1} \bs{D}e^{\bs{S}x_2} \bs{D}\dots e^{\bs{S}x_{2m}} \bs{D}e^{\bs{S}x_{2m+1}}{\bs e}
		\\& = \bs a_{\ell_0,i}(x_0)e^{\bs{S}x_1} \bs{D}e^{\bs{S}x_2} \bs{D}\dots e^{\bs{S}x_{2m}} \int_{u=0}^\infty e^{\bs Su}\cfrac{\bs \alpha e^{\bs Su}}{\bs \alpha e^{\bs Su}\bs e}\wrt u e^{\bs{S}x_{2m+1}}{\bs e}
		\\&\leq \bs a_{\ell_0,i}(x_0)e^{\bs{S}x_1} \bs{D}e^{\bs{S}x_2} \bs{D}\dots e^{\bs{S}x_{2m}} \int_{u=0}^\infty e^{\bs Su}\cfrac{\bs \alpha e^{\bs Su}}{\bs \alpha e^{\bs Su}\bs e}\wrt u {\bs e}
		\\&=\bs a_{\ell_0,i}(x_0)e^{\bs{S}x_1} \bs{D}e^{\bs{S}x_2} \bs{D}\dots e^{\bs{S}x_{2m}} \bs e.
	\end{align*}
	Repeating \(m\) more times gives the bound \(\bs a_{\ell_0,i}(x_0)\bs e=1\).\footnote{In fact, this bound holds for any initial vector \(\bs a \in\mathcal A\).} Hence, we have the bound 
	\[\bs a_{\ell_0,i}(x_0)e^{\bs{S}x_1} \bs{D}e^{\bs{S}x_2} \bs{D}\dots e^{\bs{S}x_{2m}} \bs{D}e^{\bs{S}x_{2m+1}}{\bs e}\leq 1.\]

    Therefore, (\ref{eqn:llkjhslkj}) is less than or equal to 
	\begin{align}
		& \bs e_i \left[\prod_{r=1}^m\int_{x_{2r-1}=0}^\infty \bs H^{+-}(\lambda,x_{2r-1})\nonumber
		\int_{x_{2r}=0}^\infty \bs H^{-+}(\lambda,x_{2r}) \right]
		\int_{x_{2m+1}=0}^\infty \bs H^{++}(\lambda,x_{2m+1}) \bs e_j\tr{}
		\\&\quad \wrt x_{2m+1} \dots \wrt x_1 F. \label{eqn :NNeeaefjn}
	\end{align}

	Now, for any row-vector of non-negative numbers \(\bs b\), since the elements of \(\bs H^{++}\) are non-negative and integrable, then 
	\[\bs b \int_{x_{2m+1}=0}^\infty \bs H^{++}(\lambda, x_{2m+1})\wrt x_{2m+1} \bs e_j\tr{} \leq \bs b \bs e_j\tr{} \widehat G\leq \bs b \bs e \widehat G.\]
	Observing that  
	\[\bs e_i \left[\prod_{r=1}^m\int_{x_{2r-1}=0}^\infty \bs H^{+-}(\lambda,x_{2r-1})\nonumber
	\int_{x_{2r}=0}^\infty \bs H^{-+}(\lambda,x_{2r}) \right]
	\wrt x_{2m}\dots \wrt x_1\] 
	is a vector of row-vector non-negative numbers, then (\ref{eqn :NNeeaefjn}) is less than or equal to 
	\begin{align}
		&\bs e_i \left[\prod_{r=1}^m\int_{x_{2r-1}=0}^\infty \bs H^{+-}(\lambda,x_{2r-1})
		\int_{x_{2r}=0}^\infty \bs H^{-+}(\lambda,x_{2r}) \right]
		\bs e\wrt x_{2m}\dots \wrt x_1\widehat G F\label{eqn :NNeeaefjn12}
	\end{align}

	The stochastic interpretation of the \(i\)th element of the vector \(\bs H^{+-}(\lambda,x)\bs e\) is that it is the probability density of an up-down transition at the time when the in-out fluid has increased by \(\wrt x\) and before an exponential random variable with rate \(\lambda\) occurs, given the phase is initially \(i\). There may be multiple changes of phase within \(\mathcal S_+\cup\calS_{+0}\) before the first up-down transition. The first change of phase occurs at rate (with respect to the in-out level) \(-T_{ii}/|c_i|\) and this is the lowest in-out fluid level at which it may be possible to see an up-down transition. Consider a uniformised version of the in-out fluid process with uniformisation parameter \(\gamma = \max\limits_{i\in\mathcal S_+\cup\calS_-}-T_{ii}/|c_i|\). Then the first event of the phase process of the uniformised version of the in-out fluid process occurs at rate \(\gamma\) and occurs at, or before, the first change of phase of the uniformised process. Therefore, the first uniformisation event occurs at, or before, the first up-down transition of the uniformised version of the in-out process. Hence, the first uniformisation event occurs at, or before, the first up-down transition of the original process (since they are versions of each other). This gives the bound \(\bs H^{+-}(\lambda,x)\bs e\leq \gamma e^{-(\lambda + \gamma)x}\bs e\) where the inequality is understood elementwise. Similarly, for \(\bs H^{-+}(\lambda,x)\bs e\leq \gamma e^{-(\lambda + \gamma)x}\bs e\).
	
	From the stochastic interpretation above, (\ref{eqn :NNeeaefjn12}) is less than or equal to 
	\begin{align}
	%
	&\bs e_i \bs H^{+-}(\lambda,x_1) \wrt x_1 \int_{x_2=0}^\infty \bs H^{-+}(\lambda,x_2) \bs e \wrt x_2  
				\hdots \int_{x_{2m}=0}^\infty \gamma e^{(-\gamma-\lambda)x_{2m}}\wrt x_{2m}\widehat GF\nonumber
	%
	\\&\leq \int_{x_1=0}^\infty \gamma e^{(-\gamma-\lambda)x_1}  \wrt x_1 \int_{x_2=0}^\infty \gamma e^{(-\gamma-\lambda)x_2}  \wrt x_2  
				\hdots \int_{x_{2m}=0}^\infty \gamma e^{(-\gamma-\lambda)x_{2m}}\wrt x_{2m}\widehat GF \nonumber
	%
	\\&= \left(\cfrac{\gamma}{\gamma+\lambda}\right)^{2m}\widehat GF.\label{eqn: bound ggggaaaa}
	\end{align}
	Note, for the cases with \(q\neq r\) then the equivalent bound to (\ref{eqn: bound ggggaaaa}) is \(\left(\frac{\gamma}{\gamma+\lambda}\right)^{2m-1}\widehat GF\) as there is one less factor in the expression. 

	Hence,  
	\begin{align}
		\sum_{m=M+1}^\infty \int_{x\in\calD_{\ell_0}} \widehat f^{\ell_0}_{m,+,+}(\lambda)(x,j;x_0,i)|\psi(x)|\wrt x \nonumber
		&\leq \widehat GF  \sum_{m=M+1}^\infty \left(\cfrac{\gamma}{\gamma+\lambda}\right)^{2m}\nonumber
		\\&\leq \widehat GF \left(\cfrac{\gamma}{\gamma+\lambda}\right)^{2M+2} \left(1-\left(\cfrac{\gamma}{\gamma+\lambda}\right)^2\right)^{-1}.
	\end{align}
	
	Now consider \(\widehat\mu_{m,+,+}^{\ell_0}(\lambda)( \wrt x,j;x_0,i) \) which is given by 
	\begin{align}
	\nonumber& \int_{x_1 = 0}^{\Delta-(x_0-y_{\ell_0})} \bs e_i\bs H^{+-}(\lambda,\Delta-(x_0-y_{\ell_0})-x_1) \int_{x_2 = 0}^{\Delta-x_1} \bs H^{-+}(\lambda,\Delta - x_2-x_1) \wrt x_1 
	\\\nonumber &\dots  
	\int_{x_{2m}=0}^{\Delta-x_{m-1}} \bs H^{-+}(\lambda,\Delta -x_{2m-1} - x_{2m}) \wrt x_{2m-1}
		\bs H^{++}(\lambda,\Delta -x_{2m}- (y_{\ell_0+1}- x))\bs e_j\tr{}\wrt x_{2m}\wrt x
	\\\nonumber&= \int_{x_1 = (x_0-y_{\ell_0})}^{\Delta} \bs e_i \bs H^{+-}(\lambda,\Delta-x_1) \int_{x_2 = x_1}^{\Delta} \bs H^{-+}(\lambda,\Delta -x_2) \wrt x_1 
	\dots  
	\\&\int_{x_{2m}=x_{2m-1}}^{\Delta} \bs H^{-+}(\lambda,\Delta - x_{2m}) 
	\bs H^{++}(\lambda,\Delta -x_{2m}-x_{2m-1}- (y_{\ell_0+1}- x))\bs e_j\tr{} 
	\wrt x_{2m-1}\wrt x_{2m}\wrt x.\label{eqn:m789J}
	\end{align}
	Using the bound \(\bs H^{++}(\lambda,x_{m+1})\leq G\) elementwise, then (\ref{eqn:m789J}) is less than or equal to 
	\begin{align}
		&\int_{x_1 = (x_0-y_{\ell_0})}^{\Delta} \bs e_i\bs H^{+-}(\lambda,\Delta-x_1) \int_{x_2 = x_1}^{\Delta} \bs H^{-+}(\lambda,\Delta -x_2) \wrt x_1 \nonumber
		\\&\quad\dots  
		\int_{x_{2m}=x_{2m-1}}^{\Delta} \bs H^{-+}(\lambda,\Delta - x_{2m})  
		\wrt x_{2m-1}\wrt x_{2m}\bs e G\wrt x.\label{eqn:m789J2}
	\end{align}
	The expression~(\ref{eqn:m789J2}) differs from (\ref{eqn :NNeeaefjn12}) only by a constant factor and that the integrals in the (\ref{eqn:m789J2}) are finite, hence we may bound it in the same way. Therefore, 
	\begin{align}
		\sum_{m=M+1}^\infty \int_{x\in\calD_{\ell_0}} \widehat \mu^{\ell_0}_{m,+,+}(\lambda)(\wrt x,j;x_0,i) |\psi(x)| \nonumber
		&\leq G \sum_{m=M+1}^\infty\left(\frac{\gamma}{\gamma+\lambda}\right)^{2m}\int_{x\in\calD_{\ell_0}}\psi(x)\wrt x
		\\&\leq G\Delta F\left(\cfrac{\gamma}{\gamma+\lambda}\right)^{2M+2} \left(1-\left(\cfrac{\gamma}{\gamma+\lambda}\right)^2\right)^{-1}.
	\end{align}
        
	Once again, for \(q\neq r\) the bound on \(\int_{x\in\calD_{\ell_0}} \widehat \mu^{\ell_0}_{m,+,+}(\lambda)(\wrt x,j;x_0,i) |\psi(x)|\) is \((\frac{\gamma}{\gamma+\lambda})^{2m-1}\) as there is one less factor.
	
	Analogous arguments show the same bounds for any \(i\in\calS_q\), \(j\in\calS_r\cup\calS_{r0}\), where \(q\in\{+,-\}\), \(r\in\{+,-\}\). 

	The result for \(q=0\), \(i\in\calS_0^*\), \(j\in\calS_r\cup\calS_{r0}\), \(r\in\{+,-\}\) holds after noting that 
	\begin{align}
		\widehat f_{m,0,r}^{\ell_0}(\lambda)(x,j;x_0,i)  \wrt x
		&= \sum_{q\in\{+,-\}}\sum_{k\in\calS_q}\bs e_i\vligne{\lambda \bs I - \bs T_{00}}^{-1}\bs T_{0k}\widehat f_{m+1(r\neq q),q,r}^{\ell_0}(\lambda)(x,j;x_0,k)\wrt x, \label{eqkadv2}
		\\\widehat \mu_{m,0,r}^{\ell_0}(\lambda)(\wrt x,j;x_0,i) 
		&= \sum_{q\in\{+,-\}}\sum_{k\in\calS_q}\bs e_i\vligne{\lambda \bs I - \bs T_{00}}^{-1}\bs T_{0k}\widehat \mu_{m+1(r\neq q),q,r}^{\ell_0}(\lambda)(\wrt x,j;x_0,k),\label{eqkadv22}
	\end{align}
	are a linear combination of terms which are treated the cases above, hence we may use the bounds we have established above. First, use the triangle inequality to write, 
	\begin{align}
		&\sum_{m=M+1}^\infty \left| \int_{x\in\calD_{\ell_0}} \widehat f^{\ell_0}_{m,0,+}(\lambda)(x,j;x_0,i)\psi(x) \wrt x
		-
		 \int_{x\in\calD_{\ell_0}} \widehat \mu^{\ell_0}_{m,0,+}(\lambda)(\wrt x,j;x_0,i) \psi(x) \right| \nonumber 
		 \\&\leq\sum_{m=M+1}^\infty \int_{x\in\calD_{\ell_0}} \widehat f^{\ell_0}_{m,0,+}(\lambda)(x,j;x_0,i) |\psi(x)| \wrt x \nonumber 
		 \\&\qquad{} +\sum_{m=M+1}^\infty \int_{x\in\calD_{\ell_0}} \widehat  \mu^{\ell_0}_{m,0,+}(\lambda)(\wrt x,j;x_0,i) |\psi(x)|. \label{eqn: wejf}
	\intertext{Upon substituting (\ref{eqkadv2}) and (\ref{eqkadv22}) into (\ref{eqn: wejf}) we get }
		 &\sum_{m=M+1}^\infty \int_{x\in\calD_{\ell_0}} \sum_{q\in\{+,-\}}\sum_{k\in\calS_q}\bs e_i\vligne{\lambda \bs I - \bs T_{00}}^{-1}\bs T_{0k}\widehat f_{m+1(r\neq q),q,r}^{\ell_0}(\lambda)(x,j;x_0,k)\wrt x |\psi(x)| \wrt x \nonumber 
		\\&\qquad{} +\sum_{m=M+1}^\infty \int_{x\in\calD_{\ell_0}} \sum_{q\in\{+,-\}}\sum_{k\in\calS_q}\bs e_i\vligne{\lambda \bs I - \bs T_{00}}^{-1}\bs T_{0k}\widehat \mu_{m+1(r\neq q),q,r}^{\ell_0}(\lambda)(\wrt x,j;x_0,k)|\psi(x)| \nonumber 
		\\&=\sum_{q\in\{+,-\}}\sum_{k\in\calS_q}\sum_{m=M+1}^\infty \int_{x\in\calD_{\ell_0}} \bs e_i\vligne{\lambda \bs I - \bs T_{00}}^{-1}\bs T_{0k}\widehat f_{m+1(r\neq q),q,r}^{\ell_0}(\lambda)(x,j;x_0,k)\wrt x |\psi(x)| \wrt x\nonumber 
		\\&\qquad{} +\sum_{q\in\{+,-\}}\sum_{k\in\calS_q}\sum_{m=M+1}^\infty \int_{x\in\calD_{\ell_0}} \bs e_i\vligne{\lambda \bs I - \bs T_{00}}^{-1}\bs T_{0k}\widehat \mu_{m+1(r\neq q),q,r}^{\ell_0}(\lambda)(\wrt x,j;x_0,k)|\psi(x)|\label{eqn: akvnw} 
		\intertext{where the swap of the sums and integrals is justified by Tonelli's Theorem. Now, using the bounds we found earlier, then (\ref{eqn: akvnw}) is less than or equal to}
		& \sum_{q\in\{+,-\}}\sum_{k\in\calS_q}\bs e_i\vligne{\lambda \bs I - \bs T_{00}}^{-1}\bs T_{0k}F\widehat G\left(\frac{\gamma}{\gamma+\lambda}\right)^{2M+2} \left(1-\left(\frac{\gamma}{\gamma+\lambda}\right)^2\right)^{-1}\nonumber
		\\&\qquad{} +\sum_{q\in\{+,-\}}\sum_{k\in\calS_q}\bs e_i\vligne{\lambda \bs I - \bs T_{00}}^{-1}\bs T_{0k}F\Delta G \left(\frac{\gamma}{\gamma+\lambda}\right)^{2M+2} \left(1-\left(\frac{\gamma}{\gamma+\lambda}\right)^2\right)^{-1}\nonumber
		\\&\leq \sum_{q\in\{+,-\}}\sum_{k\in\calS_q}\bs e_i\vligne{\lambda \bs I - \bs T_{00}}^{-1}\bs T_{0k}F\widehat G\left(\frac{\gamma}{\gamma+\lambda}\right)^{2M+1} \left(1-\left(\frac{\gamma}{\gamma+\lambda}\right)^2\right)^{-1}\nonumber
		\\&\qquad{} +\sum_{q\in\{+,-\}}\sum_{k\in\calS_q}\bs e_i\vligne{\lambda \bs I - \bs T_{00}}^{-1}\bs T_{0k}F\Delta G \left(\frac{\gamma}{\gamma+\lambda}\right)^{2M+1} \left(1-\left(\frac{\gamma}{\gamma+\lambda}\right)^2\right)^{-1}\nonumber
		\\&\leq F(\widehat G + \Delta G)\left(\frac{\gamma}{\gamma+\lambda}\right)^{2M+1} \left(1-\left(\frac{\gamma}{\gamma+\lambda}\right)^2\right)^{-1}\nonumber 
	\end{align}
	since 
	\begin{align*}
		\sum\limits_{q\in\{+,-\}}\sum_{k\in\calS_q}\bs e_i\vligne{\lambda \bs I - \bs T_{00}}^{-1}\bs T_{0k} 
		%
		&=\sum\limits_{q\in\{+,-\}}\sum_{k\in\calS_q}\bs e_i\int_{t=0}^\infty e^{(\bs T_{00}-\lambda \bs I)t}\wrt t\bs T_{0k}
		%
		\\&\leq\sum\limits_{q\in\{+,-\}}\sum_{k\in\calS_q}\bs e_i\int_{t=0}^\infty e^{\bs T_{00}t}\wrt t\bs T_{0k} 
		%
		\\&= \sum\limits_{q\in\{+,-\}}\sum_{k\in\calS_q}\bs e_i (-\bs T_{00})^{-1}\bs T_{0k} 
		%
		\\&= \bs e_i (-\bs T_{00})^{-1}(-\bs T_{00})\bs e\tr{} 
		%
		\\&= 1.
	\end{align*}
\end{proof}

Combining the domination in Lemma~\ref{lem: gkjljklgagjklagsjlk} and the convergence in Theorem~\ref{thm: a thm!} via the Dominated Convergence Theorem gives the following result. 
\begin{lem} \label{lem:vn4}
	For all \(x\in\calD_{\ell_0,j}\), \(x_0\in\calD_{\ell_0,i}\), \(i,j\in\calS\), \(\ell_0\in\mathcal K\), \(\lambda > 0\),  
	\begin{align}
		&\left|\int_{x\in\calD_{\ell_0}}\widehat f^{\ell_0,(p)}(\lambda)(x,j;x_0,i)\psi(x) \wrt x - \int_{x\in\calD_{\ell_0}}\widehat \mu^{\ell_0}(\lambda)(\wrt x,j; x_0,i)\psi(x) \right|\to 0  \label{eqn: akhv}
	\end{align}
	as \(p\to\infty\). 
\end{lem}
	 
\begin{rem}\label{rem: point wies}
	For a fixed \(\lambda > 0\), convergence of 
	\begin{align}
		\left|\widehat f^{\ell_0,(p)}(\lambda)(x,j;x_0,i)\wrt x - \widehat \mu^{\ell_0}(\lambda)( \wrt x,j; x_0,i) \right|
	\end{align}
	actually holds pointwise for each \(\ell_0\in\mathcal K^\circ\), and each \(i,j\in\mathcal S,\) \(x_0\in\mathcal D_{\ell_0,i}\), \(x\in\calD_{\ell_0,j}\) except at the set of points where \(x=x_0\). Specifically, the lack of pointwise convergence at this point occurs due to terms with the index \(m=0\), that is, terms where there are no up-down or down-up transitions. On these sample paths the relevant Laplace transforms of the fluid queue are discontinuous at this point. For example, 
	\begin{align*}
		\widehat \mu^{\ell_0}_{0,+,+}(\lambda)(\wrt x,j;x_0,i) &= h_{ij}^{++}(\lambda,x-x_0)1(x\geq x_0)\wrt x,
	\end{align*}
	is discontinuous at \(x=x_0\). 
\end{rem}

\section{Convergence at the time of the first orbit restart epoch, \(\tau_1\)}\label{sec: 1st change}
We conclude this chapter with a statement about a convergence of the QBD-RAP to the fluid queue \emph{at the time of the first orbit restart epoch}, \(\tau_1^{(p)}\). 

\begin{cor}\label{cor: aln222} Recall \(\bs y_{0}^{(p)} = (\ell_0, \bs a_{\ell_0,j}^{(p)}(x_0), i)\). For \(\ell_0\in\mathcal K\) \(x_0\in\mathcal D_{\ell_0,i}\), \(i\in\mathcal S_+\cup\calS_-\cup\calS_{0}^{*},\) \(j\in\calS_+\cup\calS_-\), 
	\begin{align}
		&\mathbb P(L^{(p)}(\tau_1^{(p)}) = \ell(\ell_0,j), \varphi(\tau_1^{(p)}) = j, \tau_{1}^{(p)}\leq E^\lambda 
            	 \mid \bs Y^{(p)}(0) = \bs y_0^{(p)}) \nonumber
	 	%
		\\&\to \mathbb P(\bs X(\tau_1^X) = (y_{\ell(\ell_0,j)+1(j\in\calS_-)}, j), \tau_{1}^X\leq E^\lambda 
            	 \mid \bs X(0) = (x_0,i))\label{eqn: 1421}
	\end{align}
	where \(\ell(\ell_0,j)\) can take values
	\begin{align*}
		\ell(\ell_0,j) = \begin{cases}
			\ell_0-1, &\mbox{ if } \ell_0\in\{0,1,\dots,K+1\},\, j\in\calS_-\\
			%
			\ell_0, & \mbox{ if } \ell_0 = 0, j\in\mathcal S_+, \mbox{ or }\ell_0 = K, j\in\mathcal S_-,\\
			%
			\ell_0+1, & \mbox{ if } \ell_0\in\{-1,0,1,\dots,K\},\, j\in\calS_+.
		\end{cases}
	\end{align*}
\end{cor}
\begin{proof}
	The proof follows the same structure as the proof of Theorem~\ref{thm: a thm!} however, changes are required in all the results used in the proof, as here we do not need to integrate a function \(\psi\). We give an outline of the proof only. 

	At a boundary we can model the fluid queue exactly, hence (\ref{eqn: 1421}) holds for \(\ell_0=-1\) and \(\ell_0=K+1\).

	Now consider \(i\in \calS_+,j\in\calS_+\). Partition the probability (\ref{eqn: 1421}) on the times \(\{\Sigma_n\}_{n\geq 1}\) and \(\{\Gamma_n\}_{n\geq 1}\) and, specifically, partition on the event that there are exactly \(m\) events \(\{\Sigma_n\}_{n=1}^m\) and exactly \(m\) events \(\{\Gamma_n\}_{n=1}^m\). 
	The resulting partitioned probabilities are
	\begin{align}
                 &\int_{x_1=0}^\infty \left(\bs e_i\bs H^{+-}(\lambda,x_1)\otimes \bs a_{\ell_0,i}^{(p)}(x_0)e^{\bs{S}^{(p)}x_1}\bs{D}^{(p)}\right)\wrt x_1 \nonumber
            	\\&\nonumber \quad \Bigg[\prod_{r=1}^{m-1} \int_{x_{2r}=0}^\infty \left(\bs H^{-+}(\lambda,x_{2r}) \otimes e^{\bs{S}^{(p)}x_{2r}}\bs D^{(p)}\right)\wrt x_{2r} \\&\quad \int_{x_{2r+1}=0}^\infty \left(\bs H^{+-}(\lambda,x_{2r+1}) \otimes e^{\bs{S}^{(p)}x_{2r+1}}\bs D^{(p)}\right) \wrt x_{2r+1}\Bigg] \nonumber
            	\\&
            	\quad \int_{x_{2m}=0}^\infty \left(\bs H^{-+}(\lambda, x_{2m}) \otimes e^{\bs{S}^{(p)}x_{2m}}\bs{D}^{(p)}\right) \wrt x_{2m} \nonumber
				\\&\quad \int_{x_{2m+1}=0}^\infty \left( \bs H^{++}(\lambda,x_{2m+1})\bs e_j\tr{} \otimes  e^{\bs{S}^{(p)}x_{2m+1}}\bs s^{(p)} \right) \wrt x_{2m+1}.  \label{eqn: prob ofjaiv}
	\end{align}

	To show that the terms (\ref{eqn: prob ofjaiv}) converge to 
	\begin{align}
		&\mathbb P(\bs X(\tau_1^X)=(y_{\ell_0+1},j),\tau_1^X\leq E^\lambda, \Sigma_{m}\leq \tau_1^X<\Gamma_{m+1}, \mid \bs X(0)=(x_0, i))
	\end{align}
	we can use the bounds from Corollary~\ref{cor: cond bnd 2} and Corollary~\ref{cor: a cor}. For \(m=0\) we recognise (\ref{eqn: prob ofjaiv}) as the same form as that appearing in Corollary~\ref{cor: cond bnd 2} upon choosing \(v=0\). For \(m\geq 1\), choose the closing operator to be \(\bs v(x)=e^{\bs S x}\bs s\) and set \(x=0\) in Corollary~\ref{cor: a cor}. Now take the bound from Corollary~\ref{cor: a cor} and extend it to the case of matrix functions in the same way we extended Corollary~\ref{cor: asjdajaaaaa} to the matrix case in Lemma~\ref{lem: boobies}. In this way, we have a bound for (\ref{eqn: prob ofjaiv}) which tends to \(0\) as \(p\to\infty\). Analogous arguments give the convergence for all terms \(i\in\calS_+\cup\calS_-\cup\calS_0^*,\,j\in\calS_+\cup\calS_{+0}\cup\calS_-\cup\calS_{-0}\) on a given number of up-down/down-up transitions. 

	What remains is a domination condition so that we may apply the Dominated Convergence Theorem to claim that the sum over the number of up-down and down-up transition converges (i.e.~the sum over \(m\) in (\ref{eqn: prob ofjaiv}) converges). After algebraic manipulation, (\ref{eqn: prob ofjaiv}) is 
	\begin{align}
			&\bs e_i \left[\prod_{r=1}^m\int_{x_{2r-1}=0}^\infty \bs H^{+-}(\lambda,x_{2r-1})\nonumber
			\int_{x_{2r}=0}^\infty \bs H^{-+}(\lambda,x_{2r}) \right]
			\int_{x_{2m+1}=0}^\infty \bs H^{++}(\lambda,x_{2m+1}) \bs e_j\tr{}
		\\&\quad\bs a_{\ell_0,i}(x_0) \bs N^{2m}(\lambda,x_1,\dots,x_{2m})\bs De^{\bs Sx_{2m+1}} {\bs s} \wrt x_{2m+1} \dots \wrt x_1 \label{eqn: hhhHHHaa}  
	\end{align}
	Now, since \(\left[\bs H^{++}(\lambda,x_{2m+1})\right]_{ij}\leq G\) and  
	\[\bs e_i \left[\prod_{r=1}^m\int_{x_{2r-1}=0}^\infty \bs H^{+-}(\lambda,x_{2r-1})\nonumber
	\int_{x_{2r}=0}^\infty \bs H^{-+}(\lambda,x_{2r}) \right]\] 
	is a row-vector of positive numbers, then (\ref{eqn: hhhHHHaa}) is less than or equal to 
	\begin{align}
		&\bs e_i \left[\prod_{r=1}^m\int_{x_{2r-1}=0}^\infty \bs H^{+-}(\lambda,x_{2r-1})\nonumber
		\int_{x_{2r}=0}^\infty \bs H^{-+}(\lambda,x_{2r}) \right]
		\int_{x_{2m+1}=0}^\infty  \bs e G
		\\&\quad\bs a_{\ell_0,i}(x_0) \bs N^{2m}(\lambda,x_1,\dots,x_{2m})\bs De^{\bs Sx_{2m+1}}{\bs s} \wrt x_{2m+1}\dots \wrt x_1.\nonumber
	\end{align}
	Integrating with respect to \(x_{2m+1}\) gives
	\begin{align}
		&\bs e_i \left[\prod_{r=1}^m\int_{x_{2r-1}=0}^\infty \bs H^{+-}(\lambda,x_{2r-1})\nonumber
		\int_{x_{2r}=0}^\infty \bs H^{-+}(\lambda,x_{2r}) \right]
		 \bs e G
		\\&\quad\bs a_{\ell_0,i}(x_0) \bs N^{2m}(\lambda,x_1,\dots,x_{2m})\bs D {\bs e}\wrt x_{2m} \dots \wrt x_1\nonumber
		%
		\\&\leq\bs e_i \left[\prod_{r=1}^m\int_{x_{2r-1}=0}^\infty \bs H^{+-}(\lambda,x_{2r-1})
		\int_{x_{2r}=0}^\infty \bs H^{-+}(\lambda,x_{2r}) \right]
		 \bs e G \wrt x_{2m}\dots \wrt x_1 \label{eqn :ejrvn}
	\end{align}
	the last inequality holds since, \(\bs D{\bs e}=\bs e\), and \(\bs a_{\ell_0,i}(x_0) \bs N^{2m}(\lambda,x_1,\dots,x_{2m}){\bs e}\leq 1\), as we claimed previously in the discussion after (\ref{eqn :NNeeaefjn}) in the proof of Lemma~\ref{lem: gkjljklgagjklagsjlk}. Equation~(\ref{eqn :ejrvn}) is of a similar form to (\ref{eqn :NNeeaefjn12}) (they differ only by a constant), hence the same arguments used to bound (\ref{eqn :NNeeaefjn12}) can be applied to get the desired domination result. 

	Analogous arguments give a suitable geometric bound for \(i,j\in\calS_+\cup\calS_-\). For \(i\in\calS_0^*\), the left-hand side of (\ref{eqn: 1421}) can be written as a linear combination of the left-hand side of (\ref{eqn: 1421}) for initial phases in \(\calS_+\cup\calS_-\). Hence, a bound follows along similar arguments to those used to prove the case \(i\in\calS_0^*\) in Lemma~\ref{lem: gkjljklgagjklagsjlk}.
	
	Ultimately, we can apply the Dominated Convergence Theorem to prove that the sum of the partitioned probabilities (\ref{eqn: prob ofjaiv}) converges as \(p\to\infty\). The sum of the limits is 
	\[\mathbb P(\bs X(\tau_1^X) = (y_{\ell_0+1}, j), \tau_{1}\leq E^\lambda 
            	 \mid \bs X(0) = (x_0,i)).\]
	 
	%  The results for all other cases of \(i,j\in\calS\) follow analogously.
\end{proof}