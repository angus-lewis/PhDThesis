%!TEX root = ../thesis.tex
\chapter{Convergence without ephemeral phases\label{app:extend conv}}
For completeness, we include here results needed to prove a convergence of the QBD-RAP scheme to the fluid queue without the need for the ephemeral initial states. Note that we only need to prove convergence up to, and at, the time of the first change of level of the QBD-RAP, then we can use the results from Chapter~\ref{ch: global results} to obtain global convergence results. 

For \(\varphi(0)=k\in\mathcal S_{-0}\) (or \(\varphi(0)=k\in\mathcal S_{+0}\)) the added complexity comes from the fact, upon the phase process first leaving \(\mathcal S_{-0}\) (\(\mathcal S_{+0}\)), that it is possible the phase transitions to a state in \(\mathcal S_+\) (\(\mathcal S_-\)). Since the orbit of the QBD-RAP is constant on \(\varphi(t)\in\mathcal S_{-0}\) (\(\varphi(t)\in\mathcal S_{+0}\)), then upon a first transition out of \(\mathcal S_{-0}\) (\(\mathcal S_{+0}\)) and into \(\mathcal S_+\) (\(\mathcal S_-\)) the orbit jumps to \(\bs   a_{\ell_0,i}^{(p)}(x_0)\bs D^{(p)}\). For \(k\in\mathcal S_{-0}\), \(m\geq 0\), the corresponding Laplace transform of the QBD-RAP is
\begin{align}
	&\widehat{f}_{m,-0,+}^{\ell_0,(p)}(\lambda)(x,j;x_0,k)\wrt x\nonumber 
	\\&:=\int_{x_1=0}^\infty \dots \int_{x_{2m+1}=0}^\infty  \bs e_k [\bs I - \bs T_{00}]^{-1}\bs T_{0+}\bs M^m_{++}(\lambda,x_1,\dots,x_{2m+1})\bs e_j\tr{} \nonumber
	\\& \bs a_{\ell_0,k}^{(p)}(x_0) \bs D^{(p)} \bs N^{2m+1,(p)}(\lambda,x_1,\dots,x_{2m+1}) {\bs v}_{\ell_0,j}^{(p)}(x)\wrt x_{2m+1}\dots\wrt x_2 \wrt x_1 \wrt x \nonumber
	\\& + \int_{x_1=0}^\infty \dots \int_{x_{2m+2}=0}^\infty  \bs e_k [\bs I - \bs T_{00}]^{-1}\bs T_{0-}\bs M^{m+1}_{-+}(\lambda,x_1,\dots,x_{2m+2})\bs e_j\tr{} \nonumber
	\\& \bs a_{\ell_0,k}^{(p)}(x_0)\bs N^{2m+2,(p)}(\lambda,x_1,\dots,x_{2m+2}) {\bs v}_{\ell_0,j}^{(p)}(x)\wrt x_{2m+2}\dots\wrt x_2 \wrt x_1 \wrt x.
	\label{eqn: k0jJ0}
\end{align}
% The second term in expression~(\ref{eqn: k0jJ0}) is a linear combination of \(\widehat{f}_{m+1,-,+}^{\ell_0,(p)}(\lambda)(x,j;x_0,k)\). The first term is a linear combination of terms similar to \(\widehat{f}_{m,+,+}^{\ell_0,(p)}(\lambda)(x,j;x_0,k)\), except there is a different initial vector, \(\bs a_{\ell_0,i}^{(p)}(x_0)\bs D^{(p)}\).  
The Laplace transform of the fluid queue corresponding to (\ref{eqn: k0jJ0}) is 
\begin{align}  
	\widehat \mu_{m,-0,+}^{\ell_0}(\lambda)(\wrt x,j;x_0,k) \nonumber 
	&:= \sum_{i\in\calS_+}\bs e_k\vligne{\lambda \bs I - \bs T_{00}}^{-1}\bs T_{0i}\widehat \mu_{m,+,+}^{\ell_0}(\lambda)(\wrt x,j;x_0,i) 
	\\&{}+ \sum_{i\in\calS_-}\bs e_k\vligne{\lambda \bs I - \bs T_{00}}^{-1}\bs T_{0i}\widehat \mu_{m+1,-,+}^{\ell_0}(\lambda)(\wrt x,j;x_0,i), \label{eqn: s0- actual}
\end{align}
\(m\geq 0\).

The second term of (\ref{eqn: k0jJ0}) is a linear combination of \(\widehat{f}_{m+1,-,+}^{\ell_0,(p)}(\lambda)(x,j;x_0,k)\wrt x\) which converges to \(\widehat \mu_{m,-,+}^{\ell_0}(\lambda)(\wrt x,j;x_0,i)\), so there are no issues here. The first term of (\ref{eqn: k0jJ0}) creates significantly more work. Essentially, we need to derive more bounds, analogous to the results from Chapter~\ref{sec: conv}, but with the initial vector \(\bs a_{\ell_0,i}(x_0)\bs D\). 

Analogously, for \(k\in\calS_{-0}\), \(m\geq 0\), we also have 
\begin{align*}
	\widehat{f}_{m,-0,-}^{\ell_0,(p)}(\lambda)(x,j;x_0,k)\nonumber 
	&:=\int_{x_1=0}^\infty \dots \int_{x_{2m+2}=0}^\infty  \bs e_k [\bs I - \bs T_{00}]^{-1}\bs T_{0+} \bs M^{m+1}_{+-}(\lambda,x_1,\dots,x_{2m+2})\bs e_j\tr{} \nonumber
	\\& \bs a_{\ell_0,k}^{(p)}(x_0) \bs D^{(p)} \bs N^{2m+2,(p)}(\lambda,x_1,\dots,x_{2m+2}) {\bs v}_{\ell_0,j}^{(p)}(x)\wrt x_{2m+2}\dots  \wrt x_1  \nonumber
	\\& + \int_{x_1=0}^\infty \dots \int_{x_{2m+1}=0}^\infty  \bs e_k [\bs I - \bs T_{00}]^{-1}\bs T_{0-} \bs M^{m}_{--}(\lambda,x_1,\dots,x_{2m+1})\bs e_j\tr{} \nonumber
	\\& \bs a_{\ell_0,k}^{(p)}(x_0)\bs N^{2m+1,(p)}(\lambda,x_1,\dots,x_{2m+1}) {\bs v}_{\ell_0,j}^{(p)}(x)\wrt x_{2m+1}\dots\wrt x_1.
\end{align*}
For \(k\in\calS_{+0}\), \(m\geq 0\), we have 
\begin{align*}
	\widehat{f}_{m,+0,+}^{\ell_0,(p)}(\lambda)(x,j;x_0,k)\nonumber 
	&:=\int_{x_1=0}^\infty \dots \int_{x_{2m+2}=0}^\infty \bs e_k [\bs I - \bs T_{00}]^{-1}\bs T_{0-} \bs M^{m+1}_{-+}(\lambda,x_1,\dots,x_{2m+2})\bs e_j\tr{} \nonumber
	\\& \bs a_{\ell_0,k}^{(p)}(x_0) \bs D^{(p)} \bs N^{2m+2,(p)}(\lambda,x_1,\dots,x_{2m+2}) {\bs v}_{\ell_0,j}^{(p)}(x)\wrt x_{2m+2}\dots  \wrt x_1  \nonumber
	\\& + \int_{x_1=0}^\infty \dots \int_{x_{2m+1}=0}^\infty  \bs e_k [\bs I - \bs T_{00}]^{-1}\bs T_{0+} \bs M^m_{++}(\lambda,x_1,\dots,x_{2m+1})\bs e_j\tr{} \nonumber
	\\& \bs a_{\ell_0,k}^{(p)}(x_0)\bs N^{2m+1,(p)}(\lambda,x_1,\dots,x_{2m+1}) {\bs v}_{\ell_0,j}^{(p)}(x)\wrt x_{2m+1}\dots  \wrt x_1,
\end{align*}
and 
\begin{align*}
	\widehat{f}_{m,+0,-}^{\ell_0,(p)}(\lambda)(x,j;x_0,k)\nonumber 
	&:=\int_{x_1=0}^\infty \dots \int_{x_{2m+1}=0}^\infty  \bs e_k [\bs I - \bs T_{00}]^{-1}\bs T_{0-} \bs M^m_{--}(\lambda,x_1,\dots,x_{2m+1})\bs e_j\tr{} \nonumber
	\\& \bs a_{\ell_0,k}^{(p)}(x_0) \bs D^{(p)} \bs N^{2m+1,(p)}(\lambda,x_1,\dots,x_{2m+1}) {\bs v}_{\ell_0,j}^{(p)}(x)\wrt x_{2m+1}\dots  \wrt x_1  \nonumber
	\\& + \int_{x_1=0}^\infty \dots \int_{x_{2m+2}=0}^\infty  \bs e_k [\bs I - \bs T_{00}]^{-1}\bs T_{0+} \bs M^{m+1}_{+-}(\lambda,x_1,\dots,x_{2m+2})\bs e_j\tr{} \nonumber
	\\& \bs a_{\ell_0,k}^{(p)}(x_0)\bs N^{2m+2,(p)}(\lambda,x_1,\dots,x_{2m+2}) {\bs v}_{\ell_0,j}^{(p)}(x)\wrt x_{2m+2}\dots  \wrt x_1.
\end{align*}
In general, for \(q\in \{+,-\}, \, q'\in\{+,-\}\), \(m\geq 0\),
\begin{align*}
	\widehat \mu_{m,q0,q'}^{\ell_0}(\lambda)(\wrt x,j;x_0,k) \nonumber 
	&:= \sum_{r\in\{+,-\}}\sum_{i\in\calS_r}\bs e_k\vligne{\lambda \bs I - \bs T_{00}}^{-1}\bs T_{0i}\widehat \mu_{m+1(r\neq q'),r,q'}^{\ell_0}(\lambda)(\wrt x,j;x_0,i).
\end{align*}

\begin{rem}
For technical reasons we should not have point masses at \(x_0\in{y_{\ell_0},y_{\ell_0+1}}\) when \(\varphi(0)\in\calS_{+0}\cup\calS_{-0}\). Intuitively, if \(\varphi(0)=k\in\calS_{+0}\) and \(x_0 = y_{\ell_0}\) then, upon exiting \(\calS_{+0}\), if the phase process transitions to \(\calS_-\) then the fluid queue will instantaneously leave the interval \(\calD_{\ell_0}\) upon this transition. On the same event, the orbit of the QBD-RAP will be \(\bs \alpha^{(p)} \bs D^{(p)}\) at the instant of the transition to \(\calS_-\). Roughly speaking \(\bs D^{(p)}\) maps \(\bs \alpha^{(p)}\) to approximately \(\cfrac{\bs \alpha^{(p)} e^{\bs S^{(p)}\Delta}}{\bs \alpha^{(p)} e^{\bs S^{(p)}\Delta}\bs e}\) (asymptotically). Our asymptotic arguments rely on Chebyshev's inequality, in the form of a bound in terms of the distance of the random variable \(Z^{(p)}\sim ME(\bs\alpha^{(p)},\bs S^{(p)})\) from its mean \(\Delta\). However, we cannot use such a technique to bound terms such as \(\cfrac{\bs \alpha^{(p)} e^{\bs S^{(p)}\Delta}}{\bs \alpha^{(p)} e^{\bs S^{(p)}\Delta}\bs e}e^{\bs S^{(p)}z}\bs s^{(p)}\). %Furthermore, for the QBD-RAP to capture the fact that the fluid queue instantaneously leaves \(\calD_{\ell_0}\) at the instant it the phase process transitions to \(\calS_-\), we need the intensity of a change of level of the QBD-RAP to become arbitrarily large in our limiting arguments. That is, we need \(\bs \alpha \bs D\bs s\) to become arbitrarily large. It is unknown, in general, whether that this is the case. %In these cases the initial condition \(\bs   a_{\ell_0,i}(x_0)\) has a non-negligible contribution from \(\bs \alpha e^{\bs S\Delta}/\bs \alpha e^{\bs S\Delta}\bs e\) and, to our knowledge, there is no way to ensure the denominator is bounded away from \(0\) when we construct limiting arguments. This issue is present when we try to do analysis on 
%\[\cfrac{\bs \alpha e^{\bs S\Delta}}{\bs \alpha e^{\bs S\Delta}\bs e}\bs D = \cfrac{\bs \alpha e^{\bs S\Delta}}{\bs \alpha e^{\bs S\Delta}\bs e} \int_{u=0}^\infty e^{\bs Su}\bs s \cfrac{\bs \alpha e^{\bs Su}}{\bs \alpha e^{\bs Su}\bs e}\wrt u.\]

In practice, it may be possible to avoid this issue by choosing the intervals \(\{\mathcal D_\ell\}\) so that the boundaries do not align with any point masses. Another option is to append an ephemeral class of phases to the fluid queue as previously stated. 

% Suppose we wish to have a point mass at \(x_0=y_{\ell}\), \(i\in\calS_{+0}\), and without loss of generality, assume the point mass has total mass 1. Let \(\mathcal S_{+0}^*\) with rates \(c_j=0\), \(j\in\mathcal S_{+0}^*\). These states represent a duplicated copy of \(\mathcal S_{+0}\). The initial orbit representing the point mass is \(\bs \alpha^{(p)}\), \(i\in\mathcal S_{+0}^*\). The dynamics of the QBD-RAP process regarding the ephemeral class \(\calS_{0+}^*\) is as follows. The phase process evolves according to the sub-generator matrix \(\bs T_{00}\) whilst it remains in \(\calS_{0+}^*\). Meanwhile, the orbit and level remain constant. When in phase \(k\in\calS_{0+}^*\), at rate \([\bs T_{0+}]_{kj}\) the process transitions out of \(\calS_{0+}^*\) and into phase \(j\in\mathcal S_+\). Similarly, when in phase \(k\in\calS_{0+}^*\), at rate \([\bs T_{0-}]_{kj}\) the process transitions out of \(\calS_{0+}^*\) and into phase \(j\in\mathcal S_-\). Upon a transition from \(\mathcal S_{0+}^*\) to \(j\in\mathcal S_{+}\) the orbit and level remain unchanged, and the process evolves as usual. Upon a transition from \(\mathcal S_{0+}^*\) to \(j\in\mathcal S_{-}\) the orbit remains unchanged, but the level jumps to \(\ell - 1\). Thus, the process evolves as if it has just entered level \(\ell-1\) in phase \(j\in\mathcal S_-\). 
\end{rem}

Theorem~\ref{thm: a thm2!} below is analogous to Theorem~\ref{thm: a thm!} and proves a certain convergence of the QBD-RAP scheme to the fluid queue in the case that \(\phi(0)\in\mathcal S_{+0}\cup\calS_{-0}\). Like the proof of Theorem~\ref{thm: a thm!}, the proof of Theorem~\ref{thm: a thm2!} relies on technical bounds which we establish with the remainder of this Appendix. 

\begin{thm}\label{thm: a thm2!}
	As \(p\to \infty\), for \((q,r)\in\{(+0,-),(-0,+)\},\, r\in\{+,-\}\) and \(m=0\),  
	\begin{align}\int_{x\in\calD_{\ell_0}}\widehat f_{0,q,r}^{\ell_0,(p)}(\lambda)(x,j;x_0,k)\psi(x)\wrt x \to \int_{x\in\calD_{\ell_0}}\widehat \mu_{0,q,r}^{\ell_0}(\lambda)(x,j;x_0,k)\psi(x)\wrt x.\end{align}
	For \(q\in\{+0,-0\}, r\in\{+,-\}\) and \(m\geq 1\), 
	\begin{align}\int_{x\in\calD_{\ell_0}}\widehat f_{m,q,r}^{\ell_0,(p)}(\lambda)(x,j;x_0,k)\psi(x)\wrt x \to \int_{x\in\calD_{\ell_0}}\widehat \mu_{m,q,r}^{\ell_0}(\lambda)(x,j;x_0,k)\psi(x)\wrt x.\label{eqn: thm 22}\end{align}
\end{thm}
\begin{proof}
	\textit{Cases \((q,r) \in \{(+0,-),(-0,+)\}\), and \(m=0\).} First, take \(q=-0\) and \(r=+\), then 
	\begin{align}
		&\int_{x\in\calD_k}\widehat{f}_{0,-0,+}^{\ell_0,(p)}(\lambda)(x,j;x_0,k)\psi(x)\wrt x \nonumber 
		\\&:=\int_{x_1=0}^\infty \int_{x\in\calD_k} \bs e_k [\bs I - \bs T_{00}]^{-1}\bs T_{0+} \bs M^0_{++}(\lambda,x_1)\bs e_j\tr{} \nonumber
		\bs a_{\ell_0,k}^{(p)}(x_0) \bs D^{(p)} \bs N^{1,(p)}(\lambda,x_1) 
		\\& {\bs v}_{\ell_0,j}^{(p)}(x)\psi(x)\wrt x\wrt x_{1}  \nonumber
		\\& + \int_{x_1=0}^\infty\int_{x_2=0}^\infty \int_{x\in\calD_k}  \bs e_k [\bs I - \bs T_{00}]^{-1}\bs T_{0-} \bs M^1_{-+}(\lambda,x_1,x_{2})\bs e_j\tr{} \nonumber
		\bs a_{\ell_0,k}^{(p)}(x_0)\bs N^{2,(p)}(\lambda,x_1,x_{2}) 
		\\& {\bs v}_{\ell_0,j}^{(p)}(x)\psi(x)\wrt x\wrt x_{2} \wrt x_1. \label{eqn: thm 1223}
	\end{align}
	The second term is a linear combination of \(\int_{x\in\calD_k}\widehat{f}_{1,-,+}^{\ell_0,(p)}(\lambda)(x,j;x_0,i) \psi(x)\wrt x\) terms, each of which converge to \(\int_{x\in\calD_k}\widehat{\mu}_{1,-,+}^{\ell_0,(p)}(\lambda)(\wrt x,j;x_0,i) \psi(x)\), by Theorem~\ref{thm: a thm!}. As for the first term, it is a linear combination of terms
	\[\int_{x_1=0}^\infty \int_{x\in\calD_k} \bs e_i \bs H^{++}(\lambda,x_1)\bs e_j\tr{} \nonumber
	\bs a_{\ell_0,k}^{(p)}(x_0) \bs D^{(p)} e^{\bs S^{(p)}x_1} 
	{\bs v}_{\ell_0,j}^{(p)}(x)\psi(x)\wrt x\wrt x_{1}.\]
	Lemma~\ref{lem: ppp}, below, proves that such terms converge to \(\int_{x\in\calD_k}\widehat{\mu}_{0,+,+}^{\ell_0}(\lambda)(\wrt x,j;x_0,i) \psi(x)\). Therefore, (\ref{eqn: thm 1223}) is a finite linear combination of terms, each of which converge, hence (\ref{eqn: thm 1223}) converges and converges to 
	\begin{align}
		&\int_{x_1=0}^\infty \int_{x\in\calD_k} \bs e_k \sum_{i\in\calS_+} [\bs I - \bs T_{00}]^{-1}\bs T_{0i}\int_{x\in\calD_k}\widehat{\mu}_{0,+,+}^{\ell_0}(\lambda)(\wrt x,j;x_0,i) \psi(x) \nonumber
		\\& + \int_{x_1=0}^\infty\int_{x_2=0}^\infty \int_{x\in\calD_k} \sum_{i\in\calS_-}\bs e_k [\bs I - \bs T_{00}]^{-1}\bs T_{0i}\int_{x\in\calD_k}\widehat{\mu}_{1,-,+}^{\ell_0}(\lambda)(\wrt x,j;x_0,i) \psi(x), \label{eqn: thm 1223B}
	\end{align}
	which is \(\widehat{\mu}_{0,-0,+}^{\ell_0}(\lambda)(\wrt x,j;x_0,k)\). This proves the result for \(r=+\) and \(q=-0\). Analogous arguments prove the result for \(r=-\) and \(q=+0\).

	\textit{Cases \(q\in\{+0,-0\},\,r \in \{+,-\}\) and \(m\geq 1\).} First, take \(q=+0\) and \(r=+\), then 
	\begin{align}
		&\int_{x\in\calD_k}\widehat{f}_{m,-0,+}^{\ell_0,(p)}(\lambda)(x,j;x_0,k)\psi(x)\wrt x \nonumber 
		\\&:=\int_{x\in\calD_k}\int_{x_1=0}^\infty \dots \int_{x_{2m+1}=0}^\infty   \bs e_k [\bs I - \bs T_{00}]^{-1}\bs T_{0+}\bs M^m_{++}(\lambda,x_1,\dots,x_{2m+1})\bs e_j\tr{} \nonumber
		\\& \bs a_{\ell_0,k}^{(p)}(x_0) \bs D^{(p)} \bs N^{2m+1,(p)}(\lambda,x_1,\dots,x_{2m+1}) {\bs v}_{\ell_0,j}^{(p)}(x)\psi(x)\wrt x_{2m+1}\dots\wrt x_2 \wrt x_1  \wrt x\nonumber
		\\& + \int_{x\in\calD_k}\int_{x_1=0}^\infty \dots \int_{x_{2m+2}=0}^\infty   \bs e_k [\bs I - \bs T_{00}]^{-1}\bs T_{0-}\bs M^{m+1}_{-+}(\lambda,x_1,\dots,x_{2m+2})\bs e_j\tr{} \nonumber
		\\& \bs a_{\ell_0,k}^{(p)}(x_0)\bs N^{2m+2,(p)}(\lambda,x_1,\dots,x_{2m+2}) {\bs v}_{\ell_0,j}^{(p)}(x)\psi(x)\wrt x_{2m+2}\dots\wrt x_2 \wrt x_1 \wrt x.
	\end{align}
	The second term is a linear combination of \(\int_{x\in\calD_k}\widehat{f}_{m+1,-,+}^{\ell_0,(p)}(\lambda)(x,j;x_0,i) \psi(x)\wrt x\) terms, each of which converge to \(\int_{x\in\calD_k}\widehat{\mu}_{m+1,-,+}^{\ell_0,(p)}(\lambda)(\wrt x,j;x_0,i) \psi(x)\).
	As for the first term, it is a linear combination of terms 
	\begin{align*}
		&\int_{x\in\calD_k}\int_{x_1=0}^\infty \dots \int_{x_{2m+1}=0}^\infty \bs e_i \bs M^m_{++}(\lambda,x_1,\dots,x_{2m+1})\bs e_j\tr{} \nonumber
		\\& \bs a_{\ell_0,k}^{(p)}(x_0)\bs D^{(p)}\bs N^{2m+1,(p)}(\lambda,x_1,\dots,x_{2m+1}) {\bs v}_{\ell_0,j}^{(p)}(x)\psi(x)\wrt x_{2m+1}\dots\wrt x_2 \wrt x_1 \wrt x
		%
		\\&=\int_{x\in\calD_k}\int_{x_1=0}^\infty \dots \int_{x_{2m+1}=0}^\infty \bs e_i \bs H^{+-}(\lambda,x_1)\prod_{r=1}^{m-1}\bs H^{-+}(\lambda,x_{2r}) \bs H^{+-}(\lambda,x_{2r+1}) \\&\bs H^{-+}(\lambda,x_{2m}) 
		\bs H^{++}(\lambda,x_{2m+1})\bs e_j\tr{} \nonumber
		\bs a_{\ell_0,k}^{(p)}(x_0)\prod_{r=1}^{2m} e^{\bs{S}^{(p)}x_{r}}\bs{D}^{(p)} e^{\bs{S}^{(p)}x_{2m+1}}
		\\& {\bs v}_{\ell_0,j}^{(p)}(x)\psi(x)\wrt x_{2m+1}\dots\wrt x_2 \wrt x_1 \wrt x,
	\end{align*}
	which satisfies the assumptions of Lemma~\ref{lem: boobies2}. To see this take \(n=2m+1\), \(\bs G_1(x_1) = \bs e_i\bs H^{+-}(\lambda, x_1)\), \(\bs G_{2k}(x_{2k}) = \bs H^{-+}(\lambda, x_{2k})\), \(\bs G_{2k+1}(x_{2k}) = \bs H^{+-}(\lambda, x_{2k+1})\), \(k=1,\dots,m-1\), \(\bs G_{2m}(x_{2m}) = \bs H^{-+}(x_{2m})\) and \(\bs G_{2m+1} = \bs H^{++}(\lambda,x_{2m+1})\bs e_j\tr{}\) in Equation~(\ref{eqn: KAFnnmna22G}). By the remarks following Lemma~\ref{lem: boobies2}, this gives the required convergence for this case. Analogous arguments prove the result for the remaining combinations of \((q,r)\). 
\end{proof}

\section{Technical results}
As we did in Chapter~\ref{sec: conv}, we treat the cases of \(m=0\), and \(m\geq 1\), up-down/down-up transitions separately. We start with the \(m=0\) case. The following result is analogous to Lemma~\ref{lem: Dcoajc}, though the proof is somewhat more tedious. 

\begin{lem}\label{lem: ppp}
	Let \(\psi:\calD_{\ell_0}\to \mathbb R\) be bounded \(|\psi(x)|\leq F\) and let \(\lambda >0\). For \(i\in\mathcal S_-,j\in\mathcal S_-\cup\calS_{-0}\), \(x_0\in(0,\Delta)\), 
	\begin{align}
		&\int_{x_1=0}^\infty \int_{x=0}^{\Delta}\bs k^{(p)}(x_0) \bs D^{(p)} e^{\bs S x_1} {\bs v}_{\ell_0,j}^{(p)}(x) h_{ij}^{--}(\lambda, x_1) \psi(x) \wrt x \wrt x_1  
		\to \int_{x=0}^{x_0} h_{ij}^{--}(\lambda,x_0-x)\psi(x)\wrt x, \label{eqn: LLLlllL}
	\end{align}
	as \(p\to\infty\). Similarly, for \(i\in\mathcal S_+,j\in\mathcal S_+\cup\calS_{+0}\)
	\begin{align}
		&\int_{x_1=0}^\infty \int_{x=0}^{\Delta}\bs k^{(p)}(x_0) \bs D^{(p)} e^{\bs S^{(p)} x_1} {\bs v}_{\ell_0,j}^{(p)}(x) h_{ij}^{++}(\lambda, x_1) \psi(\Delta-x) \wrt x \wrt x_1  \nonumber 
		\\&\to\int_{x=\Delta-x_0}^{\Delta} h_{ij}^{++}(\lambda,x-x_0)\psi(x)\wrt x, \label{eqn: LLLlllL2222}
	\end{align}
\end{lem}
\begin{proof}
	Assume, for simplicity and without loss of generality, that \(\ell_0=0\) so \(\calD_{\ell_0}=[0,\Delta]\). In the following we choose \(\varepsilon^{(p)}=\var(Z^{(p)})^{1/3}\) which tends to 0 as \(p\to\infty\). Therefore, assume \(p\) is sufficiently large so that \(x_0\in(2\varepsilon,\Delta-\varepsilon)\). Such a \(p<\infty\) always exists since \(x_0\in(0,\Delta)\), by assumption. 
	
	Now, upon substituting the definition of \(\bs D\) and exchanging the order of integration (justified by the Fubini-Tonelli Theorem), the difference between the left and right-hand sides of (\ref{eqn: LLLlllL}) is 
	\begin{align}
		&\Bigg|\int_{u=0}^\infty \bs k(x_0)e^{\bs Su}\bs s \int_{x=0}^{\Delta}  \int_{x_1=0}^\infty \bs k(u)  e^{\bs S x_1} {\bs v}_{\ell_0,j}(x) h_{ij}^{--}(\lambda, x_1) \psi(x) \wrt x_1 \wrt x \wrt u\nonumber
		\\&{}\qquad {} - \int_{x=0}^{x_0} h_{ij}^{--}(\lambda, x_0 - x)\psi(x)\wrt x\Bigg|.\label{eqn: aksfKJJJJJ}
	\end{align}
	We wish to apply Property~\ref{properties: 2} to the integral over \(x_1\), however, to do so requires that \(u\leq \Delta-\varepsilon\). Breaking up the integral with respect to \(u\), then (\ref{eqn: aksfKJJJJJ}) is equal to 
	\begin{align}
		&\Bigg|\int_{u=0}^{\Delta-\varepsilon} \bs k(x_0)e^{\bs Su}\bs s \int_{x=0}^{\Delta}  \int_{x_1=0}^\infty \bs k(u)  e^{\bs S x_1} {\bs v}_{\ell_0,j}(x) h_{ij}^{--}(\lambda, x_1) \psi(x) \wrt x_1 \wrt x \wrt u + d_1 \nonumber
		\\&{}  - \int_{x=0}^{x_0} h_{ij}^{--}(\lambda, x_0 - x)\psi(x)\wrt x\Bigg|
		\\&\leq \Bigg|\int_{u=0}^{\Delta-\varepsilon} \bs k(x_0)e^{\bs Su}\bs s \int_{x=0}^{\Delta}  \int_{x_1=0}^\infty \bs k(u)  e^{\bs S x_1} {\bs v}_{\ell_0,j}(x) h_{ij}^{--}(\lambda, x_1) \psi(x) \wrt x_1 \wrt x \wrt u\nonumber
		\\&{} - \int_{x=0}^{x_0} h_{ij}^{--}(\lambda, x_0 - x)\psi(x)\wrt x\Bigg| +|d_1| \label{eqnL MMMM}
	\end{align}
	where 
	\[|d_1|=\left|\int_{u=\Delta-\varepsilon}^\infty \bs k(x_0)e^{\bs Su}\bs s \int_{x=0}^{\Delta}  \int_{x_1=0}^\infty \bs k(u)  e^{\bs S x_1} {\bs v}_{\ell_0,j}(x) h_{ij}^{--}(\lambda, x_1) \psi(x) \wrt x_1 \wrt x \wrt u\right|.\]
	We show later that \(|d_1|\) can be made arbitrarily small by choosing \(Z\) with sufficiently small variance.  For now, let us focus on the first absolute value in (\ref{eqnL MMMM}). By Property~\ref{properties: 2} and swapping the order of integration (justified by the Fubini-Tonelli Theorem) the first absolute value in (\ref{eqnL MMMM}) is 
	\begin{align}
		&\Bigg|\int_{x=0}^{\Delta}\int_{u=0}^{\Delta-\varepsilon} \bs k(x_0)e^{\bs Su}\bs s \left[ h_{ij}^{--}(\lambda, \Delta - u - x)1(u+x\leq \Delta-\varepsilon) + r_{\bs v}(u,x) \right] \psi(x) \wrt u \wrt x \nonumber
		\\&{}\qquad {} - \int_{x=0}^{x_0} h_{ij}^{--}(\lambda, x_0 - x)\psi(x)\wrt x \Bigg|\nonumber
		\\&\leq \Bigg|\int_{x=0}^{\Delta}\int_{u=0}^{\Delta-\varepsilon} \bs k(x_0)e^{\bs Su}\bs s h_{ij}^{--}(\lambda, \Delta - u - x)1(u+x\leq \Delta-\varepsilon) \psi(x) \wrt u \wrt x \nonumber 
		\\&\qquad{} - \int_{x=0}^{x_0} h_{ij}^{--}(\lambda, x_0 - x)\psi(x)\wrt x \Bigg| + |d_2|,
		\label{eqn: aKK}
	\end{align}
	where 
	\[|d_2| = \left| \int_{x=0}^{\Delta}\int_{u=0}^{\Delta-\varepsilon} \bs k(x_0)e^{\bs Su}\bs s r_{\bs v}(u,x) \psi(x) \wrt u \wrt x \right|.\]
	We show later that \(|d_2|\) can be made arbitrarily small by choosing \(Z\) with sufficiently small variance. The remaining term in (\ref{eqn: aKK}) can be written as 
	\begin{align}
		&\Bigg|\int_{x=0}^{\Delta-\varepsilon}\int_{u=0}^{\Delta-x-\varepsilon} \bs k(x_0)e^{\bs Su}\bs s h_{ij}^{--}(\lambda, \Delta - u - x) \psi(x) \wrt u \wrt x 
		- \int_{x=0}^{x_0} h_{ij}^{--}(\lambda, x_0 - x)\psi(x)\wrt x \Bigg|.\label{eqn: FGHJKL}
	\end{align}
	

	Intuitively, when the variance of \(Z\) is low, we expect a significant contribution to the integral with respect to \(u\) in (\ref{eqn: FGHJKL}) to come from the portion of the integral over the interval \((\Delta-x_0-\varepsilon,\Delta-x_0+\varepsilon)\). Although, the integral with respect to \(u\) only contains this interval when \(x\) is sufficiently small. Breaking up the integral with respect to \(u\) in (\ref{eqn: FGHJKL}) into an integral over the interval \((\Delta-x_0-\varepsilon,\Delta-x_0+\varepsilon)\) and integrals over the rest, then applying the triangle inequality, (\ref{eqn: FGHJKL}) is less than or equal to 
	\begin{align}
		&\Bigg| \int_{x=0}^{\Delta-\varepsilon}\int_{u=\Delta-x_0-\varepsilon}^{\Delta-x_0+\varepsilon} \bs k(x_0)e^{\bs Su}\bs s h_{ij}^{--}(\lambda, \Delta - u - x) \psi(x) \wrt u 1(x\leq x_0-2\varepsilon) \wrt x\nonumber
		\\&\qquad{} - \int_{x=0}^{x_0} h_{ij}^{--}(\lambda, x_0 - x)\psi(x)\wrt x \Bigg| + |d_3|+|d_4|+|d_5|,\label{eqn: LLkkK} 
	\intertext{where}
		&|d_3|=\Bigg|\int_{x=0}^{\Delta-\varepsilon}\int_{u=0}^{\Delta-x-\varepsilon} \bs k(x_0)e^{\bs Su}\bs s h_{ij}^{--}(\lambda, \Delta - u - x) \psi(x) \wrt u 1(x\geq x_0)\wrt x\Bigg|,\nonumber
		\\&|d_4|=\Bigg|\int_{x=0}^{\Delta-\varepsilon}\int_{u=\Delta-x_0+\varepsilon}^{\Delta-x-\varepsilon} \bs k(x_0)e^{\bs Su}\bs s h_{ij}^{--}(\lambda, \Delta - u - x) \psi(x) \wrt u \nonumber 
		 1(x\leq x_0-2\varepsilon)\wrt x\Bigg|,\nonumber
		\\&|d_5|=\Bigg|\int_{x=0}^{\Delta-\varepsilon}\int_{u=\Delta-x_0-\varepsilon}^{\Delta-x-\varepsilon} \bs k(x_0)e^{\bs Su}\bs s h_{ij}^{--}(\lambda, \Delta - u - x) \psi(x) \wrt u \times 1(x\in[x_0-2\varepsilon,x_0))\wrt x\Bigg|.  \nonumber 
	\end{align}
	We show later that \(|d_3|,\, |d_4|\) and \(|d_5|\) can be made arbitrarily small by choosing \(Z\) with sufficiently small variance.

	In the first integral with respect to \(x\) in (\ref{eqn: LLkkK}), since \(x_0\in(2\varepsilon,\Delta-\varepsilon)\), then we can absorb the indicator function into the limits of the integral which results in
	\begin{align}
		\int_{x=0}^{x_0-2\varepsilon}\int_{u=\Delta-x_0-\varepsilon}^{\Delta-x_0+\varepsilon} \bs k(x_0)e^{\bs Su}\bs s h_{ij}^{--}(\lambda, \Delta - u - x) \psi(x) \wrt u \wrt x.
	\end{align}
	With this, and breaking up the integral over \(h_{ij}^{--}\), we can write the first absolute value in (\ref{eqn: LLkkK}) as 
	\begin{align}
		\nonumber &\Bigg| \int_{x=0}^{x_0-2\varepsilon}\int_{u=\Delta-x_0-\varepsilon}^{\Delta-x_0+\varepsilon} \bs k(x_0)e^{\bs Su}\bs s h_{ij}^{--}(\lambda, \Delta - u - x) \psi(x) \wrt u  \wrt x 
		\\\nonumber &\qquad{} - \int_{x=0}^{x_0-2\varepsilon} h_{ij}^{--}(\lambda, x_0 - x)\psi(x)\wrt x  - \int_{x=x_0-2\varepsilon}^{x_0} h_{ij}^{--}(\lambda, x_0 - x)\psi(x)\wrt x \Bigg| 
		\\\nonumber &\leq \Bigg| \int_{x=0}^{x_0-2\varepsilon}\int_{u=\Delta-x_0-\varepsilon}^{\Delta-x_0+\varepsilon} \bs k(x_0)e^{\bs Su}\bs s h_{ij}^{--}(\lambda, \Delta - u - x) \psi(x) \wrt u   \wrt x 
		\\&\qquad{} - \int_{x=0}^{x_0-2\varepsilon} h_{ij}^{--}(\lambda, x_0 - x)\psi(x)\wrt x \Bigg| + |d_6|,\label{eqn:PPPPPP}
	\end{align}
		where
	\begin{align}
		&|d_6|=\left|\int_{x=x_0-2\varepsilon}^{x_0} h_{ij}^{--}(\lambda, x_0 - x)\psi(x)\wrt x \right|.\nonumber 
	\end{align}
	We show later that \(|d_6|\) can be made arbitrarily small by choosing \(Z\) with sufficiently small variance.

	Now, since probability densities integrate to 1, then we can write 
	\begin{align*}
		\int_{x=0}^{x_0-2\varepsilon} h_{ij}^{--}(\lambda, x_0 - x)\psi(x)\wrt x
		&=\int_{x=0}^{x_0-2\varepsilon} h_{ij}^{--}(\lambda, x_0 - x)\psi(x)\wrt x\mathbb P(|Z-\Delta|\leq \varepsilon\mid Z>x_0) 
		\\&{}+ \int_{x=0}^{x_0-2\varepsilon} h_{ij}^{--}(\lambda, x_0 - x)\psi(x)\wrt x\mathbb P(|Z-\Delta|> \varepsilon\mid Z>x_0)
		\\&=\int_{x=0}^{x_0-2\varepsilon} \int_{u=\Delta-x_0-\varepsilon}^{\Delta-x_0+\varepsilon} \bs k(x_0)e^{\bs Su}\bs s h_{ij}^{--}(\lambda, x_0 - x)\psi(x)\wrt u\wrt x
		\\&{}+ \int_{x=0}^{x_0-2\varepsilon} h_{ij}^{--}(\lambda, x_0 - x)\psi(x)\wrt x\mathbb P(|Z-\Delta|> \varepsilon\mid Z>x_0).
	\end{align*}
	Therefore, the first absolute value in (\ref{eqn:PPPPPP}) can be written as 
	\begin{align}
		\nonumber &\Bigg| \int_{x=0}^{x_0-2\varepsilon}\int_{u=\Delta-x_0-\varepsilon}^{\Delta-x_0+\varepsilon} \bs k(x_0)e^{\bs Su}\bs s h_{ij}^{--}(\lambda, \Delta - u - x) \psi(x) \wrt u   \wrt x 
		\\\nonumber &\qquad{} - \int_{x=0}^{x_0-2\varepsilon} \int_{u=\Delta-x_0-\varepsilon}^{\Delta-x_0+\varepsilon} \bs k(x_0)e^{\bs Su}\bs s h_{ij}^{--}(\lambda, x_0 - x)\psi(x)\wrt u\wrt x
		\\\nonumber &\qquad{}- \int_{x=0}^{x_0-2\varepsilon} h_{ij}^{--}(\lambda, x_0 - x)\psi(x)\wrt x\mathbb P(|Z-\Delta|> \varepsilon\mid Z>x_0)\Bigg|
		\\\nonumber &=\Bigg| \int_{x=0}^{x_0-2\varepsilon}\int_{u=\Delta-x_0-\varepsilon}^{\Delta-x_0+\varepsilon} \bs k(x_0)e^{\bs Su}\bs s \left[h_{ij}^{--}(\lambda, \Delta - u - x) -  h_{ij}^{--}(\lambda, x_0 - x)\psi(x)\right]\wrt u\wrt x
		\\\nonumber &\qquad{}- \int_{x=0}^{x_0-2\varepsilon} h_{ij}^{--}(\lambda, x_0 - x)\psi(x)\wrt x\mathbb P(|Z-\Delta|> \varepsilon\mid Z>x_0)\Bigg|
		\\&\leq \Bigg| \int_{x=0}^{x_0-2\varepsilon}\int_{u=\Delta-x_0-\varepsilon}^{\Delta-x_0+\varepsilon} \bs k(x_0)e^{\bs Su}\bs s \left[h_{ij}^{--}(\lambda, \Delta - u - x) -  h_{ij}^{--}(\lambda, x_0 - x)\psi(x)\right]\wrt u\wrt x\Bigg| + |d_7| \nonumber 
		\\&\leq  \int_{x=0}^{x_0-2\varepsilon}\int_{u=\Delta-x_0-\varepsilon}^{\Delta-x_0+\varepsilon} \bs k(x_0)e^{\bs Su}\bs s \left|h_{ij}^{--}(\lambda, \Delta - u - x) -  h_{ij}^{--}(\lambda, x_0 - x)\left|\right|\psi(x)\right|\wrt u\wrt x + |d_7|, \label{eqn: ajkhvakjhdannvnnvnvn}
	\end{align}
	where 
	\[|d_7|=\left| \int_{x=0}^{x_0-2\varepsilon} h_{ij}^{--}(\lambda, x_0 - x)\psi(x)\wrt x\mathbb P(|Z-\Delta|> \varepsilon\mid Z>x_0)\right|.\]
	We show later that \(|d_7|\) can be made arbitrarily small by choosing \(Z\) with sufficiently small variance.

	Since \(h_{ij}^{--}\) is Lipschitz and \(|(\Delta-u-x)-(x_0-x)|\leq 2\varepsilon\) for all \(u\in(\Delta-x_0-\varepsilon,\Delta-x_0+\varepsilon)\), then the first absolute value of (\ref{eqn: ajkhvakjhdannvnnvnvn}) is less than or equal to 
	\begin{align}
		&\int_{u=\Delta-x_0-\varepsilon}^{\Delta-x_0+\varepsilon} \bs k(x_0)e^{\bs Su}\bs s \wrt u 2L\varepsilon \int_{x=0}^{x_0-2\varepsilon} \left|\psi(x)\right|\wrt x. \label{eqn: the last one?}
	\end{align}
	Now, \(\displaystyle \int_{u=\Delta-x_0-\varepsilon}^{\Delta-x_0+\varepsilon} \bs k(x_0)e^{\bs Su}\bs s \wrt u\leq 1\) as it is a probability, and \(\displaystyle \int_{x=0}^{x_0-2\varepsilon} \left|\psi(x)\right|\wrt x \leq F\Delta\) as \(|\psi|\leq F\) and \(x_0\in(2\varepsilon,\Delta-\varepsilon)\). Therefore, (\ref{eqn: the last one?}) is less than or equal to \(2\varepsilon LF\Delta.\)

	What remains is to bound the terms \(|d_\ell|,\, \ell=1,\dots,7\). 
	
	%% d1 %%
	Since \(|\psi|\leq F\) then 
	\begin{align}
		|d_1|&\leq \left|\int_{u=\Delta-\varepsilon}^\infty \bs k(x_0)e^{\bs Su}\bs s \int_{x=0}^{\Delta}  \int_{x_1=0}^\infty \bs k(u)  e^{\bs S x_1} {\bs v}(x) h_{ij}^{--}(\lambda, x_1)  \wrt x_1 \wrt x \wrt u\right|F.\label{eqn:  CVBNM}
	\end{align}
	From Property~\ref{properties: -2}, (\ref{eqn:  CVBNM}) is less than or equal to 
	\begin{align}
		\nonumber &\left|\int_{u=\Delta-\varepsilon}^\infty \bs k(x_0)e^{\bs Su}\bs s  \int_{x_1=0}^\infty \bs k(u)  e^{\bs S x_1} {\bs e} h_{ij}^{--}(\lambda, x_1) \wrt x_1 \wrt u\right|F
		\\\nonumber &\leq\left|\int_{u=\Delta-\varepsilon}^\infty \bs k(x_0)e^{\bs Su}\bs s  \int_{x_1=0}^\infty \bs k(u) {\bs e} h_{ij}^{--}(\lambda, x_1) \wrt x_1 \wrt u\right|F
		\\&=\left|\int_{u=\Delta-\varepsilon}^\infty \bs k(x_0)e^{\bs Su}\bs s  \int_{x_1=0}^\infty h_{ij}^{--}(\lambda, x_1) \wrt x_1 \wrt u\right|F, \label{eqb:afsgh}
	\end{align}
	since \(\bs k(u)e^{\bs Sx_1}\bs e\) is decreasing in \(x_1\) and \(\bs k(u)\bs e=1\). Now, as \(h_{ij}^{--}(\lambda,x_1)\) is integrable with \(\int_{x_1=0}^\infty h_{ij}^{--}(\lambda, x_1)\wrt x_1\leq \widehat{ G}\), then (\ref{eqb:afsgh}) is less than or equal to
	\begin{align}
		\nonumber \left|\int_{u=\Delta-\varepsilon}^\infty \bs k(x_0)e^{\bs Su}\bs s \wrt u\right|\widehat GF
		&=\mathbb P(Z>x_0+\Delta-\varepsilon\mid Z>x_0)\widehat G F
		\\&\leq\cfrac{\var{(Z)}/\varepsilon^2}{1-\var{(Z)}/\varepsilon^2} \widehat G F,\nonumber 
	\end{align}
	by Chebyshev's inequality, since \(x_0\in(2\varepsilon,\Delta-\varepsilon)\). 

	%% d2 %% 
	Since \(|\psi(x)|\leq F\), then
	\begin{align}
		|d_2| &\leq \left| \int_{x=0}^{\Delta}\int_{u=0}^{\Delta-\varepsilon} \bs k(x_0)e^{\bs Su}\bs s r_{\bs v}(u,x) \wrt u \wrt x \right|F
		%
		\\&\leq \left| \int_{u=0}^{\Delta-\varepsilon} \bs k(x_0)e^{\bs Su}\bs s  \wrt u \right|R_{\bs v,2}F
		%
		\\&\leq R_{\bs v,2}F,
	\end{align}
	where the first inequality holds from Property~\ref{properties: 2}, and the last inequality holds since \(\int_{u=0}^{\Delta-\varepsilon} \bs k(x_0)e^{\bs Su}\bs s  \wrt u=\mathbb P(Z\in (x_0,x_0+\Delta-\varepsilon)\mid Z>x_0)\leq 1\). 
	
	%% d3 %%
	Since \(|\psi(x)|\leq F\) and \(h_{ij}^{--}(\lambda,\Delta-u-x)\leq G\), then
	\begin{align}
		|d_3|&\leq \int_{u=0}^{\min(\Delta-x_0-\varepsilon,\Delta-x-\varepsilon)} \bs k(x_0)e^{\bs Su}\bs s \wrt u \Delta GF
		\\ &\leq \mathbb P(Z\leq\Delta-\varepsilon \mid Z>x_0) \wrt u \Delta GF\nonumber
		\\ &\leq \cfrac{\var(Z)/\varepsilon^2}{1-\var(Z)/\varepsilon^2} \Delta GF,\nonumber
	\end{align}
	where the last inequality holds from Chebyshev's inequality. 
	
	%% d4 %%
	Similarly, since \(|\psi(x)|\leq F\) and \(h_{ij}^{--}(\lambda,\Delta-u-x)\leq G\), then
	\begin{align} 
		|d_4|&\leq \int_{x=0}^{x_0-2\varepsilon}\int_{u=\Delta-x_0+\varepsilon}^{\Delta-x-\varepsilon} \bs k(x_0)e^{\bs Su}\bs s \wrt u \wrt x GF \nonumber
			\\& = \int_{x=0}^{x_0-2\varepsilon}\mathbb P(Z\in[\Delta+\varepsilon, \Delta-x-\varepsilon + x_0]\mid Z>x_0) \wrt x GF \nonumber
			\\& \leq \int_{x=0}^{x_0-2\varepsilon}\mathbb P(Z\in[\Delta+\varepsilon, \Delta-\varepsilon + x_0]\mid Z>x_0) \wrt x GF \nonumber
			\\& \leq \mathbb P(Z\in[\Delta+\varepsilon, \Delta-\varepsilon + x_0]\mid Z>x_0)  \Delta GF \nonumber
			\\& \leq \cfrac{\var(Z)/\varepsilon^2}{1-\var(Z)/\varepsilon^2} \Delta GF. \nonumber
	\end{align}

	%% d5 %%
	Since \(|\psi(x)|\leq F\) and \(h_{ij}^{--}(\lambda,\Delta-u-x)\leq G\), then
	\begin{align} 
		|d_5|&\leq \int_{x=0}^{\Delta-\varepsilon}\int_{u=\Delta-x_0-\varepsilon}^{\Delta-x-\varepsilon} \bs k(x_0)e^{\bs Su}\bs s \wrt u 1(x\in[x_0-2\varepsilon,x_0))\wrt xGF
		\\&\leq \int_{x=0}^{\Delta-\varepsilon} 1(x\in[x_0-2\varepsilon,x_0))\wrt xGF  \nonumber 
		\\&= 2\varepsilon GF,  \nonumber 
	\end{align}
	where the second inequality holds since \(\int_{u=\Delta-x_0-\varepsilon}^{\Delta-x-\varepsilon} \bs k(x_0)e^{\bs Su}\bs s \wrt u\leq 1\).
	
	%% d6 %% 
	Since \(|\psi(x)|\leq F\) and \(h_{ij}^{--}(\lambda,\Delta-u-x)\leq G\), then
	\begin{align}
		&|d_6| \leq \int_{x=x_0-2\varepsilon}^{x_0} \wrt x GF\nonumber 
		\\&= 2\varepsilon GF.
	\end{align}
	
	%% d7 %% 
	Since \(|\psi(x)|\leq F\) and \(h_{ij}^{--}(\lambda,\Delta-u-x)\leq G\) then
	\begin{align}
		|d_7|&\leq \mathbb P(|Z-\Delta|> \varepsilon\mid Z>x_0)\Delta GF
		\\&\leq \cfrac{\var(Z)/\varepsilon^2}{1-\var(Z)/\varepsilon^2}\Delta GF
	\end{align}
	
	Convergence follows after setting \(\varepsilon^{(p)}=\var(Z^{(p)})^{1/3}\) and observing that all the bounds \(|d_1|,\dots,|d_7|\) tend to 0, as does the bound on~(\ref{eqn: the last one?}), given by \(2\varepsilon^{(p)}LF\Delta\), as \(p\to\infty\). 
\end{proof}  

	Now we show bounds for certain Laplace transform expressions which arise when the QBD-RAP starts in phases in \(\calS_{+0}\cup\calS_{-0}\) and there is more than one up-down or down-up transition before the first change of level. These expressions have the form
	\begin{align}
		\int_{x_1=0}^\infty g_1(x_1) \bs k(x_0) \bs D e^{\bs{S}x_1}\wrt x_1\bs D 
            	\left[\prod_{n=2}^{k-1}\int_{x_n=0}^\infty g_n(x_n) e^{\bs{S}x_n} \wrt x_n
		\bs D\right]
            	\int_{x_n=0}^\infty g_{n}(x_n) e^{\bs{S}x_n} \wrt x_n {\bs v}(x). \label{eqn: c30}
	\end{align}
	Here, we ultimately wish to show that (\ref{eqn: c30}) converges to \(g_{1,n}^*(\Delta-x_0,x)\). We do not do this directly, instead, we show that (\ref{eqn: c30}) is `close' to \(w_n(\Delta-x_0,x)\), then rely on the results from Chapter~\ref{sec: conv} to get the desired convergence. 
	
	Observe that by substituting the first matrix \(\bs D\) in the expression above for its integral expression, then (\ref{eqn: c30}) is equal to 
	\begin{align}
		&\int_{x_1=0}^\infty g_1(x_1) \bs k(x_0) \int_{z_0=0}^\infty e^{\bs Sz_0}\bs s\cfrac{\bs\alpha e^{\bs Sz_0}}{\bs\alpha e^{\bs Sz_0}\bs e} \wrt z_0 e^{\bs{S}x_1}\wrt x_1\bs D 
            	\left[\prod_{n=2}^{k-1}\int_{x_n=0}^\infty g_n(x_n) e^{\bs{S}x_n} \wrt x_n
		\bs D\right]
            	\\\nonumber&\quad\times\int_{x_n=0}^\infty g_{n}(x_n) e^{\bs{S}x_n} \wrt x_n {\bs v}(x) 
	%
		\\&=\bs k(x_0) \int_{z_0=0}^\infty e^{\bs Sz_0}\bs sw_n(z_0,x)\wrt z_0.\label{eqn: kd djv}
	\end{align}
	Intuitively, when the variance of \(Z\) is low, we expect that the integral in (\ref{eqn: kd djv}) above will be approximately equal to \(w_n(\Delta-x_0,x)\). Indeed, we proved in Lemma~\ref{lemma:bound} that this is the case for functions \(g\) satisfying the Assumptions~\ref{asu: g}. However, here we do not immediately have that \(w_n(x_0,x)\) is Lipschitz in \(x_0\), which we would need for it to satisfy Assumptions~\ref{asu: g}. Instead, we can show a Lipschitz-like condition in \(x_0\) for \(w_n(x_0,x)\), which suffices. 
	
For later use, observe that 
\begin{align}
	g^*_{2,n}(u_1,x) &= \int_{u_2=0}^{\Delta-u_1}g_2(\Delta - u_2 - u_1)\wrt u_1 \dots \nonumber 
         	\int_{u_{n-1}=0}^{\Delta-u_{n-2}} g_{n-1}(\Delta - u_{n-1} - u_{n-2}) \wrt u_{n-2}
       	\\&\qquad{}g_{n}(\Delta - x-u_{n-1})1(\Delta-x-u_{n-1}\geq0)\wrt u_{n-1} \nonumber
 %
 	\\&\leq G^{n-1}\int_{u_2=0}^{\Delta-u_1}\wrt u_1 \dots\nonumber
            	\int_{u_{n-1}=0}^{\Delta-u_{n-2}}  \wrt u_{n-1}
		\\&\leq G^{n-1}\Delta^{n-2}:=G^*_n.
\end{align}
% Also, 
% \begin{align}
% 	g^*_{2,n}(u_1,x) &= \int_{u_2=0}^{\Delta-u_1}g_2(\Delta - u_2 - u_1)\wrt u_1 \dots \nonumber 
%          	\int_{u_{n-1}=0}^{\Delta-u_{n-2}} g_{n-1}(\Delta - u_{n-1} - u_{n-2}) \wrt u_{n-2}
%        	\\&\qquad{}g_{n}(\Delta - x-u_{n-1})1(\Delta-x-u_{n-1}\geq0)\wrt u_{n-1} \nonumber
%  %
%  	\\&\leq G^{n-1}\int_{u_2=0}^{\Delta-u_1}\wrt u_1 \dots\nonumber
%             	\int_{u_{n-1}=0}^{\Delta-u_{n-2}}  \wrt u_{n-1}
% 		\\&\leq G^{n-1}\Delta^{n-2}:=G^*_n.
% \end{align}

\begin{cor}\label{cor: awrg}
	For \(x_0,x\in[0,\Delta)\), \(n\geq 2\), 
	\begin{align}
		&\left| w_n(x_0,x)-w_n(z_0,x)\right| 
		\leq 2|r_5(n)| + 2|r_6(n)| + 2(n-1)|r_4(n)| + |x_0-z_0|G_n^*(G+L\Delta). \label{eqn: ssdmm}
	\end{align}
\end{cor}
\begin{proof}
	Assume, without loss of generality \(x_0<z_0\). By adding and subtracting both \(\displaystyle \int_{u_1=0}^{\Delta-x_0}g_1(\Delta - u_1 - x_0)g^*_{2,n}(u_1,x)\wrt u_1\) and \(\displaystyle \int_{u_1=0}^{\Delta-z_0}g_1(\Delta - u_1 - z_0)g^*_{2,n}(u_1,x)\wrt u_1\), we can write the left-hand side of (\ref{eqn: ssdmm}) as 
	\begin{align}
\nonumber		&\Bigg| w_n(x_0,x)
		- \int_{u_1=0}^{\Delta-x_0}g_1(\Delta - u_1 - x_0)g^*_{2,n}(u_1,x)\wrt u_1
	%
		\\\nonumber&\qquad{}- w_n(z_0,x)
		{}+ \int_{u_1=0}^{\Delta-z_0}g_1(\Delta - u_1 - z_0)g^*_{2,n}(u_1,x)\wrt u_1
		%
		\\\nonumber&\qquad {} +\int_{u_1=0}^{\Delta-x_0}g_1(\Delta - u_1 - x_0)g^*_{2,n}(u_1,x)\wrt u_1
		%
		{} - \int_{u_1=0}^{\Delta-z_0}g_1(\Delta - u_1 - z_0)g^*_{2,n}(u_1,x)\wrt u_1\Bigg| \nonumber
		%
		%
		\intertext{which, by the triangle inequality, is less than or equal to}
		\nonumber& \Bigg| w_n(x_0,x)- \int_{u_1=0}^{\Delta-x_0}g_1(\Delta - u_1 - x_0)g^*_{2,n}(u_1,x)\wrt u_1\Bigg|
		%
		\\\nonumber&\quad{}+ \Bigg|w_n(z_0,x)- \int_{u_1=0}^{\Delta-z_0}g_1(\Delta - u_1 - z_0)g^*_{2,n}(u_1,x)\wrt u_1\Bigg|
		%
		\\&\quad{} +\Bigg|\int_{u_1=0}^{\Delta-x_0}g_1(\Delta - u_1 - x_0)g^*_{2,n}(u_1,x)\wrt u_1 - \int_{u_1=0}^{\Delta-z_0}g_1(\Delta - u_1 - z_0)g^*_{2,n}(u_1,x)\wrt u_1\Bigg|.\label{eqn: kdfsdf}
		%
%		\\&{} \leq 2|r_5(n)| + 2|r_6(n)| + 2(n-1)|r_4(n)| 
%		\\&{}+\Bigg|\int_{u_1=0}^{\Delta-x_0}g_1(\Delta - u_1 - x_0)w(\Delta-u_1)\wrt u_1 - \int_{u_1=0}^{\Delta-z_0}g_1(\Delta - u_1 - z_0)w(\Delta-u_1)\wrt u_1\Bigg|.
	\end{align}
	By Corollary~\ref{cor: a cor}, the first two terms of (\ref{eqn: kdfsdf}) are less than or equal to \[|r_5(n)| + |r_6(n)| + (n-1)|r_4(n)|.\] 
	As for the last term, adding and subtracting \( \int_{u_1=0}^{\Delta-z_0}g_1(\Delta - u_1 - x_0)g^*_{2,n}(u_1,x)\wrt u_1\) gives 
	\begin{align}
%		\nonumber &\Bigg|\int_{u_1=0}^{\Delta-x_0}g_1(\Delta - u_1 - x_0)g^*_{2,n}(u_1,x)\wrt u_1 - \int_{u_1=0}^{\Delta-z_0}g_1(\Delta - u_1 - z_0)g^*_{2,n}(u_1,x)\wrt u_1\Bigg|
%		%
		\nonumber &{} = \Bigg|\int_{u_1=0}^{\Delta-x_0}g_1(\Delta - u_1 - x_0)g^*_{2,n}(u_1,x)\wrt u_1 - \int_{u_1=0}^{\Delta-z_0}g_1(\Delta - u_1 - x_0)g^*_{2,n}(u_1,x)\wrt u_1
		\\\nonumber &\qquad{} - \int_{u_1=0}^{\Delta-z_0}(g_1(\Delta - u_1 - z_0)-g_1(\Delta - u_1 - x_0))g^*_{2,n}(u_1,x)\wrt u_1\Bigg|
		%
		\\\nonumber& \leq  \Bigg|\int_{u_1=\Delta-z_0}^{\Delta-x_0}g_1(\Delta - u_1 - x_0)g^*_{2,n}(u_1,x)\wrt u_1\Bigg|
		\\\nonumber &\qquad{} + \int_{u_1=0}^{\Delta-z_0}|g_1(\Delta - u_1 - z_0)-g_1(\Delta - u_1 - x_0)|g^*_{2,n}(u_1,x)\wrt u_1
		%
		\\\label{eqn: shhhhhh}& \leq  GG^*_n|x_0-z_0|
		 + \int_{u_1=0}^{\Delta-z_0}L|x_0-z_0|G^*_n\wrt u_1,
	\end{align}
	{since \(g_1\) is Lipschitz by Assumption~\ref{asu: lipschitz}, and \(g_{2,n}^*\leq G_n^*\). Bounding the integral over \(u_1\) by \(\Delta\), then (\ref{eqn: shhhhhh}) is less than or equal to}
	\begin{align}
		&GG^*_n|x_0-z_0| + \Delta L|x_0-z_0|G^*_n.
	\end{align}
\end{proof}

% \begin{cor}\label{cor: ksjkd}
% 	Let \(g_1, g_2, \dots,\) be functions satisfying the Assumptions~\ref{asu: g} and let \(\bs v(x)\) be a closing operator with the Properties~\ref{properties: some props}, then,
% 	\begin{align}
% 		&\int_{x_1=0}^\infty g_1(x_1) \bs k(x_0) e^{\bs{S}x_1}\wrt x_1\bs D 
%             	\left[\prod_{k=2}^{n-1}\int_{x_k=0}^\infty g_k(x_k) e^{\bs{S}x_k} \wrt x_k \bs D\right] \int_{x_n=0}^\infty g_{n}(x_n) e^{\bs{S}x_n} \wrt x_n \bs v(x) \nonumber 
% 		\\&\leq \cfrac{1}{\bs \alpha e^{\bs{S}x_0}\bs e} \widehat{G}^{n-1}G(G_{\bs v}+\widehat G O(\var(Z)) \label{eqn :mmmm}
% 	\end{align}
% \end{cor}
% \begin{proof}
% 	By Corollary~\ref{cor: amammme} 
% 	\begin{align}
% 		&\int_{x_1=0}^\infty g_1(x_1) \bs k(x_0) e^{\bs{S}x_1}\wrt x_1\bs D 
%             	\left[\prod_{k=2}^{n-1}\int_{x_k=0}^\infty g_k(x_k) e^{\bs{S}x_k} \wrt x_k \bs D\right] \int_{x_n=0}^\infty g_{n}(x_n) e^{\bs{S}x_n} \wrt x_n \bs v(x)\nonumber
% 		\\&\leq \int_{x_1=0}^\infty g_1(x_1) \bs k(x_0) e^{\bs{S}x_1}\wrt x_1\bs D 
% 		\left[\prod_{k=2}^{n-1}\int_{x_k=0}^\infty g_k(x_k) e^{\bs{S}x_k} \wrt x_k \bs D\right] \int_{x_n=0}^\infty g_{n}(x_n) e^{\bs{S}x_n} \wrt x_n\bs w(x) \nonumber
% 		\\&\quad{} + O(\var(Z)). \label{eqn: LLaaNNab}
% 	\end{align}
% 	The first term of (\ref{eqn: LLaaNNab}) can be seen to be equivalent to \(\mathcal J_{1,n+1}(x_0,x_1),\) with \(g_1(x_1)=1\), and the integrability condition on \(g_1\) is not required to prove the bound. 
% \end{proof}


% \begin{cor}\label{cor: amammme}
% 	Let \(g_1, g_2, \dots,\) be functions satisfying the Assumptions~\ref{asu: g} and let \({\bs v}(x)\) be a closing operator with the Properties~\ref{properties: some props}, then,
% 	\begin{align}
% 		&\int_{x_1=0}^\infty g_1(x_1) \bs k(x_0) e^{\bs{S}x_1}\wrt x_1\bs D 
%             	\left[\prod_{k=2}^{n-1}\int_{x_k=0}^\infty g_k(x_k) e^{\bs{S}x_k} \wrt x_k \bs D\right] \int_{x_n=0}^\infty g_{n}(x_n) e^{\bs{S}x_n} \wrt x_n \widetilde{\bs w}(x) \nonumber 
% 		% \\&= \int_{x_1=0}^\infty g_1(x_1) \bs k(x_0) e^{\bs{S}x_1}\wrt x_1\bs D 
% 		% \left[\prod_{k=2}^{n-1}\int_{x_k=0}^\infty g_k(x_k) e^{\bs{S}x_k} \wrt x_k \bs D\right] \int_{x_n=0}^\infty g_{n}(x_n) e^{\bs{S}x_n} \wrt x_n {\bs w}(x) \nonumber
% 		\\&= O(\var(Z)) \label{eqn :mmmm2}
% 	\end{align}
% \end{cor}
% \begin{proof}
% 	% First observe that 
% 	% \begin{align*} 
% 	% 	\int_{x_n=0}^\infty \bs\alpha e^{\bs S(u+x_n)}{\bs v}(x)\wrt x_n 
% 	% 	&=\int_{x_n=0}^\infty \bs\alpha e^{\bs S(u+x_n)} (\bs w(x)+\widetilde{\bs w}(x))\wrt x_n
% 	% 	\\&=\int_{x_n=0}^\infty \bs\alpha e^{\bs S(u+x_n)} \bs w(x)\wrt x_n + \bs\alpha e^{\bs S(u+x_n)} \widetilde{\bs w}(x))\wrt x_n
% 	% \end{align*}
% 	% and recall that Property~\ref{properties: 0} states 
% 	% \begin{align}
% 	% 	\int_{x_n=0}^\infty \bs\alpha e^{\bs S(u+x_n)} \widetilde{\bs w}(x)\wrt x_n = \bs\alpha e^{\bs Su}(-\bs S)^{-1} \widetilde{\bs w}(x) = O(\var(Z)).
% 	% \end{align}

% 	Consider the left-hand side of (\ref{eqn :mmmm2}). Replacing the last \(\bs D\) matrix in (\ref{eqn :mmmm2}) by its integral definition, gives 
% 	\begin{align*}
% 		&\int_{x_1=0}^\infty g_1(x_1) \bs k(x_0) e^{\bs{S}x_1}\wrt x_1
%             \left[\prod_{k=2}^{n-1}\bs D \int_{x_k=0}^\infty g_k(x_k) e^{\bs{S}x_k} \wrt x_k \right] \int_{u=0}^\infty e^{\bs S u}\bs s \cfrac{\bs \alpha e^{\bs S u}}{\bs \alpha e^{\bs S u} \bs e}\wrt u 
% 			\\&\qquad{}\int_{x_n=0}^\infty g_{n}(x_n) e^{\bs{S}x_n} \wrt x_n \widetilde{\bs w}(x) 
% 		\\&=\bs k(x_0) \int_{u=0}^\infty \mathcal I_{1,n}(u) \cfrac{\bs \alpha e^{\bs S u}}{\bs \alpha e^{\bs S u} \bs e}\wrt u \int_{x_n=0}^\infty g_{n}(x_n) e^{\bs{S}x_n} \wrt x_n \widetilde{\bs w}(x) 
% 		\\&\leq \cfrac{1}{\bs \alpha e^{\bs{S}x_0}\bs e}G\widehat G^{n-1} \int_{u=0}^\infty \bs \alpha e^{\bs{S}u}\bs e \cfrac{\bs \alpha e^{\bs S u}}{\bs \alpha e^{\bs S u} \bs e}\wrt u \int_{x_n=0}^\infty g_{n}(x_n) e^{\bs{S}x_n} \wrt x_n \widetilde{\bs w}(x) 
% 	\end{align*}
% 	by Lemma~\ref{lem: lh bnd}. Integrating over \(u\) gives 
% 	\begin{align*}
% 		&\cfrac{1}{\bs \alpha e^{\bs{S}x_0}\bs e}G\widehat G^{n-1} \bs \alpha (-\bs S)^{-1} \int_{x_n=0}^\infty g_{n}(x_n) e^{\bs{S}x_n} \wrt x_n \widetilde{\bs w}(x) 
% 		\\&\leq \cfrac{1}{\bs \alpha e^{\bs{S}x_0}\bs e}G\widehat G^{n-1} \bs \alpha (-\bs S)^{-1} \int_{x_n=0}^\infty g_{n}(x_n) \wrt x_n \widetilde{\bs w}(x),
% 	\end{align*}
% 	by Property~\ref{properties: -1}. Integrating over \(x_n\), gives
% 	\begin{align*}
% 		\cfrac{1}{\bs \alpha e^{\bs{S}x_0}\bs e}G\widehat G^{n} \bs \alpha (-\bs S)^{-1} \widetilde{\bs w}(x) 
% 		&=\cfrac{1}{\bs \alpha e^{\bs{S}x_0}\bs e}G\widehat G^{n}O(\var(Z)),
% 	\end{align*}
% 	by Property~\ref{properties: 0}.
% \end{proof}


\begin{cor}\label{cor: ahjg}
	Let \(g_1,g_2,\dots,\) be functions satisfying Assumptions~\ref{asu: g} and let \({\bs v}(x)\), \(x\in[0,\Delta)\), be a closing operator with Properties~\ref{properties: some props}. For \(x_0,x\in\mathcal [0,\Delta)\), \(n\geq 2\), 
	\begin{align}
		&\left|\bs k(x_0) \int_{z_0=0}^\infty e^{\bs Sz_0}\bs sw_n(z_0,x)\wrt z_0 - w_n(\Delta - x_0,x) \right| \nonumber
		= r_8(n),
	\end{align}
	where 
	\begin{align*}
		|r_8(n)|&\leq  \left( 2|r_5(n)| + 2|r_6(n)| + 2(n-1)|r_4(n)| + \varepsilon G_n^{*}(G+L\Delta) \right) \\&\qquad{}+ 2\widehat G^{n-2}GG_{\bs v}\cfrac{\var(Z)/\varepsilon^2}{1-\var(Z)/(\Delta-x_0)^2}.
	\end{align*}
\end{cor}
\begin{proof}
	Consider
	\begin{align}
		\nonumber&\left|\bs k(x_0) \int_{z_0=0}^\infty e^{\bs Sz_0}\bs sw_n(z_0,x)\wrt z_0 - w_n(\Delta - x_0,x) \right|
		\\\nonumber&= \left|\bs k(x_0) \int_{z_0=0}^\infty e^{\bs Sz_0}\bs s(w_n(z_0,x) - w_n(\Delta - x_0,x)) \wrt z_0\right|
		%
		\\\nonumber&\leq \bs k(x_0) \int_{z_0=0}^\infty e^{\bs Sz_0}\bs s  \left|w_n(z_0,x) - w_n(\Delta - x_0,x)\right| \wrt z_0
		%
		\\\nonumber&= \bs k(x_0) \int_{z_0=0}^{\Delta-\varepsilon-x_0} e^{\bs Sz_0}\bs s  \left|w_n(z_0,x) - w_n(\Delta - x_0,x)\right| \wrt z_0
		\\\nonumber&\qquad{}+\bs k(x_0) \int_{z_0=\Delta+\varepsilon-x_0}^\infty e^{\bs Sz_0}\bs s  \left|w_n(z_0,x) - w_n(\Delta - x_0,x)\right| \wrt z_0
		\\\label{eqn: KASJF}&\qquad{}+\bs k(x_0) \int_{z_0=\Delta-\varepsilon-x_0}^{\Delta+\varepsilon-x_0} e^{\bs Sz_0}\bs s  \left|w_n(z_0,x) - w_n(\Delta - x_0,x)\right| \wrt z_0.
		%
	\end{align}
		
		Using Equations~(\ref{eqn: in here}), (\ref{eqn: J bound}) and (\ref{eqn:FGHJSjjs sj}) we can claim
		\begin{align}
			|w_n(x_0,x)| &\leq \cfrac{1}{\bs \alpha e^{\bs{S} x_0}\bs e}G^2 \widehat G^{n-2}  
			\int_{u_k=0}^\infty \bs \alpha e^{\bs{S}u_k}\bs e \wrt u_k G_{\bs v} + \cfrac{1}{\bs \alpha e^{\bs{S}x_0}\bs e}G\widehat G^{n}\widetilde G_{\bs v} \nonumber 
			\\& = \cfrac{1}{\bs \alpha e^{\bs{S} x_0}\bs e}G^2
			 \widehat G^{n-2} G_{\bs v} + \cfrac{1}{\bs \alpha e^{\bs{S}x_0}\bs e}G\widehat G^{n}\widetilde G_{\bs v} \nonumber 
			 \\&=:W_n.
		\end{align}
		
		Therefore, the sum of the first two terms in (\ref{eqn: KASJF}) is less than or equal to 
		\begin{align}
		&2W_n\left(\int_{z_0=0}^{\Delta-\varepsilon-x_0}\bs k(x_0)e^{\bs S z_0} \bs s \wrt z_0+\int_{z_0=\Delta+\varepsilon-x_0}^\infty \bs k(x_0)e^{\bs S z_0} \bs s \wrt z_0\right)\nonumber 
		\\\nonumber &= 2W \cfrac{\mathbb P(|Z-\Delta|> \varepsilon)}{\mathbb P(Z> x_0)}
		\\&\leq 2W_n\cfrac{\var(Z)/\varepsilon^2}{1-\var(Z)/(\Delta-x_0)^2}
		\end{align}
		by Chebyshev's inequality. As for the last term in (\ref{eqn: KASJF}), we can use Corollary~\ref{cor: awrg} to bound the integrand so that the last term is less than or equal to 
		\begin{align}
		\nonumber&\bs k(x_0) \int_{z_0=\Delta-\varepsilon-x_0}^{\Delta+\varepsilon-x_0} e^{\bs Sz_0}\bs s \left( 2|r_5(n)| + 2|r_6(n)| + 2(n-1)|r_4(n)| + \varepsilon G^{n-1}\Delta^{n-2}(G+L\Delta) \right) \wrt z_0
		%
		\\&\leq  \left( 2|r_5(n)| + 2|r_6(n)| + 2(n-1)|r_4(n)| + \varepsilon G^{n-1}\Delta^{n-2}(G+L\Delta) \right),\label{eqn: AASNN}
		% \\&\quad{}+2W_n\cfrac{\var(Z)/\varepsilon^2}{1-\var(Z)/(\Delta-x_0)^2} ,
	\end{align}
	since \(\displaystyle \bs k(x_0) \int_{z_0=\Delta-\varepsilon-x_0}^{\Delta+\varepsilon-x_0} e^{\bs Sz_0}\bs s\wrt z_0\leq 1\). Thus, (\ref{eqn: KASJF}) is bounded by (\ref{eqn: AASNN}).
\end{proof}

\begin{cor} \label{cor: aaaaa}
	Let \(g_1,g_2,\dots,\) be functions satisfying Assumptions~\ref{asu: g} and let \({\bs v}(x)\), \(x\in(0,\Delta)\), be a closing operator with Properties~\ref{properties: some props}. For \(x_0,x\in(0,\Delta)\), \(n\geq 2\)
	\begin{align}
		&\left|\bs k(x_0) \int_{z_0=0}^\infty e^{\bs Sz_0}\bs sw_n(z_0,x)\wrt z_0- g_{1,n}^*(\Delta-x_0,x) \right| \nonumber
		\\&\leq |r_8(n)|+|r_5(n)|+|r_6(n)| + (n-1)|r_4(n)|.\label{eqn: KAFnn}
	\end{align}
	% where 
	% \begin{align}
	% 	|r_8(n)|&\leq \left( 2|r_5(n)| + 2|r_6(n)| + 2(n-1)|r_4(n)| + \varepsilon G^{n-1}\Delta^{n-2}(G+L\Delta) \right) \\&\quad{} 
	% +2\widehat G^{n-2}GG_{\bs v}\cfrac{\var(Z)/\varepsilon^2}{1-\var(Z)/(\Delta-x_0)^2}.
	% \end{align}
\end{cor}
\begin{proof}
	Adding and subtracting \(w_n(\Delta-x_0,x)\) within the absolute value on the left-hand side of (\ref{eqn: KAFnn}) 
	\begin{align*}
		%&\Bigg| \int_{x_1=0}^\infty g_1(x_1) \bs k(x_0) \bs D e^{\bs{S}x_1}\wrt x_1\bs D 
%            	\left[\prod_{n=2}^{k-1}\int_{x_n=0}^\infty g_n(x_n) e^{\bs{S}x_n} \wrt x_n
%		\bs D\right]
%            	\int_{x_n=0}^\infty g_{n}(x_n) e^{\bs{S}x_n} \wrt x_n {\bs v}(x) \nonumber 
	%
%		\\&{}- g_{1,n}^*(\Delta - x_0,x) \Bigg| 
		&\left| \bs k(x_0) \int_{z_0=0}^\infty e^{\bs Sz_0}\bs sw_n(z_0,x)\wrt z_0 \nonumber 
	%
		 - w_n(\Delta - x_0,x) + w_n(\Delta - x_0,x) - g_{1,n}^*(\Delta - x_0,x) \right| 
		\\&\leq \left| \bs k(x_0) \int_{z_0=0}^\infty e^{\bs Sz_0}\bs sw_n(z_0,x)\wrt z_0 - w_n(\Delta - x_0,x)\right| + \left| w_n(\Delta - x_0,x) - g_{1,n}^*(\Delta - x_0,x) \right| 
	\end{align*}
	where the first absolute value is less than or equal to \(|r_8(n)|\) by Corollary~\ref{cor: ahjg} and the second absolute value is less than or equal to \(|r_5(n)|+|r_6(n)| + (n-1)|r_4(n)|\) by Corollary~\ref{cor: a cor}.
\end{proof}

\begin{cor}\label{cor: willies}
	Let \(\psi\) be bounded and Lipschitz, let \(g_1,g_2,\dots,\) be functions satisfying Assumptions~\ref{asu: g} and let \({\bs v}(x)\), \(x\in(0,\Delta)\), be a closing operator with Properties~\ref{properties: some props}. For \(x_0,x\in(0,\Delta)\), \(n\geq 2\)
	\begin{align}
		&\left| \int_{x\in[0,\Delta)}\bs k(x_0) \int_{z_0=0}^\infty e^{\bs Sz_0}\bs sw_n(z_0,x)\wrt z_0 \psi(x)\wrt x  
	% 
		- \int_{x\in[0,\Delta)} g_{1,n}^*(\Delta-x_0,x)\psi(x)\wrt x \right|\nonumber 
		\\&\leq \left(|r_8(n)|+|r_5(n)|+|r_6(n)| + (n-1)|r_4(n)|\right)F\Delta.\label{eqn: KAFnnmna2}
	\end{align}
\end{cor}
\begin{proof}
	The left-hand side of (\ref{eqn: KAFnnmna2}) is less than or equal to 
	\begin{align}
		& \int_{x\in[0,\Delta)} \left| \bs k(x_0) \int_{z_0=0}^\infty e^{\bs Sz_0}\bs sw_n(z_0,x)\wrt z_0 
	%
		- g_{1,n}^*(\Delta-x_0,x)\right| \left|\psi(x)\right|\wrt x . \label{eqn: lfj}
	\end{align}
	Now, using \(|\psi(x)|\leq F\) and Corollary~\ref{cor: aaaaa} then (\ref{eqn: lfj}) is less than or equal to 
		\[\left(|r_8(n)|+|r_5(n)|+|r_6(n)| + (n-1)|r_4(n)|\right) \Delta F.\]
\end{proof}
%
%To summarise, the main results we need are 
%\begin{align}
%	&\Bigg| \int_{x=0}^\Delta w_n(x_0,x) \psi(x) \wrt x \nonumber 
%
%	\\&{}- \int_{x=0}^\Delta \int_{u_1=0}^{\Delta-x_0}g_1(\Delta - u_1 - x_0)
%		\int_{u_2=0}^{\Delta-u_1}g_2(\Delta - u_2 - u_1)\wrt u_1  \nonumber 
%	\left[\prod_{k=2}^{n-1} \int_{u_k=0}^{\Delta-u_{k-1}} g_k(\Delta-u_k-u_{k-1})\wrt u_{k-1}\right] \nonumber 
%	%\\&{}\nonumber
%			%\int_{u_{n-1}=0}^{\Delta-u_{n-2}} g_{n-1}(\Delta - u_{n-1} - u_{n-2}) \wrt u_{n-2}
%			g_{n}(\Delta - x-u_{n-1})
%\\&\qquad{} 1(\Delta-x-u_{n-1}\geq0) \wrt u_{n-1}\psi(x) \wrt x \Bigg| \nonumber
%	\\&\leq (|r_5(n)| + |r_6(n)| + (n-1)|r_4(n)|)\Delta F. \label{eqn: rhs g 4dvfklsmv2}
%\end{align}
%and 
%\begin{align}
%	&\Bigg| \int_{x\in[0,\Delta)} \int_{x_1=0}^\infty g_1(x_1) \bs k(x_0) \bs D e^{\bs{S}x_1}\wrt x_1\bs D 
%			\left[\prod_{n=2}^{k-1}\int_{x_n=0}^\infty g_n(x_n) e^{\bs{S}x_n} \wrt x_n
%	\bs D\right] \nonumber 
%			\\&\qquad{}\times\int_{x_n=0}^\infty g_{n}(x_n) e^{\bs{S}x_n} \wrt x_n {\bs v}(x) \psi(x)\wrt x \nonumber 
%
%	\\&\qquad {}- \int_{x\in[0,\Delta)} \int_{u_1=0}^{x_0}g_1(x_0 - u_1)
%		\int_{u_2=0}^{\Delta-u_1}g_2(\Delta - u_2 - u_1)\wrt u_1  \nonumber 
%	\left[\prod_{k=2}^{n-1} \int_{u_k=0}^{\Delta-u_{k-1}} g_k(\Delta-u_k-u_{k-1})\wrt u_{k-1}\right] \nonumber 
%	%\\&{}\nonumber
%			%\int_{u_{n-1}=0}^{\Delta-u_{n-2}} g_{n-1}(\Delta - u_{n-1} - u_{n-2}) \wrt u_{n-2}
%			g_{n}(\Delta - x-u_{n-1})
%\\&\qquad{} \times1(\Delta-x-u_{n-1}\geq0)\wrt u_{n-1}\psi(x)\wrt x \Bigg| \nonumber
%	\\&\leq \left(|r_8(n)|+|r_5(n)|+|r_6(n)| + (n-1)|r_4(n)|\right)F\Delta.\label{eqn: KAFnnmna22}
%\end{align}
%
We now extend the previous results to the matrix case.
\begin{lem}\label{lem: boobies2}
	Let \(\bs G_k(x)\), \(k\in\{1,2,...\}\), be matrix functions with dimensions \(N_k \times N_{k+1}\). Further, suppose the scalar functions \([\bs G_k(x)]_{ij}\), \(k\in\{1,2,...\}\) satisfy Assumptions~\ref{asu: g}. Then, 
	\begin{align}
		&\Bigg| \int_{x\in[0,\Delta)} \int_{x_1=0}^\infty \bs G_1(x_1) \otimes \bs k(x_0) \bs D e^{\bs{S}x_1}\wrt x_1\bs D 
				\left[\prod_{k=2}^{n-1}\int_{x_k=0}^\infty \bs G_{k}(x_k) \otimes e^{\bs{S}x_k} \wrt x_k
		\bs D\right] \nonumber 
				\\&\qquad{}\times\int_{x_n=0}^\infty \bs G_{n}(x_n) \otimes e^{\bs{S}x_n} \wrt x_n {\bs v}(x) \psi(x)\wrt x \nonumber 
	%
		\\&\qquad {}- \int_{x\in[0,\Delta)} \int_{u_1=0}^{x_0}\bs G_1(x_0 - u_1)
	%		\int_{u_2=0}^{\Delta-u_1}g_2(\Delta - u_2 - u_1)\wrt u_1  \nonumber 
		\left[\prod_{k=2}^{n-1} \int_{u_k=0}^{\Delta-u_{k-1}} \bs {G}_{k}(\Delta-u_k-u_{k-1})\wrt u_{k-1}\right] \nonumber 
		%\\&{}\nonumber
				%\int_{u_{n-1}=0}^{\Delta-u_{n-2}} g_{n-1}(\Delta - u_{n-1} - u_{n-2}) \wrt u_{n-2}
				\\&\qquad{} \bs G_{n}(\Delta - x-u_{n-1})
	 		\times1(\Delta-x-u_{n-1}\geq0)\wrt u_{n-1}\psi(x)\wrt x \Bigg| \nonumber
		\\&\leq \left(|r_8(n)|+|r_5(n)|+|r_6(n)| + (n-1)|r_4(n)|\right)F\Delta \prod_{k=2}^{n}N_{k}.\label{eqn: KAFnnmna22G}
	\end{align}
	Moreover, choosing \(\varepsilon=\var(Z)\), then, for fixed \(n\), the bound is \(\mathcal O(\var(Z)^{1/3})\). 
\end{lem}
\begin{proof} 
	The proof is the same as the proof of Lemma~\ref{lem: boobies}, with Corollary~\ref{cor: willies} replacing Corollary~\ref{cor: asjdajaaaaa}.
\end{proof}
Lemma~\ref{lem: boobies2} effectively shows that, as \(p \to \infty\), then 
\begin{align}
	&\int_{x\in[0,\Delta)} \int_{x_1=0}^\infty \bs G_1(x_1) \otimes \bs k^{(p)} (x_0) \bs D^{(p)} e^{\bs{S}^{(p)}x_1}\wrt x_1\bs D^{(p)} 
			\left[\prod_{k=2}^{n-1}\int_{x_k=0}^\infty \bs G_{k}(x_k) \otimes e^{\bs{S}^{(p)}x_k} \wrt x_k
	\bs D^{(p)} \right] \nonumber 
			\\&\qquad{}\times\int_{x_n=0}^\infty \bs G_{n}(x_n) \otimes e^{\bs{S}^{(p)}x_n} \wrt x_n {\bs v}^{(p)}(x) \psi(x)\wrt x \nonumber 
%
	\\&\qquad {}\to \int_{x\in[0,\Delta)} \int_{u_1=0}^{x_0}\bs G_1(x_0 - u_1)
%		\int_{u_2=0}^{\Delta-u_1}g_2(\Delta - u_2 - u_1)\wrt u_1  \nonumber 
	\left[\prod_{k=2}^{n-1} \int_{u_k=0}^{\Delta-u_{k-1}} \bs {G}_{k}(\Delta-u_k-u_{k-1})\wrt u_{k-1}\right] \nonumber 
	%\\&{}\nonumber
			%\int_{u_{n-1}=0}^{\Delta-u_{n-2}} g_{n-1}(\Delta - u_{n-1} - u_{n-2}) \wrt u_{n-2}
			\\&\qquad{} \bs G_{n}(\Delta - x-u_{n-1})
		 \times1(\Delta-x-u_{n-1}\geq0)\wrt u_{n-1}\psi(x)\wrt x.  \nonumber
\end{align}

% Thus, we have established a type of convergence of the QBD-RAP scheme to the fluid queue on the event that the phase is initially in \(\calS_{+0}\cup\calS_{-0}\), and before the first change of level.
The technical results in this section are enough to prove Theorem~\ref{thm: a thm2!}.

\subsection{More results} 
We are not quite done yet. If we want to use Theorem~\ref{thm: a thm2!} to prove convergence before the first orbit restart epoch, we need a domination condition like that in Lemma~\ref{lem: gkjljklgagjklagsjlk}. 
\begin{lem}\label{lem: gkjljklgagjklagsjlk2}For all \(M\geq 0\), \(x\in\calD_{\ell_0,j}\), \(x_0\in\calD_{\ell_0,i}\), \(\ell_0\in\mathcal K\), \(\lambda > 0\), \(q\in\{+0,-0\}\), \(r\in\{+,-\}\), \(i\in\calS_q\), \(j\in\calS_r\cup\calS_{r0}\), and for any bounded function \(\psi\), \(|\psi|<F\), 
	\begin{align}
		\sum_{m=M+1}^\infty \left| \int_{x\in\calD_{\ell_0}} \widehat f^{\ell_0,(p)}_{m,q,r}(\lambda)(x,j;x_0,i)\psi(x)\wrt x
		-
		\int_{x\in\calD_{\ell_0}} \widehat \mu^{\ell_0}_{m,q,r}(\lambda)(\wrt x,j;x_0,i)\psi(x) \right| \leq  r_6^M
	\end{align} 
	where 
	\[ r_6^M =  F(G\Delta  + \widehat G )\left(\frac{\gamma}{\gamma+\lambda}\right)^{2M+2}\left(1-\left(\frac{\gamma}{\gamma+\lambda}\right)^2\right)^{-1}.\]
	% where 
	% \[W=\max_{k\in\calS_{0}}\sum_{i\in\calS_+\cup\calS_-} \bs e_k[\bs I - \bs T_{00}]^{-1}\bs T_{0i}.\]
\end{lem}
\begin{proof}
	The proof follows the same arguments as the proof for the case \(q=0\), \(i\in\calS_0^*\) in the proof of Lemma~\ref{lem: gkjljklgagjklagsjlk}.

	% The same arguments and results apply for all \(p\), so let us drop the dependence on \(p\). 
	
	% Consider \(i\in\calS_{-0},j\in\mathcal S_+\cup\calS_{+0}\). By the triangle inequality, 
	% \begin{align*}
	% 	&\sum_{m=M+1}^\infty \left| \int_{x\in\calD_{\ell_0}} \widehat f^{\ell_0}_{m,-0,+}(\lambda)(x,j;x_0,i)\psi(x) \wrt x
	% 	-
	% 	 \int_{x\in\calD_{\ell_0}} \widehat \mu^{\ell_0}_{m,-0,+}(\lambda)(\wrt x,j;x_0,i) \psi(x) \right|
	% 	\\&\leq\sum_{m=M+1}^\infty \int_{x\in\calD_{\ell_0}} \widehat f^{\ell_0}_{m,-0,+}(\lambda)(x,j;x_0,i) |\psi(x)| \wrt x
	% 	\\&\qquad{} +\sum_{m=M+1}^\infty \int_{x\in\calD_{\ell_0}} \widehat  \mu^{\ell_0}_{m,-0,+}(\lambda)(\wrt x,j;x_0,i) |\psi(x)|,
	% \end{align*}
	% since all terms are non-negative. 
	
	% Consider \(\displaystyle\int_{x\in\calD_{\ell_0}}\widehat f^{\ell_0}_{m,-0,+}(\lambda)(x,j;x_0,i)|\psi(x)|\wrt x\), which is less than or equal to  
    %     \begin{align}
    %     	&\int_{x\in\calD_{\ell_0}}\int_{x_1=0}^\infty \dots \int_{x_{2m+1}=0}^\infty  \bs e_k [\bs I - \bs T_{00}]^{-1}\bs T_{0+}\bs M^m_{++}(\lambda,x_1,\dots,x_{2m+1})\bs e_j\tr{} \nonumber
	% 		\\& \bs a_{\ell_0,k}^{(p)}(x_0) \bs D^{(p)} \bs N^{2m+1,(p)}(\lambda,x_1,\dots,x_{2m+1}) {\bs v}_{\ell_0,j}^{(p)}(x)\wrt x_{2m+1}\dots\wrt x_2 \wrt x_1 \wrt x F\nonumber
	% 		\\& + \int_{x\in\calD_{\ell_0}}\int_{x_1=0}^\infty \dots \int_{x_{2m+2}=0}^\infty  \bs e_k [\bs I - \bs T_{00}]^{-1}\bs T_{0-}\bs M^{m+1}_{-+}(\lambda,x_1,\dots,x_{2m+2})\bs e_j\tr{} \nonumber
	% 		\\& \bs a_{\ell_0,k}^{(p)}(x_0)\bs N^{2m+2,(p)}(\lambda,x_1,\dots,x_{2m+2}) {\bs v}_{\ell_0,j}^{(p)}(x)\wrt x_{2m+2}\dots\wrt x_2 \wrt x_1  \wrt x F\label{eqn:PPO}
	% \end{align}
	% since \(|\psi|\leq F\). To bound the second and last-lines of (\ref{eqn:PPO}) we first observe that for \(\bs a \in \mathcal A\), 
	% \begin{align}
	% 	\bs a\int_{x\in\calD_{\ell_0}}\bs De^{\bs Sx_{2m+1}}{\bs v}_{\ell_0,j}(x) 
	% 	&= \bs a\int_{x\in\calD_{\ell_0}}\int_{u=0}^\infty e^{\bs Su}\bs s\cfrac{\bs \alpha e^{\bs S u}}{\bs \alpha e^{\bs S u}\bs e}e^{\bs Sx_{2m+1}}{\bs v}_{\ell_0,j}(x)\wrt u\wrt x\nonumber
	% 	\\&\leq \bs a\int_{u=0}^\infty e^{\bs Su}\bs s\cfrac{\bs \alpha e^{\bs S u}}{\bs \alpha e^{\bs S u}\bs e}e^{\bs Sx_{2m+1}}{\bs e}\wrt u\nonumber
	% 	\\&= \bs a\bs D e^{\bs Sx_{2m+1}}{\bs e}\wrt u, \label{eqn: kkkaa2}
	% \end{align}
	% where the inequality holds from Property~\ref{properties: -2}. By definition, the second and last lines of (\ref{eqn:PPO}) are 
	% \begin{align}
	% 	&\int_{x\in\calD_{\ell_0}} \bs a \bs N^{2m+1}(\lambda,x_1,\dots,x_{2m+1}) {\bs v}_{\ell_0,j}( x)\wrt x_{2m+1}\dots\wrt x_1\nonumber
	% 	\\&=\int_{x\in\calD_{\ell_0}}\bs a_{\ell_0,i}(x_0)e^{\bs{S}x_1} \bs{D}e^{\bs{S}x_2} \bs{D}\dots e^{\bs{S}x_{2m}} \bs{D}e^{\bs{S}x_{2m+1}}{\bs v}_{\ell_0,j}(x),\label{eqn:llaaaaaa2}
	% \end{align}
	% for some \(\bs a \in\mathcal A\). Now, using (\ref{eqn: kkkaa2}), then (\ref{eqn:llaaaaaa2}) is less than or equal to 
	% \begin{align*}
	% 	&\bs a \bs{D}e^{\bs{S}x_2} \bs{D}\dots e^{\bs{S}x_{2m}} \bs{D}e^{\bs{S}x_{2m+1}}{\bs e}
	% 	\\& = \bs a e^{\bs{S}x_1} \bs{D}e^{\bs{S}x_2} \bs{D}\dots e^{\bs{S}x_{2m}} \int_{u=0}^\infty e^{\bs Su}\cfrac{\bs \alpha e^{\bs Su}}{\bs \alpha e^{\bs Su}\bs e}\wrt u e^{\bs{S}x_{2m+1}}{\bs e}
	% 	\\&\leq \bs ae^{\bs{S}x_1} \bs{D}e^{\bs{S}x_2} \bs{D}\dots e^{\bs{S}x_{2m}} \int_{u=0}^\infty e^{\bs Su}\cfrac{\bs \alpha e^{\bs Su}}{\bs \alpha e^{\bs Su}\bs e}\wrt u {\bs e}
	% 	\\&=\bs ae^{\bs{S}x_1} \bs{D}e^{\bs{S}x_2} \bs{D}\dots e^{\bs{S}x_{2m}} \bs e.
	% \end{align*}
	% Repeating \(m\) more times gives the bound \(\bs a\bs e=1\). 

    % Therefore, (\ref{eqn:PPO}) is less than or equal to 
	% \begin{align}
	% 	&\int_{x_1=0}^\infty \dots \int_{x_{2m+1}=0}^\infty  \bs e_k [\bs I - \bs T_{00}]^{-1}\bs T_{0+}\bs M^m_{++}(\lambda,x_1,\dots,x_{2m+1})\bs e_j\tr{} F \nonumber
	% 	% \\& \bs a_{\ell_0,k}^{(p)}(x_0) \bs D^{(p)} \bs N^{2m+1,(p)}(\lambda,x_1,\dots,x_{2m+1}) {\bs v}_{\ell_0,j}^{(p)}(x)\wrt x_{2m+1}\dots\wrt x_2 \wrt x_1  \nonumber
	% 	\\& + \int_{x_1=0}^\infty \dots \int_{x_{2m+2}=0}^\infty  \bs e_k [\bs I - \bs T_{00}]^{-1}\bs T_{0-}\bs M^{m+1}_{-+}(\lambda,x_1,\dots,x_{2m+2})\bs e_j\tr{} F 
	% 	% \\& \bs a_{\ell_0,k}^{(p)}(x_0)\bs N^{2m+2,(p)}(\lambda,x_1,\dots,x_{2m+2}) {\bs v}_{\ell_0,j}^{(p)}(x)\wrt x_{2m+2}\dots\wrt x_2 \wrt x_1 F 
	% 	\label{eqn :NNeeaefjn2}
	% \end{align}
	% The expressions \[\int_{x_1=0}^\infty \dots \int_{x_{2m+1}=0}^\infty \bs e_i\bs M^m_{++}(\lambda,x_1,\dots,x_{2m+1})\bs e_j\] and \[\int_{x_1=0}^\infty \dots \int_{x_{2m+2}=0}^\infty\bs e_i\bs M^{m+1}_{-+}(\lambda,x_1,\dots,x_{2m+2})\bs e_j,\] were determined to be less than or equal to \(\widehat G(\frac{\gamma}{\gamma+\lambda})^{2m}\), in the proof of Lemma~\ref{lem: gkjljklgagjklagsjlk}. Hence (\ref{eqn :NNeeaefjn2}) is less than or equal to 
	% \begin{align}
	% 	&\sum_{i\in\calS_+\cup\calS_-} \bs e_k [\bs I - \bs T_{00}]^{-1}\bs T_{0i}\widehat G\left(\frac{\gamma}{\gamma+\lambda}\right)^{2m} F \nonumber
	% 	\\&\leq W\widehat G\left(\frac{\gamma}{\gamma+\lambda}\right)^{2m} F,
	% 	\label{eqn :NNeeaefjn23}
	% \end{align}
	% where 
	% \[W=\max_{k\in\calS_{0}}\sum_{i\in\calS_+\cup\calS_-} \bs e_k[\bs I - \bs T_{00}]^{-1}\bs T_{0i}.\]

	% % Now, for any row-vector of non-negative numbers \(\bs b\), since the elements of \(\bs H^{++}\) are non-negative and integrable, then 
	% % \[\bs b \int_{x_{2m+1}=0}^\infty \bs H^{++}(\lambda, x_{2m+1})\wrt x_{2m+1} \bs e_j\tr{} \leq \bs b \bs e_j\tr{} \widehat G\leq \bs b \bs e \widehat G.\]
	% % Observing that  
	% % \[\bs e_i \left[\prod_{r=1}^m\int_{x_{2r-1}=0}^\infty \bs H^{+-}(\lambda,x_{2r-1})\nonumber
	% % \int_{x_{2r}=0}^\infty \bs H^{-+}(\lambda,x_{2r}) \right]
	% % \wrt x_{2m}\dots \wrt x_1\] 
	% % is a vector of row-vector non-negative numbers, then (\ref{eqn :NNeeaefjn}) is less than or equal to 
	% % \begin{align}
	% % 	&\bs e_i \left[\prod_{r=1}^m\int_{x_{2r-1}=0}^\infty \bs H^{+-}(\lambda,x_{2r-1})
	% % 	\int_{x_{2r}=0}^\infty \bs H^{-+}(\lambda,x_{2r}) \right]
	% % 	\bs e\wrt x_{2m}\dots \wrt x_1\widehat G F\label{eqn :NNeeaefjn12}
	% % \end{align}

	% % The stochastic interpretation of the \(i\)th element of the vector \(\bs H^{+-}(\lambda,x)\bs e\) is that it is the probability density of an up-down transition at the time when the in-out fluid has increased by \(\wrt x\) and before an exponential random variable with rate \(\lambda\) occurs, given the phase is initially \(i\). There may be multiple changes of phase within \(\mathcal S_+\cup\calS_{+0}\) before the first up-down transition. The first change of phase occurs at rate (with respect to the in-out level) \(-T_{ii}/|c_i|\) and this is the lowest in-out fluid level at which it may be possible to see an up-down transition. Consider a uniformised version of the in-out fluid process with uniformisation parameter \(\gamma = \max\limits_{i\in\mathcal S_+\cup\calS_-}-T_{ii}/|c_i|\). Then the first event of the phase process of the uniformised version of the in-out fluid process occurs at rate \(\gamma\) and occurs at, or before, the first change of phase of the uniformised process. Therefore, the first uniformisation event occurs at, or before, the first up-down transition of the uniformised version of the in-out process. Hence, the first uniformisation event occurs at, or before, the first up-down transition of the original process (since they are versions of each other). This gives the bound \(\bs H^{+-}(\lambda,x)\bs e\leq \gamma e^{-(\lambda + \gamma)x}\bs e\) where the inequality is understood elementwise. Similarly, for \(\bs H^{-+}(\lambda,x)\bs e\leq \gamma e^{-(\lambda + \gamma)x}\bs e\).
	
	% % From the stochastic interpretation above, (\ref{eqn :NNeeaefjn12}) is less than or equal to 
	% % \begin{align}
	% % %
	% % &\bs e_i \bs H^{+-}(\lambda,x_1) \wrt x_1 \int_{x_2=0}^\infty \bs H^{-+}(\lambda,x_2) \bs e \wrt x_2  
	% % 			\hdots \int_{x_{2m}=0}^\infty \gamma e^{(-\gamma-\lambda)x_{2m}}\wrt x_{2m}\widehat GF\nonumber
	% % %
	% % \\&\leq \int_{x_1=0}^\infty \gamma e^{(-\gamma-\lambda)x_1}  \wrt x_1 \int_{x_2=0}^\infty \gamma e^{(-\gamma-\lambda)x_2}  \wrt x_2  
	% % 			\hdots \int_{x_{2m}=0}^\infty \gamma e^{(-\gamma-\lambda)x_{2m}}\wrt x_{2m}\widehat GF \nonumber
	% % %
	% % \\&= \left(\cfrac{\gamma}{\gamma+\lambda}\right)^{2m}\widehat GF.\label{eqn: bound ggggaaaa}
	% % \end{align}
	% Hence,  
	% \begin{align}
	% 	&\sum_{m=M+1}^\infty \int_{x\in\calD_{\ell_0}} \widehat f^{\ell_0}_{m,-0,+}(\lambda)(x,j;x_0,i)|\psi(x)|\wrt x \nonumber
	% 	\\&\leq  F\widehat GW  \sum_{m=M+1}^\infty \left(\frac{\gamma}{\gamma+\lambda}\right)^{2m}\nonumber
	% 	\\&\leq F\widehat GW \left(\frac{\gamma}{\gamma+\lambda}\right)^{2M+2}\left(1-\left(\frac{\gamma}{\gamma+\lambda}\right)^2\right)^{-1}.
	% \end{align}
	
	% Now consider \(\widehat\mu_{m,-0,+}^{\ell_0}(\lambda)( \wrt x,j;x_0,i) \) which given by 
	% \begin{align*}
	% 	\widehat \mu_{m,-0,+}^{\ell_0}(\lambda)(\wrt x,j;x_0,i) \nonumber 
	% 	&:= \sum_{r\in\{+,-\}}\sum_{k\in\calS_r}\bs e_i\vligne{\lambda \bs I - \bs T_{00}}^{-1}\bs T_{0k}\widehat \mu_{m+1(r\neq +),r,+}^{\ell_0}(\lambda)(\wrt x,j;x_0,k).
	% \end{align*}
	% which is a linear combination of \(\widehat\mu_{m+1(r\neq +),r,+}^{\ell_0}(\lambda)( \wrt x,j;x_0,k)\) for \(k\in\calS_r\), \(r\in\{+,-\}\). From the proof of Lemma~\ref{lem: gkjljklgagjklagsjlk}, \(\widehat\mu_{m+1(r\neq +),r,+}^{\ell_0}(\lambda)( \wrt x,j;x_0,k)\) is bounded by \(G\Delta F(\frac{\gamma}{\gamma+\lambda})^{2m}\). Hence,
	% \begin{align}
	% 	&\sum_{m=M+1}^\infty \int_{x\in\calD_{\ell_0}} \widehat \mu^{\ell_0}_{m,-0,+}(\lambda)(\wrt x,j;x_0,k) |\psi(x)| \nonumber
	% 	\\&\leq G W\sum_{m=M+1}^\infty\left(\frac{\gamma}{\gamma+\lambda}\right)^{2m}\int_{x\in\calD_{\ell_0}}\psi(x)\wrt x
	% 	\\&\leq GW\Delta F\left(\cfrac{\gamma}{\gamma+\lambda}\right)^{2M+2} \left(1-\left(\cfrac{\gamma}{\gamma+\lambda}\right)^2\right)^{-1}.
	% \end{align}
        
	% Analogous arguments show the same bounds for any \(i\in\calS_{q}\), \(j\in\calS_r\cup\calS_{r0}\), where \(q\in\{+0,-0\}\), \(r\in\{+,-\}\). 
\end{proof}


The last thing we need to prove is convergence at the first change of level. Since the result in Corollary~\ref{cor: aaaaa} is pointwise in \(x\), choosing the closing operator as \(e^{\bs Sx}\bs s\) and setting \(x=0\), then we get convergence at the first change of level, on the event that are \(m>0\) up-down or down-up transitions. The only things that remain are to show convergence at the first change of level on the event that there is no up-down or down-up transitions, and a domination condition so that we may sum over the number of up-down and down-up transitions ad prove convergence at the time of the first orbit restart epoch (analogous to the domination condition in the proof of Lemma~\ref{cor: aln222}). Regarding the former, we have the following lemma.

 \begin{lem} \label{lem:tttttt}
 	Let \(g\) satisfy the Assumptions~\ref{asu: g} and \(x_0\in(2\varepsilon,\Delta-\varepsilon)\). Then
 	\begin{align}
 		\left|\int_{x=0}^\infty \bs k(x_0)\bs De^{\bs Sx}g(x)\bs s\wrt x - g(x_0)\right| \leq \cfrac{\var(Z)/\varepsilon^2}{1-\var(Z)/(\Delta-x_0)^2}4G + 3L\varepsilon+6G\cfrac{\var(Z)}{\varepsilon^2}.
 	\end{align}
 \end{lem}
 \begin{proof}
 	First rewrite the left-hand side as 
 	\begin{align}
 %		\left|\int_{x=0}^\infty \bs k(x_0)\bs De^{\bs Sx}g(x)\bs s\wrt x - g(x_0)\right| 
 		\left|\int_{x=0}^\infty \bs k(x_0)\bs De^{\bs Sx}(g(x)-g(x_0))\bs s\wrt x \right|
 		%
 		&\leq \int_{x=0}^\infty \bs k(x_0)\bs De^{\bs Sx}|g(x)-g(x_0)|\bs s\wrt x.
 	\end{align}
 	Substituting in the expression for \(\bs D\) gives,
 	\begin{align}
 		&\int_{x=0}^\infty \bs k(x_0)\int_{u=0}^\infty e^{\bs Su}\bs s \cfrac{\bs \alpha e^{\bs Su}}{\bs \alpha e^{\bs Su}\bs e} \wrt u e^{\bs Sx}|g(x)-g(x_0)|\bs s\wrt x  \nonumber 
 %		\\&= \int_{x=0}^\infty \bs k(x_0)\int_{u=0}^\infty e^{\bs Su}\bs s \cfrac{\bs \alpha e^{\bs Su}}{\bs \alpha e^{\bs Su}\bs e} \wrt u e^{\bs Sx}|g(x)-g(x_0)|\bs s\wrt x 
 		%
 		\\&= \int_{x=0}^\infty \bs k(x_0)\int_{u=0}^{\Delta-\varepsilon} e^{\bs Su}\bs s \cfrac{\bs \alpha e^{\bs Su}}{\bs \alpha e^{\bs Su}\bs e} \wrt u e^{\bs Sx}|g(x)-g(x_0)|\bs s\wrt x \nonumber
 		\\&\quad {} + \int_{x=0}^\infty \bs k(x_0)\int_{u=\Delta-\varepsilon}^\infty e^{\bs Su}\bs s \cfrac{\bs \alpha e^{\bs Su}}{\bs \alpha e^{\bs Su}\bs e} \wrt u e^{\bs Sx}|g(x)-g(x_0)|\bs s\wrt x. \label{eqn: 2nd here}
 	\end{align}
 	Since \(g\) is bounded, the second term is less than or equal to 
 	\begin{align}
 		\int_{x=0}^\infty \bs k(x_0)\int_{u=\Delta-\varepsilon}^\infty e^{\bs Su}\bs s \cfrac{\bs \alpha e^{\bs Su}}{\bs \alpha e^{\bs Su}\bs e} \wrt u e^{\bs Sx}\bs s\wrt x 2G \nonumber
 		&= \bs k(x_0)\int_{u=\Delta-\varepsilon}^\infty e^{\bs Su}\bs s \cfrac{\bs \alpha e^{\bs Su}}{\bs \alpha e^{\bs Su}\bs e} \wrt u \bs e 2G \nonumber
 		\\&= \bs k(x_0)\int_{u=\Delta-\varepsilon}^\infty e^{\bs Su}\bs s \wrt u2G \nonumber
 		\\&= \cfrac{\mathbb P(Z\geq x_0+\Delta-\varepsilon)}{\mathbb P(Z>x_0)}2G. \label{eqn" dgkjlwerhv}
 	\end{align}
 	For \(x_0\in(2\varepsilon,\Delta-\varepsilon),\) then (\ref{eqn" dgkjlwerhv}) is less than or equal to 
 	\[\cfrac{\var(Z)/\varepsilon^2}{1-\var(Z)/(\Delta-x_0)^2}2G.\]
	
 	As for the first term in (\ref{eqn: 2nd here}), it can be written as 
 	\begin{align}
 		&\int_{x=0}^\infty \bs k(x_0)\int_{u=\Delta-x_0-\varepsilon}^{u=\Delta-x_0+\varepsilon} e^{\bs Su}\bs s \cfrac{\bs \alpha e^{\bs Su}}{\bs \alpha e^{\bs Su}\bs e} \wrt u e^{\bs Sx}|g(x) - g(x_0)|\bs s\wrt x \nonumber
 		%
 		\\&\qquad{}+\int_{x=0}^\infty \bs k(x_0)\int_{u=0}^{\Delta-x_0-\varepsilon} e^{\bs Su}\bs s \cfrac{\bs \alpha e^{\bs Su}}{\bs \alpha e^{\bs Su}\bs e} \wrt u e^{\bs Sx}|g(x) - g(x_0)|\bs s\wrt x \nonumber
 		%
 		\\&\qquad{}+\int_{x=0}^\infty \bs k(x_0)\int_{u=\Delta-x_0+\varepsilon}^{\Delta-\varepsilon} e^{\bs Su}\bs s \cfrac{\bs \alpha e^{\bs Su}}{\bs \alpha e^{\bs Su}\bs e} \wrt u e^{\bs Sx}|g(x) - g(x_0)|\bs s\wrt x. \label{eqn: 201}
 	\end{align}
 	Since \(g\) is bounded, then the last two terms in (\ref{eqn: 201}) are 
 	\begin{align}
 		&2G\Bigg(\int_{x=0}^\infty \bs k(x_0)\int_{u=0}^{\Delta-x_0-\varepsilon} e^{\bs Su}\bs s \cfrac{\bs \alpha e^{\bs Su}}{\bs \alpha e^{\bs Su}\bs e} \wrt u e^{\bs Sx}\bs s\wrt x \nonumber
 		%
 		\\&\qquad{}+\int_{x=0}^\infty \bs k(x_0)\int_{u=\Delta-x_0+\varepsilon}^{\Delta-\varepsilon} e^{\bs Su}\bs s \cfrac{\bs \alpha e^{\bs Su}}{\bs \alpha e^{\bs Su}\bs e} \wrt u e^{\bs Sx}\bs s\wrt x\Bigg) \nonumber 
 		%
 		\\&=2G\Bigg(\bs k(x_0)\int_{u=0}^{\Delta-x_0-\varepsilon} e^{\bs Su}\bs s \cfrac{\bs \alpha e^{\bs Su}}{\bs \alpha e^{\bs Su}\bs e}  \bs e  \wrt u\nonumber
 		%
 		+ \bs k(x_0)\int_{u=\Delta-x_0+\varepsilon}^{\Delta-\varepsilon} e^{\bs Su}\bs s \cfrac{\bs \alpha e^{\bs Su}}{\bs \alpha e^{\bs Su}\bs e} \bs e\wrt u \Bigg) \nonumber
 		%
 		\\&=2G \cfrac{\mathbb P(Z>x_0, Z\notin(\Delta-\varepsilon, \Delta+\varepsilon))}{\mathbb P(Z>x_0)}  \nonumber 
 		%
 		\\&\leq 2G\cfrac{\var(Z)/\varepsilon^2}{1-\var(Z)/(\Delta-x_0)^2}.
 	\end{align}
 	Exchanging the order of integration for the first term in (\ref{eqn: 201}) (justified by the Fubini-Tonelli Theorem)
 	\begin{align}
 		\int_{u=\Delta-x_0-\varepsilon}^{u=\Delta-x_0+\varepsilon} \bs k(x_0) e^{\bs Su}\bs s\int_{x=0}^\infty \cfrac{\bs \alpha e^{\bs Su}}{\bs \alpha e^{\bs Su}\bs e} e^{\bs Sx}|g(x)-g(x_0)|\bs s\wrt x \wrt u, \label{eqn: fdk13897}
 	\end{align}
 	from which we see that we can apply Corollary~\ref{cor: cond bnd 2} with \(v=0\) to the integral over \(x\), implying that (\ref{eqn: fdk13897}) is less than or equal to
 	\begin{align}
 		\int_{u=\Delta-x_0-\varepsilon}^{u=\Delta-x_0+\varepsilon} \bs k(x_0) e^{\bs Su}\bs s\left(|g(\Delta-u)-g(x_0)|+r_3(u)\right)\wrt u. \label{eqn: sdkagh lkhvasfv}
 	\end{align}
 	Noting that \(\sup |r_3(u)|\leq|r_2|\) for \(u\leq \Delta-\varepsilon\), and since \(g\) is Lipschitz, then (\ref{eqn: sdkagh lkhvasfv}) is less than or equal to 
 	\begin{align}
 		\int_{u=\Delta-x_0-\varepsilon}^{u=\Delta-x_0+\varepsilon} \bs k(x_0) e^{\bs Su}\bs s\left(L\varepsilon+|r_2|\right)\wrt u \leq L\varepsilon+|r_2|. \label{eqn: sdkagh lsfv}
 	\end{align}
 	Putting all the bounds together proves the result. 
 \end{proof}

We conclude this appendix but stating the convergence results formally. 
\begin{lem} \label{lem:vn42a}
	For all \(x\in\calD_{\ell_0,j}\), \(x_0\in\calD_{\ell_0,i}\), \(i\in\calS_{0+\cup\calS_{0-}},\,j\in\calS\), \(\ell_0\in\mathcal K\), \(\lambda > 0\),  
	\begin{align}
		&\left|\int_{x\in\calD_{\ell_0}}\widehat f^{\ell_0,(p)}(\lambda)(x,j;x_0,i)\psi(x) \wrt x - \int_{x\in\calD_{\ell_0}}\widehat \mu^{\ell_0}(\lambda)(\wrt x,j; x_0,i)\psi(x) \right|\to 0  \label{eqn: akhv2a}
	\end{align}
	as \(p\to\infty\). 
\end{lem}
\begin{proof}
	The convergence in Theorem~\ref{thm: a thm2!}, the domination condition in Lemma~\ref{lem: gkjljklgagjklagsjlk2}, and the Dominated Convergence Theorem can be used to obtain the result. 
\end{proof}
	
\begin{cor}\label{cor: aln2222a} Recall \(\bs y_{0}^{(p)} = (\ell_0, \bs a_{\ell_0,j}^{(p)}(x_0), i)\). For \(\ell_0\in\mathcal K\) \(x_0\in\mathcal D_{\ell_0,i}\), \(i\in\mathcal S_{+0}\cup\calS_{0-},\) \(j\in\calS_+\cup\calS_-\), 
	\begin{align}
		&\mathbb P(L^{(p)}(\tau_1^{(p)}) = \ell(\ell_0,j), \varphi(\tau_1^{(p)}) = j, \tau_{1}^{(p)}\leq E^\lambda 
            	 \mid \bs Y^{(p)}(0) = \bs y_0^{(p)}) \nonumber
	 	%
		\\&\to \mathbb P(\bs X(\tau_1^X) = (y_{\ell(\ell_0,j)+1(j\in\calS_-)}, j), \tau_{1}^X\leq E^\lambda 
            	 \mid \bs X(0) = (x_0,i))\label{eqn: 14212a}
	\end{align}
	where \(\ell(\ell_0,j)\) can take values
	\begin{align*}
		\ell(\ell_0,j) = \begin{cases}
			\ell_0-1, &\mbox{ if } \ell_0\in\{0,1,\dots,K+1\},\, j\in\calS_-\\
			%
			\ell_0, & \mbox{ if } \ell_0 = 0, j\in\mathcal S_+, \mbox{ or }\ell_0 = K, j\in\mathcal S_-,\\
			%
			\ell_0+1, & \mbox{ if } \ell_0\in\{-1,0,1,\dots,K\},\, j\in\calS_+.
		\end{cases}
	\end{align*}
\end{cor}
\begin{proof}
	An analogue of the domination condition required in the proof of Lemma~\ref{cor: aln222} can be established by extending Lemma~\ref{lem: gkjljklgagjklagsjlk2} in the same way we extended Lemma~\ref{lem: gkjljklgagjklagsjlk} in the proof of Lemma~\ref{cor: aln222}. 

	With the aforementioned domination condition, the point-wise convergence in Corollary~\ref{cor: aaaaa}, the convergence in Lemma~\ref{lem:tttttt}, and the Dominated Convergence Theorem we can prove the result. 
\end{proof}

%
%\begin{lem}
%	Let \(f:[0,\Delta)\to \mathbb R\) be any bounded, Lipschitz continuous function with \(|\psi(x)|\leq F\). Then for \(u\leq \Delta-\varepsilon\), \(v\in[0,\Delta)\)
%	\begin{align}
%		&\int_{t=0}^\infty e^{-\lambda t}\mathbb E[f(\bar X(t))1(\varphi(t)=j, L(t)=\ell_0, L(s)=\ell_0, \varphi(s)\in\calS_+\cup\calS_{+0},s\in[0,t])\mid L(0)=\ell_0, \nonumber
%		\\&\qquad{} \bs A(0)=\bs a_{\ell_0,i}(x_0), \varphi(0)=i]\wrt t \nonumber
%		\\&= \int_{t=0}^\infty e^{-\lambda t}\mathbb E[\psi(X(t))1(\varphi(t)=j,X(s)\in\calD_{\ell_0}, \varphi(s)\in\calS_+\cup\calS_{+0},s\in[0,t])\mid X(0)=x_0, \varphi(0)=i]\wrt t \nonumber
%		\\&\qquad {} + r_{11}^{(p)}
%	\end{align}
%	where 
%	\[|r_{11}^{(p)}| \leq F R_{{\bs v},2}+ GF\varepsilon.\]
%\end{lem}
%\begin{proof}
%	Assume, without loss of generality \(\ell_0=0\) so \(\calD_{\ell_0}=[0,\Delta]\). First observe that 
%	\begin{align}
%		&\int_{t=0}^\infty e^{-\lambda t}\mathbb E[f(\bar X(t))1(\varphi(t)=j, L(t)=\ell_0, L(s)=\ell_0, \varphi(s)\in\calS_+\cup\calS_{+0},s\in[0,t])\mid L(0)=\ell_0, \nonumber
%		\\&\qquad{} \bs A(0)=\bs a_{\ell_0,i}(x_0), \varphi(0)=i]\wrt t \nonumber 
%		\\&= \int_{x_1=0}^\infty \int_{x=[0,\Delta)} \cfrac{\bs \alpha e^{\bs S(x_1+x_0+x)}\bs s}{\bs \alpha e^{\bs S x_0} \bs e}h_{ij}^{++}(\lambda, x_1)f(\Delta-x)\wrt x\wrt x_1
%	\end{align}
%	and 
%	\begin{align}
%		&\int_{t=0}^\infty e^{-\lambda t}\mathbb E[\psi(X(t))1(\varphi(t)=j,X(s)\in\calD_{\ell_0}, \varphi(s)\in\calS_+\cup\calS_{+0},s\in[0,t])\mid X(0)=x_0, \varphi(0)=i]\wrt t \nonumber
%		\\&=  \int_{x=0}^{\Delta-x_0} h_{ij}^{++}(\lambda, x)\psi(X_0 + x)\wrt x
%	\end{align}
%	In Property~\ref{properties: 2} take \(g(x) = h_{ij}^{--}(\lambda, x)\), which states 
%	\begin{align}
%		&\int_{x_1=0}^\infty \int_{x=0}^\Delta\cfrac{\bs \alpha e^{\bs S(x_1+x_0+x)}\bs s}{\bs \alpha e^{\bs S x_0} \bs e}h_{ij}^{++}(\lambda, x_1)f(\Delta-x)\wrt x\wrt x_1 \nonumber
%		\\& = \int_{x=0}^{\Delta}h_{ij}^{++}(\lambda,\Delta - x_0 - x)1(x+x_0\leq \Delta-\varepsilon)f(\Delta - x)\wrt x + \int_{x=0}^\Delta r_{\bs v}(x_0,x)\psi(x)\wrt x. 
%	\end{align}
%	The second term is less than or equal to 
%	\[F\int_{x=0}^\Delta r_{\bs v}(x_0,x)\wrt x = FR_{{\bs v},1}.\]
%	The first term is 
%	\[\int_{x=0}^{\Delta-x_0}h_{ij}^{++}(\lambda,\Delta - x_0 - x)f(\Delta - x)\wrt x - \int_{x=\Delta-x_0-\varepsilon}^{\Delta-x_0}h_{ij}^{++}(\lambda,\Delta - x_0 - x)f(\Delta - x)\wrt x.\]
%	Now 
%	\[\int_{x=\Delta-x_0-\varepsilon}^{\Delta-x_0}h_{ij}^{++}(\lambda,\Delta - x_0 - x)f(\Delta - x)\wrt x\leq GF\varepsilon\]
%	and 
%	\[\int_{x=0}^{\Delta-x_0}h_{ij}^{++}(\lambda,\Delta - x_0 - x)f(\Delta - x)\wrt x=\int_{x=0}^{\Delta-x_0}h_{ij}^{++}(\lambda,x)\psi(x_0+x)\wrt x.\]
%\end{proof}
%
%
%\begin{cor}\label{cor: Dcoajc2222}
%	Let \(\psi:\calD_{\ell_0}\to \mathbb R\) be bounded, \(|\psi(x)|\leq F\), and Lipschitz continuous. Then, for \(x_0\in(y_{\ell_0}, y_{\ell+1}),\) \(k\in\calS_{0+}\), \(i\in\calS_-\), \(j\in\calS_-\cup\calS_{-0}\), there exists \(r_{10}^{(p)}\to 0\) as \(p \to \infty\), 
%	\begin{align}
%		\left|\int_{x\in\calD_{\ell_0}} \widehat f_{0,-0,-}^{\ell_0}(x,i,j;j,x_0)\psi(x)\wrt x  \to \int_{x\in\calD_{\ell_0}} \widehat \mu_{0,-0,-}^{\ell_0}(x,i,j;k,x_0)\psi(x)\wrt x\right|\leq r_{10}^{(p)}.\label{eqn:xvasfv}
%	\end{align}
%	Similarly, for \(k\in\calS_{0-}\), \(i\in\calS_+\), \(j\in\calS_+\cup\calS_{+0}\)
%	\begin{align}
%		\left|\int_{x\in\calD_{\ell_0}} \widehat f_{0,+0,+}^{\ell_0}(x,i,j;j,x_0)\psi(x)\wrt x  - \int_{x\in\calD_{\ell_0}} \widehat \mu_{0,+0,+}^{\ell_0}(x,i,j;k,x_0)\psi(x)\wrt x\right|\leq r_{10}^{(p)}.\label{eqn:!124msfvcds}
%	\end{align}
%\end{cor}
%\begin{proof}[Proof of Corollary~\ref{cor: Dcoajc}]
%	We show the result for (\ref{eqn:xs}) only, with the result for (\ref{eqn:!124mcds}) following analogously. 
%	
%	Observe that, for \(k\in\calS_{0+}\), \(i\in\calS_-\), \(j\in\calS_-\cup\calS_{-0}\), 
%	\begin{align}
%		&\int_{x\in\calD_{\ell_0}} \widehat f_{0,-0,-}^{\ell_0,(p)}(x,i,j;j,x_0)\psi(x)\wrt x \nonumber 
%		%
%		\\&{}= \vligne{\lambda \bs I - \bs T_{00}}^{-1}_{ki}\int_{x\in\calD_{\ell_0}}\int_{x_1=0}^\infty \bs k(x_0) \bs D e^{\bs S x_1} {\bs v}(x)h_{ij}^{--}(\lambda,x_1)\wrt x_1\psi(x)\wrt x,
%	\end{align}
%	and 
%	\begin{align}
%		\int_{x\in\calD_{\ell_0}} \widehat \mu_{0,-0,-}^{\ell_0}(x,i,j;k,x_0)\psi(x)\wrt x \nonumber  = \vligne{\lambda \bs I - \bs T_{00}}^{-1}_{ki}\int_{x=0}^{x_0} h_{ij}^{--}(\lambda,x_0-x)\psi(x)\wrt x 
%	\end{align}
%	These terms appear in the proof of Corollary~\ref{lem: ppp} and using arguments applied there we can show
%	\begin{align}
%		\int_{x\in\calD_{\ell_0}} \widehat f_{0,-0,-}^{\ell_0,(p)}(x,i,j;j,x_0)\psi(x)\wrt x  \to \int_{x\in\calD_{\ell_0}} \widehat \mu_{0,-0,-}^{\ell_0}(x,i,j;k,x_0)\psi(x)\wrt x ,\label{eqn:xssgwg}
%	\end{align}
%	and therefore there exists a bound \(r_{10}^{(p)}\) such that 
%	\[\left|\int_{x\in\calD_{\ell_0}} \widehat f_{0,-0,-}^{\ell_0,(p)}(x,i,j;j,x_0)\psi(x)\wrt x  \to \int_{x\in\calD_{\ell_0}} \widehat \mu_{0,-0,-}^{\ell_0}(x,i,j;k,x_0)\psi(x)\wrt x\right|\leq r_{10}^{(p)} ,\]
%	and \(r_{10}^{(p)}\to 0\) as \(p\to\infty\). 
%\end{proof}
%
%\begin{cor}\label{corL Tta} For \(i\in\mathcal S_-,j\in\mathcal S_-\cup\calS_{-0}\), \(x_0\in[0,\Delta)\), 
%	\begin{align}
%		&\int_{t=0}^\infty e^{-\lambda t}\mathbb P(\bar X^{(p)}(t) \in \cdot, \varphi(s)\in\mathcal S_-\cup\mathcal S_{-0}, s\in[0,t], \varphi(t)=j\mid \bs A^{(p)}(0)=\bs k^{(p)}(x_0)\bs D^{(p)}, \varphi(0)=i) \wrt t \nonumber 
%		%
%		\\& \to \int_{t=0}^\infty e^{-\lambda t}\mathbb P( X(t) \in \cdot, \varphi(s)\in\mathcal S_-\cup\mathcal S_{-0}, s\in[0,t], \varphi(t)=j\mid X(0)= x_0, \varphi(0)=i) \wrt t \label{eqn: ffaA}
%	\end{align}
%	as \(p\to\infty\).
%\end{cor}
%\begin{proof}
%	From the convergence of Laplace transforms in Lemma~\ref{lem: ppp} and the Continuity Theorem~\ref{thm: ext cont thm} then 
%	\begin{align}
%		&  \mathbb E[f(\bar X^{(p)}(t))1(\varphi(s)\in\mathcal S_-\cup\mathcal S_{-0}, s\in[0,t], \varphi(t)=j) \mid \bs A^{(p)}(0)=\bs k^{(p)}(x_0)\bs D^{(p)}, \varphi(0)=i]  \nonumber 
%		\\& \to \mathbb E[\psi(X(t)) 1(\varphi(s)\in\mathcal S_-\cup\mathcal S_{-0}, s\in[0,t], \varphi(t)=j) \mid X(0)=x_0, \varphi(0)=i],
%	\end{align}
%	as \(p\to\infty\), for every bounded, Lipschitz function \(f\). By the Portmanteau Theorem, then 
%	\begin{align}
%		& \mathbb P(\bar X^{(p)}(t) \in \cdot, \varphi(s)\in\mathcal S_-\cup\mathcal S_{-0}, s\in[0,t], \varphi(t)=j\mid \bs A^{(p)}(0)=\bs k^{(p)}(x_0)\bs D^{(p)}, \varphi(0)=i)   \nonumber 
%		%
%		\\& \to \mathbb P( X(t) \in \cdot, \varphi(s)\in\mathcal S_-\cup\mathcal S_{-0}, s\in[0,t], \varphi(t)=j\mid X(0)= x_0, \varphi(0)=i) 
%	\end{align}
%	weakly as \(p\to\infty\). In addition 
%	\begin{align}
%		&\int_{t=0}^\infty e^{-\lambda t}\mathbb P(\bar X^{(p)}(t) \in \cdot, \varphi(s)\in\mathcal S_-\cup\mathcal S_{-0}, s\in[0,t], \varphi(t)=j\mid \bs A^{(p)}(0)=\bs k^{(p)}(x_0)\bs D^{(p)}, \varphi(0)=i) \wrt t\nonumber 
%		\\& \leq \int_{x=0}^\infty  h_{ij}^{--}(\lambda, x)\wrt x ,
%	\end{align}
%	which is bounded as a function of \(p\). Hence we can apply the Continuity Theorem~\ref{thm: ext cont thm} again and claim that 
%	\begin{align}
%		&\int_{t=0}^\infty e^{-\lambda t}\mathbb P(\bar X^{(p)}(t) \in \cdot, \varphi(s)\in\mathcal S_-\cup\mathcal S_{-0}, s\in[0,t], \varphi(t)=j\mid \bs A^{(p)}(0)=\bs k^{(p)}(x_0)\bs D^{(p)}, \varphi(0)=i) \wrt t \nonumber 
%		%
%		\\& \to \int_{t=0}^\infty e^{-\lambda t}\mathbb P( X(t) \in \cdot, \varphi(s)\in\mathcal S_-\cup\mathcal S_{-0}, s\in[0,t], \varphi(t)=j\mid X(0)= x_0, \varphi(0)=i) \wrt t
%	\end{align}
%	as \(p\to\infty,\) as required. 
%\end{proof}
%
%
%
%
%
%
%
