%!TEX root = ../thesis.tex
\chapter{Approximating fluid queues with the discontinuous Galerkin method \label{ch:galerkin}} 

In this chapter we introduce the discontinuous Galerkin (DG) method as applied to fluid queues and to approximate the operator-analytic expressions for fluid-fluid queues in \cite{bo2014}. 
\begin{center}
	\begin{minipage}{0.8\textwidth}
		\textit{Apart from Section~\ref{sec: slope limiting}, this chapter has been taken from Section~4 and Appendix~1 of \cite{blnos2022} with only minor changes, such as notations, so that this chapter is consistent with the rest of the thesis. I am a co-author of the paper \cite{blnos2022}. %The conceptualisation of \cite{blnos2022} was originally by Vikram Sunkara, Nigel Bean and Giang Nguyen, and the original coding was done by Vikram Sunkara. I made significant contributions to Section~3 of the paper, expressing the operator-theoretic expressions to use the same partition as the approximation scheme. I contributed Sections~4.4 and 5.1. I extended the numerical experiments in Section~6 to higher orders and made all the plots in Section~6. Appendix~A is also my original work. I did a significant proportion of the writing of the manuscript and addressed the reviewers comments and also developed code for the numerical experiments.
		}
	\end{minipage}
	\end{center}
% The DG method for B 
    % direct copy paste from the paper
% Problems with DG, non-neg/oscillations
    % solutions: filtering and limiting
% Approximating R: projection, interpolation, cell averages (eluding to QBD-RAP and unif method)
% Approximating Psi (and the stationary distribution, but this isnt really necessary)
    % constructing D
    % solving the matrix-Riccati equation, iteration/algorithm. 

% Appendix: the toy model. 
% Appendix: some properties of B?

% \section{Discontinuous Galerkin Approximation of the Generator of a Fluid Queue}
	% \label{sec:DG}
Discontinuous Galerkin (DG) methods can be used to approximate the solutions to systems of partial differential equations (PDEs). For a more thorough description of these methods see \citep{nodalDGBook}. The domain of approximation is partitioned into intervals, referred to individually as \textit{cells} and collectively as a \textit{mesh}. On each cell, we have a finite element approximation, which constructs a finite-dimensional smooth Sobolev space using piecewise-polynomial basis functions, and then projects the partial differential equations onto this space. This projection leads to a new system of equations, referred to as the \textit{weak form} of the original system of PDEs. Next, we must approximate the \textit{flux operator} which moves probability from one cell to another, in a manner similar to the underlying principle of a finite-volume approximation. This method conserves probability, and can handle discontinuities, such as jumps and point masses. Here we construct the DG approximation to the matrix of operators \(\mathbb B\) which we use later to construct a DG approximation to \(\mathbb D(s)\) then \(\mathbb\Psi(s)\), and ultimately the stationary distribution of an SFFM. 

\section{The Partial Differential Equation}
We start by introducing the PDE from which we will extract the approximation to the generator \(\mathbb B\). 

Let $f_i(x,t)$ be the joint density of $\{(X_t, \varphi_t)\}$: 
	% 
	\begin{align*} 
	    f_i(x,t) := \cfrac{\partial}{\partial x} 	\mathbb{P}\left(X_t \leq x, \varphi_t = i\right),\quad 0<x<b,\, i \in\calS,
	\end{align*} 
which satisfies the system of partial differential equations 
%
\begin{align*}
\cfrac{\partial}{\partial t} f_i(x,t)& = \sum_{j\in \mathcal{S}}  f_j(x,t)T_{ji} - c_i \cfrac{\partial}{\partial x} f_i(x,t), \quad 0<x<b,\, i\in\mathcal S,
\end{align*}
% 
subject to suitable boundary conditions. In matrix form, 
\begin{align}
\cfrac{\partial}{\partial t} \boldsymbol f(x,t) &= \boldsymbol f(x,t)\bs T -  \cfrac{\partial}{\partial x}\boldsymbol f(x,t)\bs C\label{eqn:pde_density_matrix}, 
\end{align}
where \(\boldsymbol f(x,t) = \left(f_i(x,t)\right)_{i\in\mathcal S}\) is a row-vector. 
% This system of PDEs is closely related to the generator \(\mathbb B\); \(\boldsymbol \mu(t)\) satisfies the operator differential equation
% \begin{align*}
% 	\cfrac{\wrt}{\wrt t}\boldsymbol \mu(t)(\wrt x)  = \boldsymbol \mu(t)\mathbb B(\wrt x) = \boldsymbol f(x,t)\bs T\wrt x - \cfrac{\partial}{\partial x}\boldsymbol f(x,t)\bs C\wrt x,
% \end{align*}
% on the interior of the space \([0,\infty)\). Thus, by approximating the operator on the right-hand side of Equation (\ref{eqn:pde_density_matrix}) we can approximate the infinitesimal operator \(\mathbb B\). The DG method does exactly this, by approximating the operator with a matrix.

\section{Cells, Test Functions, and Weak Formulation}
To begin with, consider an unbounded first fluid level \(\{\ddot X_t,t\geq0\}\), \(\ddot X_t\in(-\infty,\infty)\). We will eventually truncate this space so that we have a finite dimensional approximation; however, this requires a discussion on boundary conditions which we save for later. Let \(\mathcal D_k = [y_k,y_{k+1}],\, k\in\mathbb Z\) partition the domain \((-\infty,\infty)\). We call the \(\calD_k\) \textit{cells}. 

On each cell \(\calD_k\) we choose \(N_k\) linearly independent functions \(\{\phi^r_k\}_{r=1}^{N_k}\), compactly supported on \(\calD_k\) (i.e.~\(\phi^r_k(x)=0\) for \(x\notin\calD_k\)) to form a basis for the space \(W_k\), in which we formulate the approximation. Here, as is standard in DG methods \citep{nodalDGBook}, we take \(\{\phi^r_k\}_{r=1}^{N_k}\) to be the space of polynomials of degree \(N_k-1\). It is convenient in this work to take \(\{\phi^r_k\}_{r=1}^{N_k}\) as a basis of Lagrange interpolating polynomials defined by the Gauss-Lobatto quadrature points, since our approximations inherit nice properties from this \citep{nodalDGBook}. However, the constructions presented here are general, and any basis can be used. For the sake of illustration, the reader may think of \(\{\phi^r_k\}_{r=1}^{N_k}\) as the Lagrange polynomials. On each cell \(\mathcal D_k\) we approximate 
\[f_i(x,t)\approx u_{k,i}(x,t)=\sum\limits_{r=1}^{N_k}a_{k,i}^r(t)\phi^r_k(x),\] 
where \(a_{k,i}^r(t)\) are yet-to-be-determined time-dependent coefficients. We refer to \(u_{k,i}\) as the \textit{local} approximation on \(\calD_k\), while the \textit{global} approximation is given by \(\sum\limits_{k\in\mathbb Z}u_{k,i}\) on the whole domain. The whole approximation space is \(\bigoplus\limits_{k\in\mathbb Z} W_k\).

Let \(\mathcal N_k := \left\{1,\dots,N_k\right\},\, k \in \mathbb Z\). For \(k\in\mathbb Z,\) define \textit{local} row-vectors 
\[\boldsymbol \phi_k(x) = (\phi^r_k(x))_{r\in\mathcal N_k}, \quad \boldsymbol a_{k,i}(x) = (a_{k,i}^r(x))_{r\in\mathcal N_k},\,i\in\mathcal S.\]
%Also define \textit{global} row-vectors 
%\[\boldsymbol \phi(x) = (\boldsymbol\phi^k(x))_{k\in\{1,...,K\}},\quad \boldsymbol a_i(x) = (\boldsymbol a_{k,i}(x))_{k\in\{1,...,K\}},\,i\in\mathcal S.\]
Note that we will always use the letter \(r\) to index the basis function within each cell.

The DG method proceeds by first considering the \textit{weak-formulation} of the PDE, which is constructed from the \textit{strong-form} of the PDE, Equation (\ref{eqn:pde_density_matrix}). In general, to construct the weak-form we need a set of test functions, say \(W\). Now, take the strong form of the PDE, multiply it by some test function \(\psi(x)\in W\), integrate with respect to \(x\), and apply integration by parts to the derivative with respect to \(x\), to get 
\begin{align}
\begin{multlined}[t]
	\int_{x\in\mathbb R}\cfrac{\partial}{\partial t} f_j(x,t)\psi(x)\wrt x = \int_{x\in\mathbb R} \sum_{i\in\calS}f_i(x,t)T_{ij}\psi(x)\wrt x 
	%
	+  \int_{x\in\mathbb R} f_j(x,t)c_j\cfrac{\wrt}{\wrt x}\psi(x)\wrt x \\{}- [ f_j(x,t)c_j\psi(x)]_{x=-\infty}^{x=\infty}, \end{multlined}\label{eqn:weak form}
\end{align}
for \(j\in\calS\). It is common to choose \(W\) such that \(\psi(-\infty)=\psi(\infty)=0\), in which case the last term on the right is zero. Requiring (\ref{eqn:weak form}) to hold for every \(\psi\in W\) gives the weak-formulation of the PDE. For a sufficiently rich set of test functions \(W\) the weak and strong forms of the PDE are equivalent. Solutions to (\ref{eqn:weak form}) are known as \textit{weak} solutions and generalise the concept of a solution of the PDE. For example, this may allow discontinuities with respect to \(x\) in the solution -- something which is ill-defined for the strong form.

For the purpose of DG, we take the set of test functions to be \(W = \bigoplus\limits_{k\in\mathbb Z} W^k\), the same as the set of basis functions of our solution space. Proceeding as described above, the weak formulation is 
\begin{align*}
	\begin{multlined}[t]\int_{x\in\calD_k}\cfrac{\partial}{\partial t} f_j(x,t)\phi^r_k(x)\wrt x = \int_{x\in\calD_k}\sum_{i\in\calS} f_i(x,t)T_{ij}\phi^r_k(x)\wrt x  
	%
	+  \int_{x\in\calD_k} f_j(x,t)c_j\cfrac{\wrt}{\wrt x}\phi^r_k(x)\wrt x \\ {}- [f_j(x,t)c_j\phi^r_k(x)]_{x=y_k}^{x=y_{k+1}}, \end{multlined}
\end{align*}
since \(\phi^r_k\) is compactly supported on \(\calD_k\), for all \(j\in\mathcal S,\,r\in\mathcal N_k\), \(k\in\mathbb Z.\) Now, note that any function \(g(x)\) can be decomposed as \(g(x) = g^{W}(x)+g^\perp(x)\) where \(g^{W}\in W\) and \(g^\perp \in W^\perp\), and \(W^\perp\) is the orthogonal complement of \(W\). Since \(g^\perp\) is orthogonal to \(W\), \(\displaystyle\int_{x}g^\perp(x)\phi^r_k(x)\wrt x=0\) for \(r\in\mathcal N_k,\,k\in\mathbb Z\). Also, note that \(\cfrac{\wrt}{\wrt x}\phi^r_k(x)\in W\). Using this, we can write 
\begin{align*}
	\begin{multlined}[t]\int_{x\in\calD_k}\cfrac{\partial}{\partial t} \left(f_j^W(x,t)+f_j^\perp(x,t)\right)\phi^r_k(x)\wrt x 
	= \int_{x\in\calD_k}\sum_{i\in\calS} \left(f_i^W(x,t)+f_i^\perp(x,t)\right)T_{ij}\phi^r_k(x)\wrt x  
	%
	\\ {}
	+  \int_{x\in\calD_k} \left(f_j^W(x,t)+f_j^\perp(x,t)\right)c_j\cfrac{\wrt}{\wrt x}\phi^r_k(x)\wrt x - [f_j(x,t)c_j\phi^r_k(x)]_{x=y_k}^{x=y_{k+1}}, \end{multlined}
\end{align*}
	which is equivalent to
\begin{align}
	\begin{multlined}[t]\int_{x\in\calD_k}\cfrac{\partial}{\partial t} f_j^W(x,t)\phi^r_k(x)\wrt x = \int_{x\in\calD_k}\sum_{i\in\calS} f_i^W(x,t)T_{ij}\phi^r_k(x)\wrt x  
	%
	+  \int_{x\in\calD_k} f_j^W(x,t)c_j\cfrac{\wrt}{\wrt x}\phi^r_k(x)\wrt x \\{} - [f_j(x,t)c_j\phi^r_k(x)]_{x=y_k}^{x=y_{k+1}}. \end{multlined} \label{eqn:hash}
\end{align}
Now, \(f_j^W(x,t)\in W\) so, on \(\mathcal D_k\), it can be expressed as \(u_{k,j}(x,t):=\boldsymbol a_{k,j}(t) \boldsymbol \phi_k(x)\tr{}\), which we now substitute into (\ref{eqn:hash})
%\begin{align*}
%	\begin{multlined}[t]\int_{x\in\calD_k}\cfrac{\wrt}{\wrt t} \boldsymbol a_{k,j}(t) \boldsymbol \phi^k(x)\tr{}\phi^r_k(x)\wrt x = \int_{x\in\calD_k}\sum_{i\in\calS} \boldsymbol a_{k,i}(t) \boldsymbol \phi^k(x)\tr{}T_{ij}\phi^r_k(x)\wrt x  
%	%
%	\\ {}+  \int_{x\in\calD_k}\boldsymbol a_{k,j}(t) \boldsymbol \phi^k(x)\tr{}c_j\cfrac{\wrt}{\wrt x}\phi^r_k(x)\wrt x - c_j[f_j(x,t)\phi^r_k(x)]_{x=y_k}^{x=y_{k+1}}.\end{multlined}
%\end{align*}
and repeat this for all test functions \(\phi^r_k(x)\), \(r=1,...,N_k\), to get the following system of equations,
\begin{align}
	\begin{multlined}[t]\int_{x\in\calD_k}\cfrac{\wrt}{\wrt t} \boldsymbol a_{k,j}(t) \boldsymbol \phi_k(x)\tr{}\boldsymbol \phi_k(x)\wrt x = \int_{x\in\calD_k}\sum_{i\in\calS} \boldsymbol a_{k,i}(t) \boldsymbol \phi_k(x)\tr{}T_{ij}\boldsymbol\phi_k (x)\wrt x  
	%
	\\{}+  \int_{x\in\calD_k}\boldsymbol a_{k,j}(t) \boldsymbol \phi_k(x)\tr{}c_j\cfrac{\wrt}{\wrt x}\boldsymbol\phi_k(x)\wrt x - c_j[f_j(x,t) \boldsymbol\phi_k(x)]_{x=y_k}^{x=y_{k+1}},\quad k\in\mathbb Z. \end{multlined} \label{eqn:DG system}
\end{align}

\section{Mass, Stiffness, and Flux}
For \(k\in\mathbb Z\), define local \textit{mass} and \textit{stiffness} matrices \(\bs M_k\) and \(\bs G_k\) by 
\[\bs M_k := \int_{x\in\calD_k}\boldsymbol \phi_k(x)\tr{}\boldsymbol \phi_k(x)\wrt x,\quad \bs G_k := \int_{x\in\calD_k}\boldsymbol \phi_k(x)\tr{}\cfrac{\wrt}{\wrt x} \boldsymbol \phi_k(x)\wrt x.\]
%and global mass and stiffness matrices by
%\begin{align*}
%	M &:= \left[\begin{array}{ccc} M_1 & & \\ & \ddots & \\ & & M_K \end{array}\right] = \int_{x\inb}\boldsymbol \phi(x)\tr{}\boldsymbol \phi(x)\wrt x,
%	\\G &:= \left[\begin{array}{ccc} G_1 & & \\ & \ddots & \\ & & G_K \end{array}\right] = \int_{x\inb}\boldsymbol \phi(x)\tr{}\cfrac{\wrt}{\wrt x}\boldsymbol \phi(x)\wrt x,
%\end{align*}
We can write (\ref{eqn:DG system}) as 
\begin{align}
	&\cfrac{\wrt}{\wrt t} \boldsymbol a_{k,j}(t) \bs M_k = \sum_{i\in\calS} \boldsymbol a_{k,i}(t)\bs M_kT_{ij} 
	%
	+  c_j\boldsymbol a_{k,j}(t) \bs G_k - c_j[f_j(x,t)\boldsymbol\phi_k(x)]_{x=y_k}^{x=y_{k+1}}.\label{eqn:DG matrix}
\end{align}

It remains to approximate the \textit{flux}, \(f_j(x,t)\) at the cell edges \(y_k,\,k\in\mathbb Z\), so that we may evaluate the terms \([f_j(x,t)\phi^r_k(x)]_{x=y_k}^{x=y_{k+1}}\), \(r=1,...,N_k,\,k\in\mathbb Z\). This is the key for DG -- it joins the local approximations on each cell \(\calD_k\), into a global approximation on the whole domain of approximation. The flux is the instantaneous rate (with respect to time) at which density moves across the boundaries \(y_k,\,k\in\mathbb Z\). There are different choices for the flux, and we refer the reader to \citep{c99,nodalDGBook}, and references therein, for some discussion of the topic. Here, we choose the \textit{upwind} scheme, which, as we shall see, closely resembles the flux terms from the generator \(\mathbb B\). The approximate flux, also known as the \textit{numerical flux}, is given by 
\[f^*_j(x,t) = sign(c_j)\lim_{\varepsilon\to0^+}u_j(x-\varepsilon c_j,t),\]
at each \(x=y_k,\,k\in\mathbb Z\). 
Intuitively, the upwind flux takes the value of the density immediately on the upwind side of each \(y_k\). 

Denote by \(x^-\) and \(x^+\) the left and right limits at \(x\), respectively. Assume first \(c_j>0\), then 
\begin{align*}
	-c_j[f_j(x,t)\phi^r_k(x)]_{x=y_k}^{x=y_{k+1}} & \approx -c_j[f_j^*(x,t)\phi^r_k(x)]_{x=y_k}^{x=y_{k+1}}
	\\& = -c_jf_j^*(y_{k+1},t)\phi^r_k(y_{k+1}) + c_jf_j^*(y_{k},t)\phi^r_k(y_{k})
	\\& = -c_ju_j(y_{k+1}^-,t)\phi^r_k(y_{k+1}) + c_ju_j(y_{k}^-,t)\phi^r_k(y_{k})
	\\& = -c_ju_{k,j}(y_{k+1}^-,t)\phi^r_k(y_{k+1}) + c_ju_{k-1,j}(y_{k}^-,t)\phi^r_k(y_{k})
	\\& = -c_j\boldsymbol a_{k,j}(t)\boldsymbol \phi_k(y_{k+1}^-)\tr{}\phi^r_k(y_{k+1}) + c_j\boldsymbol a_{k-1,j}(t)\boldsymbol \phi_{k-1}(y_{k}^-)\tr{}\phi^r_k(y_{k}).
\end{align*}
In matrix form,  
\begin{align*}
	-c_j[f_j(x,t)\boldsymbol\phi_k(x)]_{x=y_k}^{x=y_{k+1}} & \approx -c_j[f_j^*(x,t)\boldsymbol\phi_k(x)]_{x=y_k}^{x=y_{k+1}}
	\\& = -c_j\boldsymbol a_{k,j}(t)\boldsymbol \phi_k(y_{k+1}^-)\tr{}\boldsymbol\phi_k(y_{k+1}) + c_j\boldsymbol a_{k-1,j}(t)\boldsymbol \phi_{k-1}(y_{k}^-)\tr{}\boldsymbol\phi_k(y_{k})
	\\& = c_j\boldsymbol a_{k,j}(t)F_j^{k,k}+c_j\boldsymbol a_{k-1,j}(t)F_j^{k-1,k},
\end{align*}
where, for \(j\in \calS\) with \(c_j>0\), we define \(\bs F_j^{k,k} := -\boldsymbol \phi_k(y_{k+1}^-)\tr{}\boldsymbol\phi_k(y_{k+1}),\,k\in\mathbb Z\) and \(\bs F_j^{k-1,k} := \boldsymbol \phi_{k-1}(y_{k}^-)\tr{}\boldsymbol\phi_k(y_{k}),\,k\in\mathbb Z\).

Now proceed similarly for \(c_j<0\) to get the approximation 
\begin{align*}
	-c_j[f_j(x,t)\boldsymbol\phi_k(x)]_{x=y_k}^{x=y_{k+1}} & \approx -c_j[f_j^*(x,t)\boldsymbol\phi_k(x)]_{x=y_k}^{x=y_{k+1}}
	\\& = -c_j\boldsymbol a_{k+1,j}(t)\boldsymbol \phi_{k+1}(y_{k+1}^+)\tr{}\boldsymbol\phi_k(y_{k+1}) + c_j\boldsymbol a_{k,j}(t)\boldsymbol \phi_{k}(y_{k}^+)\tr{}\boldsymbol\phi_k(y_{k})
	\\& = c_j\boldsymbol a_{k+1,j}(t)\bs F_j^{k+1,k}+c_j\boldsymbol a_{k,j}(t)\bs F_j^{k,k},
\end{align*}
where, for \(j\in \calS\) with \(c_j<0\), we define \(\bs F_j^{k+1,k} := -\boldsymbol \phi_{k+1}(y_{k+1}^+)\tr{}\boldsymbol\phi_k(y_{k+1}),\,k\in\mathbb Z,\) and \(\bs F_j^{k,k} := \boldsymbol \phi_{k}(y_{k}^+)\tr{}\boldsymbol\phi_k(y_{k}),\,k\in\mathbb Z\).

The matrices \(\bs F_j^{k-1,k},\,\bs F_j^{k,k},\) and \(\bs F_j^{k+1,k}\) are the local flux matrices. For convenience, we also define the matrices \(\bs F_j^{k,k+1}=0\) for \(c_j<0\) and \(\bs F_j^{k,k-1}=0\) for \(c_j>0\), \(k\in\mathbb Z\).


% from which we can assemble the global flux matrices
%\begin{align*}
%	F_j &= \left[\begin{array}{cccccc}
%		F_j^{1,1} & F_j^{1,2} & & & & \\
%		 & F_j^{2,2} & F_j^{2,3} & & & \\
%		& & \ddots & \ddots & & \\
%		& & & \ddots & \ddots & \\
%		& & & & F_j^{K-1,K-1} & F_j^{K-1,K} \\
%		& & & & & F_j^{K,K}
%	\end{array}\right],\quad c_j\geq0,
%	%
%	\\F_j &= \left[\begin{array}{cccccc}
%		F_j^{1,1} & & & & & \\
%		F_j^{2,1} & F_j^{2,2} & & & & \\
%		& \ddots & \ddots & & & \\
%		& & \ddots & \ddots & & \\
%		& & & F_j^{K-1,K-2} & F_j^{K-1,K} & \\
%		& & & & F_j^{K-1,K} & F_j^{K,K}
%	\end{array}\right],\quad c_j<0.
%\end{align*}
%
%We can now write 
%\begin{align*}
%	&\cfrac{\wrt}{\wrt t} \boldsymbol a_j(t) M = \sum_{i\in\calS} \boldsymbol a_j(t)MT_{ij} 
%	%
%	+  c_j\boldsymbol a_j(t) (G + F_j).
%\end{align*}
%
%Defining \(\boldsymbol a(t) = (\boldsymbol a_j(t))_{i\in\calS}\), then 
%\begin{align}\label{eqn:DG ODE}
%	&\cfrac{\wrt}{\wrt t} \boldsymbol a(t)(I_{N_\calS}\otimes M) = \boldsymbol a(t){  B,}
%\end{align}
%where 
%\[{  B} = \left[(T\otimes M) +
%	\left[\begin{array}{ccc}
%		c_1(G+F_1) & &  \\
%		& \ddots & \\
%		& & c_{N_\calS}(G+F_{N_\calS})\\
%	\end{array}\right] \right],\]
%\(\otimes\) is the kronecker product and \(I_{N_\calS}\) is the \(N_\calS\times N_\calS\) identity matrix.

%So far, this DG approximation has been constructed according to the lexicographic ordering of \(\calS\times\{1,...,K\}\). A reordering of this construction according to the lexicographic ordering of \(\{1,...,K\}\times\calS\) helps elucidate the connection with the partitioned operator \(\mathbb B\) and eases notation slightly for the next discussion on boundary conditions. 

%Also define \textit{global} row-vectors 
%\[\boldsymbol \phi(x) = (\boldsymbol\phi^k(x))_{k\in\{1,...,K\}},\quad \boldsymbol a_j(x) = (\boldsymbol a_{k,i}(x))_{k\in\{1,...,K\}},\,i\in\mathcal S.\]

To write this out as a \textit{global} system, define the row-vectors 
\[\boldsymbol a_k(t) = (\boldsymbol a_{k,i}(t))_{i\in\mathcal S},\quad \boldsymbol a(t) = (\boldsymbol a_k(t))_{k\in\mathbb Z},\]
and the block-tridiagonal matrix 
\begin{align*}
% \ddot{\bs M} &= \left[\begin{array}{ccc}\ddots&&\\&\bs I_{N_\calS}\otimes \bs M_k&\\&&\ddots\end{array}\right],
% \intertext{where \(N_\calS=|\calS|\), \(\otimes\) is the Kronecker product, and the block-tridiagonal matrix}
\ddot{\bs B} &= \left[\begin{array}{ccccc}
	\ddots & \ddots & \ddots & & \\
	& \ddot{\bs B}^{k,k-1} & \ddot{\bs B}^{k,k} & \ddot{\bs B}^{k,k+1} & \\
	& & \ddots & \ddots & \ddots 
\end{array}\right],
\end{align*}
where, for \(k\in\mathbb Z\), 
\begin{align*}
    \ddot{{\bs B}}^{kk}&=\left[\begin{array}{ccc}T_{11}\bs I + c_1(\bs F_1^{kk}+\bs G_k)\bs M_k^{-1} & T_{12}\bs I & T_{1N_\calS}\bs I  \\ T_{21}\bs I & & \\ \vdots &\ddots & \vdots \\ & &   T_{N_\calS-1,N_\calS}\bs I \\  T_{N_\calS1}\bs I &  T_{N_\calS,N_\calS-1}\bs I & T_{N_\calS,N_\calS}\bs I +c_{N_\calS}(\bs F_{N_\calS}^{kk}+\bs G_k)\bs M_k^{-1}\end{array}\right],\\
	\ddot{{\bs B}}^{k,k+1}&=\left[\begin{array}{ccc}c_1\bs F_1^{k,k+1}\bs M_{k+1}^{-1}&  & \\  &\ddots & \\  &  &c_{N_\calS}\bs F_{N_\calS}^{k,k+1}\bs M_{k+1}^{-1} \end{array}\right],\\
	\ddot{{\bs B}}^{k,k-1}&=\left[\begin{array}{ccc}c_1\bs F_1^{k,k-1}\bs M_{k-1}^{-1}&  & \\  &\ddots &  \\  &  &c_{N_\calS}\bs F_{N_\calS}^{k,k-1}\bs M_{k-1}^{-1} \end{array}\right].
\end{align*} 
The global system of equations is 
\begin{align}\label{eqn:DG ODE}
	&\cfrac{\wrt}{\wrt t} \boldsymbol a(t) = \boldsymbol a(t){\ddot{{\bs B}}}.
\end{align}

\section{Boundary conditions}\label{subsec: boundary DG}
To enable computation, this numerical approximation has to take place on a finite interval, which means we must consider a bounded domain and specify boundary conditions. Recall that we wish to impose lower and upper boundaries at \(0\) and \(b\), respectively. %In the numerics chapter, Chapter~\ref{sec: numerics}, we wish to approximate a fluid-fluid queue where the first fluid level, \(\dot X_t\), is bounded below at 0, only: in that case the first step in the approximation scheme is to approximate \(\dot X_t\) by \( X_t\). The truncation of \(X_t\) to \(\overline X_t\) will result in an artificial point mass at the upper bound, which we have to address properly. It is important to choose an \(b\) sufficiently large to control the error induced by the artificial upper bound, however, with larger \(b\) there comes increased computational burden. 

Let $[0,b]$ be the domain of the approximation, where $b < \infty$. We partition the space $[0,b]$ into \(\mathcal D_{-1}=\{0\},\) \(\mathcal D_{K+1}=\{b\},\) and \(K\) non-trivial intervals, \(\calD_k=[y_k,y_{k+1}]\setminus \{\{0\}\cup\{b\}\},\, y_k<y_{k+1},\, k\in\mathcal K^\circ\), \(y_0=0,\,y_{K+1}=b\) and define \(\Delta_k := y_{k+1}-y_k\). 

For states with \(c_i\leq 0\), there is the possibility of point mass accumulating at the boundary at~\(0\). Denote these point masses by \(q_{{-1},i}(t)\) for \(i\in\mathcal S_{-1}\). For states with \(c_i>0\) there is no possibility of a point mass at \(0\). Similarly, for \(c_i\geq 0\) there is the possibility of a point mass at \(b\). Denote these point masses by \(q_{{K+1},i}(t)\), for \(i\in\mathcal S_{K+1}\). For states with \(c_i<0\) there is no possibility of a point mass at \(b\). Let \(\bs q_{-1}(t)=(q_{{-1},i}(t))_{i\in\calS_{-1}}\) and \(\bs q_{K+1}(t) = (q_{{K+1},i}(t))_{i\in\calS_{K+1}}\) and \(\bs f_m(x,t) = (f_i(x,t))_{i\in\calS_m}\), \(m\in\{+,-,0\}\). 

Let us first consider the boundary at \( X_t=0\). The following boundary conditions describe the evolution of probability/density of a stochastic fluid model with a boundary at \(0\);
\begin{align}\label{eqn:BC1}
\cfrac{\wrt}{\wrt t}\bs q_{-1}(t) &= \bs q_{-1}(t) T_{{-1} {-1}} - \boldsymbol f_-(0,t)\bs C_-\bs P_{-{-1}},
\\\label{eqn:BC2}
\bs q_{-1}(t)T_{{-1}+}&=\bs f_+(0,t)C_+,
\end{align}
where \(\bs P_{-{-1}} = \left[p_{ij}^{-1}\right]_{i\in\calS_-,j\in\calS_{-1}}\). Equation (\ref{eqn:BC1}) states that point mass moves between phases according to the sub-generator matrix \(T_{{-1},{-1}}\), and that the flux of probability density into the point masses is \(- \boldsymbol f_-(0,t)\bs P_{-{-1}}C_{-1}\). Substituting the DG approximation for \(\boldsymbol f_-(0,t)\) into (\ref{eqn:BC1}) gives, for \(j\in\calS_{-1}\), 
\begin{equation*}\label{eqn:DGBC1}
\cfrac{\wrt}{\wrt t} q_{{-1},j}(t) = \sum_{i\in\calS_{-1}} q_{{-1},i}(t) T_{i j} - \sum_{i\in\calS_-}\boldsymbol a_{0,i}(t)\boldsymbol \phi_0(0)\tr{}p_{ij}^{-1} c_i.
\end{equation*}
Equation (\ref{eqn:BC2}) describes the flux of probability mass to density upon a transition from a phase in \(\calS_{-1}\) to a phase in \(\calS_+\). Thus, the flux into the left-hand edge of \(\calD_0\) in phase \(j\in\calS_+\) is, \(\sum\limits_{i\in\calS_{-1}} q_{{-1},i}(t)T_{ij}\). Therefore, we can now evaluate 
\begin{align*}
	-c_j[f_j(x,t)\boldsymbol\phi_0(x)]_{x=0}^{x=y_{0}} & =  -c_jf_j(y_0,t)\boldsymbol\phi_0(y_0)+c_jf_j(0,t)\boldsymbol\phi_0(0)
	\\& \approx -c_j(f_j^*(y_0,t)\boldsymbol\phi_0(y_0)+\sum_{i\in\calS_{-1}}q_{{-1},i}(t)T_{ij}\boldsymbol\phi_0(0)
	\\& = c_j\boldsymbol a_{0,j}(t)F_j^{0,0}+\sum_{i\in\calS_{-1}}q_{{-1},i}(t)T_{ij}\boldsymbol\phi_0(0),
\end{align*}
for \(j \in \calS_+\). 
Thus, the DG approximation of the flux into point masses \(q_{{-1},j}(t)\) is \[-\sum_{i\in\calS_-}\boldsymbol a_{0,i}(t)\boldsymbol \phi_0(0)\tr{} p_{ij}^{-1} c_i,\,j\in\calS_-,\] the rate of transition of point mass within \(\bs q_{{-1}}(t)\) is \(T_{{-1},{-1}}\), and the DG approximation of the transition of point mass to density is \(\sum\limits_{i\in\calS_{-1}}q_{{-1},i}(t)T_{ij}\boldsymbol\phi_0(0),\,j\in\calS_+\). 

Similarly, for the upper boundary at \(b\) the boundary conditions are 
\begin{align*}
\cfrac{\wrt}{\wrt t}\bs q_{K+1}(t) &= \bs q_{K+1}(t) T_{{K+1} {K+1}} + \boldsymbol f_+(b,t)\bs C_+\bs P_{+{K+1}},\\
\bs q_{K+1}(t)T_{{K+1}-}&=-\bs f_-(b,t)C_-.
\end{align*}
Using the same arguments as above, 
\begin{align*}
\cfrac{\wrt}{\wrt t} q_{{K+1},j}(t) &= \sum_{i\in\calS_{K+1}}q_{{K+1},i}(t) T_{ij} + \sum_{i\in\calS_+}\boldsymbol a_{K,i}(t)\boldsymbol \phi_K(b)\tr{}p_{ij}^{K+1} c_i,
\\-c_j[f_j(x,t)\boldsymbol\phi_K(x)]_{x=y_K}^{x=b} & \approx c_j\boldsymbol a_{K,j}(t)F_j^{K,K}+\sum_{i\in\calS_{K+1}}q_{{K+1},i}(t)T_{ij}\boldsymbol\phi_K(b),
\end{align*}
for \(j\in\calS_-\). 
Thus, the DG approximation of the flux into the point mass \(q_{{K+1},j}(t)\) is \(\sum_{i\in\calS_+}\boldsymbol a_{K,i}(t)\boldsymbol \phi_K(0)\tr{} p_{ij}^{K+1} c_i\), \(j\in\calS_+\), the rate of transition of point mass within \(\bs q_{{K+1}}(t)\) is \(T_{{K+1},{K+1}}\), and the DG approximation of the transition of point mass to density is \(\sum\limits_{i\in\calS_{K+1}}q_{{K+1},i}(t)T_{ij}\boldsymbol\phi_K(b)\), \(j\in\calS_-\). 

To include this behaviour in the DG generator we truncate the doubly-infinite matrix \(\ddot{{\bs B}}\) at \(k=0\) and \(k=K\), then append \(|\mathcal S_{-1}|\) rows and columns to the top and left, and \(|\mathcal S_{K+1}|\) rows and columns to the bottom and right. These represent the point masses \(\bs q_{-1}(t)\) and \(\bs q_{K+1}(t)\), respectively. Given the discussion above, the truncated matrix is
\[{{\bs B}} = \left[\begin{array}{llllll}
	\bs T_{{-1},{-1}}& {{\bs B}}^{{-1}0} & & & & \\
	{{\bs B}}^{0{-1}} & {{\bs B}}^{00} & {{\bs B}}^{01} & & & \\
	& {{\bs B}}^{10} & {{\bs B}}^{11} & {{\bs B}}^{12} & & \\
	& & \ddots & \ddots & \ddots & \\
	& & {{\bs B}}^{K-1,K-2} &{{\bs B}}^{K-1,K-1} & {{\bs B}}^{K-1,K} & \\
	& & &{{\bs B}}^{K,K-1} & {{\bs B}}^{K,K} & {{\bs B}}^{K,{K+1}} \\
	& & & & {{\bs B}}^{{K+1}, K} & \bs T_{{K+1},{K+1}}
\end{array}\right],\]
where 
\begin{align*}
	{{\bs B}}^{k\ell} &= \ddot{{\bs B}}^{k\ell}, \mbox{ for \(k\in\mathcal K^\circ\), \(\ell\in\{k-1,k,k+1\}\), \(k=\ell\neq 0\) or \(k=\ell\neq K\),}
	\\ \bs B^{00} &= \ddot{\bs B}^{00} + \left[-c_ip_{ij}^{-1} 1(c_i<0, c_j>0)\right]_{i\in\calS,j\in\calS}\otimes \bs \phi^0(0)\tr{}\bs \phi^0(0)\bs M_0^{-1},
	\\ \bs B^{KK} &= \ddot{\bs B}^{KK} + \left[c_ip_{ij}^{-1} 1(c_i>0, c_j<0)\right]_{i\in\calS,j\in\calS}\otimes \bs \phi^K(b)\tr{}\bs \phi^K(b)\bs M_K^{-1}, 
	\\ {{\bs B}}^{{-1}0} &:= \bs T_{{-1}+}\otimes \boldsymbol\phi^0(0), 
	\\ {{\bs B}}^{0{-1}} &:=-\left[c_ip_{ij}^{-1}\mathbb 1_{(c_i<0)}\right]_{i\in\calS,j\in\calS_{{-1}}} \otimes \boldsymbol \phi^0(0)\tr{}, 
	\\ {{\bs B}}^{{K+1} K} &:= \bs T_{{K+1}-}\otimes \boldsymbol\phi^K(b),
	\\ {{\bs B}}^{K,{K+1}} &:= \left[c_ip_{ij}^{K+1}\mathbb 1_{(c_i>0)}\right]_{i\in\calS,j\in\calS_{K+1}} \otimes \boldsymbol \phi^K(b)\tr{},
\end{align*} 
and \(\otimes\) is the Kronecker product. 

For future reference, we also define the matrices \({{\bs B}}^{k\ell}_{ij}\) for \(k\in\{2,\dots,K-1\}\) by
\begin{align*}
	{{\bs B}}^{kk}_{ij} &= \begin{cases}T_{ij}\bs I_{N_k} + c_i(\bs F_i^{kk}+\bs G_k)\bs M_k^{-1} & i=j,\\T_{ij} \bs I_{N_k}& i\neq j,\end{cases}
\\	{{\bs B}}^{k\ell}_{ij} &= \begin{cases}c_i\bs F_i^{k\ell}\bs M_\ell^{-1} & i=j,\\\bs 0 & i\neq j,\end{cases}\quad \ell \in \{k-1,k\}
\end{align*}
and
\begin{align*}
	{{\bs B}}^{00}_{ij} &= \begin{cases}T_{ij}\bs I_{N_k} + c_i(\bs F_i^{00}+\bs G_0)\bs M_0^{-1} & i=j,\\T_{ij}\bs I_{N_k} - 1(c_i<0,c_j>0)c_ip_{ij}^{-1} \bs F_i^{00}\bs M_0^{-1} & i\neq j,\end{cases}
\\	{{\bs B}}^{KK}_{ij} &= \begin{cases}T_{ij}\bs I_{N_k} + c_i(\bs F_i^{KK}+\bs G_K)\bs M_K^{-1} & i=j,\\T_{ij}\bs I_{N_K} + 1(c_i>0,c_j<0)c_ip_{ij}^{K+1} \bs F_i^{KK}\bs M_K^{-1} & i\neq j.\end{cases}
\end{align*}

After the addition of the boundary conditions, the system of ODEs (\ref{eqn:DG ODE}) can now be written as 
\begin{align}\label{eqn: DG ODE w BCs}
	\cfrac{\wrt}{\wrt t} \vligne{\boldsymbol q_{{-1}}(t) & {\boldsymbol a}(t) & \boldsymbol q_{K+1}(t)} 
	% 
	= \vligne{\boldsymbol q_{{-1}}(t) & {\boldsymbol a}(t) & \boldsymbol q_{K+1}(t)} \bs B.
\end{align}

Approximations \( \bs B^{mn}_{ij}\), \( \bs B_{ij}\), and \( \bs B^{mn}\) to \(\mathbb B^{mn}_{ij}\), \(\mathbb B_{ij}\), and \(\mathbb B^{mn}\), \(i,j\in\calS,\,m,n\in\{+,-,0\}\), are constructed from the block-matrices \({  \bs B}^{k\ell}_{ij}\), \(i,j\in\calS\), \(k,\ell\in\mathcal K\), as
\begin{align*}
	{  \bs B}_{ij}^{m n} &= \left[{  \bs B}_{ij}^{k \ell}\right]_{k\in\mathcal K_i^m,\ell\in\mathcal K_j^n},\quad i,j\in\calS,\,m,n\in\{+,-,0\},
%\intertext{an approximation \({  B}_{ij}\) to \(\mathbb B_{ij}\), \({i,j\in\calS},\) is}
\\	{  \bs B}_{ij} &= \left[{  \bs B}_{ij}^{k\ell}\right]_{k,\ell\in\mathcal K},\,{i,j\in\calS},
%\intertext{and an approximation \({  B}^{mn}\) to \(\mathbb B^{mn}\), \(m,n\in\{+,-,0\}\) is}
\\	{  \bs B}^{m n} &= \left[\left[{  \bs B}_{ij}^{k\ell}\right]_{i\in\calS_k^m,j\in\calS_\ell^n}\right]_{k\in\mathcal K^m,\ell\in\mathcal K^n},\,m,n\in\{+,-,0\}.
\end{align*}
% where \(\bs B = \widehat{{\bs B}}\widehat{\bs M}^{-1}\) and \(\widehat{\bs M} = \left[\begin{array}{ccccc}
% 	\bs I_{|\calS_{-1}|} & & & & \\
% 	& \bs I_{N_S}\otimes \bs M_1 & & & \\
% 	& & \ddots & & \\
% 	& & & \bs I_{N_S}\otimes \bs M_K & \\
% 	& & & & \bs I_{|\calS_{K+1}|}
% \end{array}\right]\), that is, 
% \[\bs B = \left[\begin{array}{llllll}
% 	\bs T_{{-1},{-1}}& \widehat{{\bs B}}^{{-1}1}\bs M_1^{-1} & & & & \\
% 	\widehat{{\bs B}}^{1{-1}} & \ddot{{\bs B}}^{11}\bs M_1^{-1} & \ddot{{\bs B}}^{12}\bs M_2^{-1} & & & \\
% 	& \ddot{{\bs B}}^{21}\bs M_1^{-1} & \ddot{{\bs B}}^{22}\bs M_2^{-1} & \ddot{{\bs B}}^{23}\bs M_3^{-1} & & \\
% 	& & \ddots & \ddots & \ddots & \\
% 	& & \ddot{{\bs B}}^{K-1,K-2}\bs M_{K-2}^{-1} &\ddot{{\bs B}}^{K-1,K-1}\bs M_{K-1}^{-1} & \ddot{{\bs B}}^{K-1,K}\bs M_{K}^{-1} & \\
% 	& & &\ddot{{\bs B}}^{K,K-1}\bs M_{K-1}^{-1} & \ddot{{\bs B}}^{K,K}\bs M_{K}^{-1} & \widehat{{\bs B}}^{K,{K+1}} \\
% 	& & & & \widehat{{\bs B}}^{{K+1}, K}\bs M_{K}^{-1} & \bs T_{{K+1},{K+1}}
% \end{array}\right].\]

% Regarding our notational convention, we use regular mathematics fonts to represent DG approximations to operators, i.e.~\(\bs B\) is a DG approximation to \(\mathbb B\) and \(\bs \Psi\) is an approximation to \(\mathbb \Psi\). 

We prove the following result in Appendix\ref{sec:properties}.
%This transformation is not strictly necessary and all the following calculations can be done with either form, as long as week keep a consistent interpretation of the DG generator and related coefficients. However, it is convenient to prove properties of the generator.
\begin{cor}
	The approximate generator \( \bs B\) conserves probability. That is, for all \(t\geq 0\),
	\begin{align*}
	\begin{multlined}[t]\sum_{i\in\calS_{-1}}q_{{-1},i}(t)+\sum_{i\in\calS_{K+1}}q_{{K+1},i}(t)+\sum_{i\in\calS} \int_{x\in[0,b]}u_i(x,t)\wrt x 
	%
	\\= \sum_{i\in\calS_{-1}}q_{{-1},i}(0)+\sum_{i\in\calS_{K+1}}q_{{K+1},i}(0)+\sum_{i\in\calS} \int_{x\in[0,b]}u_i(x,0)\wrt x.\end{multlined}
	\end{align*}
\end{cor}


\section{Application to an SFFM}\label{sec:DGSFFM}
Given our approximation \(\bs B\) to the generator \(\mathbb B\) we now follow the recipe from \cite{bo2014}, replacing the actual generator \(\mathbb B\) with its approximation \({  \bs B}\), to construct approximations, \(\bs \pi\) and \(\bs p\), to the stationary operators, \(\bbpi\) and \(\mathbb p\).

It may be convenient to think of our approximations in terms of approximations of kernels. Recall that the operators in \citep{bo2014} can be thought of in terms of kernels. That is, for some function \(\bs g = (g_i(x))_{i\in\calS}\), we can write \(\bs \mu \mathbb B \boldsymbol g\tr{} = \displaystyle \sum\limits_{k,\ell\in\mathcal K}\sum\limits_{i,j\in\calS} \displaystyle \int_{x,y}\wrt \mu_i(x) \mathbb B_{ij}^{k\ell}(x,\wrt y)g_j(y)\) where \(\mathbb B_{ij}^{k\ell}(x,\wrt y)\) is the kernel of the operator \(\mathbb B_{ij}^{k\ell}\). 

Let \(\boldsymbol a_{-1}(t):=\boldsymbol q_{-1}(t)\) and \(\boldsymbol a_{K+1}(t):=\boldsymbol q_{K+1}(t)\), and define basis functions \(\bs\phi_{-1}(x) = \phi_{-1}^1(x) = \delta(x)\) and \(\bs\phi_{K+1}(x) = \phi_{K+1}^1(x) = \delta(x-b)\), where \(\delta\) is the Dirac delta, \(N_{-1} = N_{K+1} = 1\), and \(\mathcal N_{-1} = \mathcal N_{K+1} = \{1\}\). Also define \({\bs M}_{-1}=\bs I_{|\calS_{-1}|}\), \({\bs M}_{K+1}=\bs I_{|\calS_{K+1}|}\), the block-diagonal matrix \(\bs M=diag(\bs M_k,k\in\mathcal K)\), and row-vectors 
\[\boldsymbol \phi(x) = (\bs\phi_k(x))_{k\in\mathcal K}, \quad \boldsymbol a_i(t) = (\bs a_{k,i}(t))_{k\in\mathcal K},\,i\in\calS.\]

To pose the approximation \(\bs B\) in kernel form let \(\boldsymbol a_i \boldsymbol \phi(x)\tr{}\in W,\,i\in\calS\) be the initial density of the process, and \(\boldsymbol \phi(x)\bs b_i\tr{}\in W,\,i\in\calS\) be a test function. Observe that, from our DG construction earlier and the definition of \({\bs M}\), 
\[\sum_{i,j\in\calS}\int_{x,y\in[0,b]} \boldsymbol a_i \boldsymbol \phi(x)\tr{} \boldsymbol \phi (x) \wrt x {\bs M}^{-1}{  \bs B}_{ij} \bs \phi(y)\tr{} \bs \phi(y)\bs b_j \wrt y= \sum_{i,j\in\calS} \boldsymbol a_i {  \bs B}_{ij} {\bs M} \bs b_j .\]
Thus, we can think of 
\[\boldsymbol \phi (x) {\bs M}^{-1}{  \bs B}_{ij} \bs \phi(y)\tr{}\wrt y,\]
as an approximation to the kernel \(\mathbb B_{ij}(x,\wrt y)\). This concept can be extended to all the approximations of operators considered in this work. 

\subsection{Approximating the operator \(\mathbb R\)}\label{sec: approx r}
Recall the operator \(\mathbb R^k\) from Lemma~\ref{lemma: D(s)}. Essentially, the operator \(\mathbb R^k\) takes an initial measure \(\boldsymbol \mu_k\) and multiplies each element by \(1/|r_i(x)|\) on cells \(\calD_k\) where \(r_i(x)\neq 0\). In the context of DG the initial distribution is given by \(\boldsymbol a_i \boldsymbol \phi(x)\tr{}\in W,\,i\in\calS\). Thus, for \(k\in\mathcal K\) such that \(r_i(x)\neq0\) on \(\calD_k\), we have 
\[\boldsymbol a_{k,i} \boldsymbol \phi_k(x)\tr{}\mathbb R^k_i = \cfrac{\boldsymbol a_{k,i} \boldsymbol \phi_k(x)\tr{}}{|r_i(x)|}.\]
Decompose the right-hand side into a component which lies in \(W\) and another orthogonal to \(W\): 
\[\cfrac{\boldsymbol a_{k,i} \boldsymbol \phi_k(x)\tr{}}{|r_i(x)|} = \bs \rho_{k,i} \bs \phi_k(x)\tr{} + g_i^\perp(x),\] where \(\bs \rho_{k,i} \bs\phi_k(x)\tr{}\in W\), \(g_i^\perp \in W^\perp\). Now, multiply by test functions \(\{\phi^r_k(x)\}_{r=1}^{N_k}\) and integrate over \([0,b]\):
\begin{align*}
	\boldsymbol a_{k,i}\int_{x\in[0,b]} \cfrac{ \boldsymbol \phi_k(x)\tr{}\bs \phi_k(x)}{|r_i(x)|}\wrt x
	&=\bs \rho_{k,i} \int_{x\in[0,b]}\bs \phi_k(x)\tr{}\bs \phi_k(x)\wrt x + \int_{x\in[0,b]} g_i^\perp(x)\bs\phi_k(x)\wrt x 
	%
	\\&= \bs \rho_{k,i} \int_{x\in[0,b]}\bs \phi_k(x)\tr{}\bs \phi_k(x)\wrt x = \bs \rho_{k,i}\bs M_k,
\end{align*}
since \(g_i(x)^\perp\in W^\perp\). Define the matrix \(\bs M_k^r := \displaystyle\int_{x\in[0,b]} \cfrac{ \boldsymbol \phi_k(x)\tr{}\bs \phi_k(x)}{|r_i(x)|}\wrt x\), then 
\(
	\boldsymbol a_{k,i}\bs M_k^r
	= \bs \rho_{k,i} M_k,
\)
which implies
\(
	\bs \rho_{k,i}  = \boldsymbol a_{k,i}\bs M_k^r\bs M_k^{-1}.
\)
Thus, we have the approximation 
\[\boldsymbol a_{k,i} \boldsymbol \phi_k(x)\tr{}\mathbb R^k_i = \cfrac{\boldsymbol a_{k,i} \boldsymbol \phi_k(x)\tr{}}{|r_i(x)|}\approx \boldsymbol a_{k,i}\bs M_k^r\bs M_k^{-1}\bs\phi_k(x)\tr{}.\]
Since \(\boldsymbol a_{k,i}\) is arbitrary, we see that we approximate \(\mathbb R_{k,i}\) by \(  \bs R_{k,i} = \bs M_k^r\bs M_k^{-1},\)
and \(\mathbb R^k\) by \(  \bs R^k = \diag(  \bs R_{k,i})_{i\in\calS^\bullet_k}\).

In practice, we implement a Gauss-Lobatto quadrature approximation to compute the elements of \(\bs M_k^r\).

\begin{rem}
	We could also use interpolation to approximate \(\mathbb R\). 
\end{rem}

\subsection{Approximating the operator \(\bs D\) and the Riccati equation}
Recalling Lemma~\ref{lemma: D(s)} and replacing the operators \(\mathbb R^k\) and \(\mathbb B^{\ell m}\), by their approximations we have the following approximation to \(\mathbb D^{mn}(s)\)
\begin{align*}
		 {\bs D}^{mn}(s) = \left[ \bs {R}^{m}\left(
		{ {\bs B}}^{mn} - s{\bs I} + { {\bs B}}^{m0 }\left({ {\bs B}}^{00}- s\bs I \right)^{-1} { {\bs B}}^{0n}\right)\right],\quad m,n \in\{+,-\}.
\end{align*} 

Let \(\bs\phi_k(x){\bs M}_k^{-1}\bs \Psi_{ij}^{k\ell}(s)\bs\phi_\ell(y)\tr{}\wrt y\), \(i,j\in\calS,\) \(k\in\mathcal K_i^+\,, \ell\in\mathcal K_j^-\) be a finite-dimensional approximation of the operator kernel \(\mathbb\Psi_{ij}^{k\ell}(s)(x,\wrt y)\), where \(\bs \Psi_{ij}^{k\ell}(s)\) is a matrix of constants for a given \(s\). Construct an approximation to \(\mathbb\Psi(s)(x,\wrt y)\) by 
\[\bs\phi^+(x){\bs M}_+^{-1}\bs \Psi(s)\bs\phi^-(y)\tr{}\wrt y = \bs \phi^+(x)\bs M_+ \left[\left[\bs \Psi_{ij}^{k\ell}\right]_{i\in\calS_k^+,j\in\calS_\ell^-}\right]_{k\in\mathcal K^+,\ell\in\mathcal K^-}\bs \phi^-(y)'\wrt y\]%\left[\left[\bs\phi^k(x)\bs {M}_k^{-1}\bs \Psi_{ij}^{k\ell}(s)\bs\phi^\ell(y)\tr{}\wrt y\right]_{i\in\calS_k^+,j\in\calS_\ell^-}\right]_{ k\in\mathcal K^+, \ell\in\mathcal K^-},\]
where \(\bs\phi^+(x) = (\bs\phi_k(x))_{i\in\calS_k^+,k\in\mathcal K^+}\) and \(\bs\phi^-(y) = (\bs\phi_k(y))_{i\in\calS_k^-,k\in\mathcal K^-}\) are row-vectors, \(\bs \Psi(s)\) is a matrix of constants for a given \(s\) with the same size as \(\bs D^{+-}\), and \({\bs M}_m,\) \(m\in\{+,-,0\}\) is a block diagonal matrix \({\bs M}_m = \diag\left( \left( \bs M_k\right)_{i\in\calS_k^m}\right)_{k\in\mathcal K^m}\), \(m\in\{+,-,0\}\). Now replace the theoretical kernels in Theorem~\ref{theo:Psi} by their DG approximations to get 
\begin{align*}
&\bs\phi^+(x){\bs M}_+^{-1}  \bs D^{+-}(s)\bs\phi^-(y)\tr{}\wrt y
\\&{}+ \int_{z_1,z_2}\bs\phi^+(x){\bs M}_+^{-1}\bs \Psi(s)\bs\phi^-(z_1)\tr{}\bs\phi^-(z_1) {\bs M}_-^{-1}  \bs D^{-+}(s)\bs\phi^+(z_2)
\\&{}\quad\times\bs\phi^+(z_2){\bs M}_+^{-1}\bs \Psi(s)\bs\phi^-(y)\tr{}\wrt z_1\wrt z_2\wrt y
\\&{}+ \int_{z_1}\bs\phi^+(x){\bs M}_+^{-1}  \bs D^{++}(s)\bs\phi^+(z_1)\tr{}\bs\phi^+(z_1) {\bs M}_+^{-1}\bs \Psi(s)\bs\phi^-(y)\tr{}\wrt z_1\wrt y
\\&{}+ \int_{z_1}\bs\phi^+(x){\bs M}_+^{-1}\bs \Psi(s)\bs\phi^-(z_1)\tr{}\bs\phi^-(z_1){\bs M}_-^{-1}  \bs D^{--}(s)\bs\phi^-(y)\tr{}\wrt z_1\wrt y
= 0.
\end{align*}
Multiplying on the left by \(\bs\phi^+(x)\tr{}\) and on the right by \(\bs\phi^-(y)\), integrating over both \(x\) and \(y\), then post-multiplying by \({\bs M}^{-1}_-\) gives the following matrix Ricatti equation
\begin{align}\label{eqn:RiccatiPsi}
    \bs D^{+-}(s)
+ \bs \Psi(s)   \bs D^{-+}(s)\bs \Psi(s)
+   \bs D^{++}(s)\bs \Psi(s)
+ \bs \Psi(s)  \bs D^{--}(s)
= \bs 0.
\end{align}
Thus, we may find \(\bs \Psi(s)\) by solving \eqref{eqn:RiccatiPsi}, using one of the methods in \citep{bot08}. Here, we use the Newtons method. 

\begin{rem}
	Given the stochastic interpretation of \(\mathbb\Psi(0)\) we know that \( \bs \mu^+ \mathbb\Psi(0)([0,\infty))=1\) for every vector of measures \( \bs \mu^+\) such that \( \bs \mu^+([0,\infty)\boldsymbol 1 = 1\), when an SFFM is recurrent. It appears that this result carries over to the matrix \(\bs \Psi(0)\) giving the property that \(\displaystyle\int_{y\in[0,b]} \bs \Psi(0)\bs\phi^-(y)\tr{}\wrt y = \bs 1\). However, we have only observed this numerically and have no proof of this property. 
\end{rem}

\subsection{Putting it all together: constructing an approximation to the stationary distribution}
We find an approximation to the stationary distribution by replacing the theoretical operators in Theorem~\ref{theo:density} with their approximations. Table \ref{table:notations} defines the notation we use for the DG approximations to stationary operators. 

 \begin{table}[h!]
 \centering
 \begin{tabular}{c|c|c|c}
	\begin{tabular}{c}Exact \\operator\end{tabular} & Operator indices & \begin{tabular}{c}Approximation \\ notation\end{tabular} & Approximations \\\hline 
	%
	\( \bbxi_{k,i} \) & \(i\in\calS_k^-,\,k\in\mathcal K^-\)  & \(\bs \xi_{k,i} := (\xi_{k,i}^r)_{r\in\mathcal N_k}\) & 
	\(%\begin{array}{c}
	\bbxi_{k,i}(\wrt x)\approx  \bs{\xi}_{k,i} \bs \phi^k(x)\tr{}\wrt x,
	 %\\\bbxi_i^{-1}(\{0\}) %:=\lim\limits_{n\to\infty}\mathbb{P}\left[X_{\theta_n} = 0, \varphi_{\theta_n} = i\right]
	%\approx \bs\xi_{i}^{-1},
	%\\\bbxi_i^{K+1}(\{b\}) %:= \lim\limits_{n\to\infty}\mathbb{P}\left[X_{\theta_n} = b, \varphi_{\theta_n} = i\right]
	%\approx \bs\xi_{i}^{K+1}.%\end{array}
	\)
	 \\\hline
%
%
	\(\mathbb p_{k,i}\) & \(\begin{array}{c}i\in\calS_k^-\cup\calS_k^0,\\k\in\bigcup\limits_{m\in\{-,0\}}\mathcal K_m\end{array}\) & \(\bs p_{k,i} := (p_{k,i}^r)_{r\in\mathcal N_k}\) & \(%\begin{array}{c}
	\mathbb p_{k,i}(\wrt x) %:= \lim\limits_{t\to\infty}\mathbb{P}\left[Y_t = 0, X_{t} \in \wrt x, \varphi_{t} = i\right]\\
	\approx\bs{p}_{k,i} \bs\phi_k(x)\tr{}\wrt x
	%\\ \mathbb p_{i}^{-1}(\{0\}) %:= \lim\limits_{t\to\infty}\mathbb{P}\left[Y_t=0, X_{t} = 0, \varphi_{t} = i\right]
	%\approx \bs p_{i}^{-1},\\ \mathbb p_{i}^{K+1}(\{b\}) %:= \lim\limits_{t\to\infty}\mathbb{P}\left[Y_t=0, X_{t} = b, \varphi_{t} = i\right]
	%\approx \bs p_{i}^{K+1}.% \end{array}
	\)\\\hline
%
%
	 \(\bbpi_{k,i}(y)\)  & \(\begin{array}{c}i\in\calS,\\k\in\mathcal K\end{array}\) & \(\bs \pi_{k,i}(y) := (\pi_{k,i}^r(y))_{r\in\mathcal N_k}\) & \(%\begin{array}{c}
	 \bbpi_{k,i}(y)(\wrt x) %:= \lim\limits_{t\to\infty}\mathbb{P}\left[Y_t \in \wrt y, X_{t} \in \wrt x, \varphi_{t} = i\right]\\
	\approx\bs{\pi}_{k,i}(y) \bs \phi_k(x)\tr{}\wrt x
	%\\ \bbpi_{i}^{-1}(y)(\{0\}) %:= \lim\limits_{t\to\infty}\mathbb{P}\left[Y_t=0, X_{t} = 0, \varphi_{t} = i\right]
	%\approx \bs \pi_{i}^{-1}(y),\\ \bbpi_{i}^{K+1}(y)(\{b\}) %:= \lim\limits_{t\to\infty}\mathbb{P}\left[Y_t=0, X_{t} = b, \varphi_{t} = i\right]
	%\approx \bs \pi_{i}^{K+1}(y). \end{array}
	\)\\\hline
 \end{tabular}
 \caption{Notation for the approximation of the stationary operators of an SFFM. The first column contains the operators which we are approximating, the second column contains indices for which the operators are defined, the third column defines the notation we use for the coefficients of the approximation, and the last column defines how the coefficients and basis functions are used to approximate the operators. \label{table:notations}}
 \end{table}

%  \begin{table}[h!]
% \centering
% \begin{tabular}{c|c|c|c}
%	\begin{tabular}{c}Exact \\Operator\end{tabular} & Notation & Set of indices & Approximation \\\hline 
%	%
%	\( \bbxi_{k,i}(\wrt x) \) & \(\xi_{k,i}^r\) & \(\begin{array}{l}i\in\calS,\,r\in\{1,...,N_k\},\\k\in\mathcal K_i^-\setminus \{{-1}\cup{K+1}\}.\end{array}\) & \(%:= \lim\limits_{n\to\infty}\mathbb{P}\left[X_{\theta_n} \in \wrt x, \varphi_{\theta_n} = i\right]\\
%	 \displaystyle\sum\limits_{r=1}^{N_k} {\xi}_{k,i}^r \phi^r_k(x)\wrt x,\,x\in\calD_k.\)\\\hline
%
%	&\(\xi_{k,i}^r\) & \(\begin{array}{l}i\in\calS,\,r=1,\\k\in\mathcal K_i^-\cap \{{-1}\cup{K+1}\}.\end{array}\) & \(\begin{array}{l}\bbxi_{k,i}(\{0\}) %:=\lim\limits_{n\to\infty}\mathbb{P}\left[X_{\theta_n} = 0, \varphi_{\theta_n} = i\right]
%	\approx \xi_{i,r}^{-1},
%	\\\bbxi_{k,i}(\{b\}) %:= \lim\limits_{n\to\infty}\mathbb{P}\left[X_{\theta_n} = b, \varphi_{\theta_n} = i\right]
%	\approx \xi_{i,r}^{K+1}.\end{array}\)\\\hline 
%
%	&\(p_{k,i}^r\) & \(\begin{array}{l}i\in\calS,\,r\in\{1,...,N_k\},\\k\in\bigcup\limits_{m\in\{-,0\}}\mathcal K_i^m\setminus \{{-1}\cup{K+1}\}.\end{array}\) & \(\mathbb p_{k,i}(\wrt x) %:= \lim\limits_{t\to\infty}\mathbb{P}\left[Y_t = 0, X_{t} \in \wrt x, \varphi_{t} = i\right]\\
%	\approx\displaystyle\sum\limits_{r=1}^{N_k} {p}_{k,i}^r \phi^r_k(x)\wrt x,\,x\in\calD_k.\)\\\hline
%
%	&\(p_{k,i}^r\) & \(\begin{array}{l}i\in\calS,\,r=1,\\k\in\bigcup\limits_{m\in\{-,0\}}\mathcal K_i^m\cap \{{-1}\cup{K+1}\}.\end{array}\) & \(\begin{array}{l} \mathbb p_{k,i}(\{0\}) %:= \lim\limits_{t\to\infty}\mathbb{P}\left[Y_t=0, X_{t} = 0, \varphi_{t} = i\right]
%	\approx p_{i,r}^{-1},\\ \mathbb p_{k,i}(\{b\}) %:= \lim\limits_{t\to\infty}\mathbb{P}\left[Y_t=0, X_{t} = b, \varphi_{t} = i\right]
%	\approx p_{i,r}^{K+1}. \end{array}\) \\\hline
%
%	&\(\pi_{k,i}^r(y)\) & \(\begin{array}{l}i\in\calS,\,r\in\{1,...,N_k\},\\k\in\bigcup\limits_{m\in\{+,-,0\}}\mathcal K_i^m\setminus \{{-1}\cup{K+1}\}.\end{array}\) & \(\bbpi_{k,i}(y)(\wrt x) %:= \lim\limits_{t\to\infty}\mathbb{P}\left[Y_t \in \wrt y, X_{t} \in \wrt x, \varphi_{t} = i\right]\\
%	\approx\displaystyle\sum\limits_{r=1}^{N_k} {\pi}_{k,i}^r(y) \phi^r_k(x)\wrt x,\,y>0,\, x\in\calD_k.\)\\\hline
% \end{tabular}
% \caption{Notation for the approximation of the stationary operators of an SFFM.\label{table:notations}}
% \end{table}
 
With the notation in Table \ref{table:notations} define row-vectors 
 \begin{align*}
	%{\bs{\xi}}_{k,i} &= ( \xi_{k,i}^r)_{r\in\{1,...,N_k\}}, \quad  { i\in\calS,\,k\in\mathcal K_i^-},
	%
	{\bs{\xi}}_k &:= ( \bs\xi_{k,i})_{i\in\calS_k^-}, \quad  {k\in\mathcal K_i^-},
	%
	\\ {\bs{\xi}}& := ( \bs \xi_k)_{k\in\mathcal K^-},
	% 
	%\\{\bs{p}}_{k,i} &= (  p_{k,i}^r)_{r\in\{1,...,N_k\}}, \quad  {i\in\calS,\,k\in\bigcup\limits_{m\in\{-,0\}}\mathcal K_i^m},
	%
	\\\bs{p}^{m}_k &:= (  \bs p_{k,i})_{i\in\calS_k^m}, \quad  k\in{\mathcal K^m},\, m\in\{-,0\},
	%
	\\ {\bs{p}^m} &:= (  \bs p^{m}_k)_{k\in\mathcal K^m},\quad m\in\{-,0\},
	%
	\\ {\bs{p}} &:= (  \bs p^m)_{m\in\{-,0\}},
	%
	%\\{\bs{\pi}}_{k,i}(y) &= (  \pi_{k,i}^r(y))_{r\in\{1,...,N_k\}},\quad  i\in\calS,\,k\in\bigcup\limits_{m\in\{+,-,0\}}\mathcal K_i^m,
	%
	\\{\bs{\pi}}^{k}_m(y) &:= (  \bs \pi_{k,i}(y))_{i\in\calS_k^m},\quad  k\in\mathcal K,\,m\in\{+,-,0\},
	%
	\\{\bs{\pi}}^m(y) &:= (\bs \pi^{k}_m(y))_{k\in\mathcal K^m},\quad  m\in\{+,-,0\},
	%
	\\ {\bs{\pi}}(y) &:= (  \bs \pi^m(y))_{m\in\{+,-,0\}}.
 \end{align*}
 
Proceeding similarly to the derivation of the Ricatti equation (\ref{eqn:RiccatiPsi}), we can argue that the coefficients \( {\boldsymbol{\xi}}\) are the solution to the matrix system 
% 	 
	\begin{align*}
		\vligne{ {\boldsymbol{\xi}}  & \boldsymbol{0}}\left(-\left[\begin{array}{ll} 
			 { {\bs B}}^{--} &  { {\bs B}}^{-0} \\
                         { {\bs B}}^{0-} &  { {\bs B}}^{00} 
		\end{array} \right]\right)^{-1}\left[\begin{array}{l} 
			 { {\bs B}}^{-+} \\ 
			 { {\bs B}}^{0+}
		\end{array} \right]\bs \Psi(0) & =  {\boldsymbol{\xi}}, \\ 
		\int_{x\in[0,b]} {\bs \xi}\left[\begin{array}{c}\bs \phi^-(x)\tr{} \\ \bs \phi^0(x)\tr{}\end{array}\right]\wrt x \boldsymbol 1 & = 1. 
	\end{align*} 
Essentially, we replace the theoretical operators in (\ref{eqn:xi1}) and (\ref{eqn:xi2}) with their DG counterparts. 

Similarly, the coefficients \( {\boldsymbol{p}}\) are given by 
	\begin{equation}\vligne{\bs{p}^{-}  & \bs{p}^{0}} = z \vligne{{\bs\xi} & \bs{0}} 
	\left(-\left[\begin{array}{ll} 
		{\bs B}^{--} & {\bs B}^{-0} \\
		{\bs B}^{0-} & {\bs B}^{00} 
		\end{array} \right] \right)^{-1},\label{eqn:pisystem1}\end{equation}
		where \(z\) is a normalising constant. The coefficients \(\bs \pi(y)\) are given by 
\begin{align} 
	\;  {\bs{\pi}}^{0}(y) &= \vligne{ {\bs{\pi}}^{+}(y) &  {\bs{\pi}}^{-}(y)}\left[\begin{array}{l} { {\bs B}}^{+0} \\ { {\bs B}}^{-0} \end{array} \right]\left(-{ {\bs B}}^{00}\right)^{-1}, \\ 
	% 
	 \vligne{ {\bs{\pi}}^{+}(y) &  {\bs{\pi}}^{-}(y)}& = \vligne{ {\bs{p}}^{-} &  {\bs{p}}^{0}}\left[\begin{array}{l} { {\bs B}}^{-+} \\ { {\bs B}}^{0+} \end{array} \right]\vligne{e^{ {\bs K}y} & e^{ {\bs K}y}\bs \Psi(0)}\left[\begin{array}{cc}  {\bs R}^{+} & 0 \\ 0 &  {\bs R}^{-}\end{array}\right], \\
	%  
	 \sum_{i \in \mathcal{S}}\sum_{k \in \mathcal K} & \int_{y = 0}^{\infty} \int_{x \in [0,b]}  \bs \pi_{k,i}(y)\bs\phi_k(x)\tr{}\wrt x\wrt y 
	 %
	 \\&{}+  \sum_{i \in \mathcal{S}}\sum_{\ell \in \{-,0\}}\sum_{k\in \mathcal K_i^\ell}  \int_{x \in [0,b]}  \bs p_{k,i}\bs\phi_k(x)\tr{}\wrt x = 1,\label{eqn:pisystem2}
	\end{align}
	% 
	where $ {\bs K} :=  {\bs D}^{++}(0) + \bs \Psi(0) {\bs D}^{(-+)}(0)$, and $z$ is a normalising constant.

%For \(i,k,r\) such that \(i\in\calS,\,r\in\{1,...,N_k\},\,k\in\mathcal K_i^-\setminus \{{-1}\cup{K+1}\}\), define coefficients \(  \xi_{k,i}^r\) to be determined later such that 
%\[\displaystyle\sum\limits_{r\in\{1,...,N_k\}} {\xi}_{k,i}^r \phi^r_k(x),\quad x \in \calD_k,\]
%is an approximation to the density of \(\lim\limits_{n\to\infty}\mathbb{P}\left[X_{\theta_n} \in \wrt x, \varphi_{\theta_n} = i\right]\) on \(x\in\calD_k\). For \(i,k,r\) such that \(i\in\calS,\,r=1,\,k\in\mathcal K_i^-\cap \{{-1}\cup{K+1}\},\) define point masses \(  \xi_{k,i}^r\) such that \( \xi_{i,r}^{-1}\) and \( \xi_{i,r}^{K+1}\) approximate the point masses \(\lim\limits_{n\to\infty}\mathbb{P}\left[X_{\theta_n} = 0, \varphi_{\theta_n} = i\right]\) and \(\lim\limits_{n\to\infty}\mathbb{P}\left[X_{\theta_n} = b, \varphi_{\theta_n} = i\right]\), respectively. In vector form, for \(i,k\) such that \(i\in\calS,\,k\in\mathcal K_i^-\), let \( {\bs{\xi}}_{k,i} = ( \xi_{k,i}^r)_{r\in\{1,...,N_k\}}\), for \(k\in\bigcup\limits_{i\in\calS}\mathcal K_i^-\), let \( {\bs{\xi}}^k = ( \xi_{k,i})_{\{i\in\calS\mid r_i(x)<0,x\in\calD_k\}}\) and let \( {\bs{\xi}} = ( \xi^k)_{k\in\bigcup\limits_{i\in\calS}\mathcal K_i^-}\). 
%
%Proceeding similarly to the derivation of the Ricatti equation (\ref{eqn:RiccatiPsi}), we can argue that the DG approximation to \( {\boldsymbol{\xi}}\) is the solution to the matrix system 
%% 	 
%	\begin{align}
%		\vligne{ {\boldsymbol{\xi}}  & \boldsymbol{0}}\left(-\left[\begin{array}{ll} 
%			 { {B}}^{--} &  { {B}}^{-0} \\
%                         { {B}}^{0-} &  { {B}}^{00} 
%		\end{array} \right]\right)^{-1}\left[\begin{array}{l} \label{eqn:xi1}
%			 { {B}}^{-+} \\ 
%			 { {B}}^{0+}
%		\end{array} \right]\Psi(0) & =  {\boldsymbol{\xi}}, \\ 
%		\int_y  {\bs \xi}\left[\begin{array}{c}\bs \phi^-(x)\tr{} \\ \bs \phi^0(x)\tr{}\end{array}\right]\wrt x \boldsymbol 1 & = 1. \label{eqn:xi2}
%	\end{align} 
%Essentially, we replace the theoretical operators in (\ref{eqn:xi1}) and (\ref{eqn:xi2}) with their DG counterparts. 
%%	We write 
%%	\( {\bs\xi} = \vligne{ {\bs\xi}_{-1} &  {\bs\xi}^\circ &  {\bs\xi}_{K+1}}\), where \( {\bs\xi}_{-1}=( {\xi}_{{-1},i})_{\{\i\in\calS_{-1}\mid r_i(0)<0\}}\) and \( {\bs\xi}_{K+1}=( {\xi}_{{K+1},i})_{\{\i\in\calS_{K+1}\mid r_i(0)<0\}}\) are associated with the boundaries at \(0\) and \(b\), respectively, and \( {\bs\xi}^\circ\) is associated with the interior \(x\in(0,b)\). Let us denote the elements of \( {\bs{\xi}}^\circ\) as \( {\xi}_{k,i}^r\), \(i\in\calS,\,r\in\{1,...,N_k\},\,k\in\mathcal K_i^-\) where the elements are ordered by \(k\) then \(i\) and then \(r\).
%%	We construct an approximation to the density of \(\lim\limits_{n\to\infty}\mathbb{P}\left[X_{\theta_n} \in \wrt x, \varphi_{\theta_n} = i\right]\) as \(\displaystyle\sum\limits_{\substack{k\in\mathcal K_i^- \\r\in\{1,...,N_k\}}} {\xi}_{k,i}^r \phi^r_k(x)\). An approximation to \(\lim\limits_{n\to\infty}\mathbb{P}\left[X_{\theta_n} = 0, \varphi_{\theta_n} = i\right]\) is \( {\xi}_{{-1},i}\) and an approximation to the artificial point mass \(\lim\limits_{n\to\infty}\mathbb{P}\left[X_{\theta_n} = b, \varphi_{\theta_n} = i\right]\) is \( {\xi}_{{K+1},i}\).
%	
%	Next, we seek to approximate the stationary operators \(\bs p\) defined in (\ref{eqn:jointmass}). For \(i,k,r\) such that \(i\in\calS,\,r\in\{1,...,N_k\},\,k\in\bigcup\limits_{m\in\{-,0\}}\mathcal K_i^m\setminus \{{-1}\cup{K+1}\}\), coefficients \(p_{k,i}^r\) to be determined later, such that 
%\[\displaystyle\sum\limits_{r\in\{1,...,N_k\}}p_{k,i}^r \phi^r_k(x),\quad x \in \calD_k,\]
%is an approximation to the density of \(\lim\limits_{t\to\infty}\mathbb{P}\left[Y_t = 0, X_{t} \in \wrt x, \varphi_{t} = i\right]\) on \(x\in\calD_k\). For \(i,k,r\) such that \(i\in\calS,\,r=1,\,k\in\bigcup\limits_{m\in\{-,0\}}\mathcal K_i^m\cap \{{-1}\cup{K+1}\},\) define point masses \(p_{k,i}^r\) such that \(p_{i,r}^{-1}\) and \(p_{i,r}^{K+1}\) approximate the point masses \(\lim\limits_{t\to\infty}\mathbb{P}\left[Y_t=0, X_{t} = 0, \varphi_{t} = i\right]\) and \(\lim\limits_{t\to\infty}\mathbb{P}\left[Y_t = 0, X_{t} = b, \varphi_{t} = i\right]\), respectively. In vector form, for \(i,k\) such that \(i\in\calS,\,k\in\bigcup\limits_{m\in\{-,0\}}\mathcal K_i^m\), let \( {\bs{p}}_{k,i} = (  p_{k,i}^r)_{r\in\{1,...,N_k\}}\), for \(k\in\bigcup\limits_{i\in\calS}\bigcup\limits_{m\in\{-,0\}}\mathcal K_i^m\), let \( {\bs{p}}^k = (  p_{k,i})_{\{i\in\calS\mid r_i(x)\leq 0,x\in\calD_k\}}\) and let \( {\bs{p}} = (  p^k)_{k\in\bigcup\limits_{i\in\calS}\bigcup\limits_{m\in\{-,0\}}\mathcal K_i^m}\). Again, by proceeding similarly to the derivation of (\ref{eqn:RiccatiPsi}), that the DG approximation to \( {\boldsymbol{p}}\) is 
%	\[\vligne{\bs{p}^{-}  & \bs{p}^{0}} = z \vligne{{\bbxi} & \bs{0}} 
%	\left(-\left[\begin{array}{ll} 
%		{B}^{--} & {B}^{-0} \\
%		{B}^{0-} & {B}^{00} 
%		\end{array} \right] \right)^{-1},\]
%		where \(z\) is a normalising constant. 
%That is, we substitute the DG approximations of the operators into the corresponding equation in Theorem~\ref{theo:density}.
%%	
%%	For \(i\in\calS_{-1}, m\in\{-,0\}\) define \(  p^m_{{-1},i}\) as the approximation to \( \lim\limits_{t\to\infty}\mathbb{P}\left[X_{t} =0, Y_t=0, \varphi_{t} = i\right]\).
%%	
%%	 \( {\bs p}^-_{{-1}}=(  p^-_{{-1},i})_{\{i\in\calS_{-1}\mid r_i(0)<0\}}\) and \( {\bs p}^0_{{-1}}=(  p^0_{{-1},i})_{\{i\in\calS_{-1}\mid r_i(0)=0\}}\). 
%%	Now define \( {\bs p}^m_{{-1}},\, {\bs p}^m_{\circ},\, {\bs p}^m_{{K+1}}\) as the approximation to \(\bs p\) at the left-hand boundary, interior, and on the right-hand boundary of \([0,b]\), respectively.
%%	\[ {\bs p}^m := \vligne{ {\bs p}^m_{-1}(y) &  {\bs p}^m_\circ(y) &  {\bs p}^m_{K+1}(y)},\quad m\in\{-,0\},\]
%%where Specifically,  
%
%	
%	Next we approximate the operator \({{\bbpi}}(y)\). For \(i,k,r\) such that \(i\in\calS,\,r\in\{1,...,N_k\},\,k\in\bigcup\limits_{m\in\{+,-,0\}}\mathcal K_i^m\setminus \{{-1}\cup{K+1}\}\), define \(\pi_{k,i}^r(y)\), to be determined later, such that 
%\[\displaystyle\sum\limits_{r\in\{1,...,N_k\}}\pi_{k,i}^r(y) \phi^r_k(x),\quad x \in \calD_k,\]
%is an approximation to the density of \(\lim\limits_{t\to\infty}\mathbb{P}\left[Y_t \in\wrt y, X_{t} \in \wrt x, \varphi_{t} = i\right]\) on \(x\in\calD_k\). For \(i,k,r\) such that \(i\in\calS,\,r=1,\,k\in\bigcup\limits_{m\in\{+,-,0\}}\mathcal K_i^m\cap \{{-1}\cup{K+1}\}\) define point masses \(\pi_{k,i}^r(y)\) such that \(\pi_{i,r}^{-1}(y)\) and \(\pi_{i,r}^{K+1}(y)\) approximate the point masses \(\lim\limits_{t\to\infty}\mathbb{P}\left[Y_t\in\wrt y, X_{t} = 0, \varphi_{t} = i\right]\) and \(\lim\limits_{t\to\infty}\mathbb{P}\left[Y_t \in\wrt y, X_{t} = b, \varphi_{t} = i\right]\), respectively. In vector form, for \(i,k\) such that \(i\in\calS,\,k\in\bigcup\limits_{m\in\{+,-,0\}}\mathcal K_i^m\), let \( {\bs{\pi}}_{k,i}(y) = (  \pi_{k,i}^r(y))_{r\in\{1,...,N_k\}}\), for \(k\in\bigcup\limits_{i\in\calS}\bigcup\limits_{m\in\{+,-,0\}}\mathcal K_i^m\), let \( {\bs{\pi}}^k(y) = (  \pi_{k,i}(y))_{\{i\in\calS\mid r_i(x)\leq 0,x\in\calD_k\}}\) and let \( {\bs{\pi}}(y) = (  \pi^k(y))_{k\in\bigcup\limits_{i\in\calS}\bigcup\limits_{m\in\{+,-,0\}}\mathcal K_i^m}\). Again, we can argue that the DG approximation to \( {\boldsymbol{\pi}}(y)\) is given by substituting the DG operators for their theoretical counterparts in Theorem~\ref{theo:density}
%	\begin{align} 
%	& \;  {\bs{\pi}}^{0}(y) = \vligne{ {\bs{\pi}}^{+}(y) &  {\bs{\pi}}^{-}(y)}\left[\begin{array}{l} { {B}}^{+0} \\ { {B}}^{-0} \end{array} \right]\left(-{ {B}}^{00}\right)^{-1}, \label{eqn:pisystem1} \\ 
%	% 
%	&  \vligne{ {\bs{\pi}}^{+}(y) &  {\bs{\pi}}^{-}(y)} = \vligne{ {\bs{p}}^{-} &  {\bs{p}}^{0}}\left[\begin{array}{l} { {B}}^{-+} \\ { {B}}^{0+} \end{array} \right]\vligne{e^{ {K}y} & e^{ {K}y}\Psi(0)}\left[\begin{array}{cc}  {R}^{+} & 0 \\ 0 &  {R}^{-}\end{array}\right], \\
%	%  
%	 &\sum_{r \in \{{-1},1,...,K,{K+1}\}}\sum_{i \in \mathcal{S}_{\ell}} \int_{y = 0}^{\infty}  \pi_{i,r}^{k}(y)\phi_{k}^r(x)\wrt x\wrt y + \sum_{\ell \in \{-,0\}} \sum_{i \in \mathcal{S}}\sum_{r\in \mathcal K_i^\ell}   p^{r}_i(\mathcal{F}^{\ell}_i) = 1,\label{eqn:pisystemend}
%	\end{align}
%	% 
%	where $ {K} :=  {D}^{++}(0) + \Psi(0) {D}^{(-+)}(0)$, and $z$ is a normalising constant.
%	%
%%Also \( {\bs p}^-_{{K+1}}=(  p^-_{{K+1},i})_{\{i\in\calS_{K+1}\mid r_i(b)<0\}}\) and \( {\bs p}^0_{{K+1}}=(  p^0_{{K+1},i})_{\{i\in\calS_{K+1}\mid r_i(b)=0\}}\) where \(  p^m_{{K+1},i},\,m\in\{-,0\}\) are approximations to \( \lim\limits_{t\to\infty}\mathbb{P}\left[X_{t} =b, Y_t=0, \varphi_{t} = i\right]\). On the interior, we have 
%%\( {\bs p}^m_{\circ}=( {\bs p}_{k,i})_{i\in\calS,k\in\mathcal K_i^m}\) where \( {\bs p}_{k,i}=( {p}_{k,i}^r)_{r=1,...,N_k}\). On \(\mathcal D_k\) an approximation to the density of \(\lim\limits_{t\to\infty}\mathbb{P}\left[X_{t} =\wrt x, Y_t=0, \varphi_{t} = i\right]\) is given by \( {\bs p}_{k,i}\bs \phi^k(x)\tr{}\).
%%
%%Similarly for \( {\bs\pi}\) we write \[ {\bs \pi}^m =\vligne{ {\bs \pi}^m_{-1}(y) &  {\bs \pi}^m_\circ(y) &  {\bs \pi}^m_{K+1}(y)},\quad m\in\{+,-,0\},\]
%%where \( {\bs \pi}^m_{{-1}}(y),\, {\bs \pi}^m_{\circ}(y),\, {\bs \pi}^m_{{K+1}}(y)\) correspond to the approximation at the left-boundary, interior, and on the right boundary of \(x\in[0,b]\), respectively. 
%%%
%%Specifically, \( {\bs \pi}^+_{{-1}}(y)=(  \pi^+_{{-1},i}(y))_{\{i\in\calS_{-1}\mid r_i(0)>0\}}\), \( {\bs \pi}^-_{{-1}}(y)=(  \pi^-_{{-1},i}(y))_{\{i\in\calS_{-1}\mid r_i(0)<0\}}\) and \( {\bs \pi}^0_{{-1}}(y)=(  \pi^0_{{-1},i}(y))_{\{i\in\calS_{-1}\mid r_i(0)=0\}}\) where \(  \pi^m_{{-1},i}(y),\,m\in\{+,-,0\}\) are approximations to \( \lim\limits_{t\to\infty}\mathbb{P}\left[X_{t} =0, Y_t=\wrt y, \varphi_{t} = i\right]\). 
%%%
%%Also \( {\bs \pi}^+_{{K+1}}(y)=(  \pi^+_{{K+1},i}(y))_{\{i\in\calS_{K+1}\mid r_i(b)>0\}}\), \( {\bs \pi}^-_{{K+1}}(y)=(  \pi^-_{{K+1},i}(y))_{\{i\in\calS_{K+1}\mid r_i(b)<0\}}\) and \( {\bs \pi}^0_{{K+1}}(y)=(  \pi^0_{{K+1},i}(y))_{\{i\in\calS_{K+1}\mid r_i(b)=0\}}\) where \(  \pi^m_{{K+1},i}(y),\,m\in\{+,-,0\}\) are approximations to \( \lim\limits_{t\to\infty}\mathbb{P}\left[X_{t} =b, Y_t\in\wrt y, \varphi_{t} = i\right]\). 
%%%
%%On the interior, we have 
%%\( {\bs \pi}^m_{\circ}(y)=( {\bs \pi}_{k,i}(y))_{i\in\calS,k\in\mathcal K_i^m}\) where \( {\bs \pi}_{k,i}(y)=( {\pi}_{k,i}^r(y))_{r=1,...,N_k}\). On \(\mathcal D_k\) an approximation to the density of \(\lim\limits_{t\to\infty}\mathbb{P}\left[X_{t} =\wrt x, Y_t\in\wrt y, \varphi_{t} = i\right]\) is given by \( {\bs \pi}_{k,i}(y)\bs \phi^k(x)\tr{}\).

To assist the reader in understanding these constructions and the notation we provide an explicitly worked toy-example in Appendix~\ref{appendix:example}.

% \section{A stochastic interpretation of the simplest DG scheme}
% links between DG and the uniformisation scheme
% positivity preservation
% stochastic interpretations
% how I use/analyse it in the numerics section

\section{Other problems we can solve with the DG method}
The utility of the DG approximation scheme for fluid queues in not limited to approximating the first return and stationary operators of fluid-fluid queues. For example, we can use a DG scheme to approximate the transient distribution of the fluid queue at time \(t\), which requires us to find the coefficients \(\vligne{\bs q_{-1}(t) & \bs a(t) & \bs q_{{K+1}}(t)}\) via the differential equation (\ref{eqn: DG ODE w BCs}) given an initial condition, typically by numerically integrating (\ref{eqn: DG ODE w BCs}). The stationary distribution of the fluid queue can be found by solving 
\begin{align*}
	\bs b \bs B &= \bs 0,
	\\ \bs b \bs 1 &= \bs 1,
\end{align*}
for the coefficients \(\bs b\). We can also approximate first hitting times. For example, given an initial condition of the cell \(\calD_k\), we can project the initial condition onto the polynomial basis with coefficients \(\bs a_k(0)\), then approximate the probability that the level process first hits \(\{y_k,y_{k+1}\}\) after time \(t_0\) by finding \(\bs a_k(0)e^{B^{kk} t_0}\int_{\calD_k}\bs \phi_k(x)\wrt x\) where the coefficients \(\bs a_k(t_0)\) can be found bi integrating the differential equation 
\[\cfrac{\wrt}{\wrt t}\bs a_k(t)=\bs a_k(t)\bs B^{kk},\]
over time. We now briefly introduce time integration schemes which we use in this thesis, and some issues pertaining to oscillatory approximations. 

% APPLICATION TO INTERGRATING OVER TIME
\section{Time-integration schemes}\label{sec: time integration}
In this thesis we numerically integrate ODEs of the form
\begin{align}\label{eqn: DG ODE w BCs 3}
	\cfrac{\wrt}{\wrt t} \boldsymbol a(t)
	% 
	= \boldsymbol a(t) \bs Q
\end{align}
where \(\bs a(t)\) is a vector of coefficients and \(\bs Q\) a matrix. To do so, we employ the strong stability preserving Runge-Kutta (SSPRK) scheme of order 4 with 5 stages \citep{sr2002}. SSPRK methods are a family of Runge-Kutta methods which, to quote Section~5.7 of \cite{nodalDGBook}, ``can be used with advantage for problems with strong shocks and discontinuities, as they guarantee that no additional oscillations are introduced as part of the time-integration process.'' 

In our context, the SSPRK method with \(s\) stages has the form 
\begin{align*}
	&\bs v^{(0)} = \bs a(t),\\
	%
	&\bs v^{(\ell)} = \sum_{k=0}^{\ell-1} \alpha_{\ell k}\bs v^{(k)} + h\beta_{\ell k}  \bs v^{(k)}\bs Q,\, \ell = 1,\dots,s,\\
	%
	&\bs a(t+h) =\bs v^{(s)},
\end{align*}
where \(\alpha_{\ell k}\) and \(\beta_{\ell k}\) are coefficients which define the scheme and \(h\) is the \(t\)-step size. The coefficients \(\alpha_{\ell k}\) and \(\beta_{\ell k}\) are chosen to be positive so that \(\bs v^{(s)}\) is a convex combination of forward-Euler steps. Moreover, the coefficients \(\alpha_{\ell k}\) and \(\beta_{\ell k}\) are optimised so that the allowable \(t\)-step size is as large as possible. The maximum allowable \(t\)-step size is \(\Delta_{RK}\), and we require 
\[\Delta_{RK} \leq \min_{\ell k}\cfrac{\alpha_{\ell k}}{\beta_{\ell k}}\Delta_E,\]
where \(\Delta_E\) is the maximum allowable \(t\)-step size for the forward-Euler scheme. 

The \(t\)-step size needs to be chosen to be sufficiently small so that the time integration is stable. Here, for an order \(p\) DG scheme we choose the time step to be less than 
\[\max_{i\in\calS}|c_i|\min_{r\in{2,...,p+1}}\left(z_{r}-z_{r-1}\right) \cfrac{\min_{k}{K+1}_k}{2}\]
where \(z_{r-1}\leq z_r\), \(z_1=-1\), \(z_{p+1}=1\) and \(z_r, r=2,...,p,\) are the \(p-1\) zeros of the first derivative of \(P_{p}(x)\), the order \(p\) Legendre polynomial. This follows advice from \cite{nodalDGBook}. 

\section{Slope limiters}\label{sec: slope limiting}
A well-know problem with DG schemes is that, in the presence of steep gradients or discontinuities, spurious oscillations in the approximate solution can occur and positivity is not guaranteed (see \cite{nodalDGBook}~Section~5.6 and \cite{koltai2011}~Section~3.3, for example). To rectify this, in some contexts, slope limiting can be used (see \cite{c99}, or \cite{nodalDGBook},~Section~5.6.2 and references therein). Slope limiting effectively alters the DG operator in regions where oscillations are detected to ensure non-oscillatory and non-negativity by reducing the order of the approximation to linear in these regions. However, limiting does not distinguish between natural oscillations, which are a fundamental feature of the solution, and spurious oscillations, which are caused by the approximation scheme. As a result, the limiter may unnecessarily reduce the accuracy of the scheme in the presence of natural oscillations (see \citep{nodalDGBook},~Example~5.8). Furthermore, slope limiting adds an extra computational cost on top of the approximation scheme and means that the resultant approximation to certain operators are no longer linear operators.

We now describe a simple slope limiter known as the \emph{Generalised MUSCL} scheme \citep{c99} (see also \cite[Section~5.6.1]{nodalDGBook}). Define the \emph{minmod} function 
\begin{align*}
	m(a_1,a_2,a_3)=\begin{cases}
		s\min_{i\in\{1,2,3\}}|a_i|, & |s|=1, \\
		0, &\mbox{otherwise},
	\end{cases}
\end{align*}
where \(s=\frac{1}{3}\sum\limits_{i=1}^{3}\mbox{sign}(a_i)\). When all three arguments, \(a_1,\,a_2\) and \(a_3\) have the same sign, the minmod function returns the smallest of the three arguments with the correct sign, otherwise the signs of the three arguments differ and the minmod function returns zero. In the context of slope-limiting, the three arguments \(a_1,\,a_2\) and \(a_3\) are the gradients of the approximate solution in three cells. When the signs of \(a_1,\,a_2\) and \(a_3\) differ, this suggests an oscillation in the approximate solution. 

Now, define \(\overline u_{k}\) as the average value of the approximate solution on cell \(k\), then the slope limited solution on cell \(k\) is the linear approximant 
\begin{align*}
	\Pi_{k}^{lim}(u_k)=\overline u_{k} + (x-\overline y_k)m\left(\Pi^1_{\partial x}u_k,\cfrac{\overline u_{k+1}-\overline u_{k}}{\Delta_k},\cfrac{\overline u_{k}-\overline u_{k-1}}{\Delta_k}\right),
\end{align*}
where \(\Pi^1_{\partial x}u_k\) is the slope of the linear projection of \(u_k\) on \(\calD_k\) and \(\overline y_k=(y_k+y_{k+1})/2\) is the centre of \(\calD_k\). 

In non-oscillatory regions of the solution the slope limiter is unnecessary and reduces accuracy, so we only apply the limiter to cells \(k\) where oscillations are detected. To determine which cells need limiting we compute 
\begin{align*}
	v_k^L &:= \overline{u}_{k} - m(\overline{u}_{k} - {u}_{k}^L,\overline{u}_{k}-\overline{u}_{k-1},\overline{u}_{k+1}-\overline{u}_{k}),
	\\v_k^R &:= \overline{u}_{k} + m(\overline{u}_{k} - {u}_{k}^R,\overline{u}_{k}-\overline{u}_{k-1},\overline{u}_{k+1}-\overline{u}_{k}),
\end{align*}
where \({u}_{k}^L\) and \({u}_{k}^R\) are the values of the approximate solution on cell \(k\) evaluated at the left-hand and right-hand edges of the cell. We apply the slope limiter to cell \(k\) if \(v_k^L\neq u_k^L\) or \(v_k^R\neq u_k^R\), otherwise the approximation on cell \(k\) is left unaltered. 

In the context of approximating the first return operator of a fluid-fluid queue there is no direct equivalent of slope limiting, other than, perhaps, to re-compute the solution with constant basis function if a higher-order approximation happens to be oscillatory, but this is a post-hoc solution which would essentially require computing two solutions. 

\subsection{Slope limiters and time integration}
To use slope limiters within the SSPRK scheme we first project the initial condition on to \(W\), the set of polynomial basis functions, and apply the slope limiter to the projection to determine the initial coefficients \(\bs a(0)\).

For example, to find the coefficients for the approximate solution to the transient distribution at time \(t_0\) we take the initial condition \(\bs a(0)\) and evolve it forward in time until \(t_0\) via numerical integration of the differential equation (\ref{eqn: DG ODE w BCs 3}). At each stage of the time-integration we apply the slope limiter, i.e.~the scheme above with a slope limiter is 
\begin{align*}
	&\bs v^{(0)} = \vligne{\bs q_{-1}(t) & \bs a(t) & \bs q_{{K+1}}(t)},\\
	%
	&\bs v^{(\ell)} = \Pi^{lim}\left(\sum_{k=0}^{\ell-1} \alpha_{\ell k}\bs v^{(k)} + h\beta_{\ell k} \bs B \bs v^{(k)}\right),\, \ell = 1,\dots,s,\\
	%
	&\vligne{\bs q_{-1}(t+h) & \bs a(t+h) & \bs q_{{K+1}}(t+h)} =\bs v^{(s)},
\end{align*} 
where \(\Pi^{lim}\) is the slope limiter function.

\paragraph{Consequences of slope limiting} We have already mentioned that the Generalised MUSCL slope limiter reduces the order to linear in regions where oscillatory solutions are detected. When slope limiting is used in conjunction with a time-integration scheme the reduction in order may not be local in time and may persist past the time when the limiter first detected oscillations. For example, consider a problem with an initial condition which introduces oscillations into the numerical approximation. If the slope limiter is applied to the initial condition then, in the region around the oscillations, the approximation to the initial condition will be linear. Even if the slope limiter is not applied (or not needed) after this point, the initial condition has still been altered, perhaps significantly, from the original approximation and this can affect the accuracy of transient approximations. This is not to say that the slope-limited-regions of the approximation will \emph{always} remain linear. When no oscillations are detected, the DG scheme can use the full power of the high-order polynomial basis to approximate the solution, and this is one of the advantages of Generalised slope limiter described. Instead, we want to point out that once the limiter is applied at one time point, the approximation becomes linear which will effect the accuracy of subsequent computations. 

\subsection{Another slope limiting scheme}\label{sec: anoth pos pres}
The advantage of the approach above is that, in areas of the approximation where the solution is smooth, then the high-order accuracy of the method can be retained, while maintaining positivity where necessary. However, if there are no regions of the solution of interest which are smooth, then there is no real advantage in this approach as the limiter will reduce the approximate solution to linear in all regions of interest. Hence, we may as well just use a linear scheme and save on some computation. Perhaps a better approach would be to use a linear scheme with a smaller cell width, such that the computational work remains approximately the same. 

For example, for a DG scheme with \(2p\) basis functions on a cell, \(\mathcal D_k\), of width \(\Delta\), say, we construct block matrices of dimension \(2p\times 2p\) (i.e.~the mass, stiffness and flux matrices, \(\bs M_k\), \(\bs G_k\), and \(\bs F^{k\ell}\), respectively) and compute the approximate solution using these matrices. Instead, we could break the cell, \(\mathcal D_k\), up into \(p\) sub-cells of width \(\Delta/p\) and use a DG scheme with \(2\) basis functions on each sub-cell (i.e.~a linear basis on each cell). With this approach, to approximate the solution on the original interval \(\mathcal D_k\), we would also construct block matrices of dimension \(2p\times 2p\) and compute the approximate solution on \(\calD_k\) using these matrices in the same way as before. If we know \emph{a priori} that we will apply the slope limiter to cell \(\calD_k\), then intuitively we expect that the latter scheme will be far more accurate; it will approximate the solution on \(\calD_k\) by a piecewise linear function with \(p\) pieces, whereas the DG scheme with \(2p\) basis functions on cell \(\calD_k\) will, after limiting, represent the solution as a single linear function on \(\calD_k\). 

\subsection{Briefly on accuracy}
As a rough guide on accuracy we now paraphrase some know results on the accuracy of the DG method for a similar kind of problem to the one under study here. For certain scalar conservation-law problems, under certain regularity conditions on the continuity and bounded-ness of the solution and its derivatives, on the grid used, on the flux function, and on the approximation to the flux, then one can expect accuracy proportional to \(\Delta^{p}\) for the DG approximation of the solution at time \(t\) provided the DG scheme uses a polynomial basis with \(p\) functions on each cell, a regular grid of \(\Delta = \max \Delta_k\) is used and a second-order SSPRK method is used to integrate over time \citep[Sections~5.5,~5.8, and references therein]{nodalDGBook}. Applying this result to the type of positivity preserving scheme in \ref{sec: anoth pos pres}, we might expect accuracy of order \((\Delta/p)^2\) under the same regularity conditions.

% To quote Section~5.7 of \cite{nodalDGBook} SSPRK methods ``can be used with advantage for problems with strong shocks and discontinuities, as they guarantee that no additional oscillations are introduced as part of the time-integration process.''

% why the MUSCL limiter?
%   it requires no tuning and guarentees non-negativity
%   some other positivity-preserving methods require tuning and can still produce negative results if not tuned properly
%   ^ same goes for filtering
