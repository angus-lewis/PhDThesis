%!TEX root = ../thesis.tex
\chapter{Conclusion\label{ch: conclusion}} 
A fluid-fluid queue is a stochastic fluid queue, where the driving process is a fluid queue itself. Fluid queues provide a model for a single continuous performance measure of a system in the presence of a random environment. Fluid queues have found a wide variety of applications including risk processes, telecommunications, and environmental modelling, among others. Given the success of fluid queues it is plausible that the extension to fluid-fluid queues, which enable us to track two continuous performance measures of a system, will also find success. \cite{bo2014} provide an analysis of fluid-fluid queues and derive operator-analytic expressions for the first-return operator, and limiting distribution of a fluid-fluid queue. Motivated by the computation of the theoretical operators for stochastic fluid-fluid queues in \cite{bo2014}, this thesis investigates approximations method for the generator of stochastic fluid queues. Cell-based approximation methods are particularly useful as they allow flexible partitioning of the approximate operators, which is required in the analysis of fluid-fluid queues. 

In Chapter~\ref{ch:galerkin} we introduced a discontinuous Galerkin (DG) scheme for the approximation of the generator of a fluid queue and show how to approximate the operators from \cite{bo2014}. High-order DG schemes are known to perform well for smooth problems, but problems, such as negative probability estimates, can occur in the presence of discontinuities. In some contexts, slope limiting can be used to rectify this, but they result in approximations to operators which are non-linear which complicates the computation of the operators in \cite{bo2014}. The uniformisation scheme of \cite{bo2013}, which is equivalent to the simplest DG scheme, does not give erroneous approximations in the presence of discontinuities but can converge slowly compared to high-order DG schemes. 

In Chapter~\ref{sec: construction and modelling}, we introduce a new approximation to a fluid queue. Inspired by the uniformisation scheme (which is a QBD process) and its ability to handle discontinuous solutions, we construct a new discretisation of a fluid queue in the form of a QBD-RAP. Chapter~\ref{sec: construction and modelling} describes the construction of the approximation and the intuition behind it. 

In Chapters~\ref{sec: conv} and~\ref{ch: global results} we prove that the QBD-RAP approximation scheme converges weakly (in the spatial and temporal variables) to the distribution of the fluid queue. Chapters~\ref{sec: conv} uses matrix-exponential-specific arguments in its proof and ultimately proves the convergence of the QBD-RAP scheme to the fluid queue when the latter remains within a given interval. Chapter~\ref{ch: global results} uses more traditional Markov process arguments to prove a global convergence result for the approximation scheme. 

Chapter~\ref{sec: numerics} investigates, via numerical experiments, the performance of the DG, QBD-RAP, and uniformisation schemes. In some contexts, we also implement two positivity preserving DG schemes which utilise a limiter. For smooth problems the DG scheme was superior, however, for discontinuous problems, the DG scheme exhibited oscillations and produced negative probability estimates. One of the slope limited DG schemes, in which a slope limiter is applied to a high-order DG scheme (which we title the DG-lim scheme), performed very poorly due to the limiter reducing the order of the scheme to linear near discontinuities. The QBD-RAP scheme performed similarly to the linear DG approximation on a finer grid with a slope limiter (the DG-lin-lim scheme). %The QBD-RAP scheme performed slightly better on problems with point masses, and the DG-lin-lim scheme performed slightly better on the smoother problems. 
However, unlike the slope limited DG-lin-lim scheme, the QBD-RAP scheme constructs an approximation to the generator of the fluid queue which is linear and this is advantageous for the application to fluid-fluid queues. The uniformisation scheme performed better than the DG-lim scheme but worse than all the others. 

In summary, this thesis investigated three main computational frameworks for the analysis of fluid queues and fluid-fluid queues, namely, the uniformisation, DG, and QBD-RAP schemes. The DG method is well known, but its application to fluid and fluid-fluid queues has not been well-studied. The QBD-RAP scheme is a novel approach which overcomes some known issues of the DG scheme, and converges faster than the uniformisation scheme. Analysis of the QBD-RAP scheme proves that it is convergent. Numerical experiments demonstrate the effectiveness of the approximation schemes for various problems. 