\documentclass[a4paper]{article}
\usepackage{amsmath}
\title{Summary of thesis}
\author{Angus Lewis}

\begin{document}
\maketitle
The stochastic fluid-fluid model (SFFM) is a Markov process {\(\{(X_t,Y_t,\varphi_t)\}_{t\geq 0}\),} where {\(\{\varphi_t\}_{t\geq 0}\)} is a continuous-time Markov chain, the first fluid, {\(\{X_t\}_{t\geq 0}\),} is a classical stochastic fluid process driven by {\(\{\varphi_t\}_{t\geq0}\),} and the second fluid, {\(\{Y_t\}_{t\geq0}\),} is driven by the pair {\(\{(X_t,\varphi_t)\}_{t\geq 0}\).} Operator-analytic expressions for performance measures of fluid-fluid queues, such as the stationary distribution and first return distribution of the SFFM, have been derived \cite{bo2014}. These performance measures are in terms of the infinitesimal generator of the process {\(\{(X_t,\varphi_t)\}_{t\geq 0}\)}, which is a differential operator. Thus, the operator-analytic expressions do not lend themselves to direct computation. 

In this thesis we investigate approximations to the infinitesimal generator of the process {\(\{(X_t,\varphi_t)\}_{t\geq 0}\)} so that one may approximate the performance measures derived in \cite{bo2014}. As a first step we first introduce modified versions of the operators derived in \cite{bo2014} which are partitioned so that we can directly correspond elements of the true and approximated operators.

Perhaps the simplest approximation is to approximate the continuous fluid process \(\{(X_t,\varphi_t)\}_t\) by a discrete-time Markov chain. This was considered in \cite{bo2013} where they use a quasi-birth-and-death process (QBD) to approximate \(\{(X_t,\varphi_t)\}_t\). Another approach is to use a finite-element method to approximate the infinitesimal generator of \(\{(X_t,\varphi_t)\}_t\). In Chapter~2 we show how we can use the discontinuous Galerkin (DG) method to derive approximations to the infinitesimal generator of \(\{(X_t,\varphi_t)\}_t\), and then use these to approximate the operator-analytic expressions from \cite{bo2014}. When discontinuities or steep gradients are present in the distribution of \(\{(X_t,\varphi_t)\}_t\), the DG method can result in approximations to the cumulative distribution function (CDF) which oscillate and, in the worst cases, are non-monotonic and can take values outside the interval \([0,1]\). This is clearly undesirable. In contrast, the QBD approximation of \cite{bo2013} does not suffer the same problems due to its interpretation as a stochastic process, which ensures that the resulting approximations to the CDF are well-behaved. However, the rate of convergence of the QBD approximation is, in general, much slower than the DG approximation.

Motivated by this, the thesis proceeds to derive a new approximation to a fluid queue. The approximation is inspired by the observation that the QBD approximation of \cite{bo2013} effectively uses an Erlang distribution to model the sojourn time of \(\{(X_t,\varphi_t)\}_t\) in a given interval on the event that the phase of the fluid is constant. The aforementioned sojourn time is a deterministic event, and it is known that the Erlang distribution is the least-variable Phase-type distribution so, in this sense, the Erlang distribution is the best approximation to the distribution of this deterministic sojourn time. Thus, we argue that the approximation of \cite{bo2013} is the best-possible Markov chain approximation to the fluid queue \(\{(X_t,\varphi_t)\}_t\). In light of this, we look to a broader class of models known as quasi-birth-and-death-processes with rational-arrival-process components (QBD-RAPs) \cite{bn2010} to construct our approximation. QBDs have Phase-type distributed inter-event times. QBD-RAPs extend QBDs to allow \emph{matrix-exponentially} distributed inter-event times. Matrix exponential distributions have the same functional form as Phase-type distributions without the restriction that the distribution has an interpretation in terms of the absorption time of a continuous-time Markov chain. Recently, there has been much work on a class of \emph{concentrated matrix exponential distributions} \cite{hhat2020} which are postulated to be the least-variable matrix exponential distribution.  Thus, by using concentrated matrix exponential distributions, Chapter~3 construct a QBD-RAP which better captures the dynamics of the fluid queue than the QBD approximation in \cite{bo2013}. As the QBD-RAP has a stochastic interpretation then the approximations of CDFs it produces are guaranteed to be monotonic, non-decreasing and take values in \([0,1]\).

The thesis then moves on to mathematical analysis of the constructed QBD-RAP process. Chapter~4 proves a convergence of the QBD-RAP process to the fluid queue up to the time that the fluid queue leaves a given interval. Chapter~5 then uses the results of Chapter~4 to prove a global convergence results of the QBD-RAP approximation scheme to the fluid queue. Chapter~4 uses more matrix-exponential-specific arguments to show its convergence result, while Chapter~5 uses more traditional Markov process arguments, such as partitioning on the position of the process at convenient stopping times, then exploiting the strong Markov property and time homogeneity. 

Chapter~6 numerically investigates the performance of the DG, QBD-RAP and QBD approximations. The numerics demonstrate that, if the solution is smooth enough, then the discontinuous Galerkin method is highly-effective and displays rapid convergence to the solution as the number of basis functions is increased. However, when the solution is sufficiently non-smooth, oscillations and negative approximations may occur when using the discontinuous Galerkin method, leading to nonsense solutions. In these cases, the QBD-RAP approximation performs relatively well compared to the other positivity preserving schemes investigated.

\begin{thebibliography}{9}
    \bibitem{bn2010}
    Bean, N.~G.~and Nielsen, B.~F. (2010). {`Quasi-birth-and-death processes with rational arrival process components.}' \textit{Stochastic Models}, 26(3), 309-334.

    \bibitem{bo2013}
    Bean, N.~G.~and O'Reilly, M.~M. (2013). {`Spatially-coherent uniformisation of a stochastic fluid model to a quasi-birth-and-death-process.'} \textit{Performance Evaluation}, 70(9):578 -- 592.

    \bibitem{bo2014}
    Bean, N.~G.~and O'Reilly, M.~M. (2014). {`The stochastic fluid-fluid model: A stochastic fluid model driven by an uncountable-state process, which is a stochastic fluid itself.'} \textit{Stochastic Processes and their Applications}, 124:1741 -- 1772.

    \bibitem{hhat2020}
    Horv\'ath, I.~and S\'af\'ar, O.~and Telek, M.~and Zamb\'o, B. (2020). `{High order concentrated matrix-exponential distributions.}' \textit{Stochastic Models}, 36(2), 176 -- 192.
\end{thebibliography}
\end{document}