%!TEX root = ../thesis.tex
\chapter{Introduction\label{ch:intro}} 
% intro to intro
% contextual background
	% Fluid queues 
		% definition
		% applications 
		% existing results 
	% Fluid-fluid queues
		% definition 
		% basic analysis in terms of generators
		% the need for approximation and how the old work doesn't fit here
	% The DG method
		% pertition into cells
		% project on to a basis
		% problems: oscillations
		% soutions and why they dont fit
		% observation about constant-basis methods
		% this thesis asks the question: motivated by the constant basis being a probability model, can we dream up a more accurate probability model to approximate a fluid queue?
	% On the structure of the constant basis/uniformisation method (QBD)
	% least variable is PH, so what about MEs?
% structure of the thesis
	% rest of chapter 1 gives mathematical preliminaries
	% chapter 2 explains the DG method, problems/oscillations, slope limiting, and appllication to SFFMs
	% inspired by the structure of the order-1 scheme, and to solve the negativity problem, chapter 3 introduces a new approximation method; The QBD-RAP. In this chapter we derive the intuitively of the model and explain it's dynamics.
	% we then move to convergence, with chapter 4 proving a convergence of the QBD-RAP up to the first hitting time of the fluid queue on the edges of an interval. technical results and extensions are left to the appendix: certain properties of closing vectors, convergence without ephemeral phases, and certain matrix algebraic manipulations. 
	% chapter 5 stitches together the results of chapter 4 to prove a global result about convergence
	% chapter 6 investigate, numerically, the performance of the DG, DG with limiter, uniformisation and QBD-RAP schemes. 
	% chapter 7  makes concluding remarks.
% mathematical preliminaries 
	% CTMCs
	% fluid queues
	% fluid-fluid queues and operator-analytic expressions (all of them)
		% the equations of bo2014
		% meets the partition of the DG/QBD-RAP, and reconstructing the Fil sets 
    % why we cant solve the equations, what we need to approximate to solve it, i.e. B, R, then D, Psi
	% projections
	% phase-type distributions 
		% definition
		% least variable property
	% matrix exponentials
		% definition 
		% properties 
		% verification that parameters give an ME
		% CMEs
	% QBD-RAPs, then QBDs as a special case
		% orbit processes, their interpretation for PH and how they differ for MEs
	% convergence theorems 
		% DCT 
		% Continuity of Laplace transforms

		
% THE ROUGH IDEA & CONTEXT
% 	THE CONTEXT: FLUID-FLUID QUEUES
%	PDEs 
%	NON-NEGATIVITY & CONSERVATION 
% MATHEMATICAL PRELIMINARIES

A fluid queue is a two-dimensional stochastic process \(\{{\bs X}(t)\} = \{( X(t),\varphi(t))\}_{t\geq0}\). The phase process, also known as the driving process, \(\{\varphi(t)\}_{t\geq0}\), is a continuous-time Markov chain (CTMC) and determines the rate at which \(\{ X(t)\}\) moves. The level process, \(\{ X(t)\}_{t\geq0}\), is real-valued, continuous, piecewise linear and deterministic given \(\{\varphi(t)\}\). 

Typically, the phase, \(\varphi(t)\), represents the underlying operating state or environment of a system at time \(t\). The level variable, \(X(t)\), describes some continuous variable of the system. Stochastic fluid queues have found a variety of applications such as telecommunications (see \cite{anick1982}, a canonical application in this area), power systems \citep{hydro}, risk processes \citep{betal2005}, environmental modelling \citep{wurm2020}, dam storage \citep{loynes1962}, and more. For example, \cite{anick1982} model a \emph{data handling switch} in which there are \(N\) information sources which asynchronously switch from ON to OFF, or OFF to ON. The phase process \(\varphi(t)\) models the number of data sources which are ON at time \(t\), and the rate at which data is received from a single source when the buffer is on is assumed to be 1 (without loss of generality). The switch stores data in a buffer and processes the data at rate \(c\), thus the net rate of change of the buffer at time \(t\), when there are \(r\) information sources which are ON, is \(r-c\). The level process, \(X(t)\), models the amount of data in the buffer at time \(t\), and the rate at which \(X(t)\) changes when \(\varphi(t)=r\) is \(c_r=(r-c)\times 1(X(t)>0)\). The model described above could be used to aid in the analysis and design of the data handling switch, for example, by determining the minimum processing rate \(c\) for which the system is \emph{stable} in steady state. In practical terms, the minimum processing rate \(c\) determines the minimum processing power of the switch so that it is sufficient to prevent the amount of data in the buffer growing without bound. 

The success of fluid queues largely lies in their mix of flexibility, any (finite) number of phases can be used, and their parsimony and analytic tractability which enables effective transient and stationary analysis (such as the analyses \cite{ajr2005,ar2003,ar2004,bean2005b,bean2005,bot08,bean2009,dasilva2005,latouche2018,bnp2018}). Fluid queues are relatively well studied. Largely, the analysis of fluid queues falls into two categories, matrix-analytic methods e.g.~\cite{ajr2005,ar2003,ar2004,bean2005b,bean2005,bot08,bean2009,dasilva2005,latouche2018}, and differential equation-based methods \cite{anick1982,kk1995,blnos2022}. %For example, Ramaswami CITE, analysed fluid queues by mapping them to a quasi-birth-and-death process (QBD), after which they applied known matrix-analytic methods for QBDs to compute quantities of interest. Anick Mitra Sondhi CITE, analysed fluid queues using a more direct differential equation-based method. Since Ramaswami's CITE initial work, there has been significant developments in the analysis of fluid queues CITE and related algorithms CITE. 

% Of particular relevance to this thesis is the matrix-analytic analysis of fluid queues in \cite{bean2005}. Sample paths of fluid queues can be broken up into contiguous (random) intervals of time on which either, the phase remains in some subset of phases, \(\varphi(t) \in \calS_+'\) say, and the level process of the fluid queue is non-decreasing, or the phase remains in some subset of phases, \(\varphi(t) \in \calS_-'\) say, and the level process of the fluid queue is non-increasing. For \(m\in\{+,-\}\), the distribution of 
% \begin{align}
% 	&\mathbb P(X(t)\geq x, \varphi(t)=j\in\calS_m', t\leq E^\lambda \mid X(0)=x_0, \varphi(0)=i\in\calS_m') \label{eqn: LLLppL}
% \end{align}
% where \(E^\lambda\) is an exponential random variable with rate \(\lambda\) is readily computable as a Phase-type distribution. In particular, the probability (\ref{eqn: LLLppL}) is characterised in terms of the \emph{generator} of the phase process, and the rates at which the level process moves in each phase. %Moreover, taking the Laplace transform with respect to time of the distribution yields 
% \begin{align*}
% 	\mathbb P(X(t)\geq x, \varphi(u)\in\calS_+, u\in[0,t), \varphi(t)=j\in\calS_+, t\leq E^\lambda \mid \varphi(0)=i\in\calS_+)
% \end{align*}

More recently, \cite{bo2014} extend fluid queues to so-called \emph{stochastic fluid-fluid queues}. In a fluid-fluid queue there is a second level process, \(\{W(t)\}_{t\geq0}\) which is itself driven by a fluid queue, \(\{(X(t),\varphi(t))\}_{t\geq0}\), so \((X(t),\varphi(t))\) determines the rate at which \(\{W(t)\}\) moves. The addition of the second level process \(\{W(t)\}\) increases the modelling potential. For example, \cite{hydro} describe a fluid queue model for a hydroelectric power generator which can be operated `on design' and `off design'. When the system operates `off design' the system is less efficient and there is more wear, but may be used to optimise overall system performance. Further, maintenance of the generator takes place periodically to improve its performance and lifespan. The system can plausibly be modelled as a fluid queue, \(\{(\varphi(t),X(t))\}\), where the state space of \(\{\varphi(t)\}\) consists of states representing the systems states `start', `stop', `on design', `off design', `maintenance', and `idle'. The level variable \(\{X(t)\}\) take values in \([0,1]\) and represents the deterioration level of the system with 0 being perfect working order and 1 meaning the system needs replacement. The second level process \(\{W(t)\}\) could be used to model the revenue of the power generator, which, logically, should depend on the state of the system, \(\{\varphi(t)\}\). Moreover, the revenue could also depend on the deterioration level \(\{X(t)\}\), as, for example, when the system is brand new it may operate more efficiently, and when it is near broken it may be less efficient. 

% In this paper, we are interested in the following modeling potential. Suppose that, besides
% the level X (t) at time t, we would like to model some other continuous performance measure
% W (t), such as the net profit at time t for example. To illustrate this, consider a hydro-power
% generator, which can be operated “on design” and “off design”. The latter causes increased
% wear, and is less efficient, but can be useful to optimize overall system operation. The generator
% must be periodically maintained, in order to improve its performance and prolong its lifespan.
% An important related problem is the evaluation of maintenance strategies and the impact of
% [1–6,10,12–16,28,29][7,12,21]
% N.G. Bean, M.M. O’Reilly / Stochastic Processes and their Applications 124 (2014) 1741–1772 1743
% maintenance on reliability. This problem can be modeled as a bounded SFM {(φ(t), X (t)), t ≥ 0}
% with S consisting of operating phases (start, stop, on design, off design, maintenance, idle) and
% deterioration level X (t) ∈ [0, 1], where 0 means the generator is brand new and 1 means it needs
% replacement, and it is assumed that once level 1 is reached, the process {(φ(t), X (t)), t ≥ 0}
% restarts itself from level 0 [12].
% Now, in order to describe the continuous revenue process {W (t), t ≥ 0} associated with
% the process {(φ(t), X (t)), t ≥ 0}, it may be plausible to use a SFM driven by {φ(t), t ≥ 0}.
% This is because the rate at which the revenue is increasing/decreasing at time t, ideally should
% depend on the phase φ(t). However, for practical reasons, we may additionally require that the
% revenue W (t) at time t depends on the operating level X (t). For example, if the generator is
% newer, it may operate more efficiently, produce more energy and require less-costly maintenance.
% Hence, the rates at which the revenue level W (t) is increasing/decreasing may depend both on
% the phase i ∈ S and the operating level X (t) = x at time t, and so to reflect this, we denote
% these rates by ri (x). Summarizing this, we propose a new class of fluid models, denoted by
% {(φ(t), X (t), W (t)), t ≥ 0}. We refer to this class as the stochastic fluid–fluid model, since it is
% a stochastic fluid model with level W (t) driven by {(φ(t), X (t)), t ≥ 0}, which is also a stochastic
% fluid model with level X (t). The buffers collecting fluid from the process {X (t), t ≥ 0} and
% {W (t), t ≥ 0} are referred to as X and W , respectively.


The addition of the second level process \(\{W(t)\}\) increases the modelling potential, however, it comes at a cost to the simplicity of the analysis of the model. For both fluid-fluid queues and classical fluid queues, performance measures, such as the stationary distribution or the Laplace-transforms of hitting times and return times, can be derived in terms of the infinitesimal generator of the modulating process. For fluid queues the modulating process is a CTMC and the infinitesimal generator is a matrix. Thus, expressions for the performance measures of fluid queues are matrix expressions which are readily computable. For fluid-fluid queues the modulating processes is a fluid queue and the infinitesimal generator is a matrix of differential operators. Thus, expressions for the performance measures of fluid-fluid queues are matrices of differential operators which, in all but the simplest of cases, are not readily computable. This thesis investigates approximations to these differential operator expressions. 

The analysis of fluid-fluid queues in \cite{bo2014} is, in principle, similar to the matrix-analytic methods of \cite{bean2005}, and derives results about the second level process \(\{W(t)\}_{t\geq0}\) in terms of the generator of the driving fluid queue, \(\{(X(t),\varphi(t))\}_{t\geq0}\). For practical computation of the results of \cite{bo2014}, a discretisation of the infinitesimal generator of the fluid queue can be used. To this end, two discretisations have been suggested. Taking a differential equations-based approach, \cite{blnos2022} use the discontinuous Galerkin (DG) method to discretise the operator, while \cite{bo2013} take a stochastic modelling approach to approximate the fluid queue by a quasi-birth-and-death (QBD) process. The QBD approximation of \cite{bo2013} is derived via a uniformisation argument, so we refer to it as the uniformisation scheme throughout this thesis. Both approaches are insightful and offer different tools and perspectives with which to analyse the resulting approximations. It turns out that the uniformisation scheme of \cite{bo2013} is a subclass of the DG scheme; the uniformisation scheme can be viewed as the simplest DG scheme where cells all have equal width, the operator is projected onto a basis of piecewise constant functions, and an upwind flux is used to approximate the flow of mass between cells. This is also equivalent to a finite-volume scheme.

%A QBD can be viewed as a two-dimensional CTMC, \(\{(L(t),\varphi(t))\}_{t\geq0}\), where \(\{L(t)\}\) is the discrete level process, and \(\{\varphi(t)\}_{t\geq0}\) is the phase process. The level process \(\{L(t)\}\) is skip free, meaning that, given the process is at level \(L(t)=\ell\), is may only jump to \(\ell+1\) or \(\ell-1\) at jump epochs. The sojourn time of \(\{L(t)\}\) in a given level follows a phase-type distribution 

One drawback of the uniformisation scheme is that convergence can be relatively slow compared to higher-order DG schemes. However, in the context of approximating fluid queues, one advantage of the uniformisation scheme is that the uniformisation scheme guarantees probabilities computed from the approximation are positive \citep[Section~3.3]{koltai2011}. One justification for the positivity preserving property of the uniformisation scheme is from its interpretation as a stochastic process. 

For higher order DG schemes there is no such interpretation and positivity is not guaranteed \citep[Section~3.3]{koltai2011}. Higher-order DG approximation schemes may produce negative and oscillatory solutions, particularly when discontinuities or steep gradients are present. This is problematic for probabilistic problems as we know that probabilities must be positive. Methods to navigate the problem of negative and oscillatory solutions have been developed, such as filtering and slope limiting (see \cite{c99}, or \cite{nodalDGBook},~Section~5.6 and references therein). Slope limiting alters the discretised operator in regions where oscillations are detected and reduces the order of the approximation to linear in these regions but ensures the solution will be non-negative. Filtering is a post-hoc method which looks to recover an accurate solution, given an oscillatory approximation. 

Depending on the context, filtering the approximate solution to remove oscillations may not necessarily guarantee a strictly non-negative approximation, or may result in severe smearing of the solution at discontinuities or regions with steep gradients \citep[Section~5.6.1]{nodalDGBook}. Moreover, filtering requires us to make a trade-off between filtering enough of the oscillations away while maintaining sufficient accuracy -- a choice which may not be obvious \emph{a priori}. Slope limiting does guarantee positivity but reduces the approximation to linear where oscillations in the approximate solution are detected \citep[Section~5.6.1]{nodalDGBook}. Limiting and filtering do not distinguish between natural oscillations, which are a fundamental feature of the solution, and spurious oscillations, which are caused by the approximation scheme, and they may remove both from the approximation. This can lead to an unnecessary loss of accuracy in the approximation (see \citep{nodalDGBook},~Example~5.8). 

Further, slope limiting is a non-linear operator which means it cannot be represented as matrices and this hinders its application to fluid-fluid queues. %In addition, filtering and limiting also incur an additional computational cost on top of the DG scheme.
% In the context of approximating the first return operator of a fluid-fluid queue, the application of a slope limiter is not possible since we need the approximated operators to be linear. 
In the process of approximating the first return operator of a fluid-fluid queue we approximate an operator-Riccati equation with a matrix-Riccati equation by substituting in matrix approximations to the operators. The operators are related to the generator of the fluid queue and approximations to them can be constructed via the DG scheme \citep{blnos2022}. We then solve the matrix-Riccati equation via an iterative procedure \citep{bean2005b,blnos2022} and the solution is an approximation to the first-return operator of the fluid-fluid queue. Since the operators constructed with limiters are non-linear (not matrices) then we cannot apply them in this case. %As an alternative, we could apply a post-processing based on slope limiters, or filtering, but we do not consider this approach in this thesis. %Both of these post-processing methods incur a computational cost and need to be applied on a case-by-case basis depending on the initial condition. 

The approach described above relies on the intuition that substituting the DG approximation in place of the true operators in the Riccati equation is a reasonable thing to do, however there is limited theory (to the authors' knowledge) on why this approach works. In contrast, for the QBD approximation of \cite{bo2013} the approximate matrix-Riccati equation is theoretically justified and in it can be derived by considering the first-return probabilities for a fluid queue driven by the approximating QBD, where the rates of the fluid are determined by a piecewise constant approximation to \(r_i(x)\) \citep{bean2005}. 

Hence, for the approximation of fluid queues with application to fluid-fluid queues, on the one hand we can use the DG scheme which can produce high-order approximations when the problem is smooth, but can display spurious oscillations and result in negative probability estimates in the presence of discontinuities. On the other hand, we can use the uniformisation scheme for the approximation which guarantees non-negative probability estimates and has a sound theoretical justification, but convergence is relatively slower. 

Motivated by this, this thesis derives a new approximation to a fluid queue. The construction of the approximation is inspired by the observation that the Markov chain approximation of \cite{bo2013} effectively uses Erlang distributed random variables to approximate the sojourn time of the fluid level in a given interval on the event that the phase of the fluid is constant. On the event that the phase of the fluid is constant, the sojourn time in a given interval is a deterministic event. It is known that the Erlang distribution is the least-variable Phase-type distribution of a given order. In this sense, in the class of all Phase-type distributions of a given order, an Erlang-distribution is the best-possible approximation to the distribution of the deterministic sojourn time. Thus, it appears that the approximation of \cite{bo2013} is the best-possible Markov chain approximation to a fluid queue. Recently, \cite{hhat2020} have developed the class of \emph{concentrated matrix exponential distributions} which are postulated to be the least-variable matrix exponential distributions of a given order. Matrix exponential distributions generalise Phase-type distributions; they have the same functional form, without the restriction that the distribution has an interpretation in terms of the absorption time of a continuous-time Markov chain. A class of stochastic processes, known as quasi-birth-and-death-processes with rational-arrival-process components (QBD-RAPs) \citep{bn2010}, extends QBDs, which have Phase-type inter-event times, to allow matrix-exponentially distributed inter-event times. Thus, by using concentrated matrix exponential distributions, one aim of this thesis is to construct a QBD-RAP to approximate a fluid queue with the idea that it will be more accurate than the QBD approximation in \cite{bo2013}, while still retaining a stochastic interpretation which ensures positivity and aids in the application to fluid-fluid queues. 

As a QBD-RAP has a stochastic interpretation the resulting approximations are guaranteed to have non-negative density functions and give non-negative estimates of probabilities. This non-negativity is guaranteed for any initial condition without any further computation or post-processing. Further, for the application to fluid-fluid queues, the matrix-Riccati equation that we solve to approximate the first-return distribution can be derived by considering the first-return probabilities for a RAP-modulated fluid queue driven by the QBD-RAP \citep{p2019,bgnp2021} and hence it has a sound theoretical justification. In addition, the stochastic interpretation can be leveraged to aid in the analysis of the approximation. %Another attractive property is that the approximations to operators obtained via the QBD-RAP scheme are linear, which is necessary for the application to approximating the first-return operator of fluid-fluid queues and may also provide other computational and analytical benefits. 

The structure of this thesis is as follows. The next chapter, Chapter~\ref{ch:math}, is dedicated to mathematical preliminaries and introduces the main mathematical objects and tools that we will need. In particular, we go into detail describing operators arising in the analysis of fluid-fluid queues and introducing them in such a way that makes clear exactly how the approximations correspond to the theoretical operators. %Chapter~\ref{ch:math} also introduces the discontinuous Galerkin (DG) method for approximating solutions to differential equations. Due to issues with oscillations in the approximations produced by the DG scheme, at the end of Chapter~\ref{ch:math} we also describe \emph{slope limiting} -- a method which can be used to prevent oscillations and negativity. 
Chapter~\ref{ch:math} also introduces the existing literature and gives more context to this thesis.

Chapter~\ref{ch:galerkin} demonstrates a way to use the discontinuous Galerkin method to approximate certain operators and distributions of fluid and fluid-fluid queues. The DG method provides an effective tool for approximating performance measures of fluid-fluid queues. However, it can exhibit oscillations in the presence of discontinuities and these oscillations can cause approximations to certain probabilities to be negative. This motivates the next chapter.

In Chapter~\ref{sec: construction and modelling}, we develop a new QBD-RAP approximation scheme which, due to its interpretation as a stochastic process, ensures all estimates of probabilities will be non-negative. The chapter takes a stochastic modelling approach to developing the approximation scheme. 

Once we have established the new approximation scheme, we then prove its convergence in Chapters~\ref{sec: conv}~and~\ref{ch: global results}. In Chapter~\ref{sec: conv} we consider the QBD-RAP up until the first \emph{orbit restart epoch} (which is the same as a change of level, except in the case that the process hits a boundary and is reflected). Ultimately, Chapter~\ref{sec: conv} proves that, on the event that an \emph{orbit restart epoch} has not-yet occurred, certain Laplace transforms with respect to time of the QBD-RAP scheme converge to corresponding Laplace transforms with respect to time of the fluid queue. This is a local convergence result as it relates only to convergence to the fluid queue in an interval. 

Chapter~\ref{ch: global results} then extends the local convergence results of Chapter~\ref{sec: conv} to a global convergence result on the whole domain of approximation. Chapter~\ref{ch: global results} uses more traditional Markov process arguments which rely on properties such as the strong Markov property, time-homogeneity and the law of total probability. In Chapter~\ref{ch: global results} we first consider the discrete-time process which is the process embedded in the QBD-RAP at orbit restart epochs. We prove that the embedded process converges in distribution to a corresponding embedded process of the fluid queue. The rest of Chapter~\ref{ch: global results} proves a global convergence result. The main result is Theorem~\ref{thm: big thm} which states that the QBD-RAP approximation scheme converges weakly (with respect to the spatial and temporal variables) to the distribution of the fluid queue. 

Once we have established methods for approximating fluid queues, Chapter~\ref{sec: numerics} then numerically explores some properties of the approximations. Given that the QBD-RAP scheme was developed so that it guarantees positivity of the approximation, we largely focus on problems with discontinuities. Chapter~\ref{sec: numerics} can be read independently of Chapters~\ref{sec: conv}~and~\ref{ch: global results}, which are somewhat technical and could be skipped on a first-pass if the reader wishes. Although the construction of the QBD-RAP and the proof that it converges is somewhat technical, the actual computations involved in Chapter~\ref{sec: numerics} are relatively straightforward once we have the generator of the QBD-RAP and knowledge of how the level process of the QBD-RAP corresponds to the level process of the fluid queue. 

Finally, Chapter~\ref{ch: conclusion} makes concluding remarks. 

Some of the more general mathematical background has been reserved for Appendix~\ref{app: math background}. To help the reader understand the notation used in the DG scheme, we provide a small toy model example in Appendix~\ref{appendix:example}. Appendix~\ref{sec:properties} provides a proof that the DG scheme conserves probability. The rest of the appendices contain technical results relating to proving the convergence of the QBD-RAP approximation scheme. Appendix~\ref{appendix: sec: 2} proves that the \emph{closing operators} introduced as part of the QBD-RAP scheme have the properties we claim they do when we are proving convergence in Chapter~\ref{sec: conv}. Appendix~\ref{app:extend conv} extends some results from Chapter~\ref{sec: conv} to a setting which requires slightly less computation. Lastly, Appendix~\ref{appendix: kronecker} provides some algebraic results which help us to manipulate certain Laplace transforms from Chapter~\ref{sec: conv}.
