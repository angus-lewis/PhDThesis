%!TEX root = ../thesis.tex
\chapter{Introduction \label{ch:intro}}
A fluid queue is a two-dimensional stochastic process \(\{(X(t),\varphi(t))\}_{t\geq0}\). The phase process, also known as the driving process, \(\{\varphi(t)\}_{t\geq0}\), is a continuous-time Markov chain (CTMC). The level process, \(\{X(t)\}_{t\geq0}\), is a real-valued, continuous, and piecewise linear. 

Stochastic fluid queues have found a variety of applications such as telecommunications (see \cite{anick1982} as a canonical application in this area), power systems \cite{hydro}, risk processes \cite{betal2005} and environmental modelling \cite{wurm2020}. Fluid queues are relatively well studied. Largely, the analysis of fluid queues falls into two categories, matrix-analytic methods \cite{ajr2005,ar2003,ar2004,bean2005b,bean2005,bot08,bean2009,dasilva2005,latouche2018}, and differential equation-based methods \cite{anick1982,kk1995,beanetal2019}. %For example, Ramaswami CITE, analysed fluid queues by mapping them to a quasi-birth-and-death process (QBD), after which they applied known matrix-analytic methods for QBDs to compute quantities of interest. Anick Mitra Sondhi CITE, analysed fluid queues using a more direct differential equation-based method. Since Ramaswami's CITE initial work, there has been significant developments in the analysis of fluid queues CITE and related algorithms CITE. 

More recently, Bean and O'Reilly \cite{bo2014} extended fluid queues to \emph{so-called} stochastic fluid-fluid queues. In a fluid-fluid queue there is a second level process, \(\{Y(t)\}_{t\geq0}\) which is itself driven by a fluid queue, \(\{(X(t),\varphi(t))\}_{t\geq0}\). The analysis, Bean and O'Reilly \cite{bo2014}, is in principal similar to the matrix-analytic methods of \cite{bean2005}, and derives results about the second level process \(\{Y(t)\}_{t\geq0}\) in terms of the infinitesimal generator (a differential operator) of the fluid queue, \(\{(X(t),\varphi(t))\}_{t\geq0}\). For practical computation of the results \cite{bo2014} a matrix-discretisation of the infinitesimal generator of the fluid queue can be used. To this end, to date, two possible discretisation have been suggested. Taking a differential equations-based approach, Bean \emph{et al.}~\cite{beanetal2019} use the discontinuous Galerkin (DG) method to discretise this operator, while Bean and O'Reilly \cite{bo2013} take a stochastic modelling and matrix-analytic methods approach to approximate the fluid-queue by a quasi-birth-and-death (QBD) process. Both approaches are insightful and offer different tools and perspectives with which to analyse the resulting approximations. It turns out that the latter approach is a sub-class of the former; the QBD can be viewed as the simplest DG scheme where the operator is projected onto a basis of piecewise constant functions.

%A QBD can be viewed as a two-dimensional CTMC, \(\{(L(t),\varphi(t))\}_{t\geq0}\), where \(\{L(t)\}\) is the discrete level process, and \(\{\varphi(t)\}_{t\geq0}\) is the phase process. The level process \(\{L(t)\}\) is skip free, meaning that, given the process is at level \(L(t)=\ell\), is may only jump to \(\ell+1\) or \(\ell-1\) at jump epochs. The sojourn time of \(\{L(t)\}\) in a given level follows a phase-type distribution 

In the context of approximating fluid queues, one advantage of the QBD discretisation and, equivalently, DG schemes with constant basis functions, is that they guarantee probabilities computed from the approximation are positive \cite[Section 3.3]{koltai2011}, see also \cite{nodalDGBook} and references therein. One justification for the positivity preserving property is from the interpretation of the discretisation as a stochastic process thereby ensuring positivity. For higher order DG schemes there is no such interpretation. Moreover, higher-order DG approximation schemes may produce negative, or highly oscillatory solutions, particularly when discontinuities or steep gradients are present. Methods to navigate the problem of negative or highly oscillatory solutions have been developed, such as slope limiters, and filtering (see \cite[Section 6.5]{nodalDGBook} and references therein). Slope limiting effectively alters the discretised operator in regions where oscillations are detected and lowers the order of the approximation in theses regions. Filtering is a post-hoc method which looks to recover an accurate solution, given an oscillatory approximation. 

\subsection{Fluid queues}
A fluid queue is a two-dimensional stochastic process \(\{(X(t),\varphi(t))\}_{t\geq0}\) where \(\{\varphi(t)\}_{t\geq0}\) is known as the phase or driving process, and \(\{X(t)\}_{t\geq0}\) is known as the level process or buffer. The phase process \(\{\varphi(t)\}_{t\geq0}\), is an irreducible continuous-time Markov chain (CTMC) with finite state space, which we we assume to be \(\mathcal S=\{1,2,\dots,N\}\) without loss of generality, and infinitesimal generator \(\bs T= [T_{ij}]_{i,j\in\mathcal S}\). We assume that \(\bs T\) is \emph{conservative}. Associated with states \(i\in\mathcal S\) are real-valued \emph{rates} \(c_i\in\mathbb R\). 

Partition the state space \(\calS\) into \(\calS_+ = \{i\in\calS\mid c_i>0\}\), \(\calS_- = \{\i\in\calS\mid c_i<0\}\) and \(\calS_0 = \{i\in\calS\mid c_i=0\}\). We assume, without loss of generality, that the generator \(\bs T\) is partitioned into sub-matrices 
\[\bs T = \left[\begin{array}{ccc}\bs T_{++} & \bs T_{+-} & \bs T_{+0} \\ \bs T_{-+} & \bs T_{--} & \bs T_{-0} \\ \bs T_{0+} & \bs T_{0-} & \bs T_{00}  \end{array}\right],\]
where \(\bs T_{mn} = [T_{ij}]_{i\in\mathcal S_m, j\in\mathcal S_n}\), \(m,n\in\{+,-,0\}\). Also define the diagonal matrices 
\begin{align*}
	\bs C &= \left[\begin{array}{ccc} \bs C_+ && \\ &\bs C_-& \\ && \bs 0\end{array}\right], && \bs C_+ = diag(c_i,i\in\calS_+), && \bs C_- = diag(|c_i|,i\in\calS_-),
\end{align*}
where \(diag(a_i,i\in\mathcal I)\) denotes a diagonal matrix with entries \(a_i\) down the diagonal. 

When no boundary conditions are imposed, the level process is given by 
\[X(t) = X(0) + \int_{s=0}^t c_{\varphi(s)}\wrt s.\]
Sample paths of \(\{X(t)\}\) are continuous and piecewise linear, with \(\cfrac{\wrt }{\wrt t} X(t) = c_\varphi(t)\), when \(X(t)\) is differentiable. Given sample paths of \(\{\varphi(t)\}\), then \(\{X(t)\}\) is deterministic, and in this sense, \(\{\varphi(t)\}\) is the only stochastic element of the fluid queue. 

Often, boundary conditions are imposed. Here, we consider a mixture of \emph{regulated} and \emph{reflecting} boundary conditions. Upon hitting a boundary we suppose that, with probability \(p_{ij},\,i,j\in\mathcal S\), the phase process instantaneously transitions from phase \(i\) to phase \(j\) (note that we might have \(i=j\) i.e.~no transition). At a lower boundary, if \(j\in\calS_0\cup\calS_-\), then \(\cfrac{\wrt}{\wrt t} X(t) = 0\), and the phase process continues to evolve according to the sub-generator 
\[\left[\begin{array}{cc} \bs T_{--} & \bs T_{-0} \\ \bs T_{0-} & \bs T_{00}  \end{array}\right],\]
until such a time that \(\varphi(t)\) transitions to a phase \(k\in\calS_+\), at which time \(X(t)\) leaves the boundary. Similarly, at an upper boundary if \(j\in\calS_0\cup\calS_+\), then \(\cfrac{\wrt}{\wrt t} X(t) = 0\) and the phase process continues to evolve according to the sub-generator 
\[\left[\begin{array}{cc} \bs T_{++} & \bs T_{+0} \\ \bs T_{0+} & \bs T_{00}  \end{array}\right],\]
until such a time that \(\varphi(t)\) transitions to a phase \(k\in\calS_-\) at which time \(X(t)\) leaves the boundary. Without loss of generality, we assume the lower and upper boundaries (when present) are at \(x=0\) and \(x=M>0\), respectively.

In summary, the evolution of the level can be expressed as 
\[\cfrac{\wrt}{\wrt t} X(t) = \begin{cases} c_{\varphi(t)}, & \mbox{ if } X(t)>0, \\ \max\{0,c_{\varphi(t)}\}, & \mbox{ if } X(t)=0, \\ \min\{0,c_{\varphi(t)}\}, & \mbox{ if } X(t)=M.  \end{cases}\]

Let \(\bs f(x,t) = \vligne{f_i(x,t)}_{i\in\calS}\) be a row-vector function where \(f_i(x,t)\) is the density of \(\mathbb P(X(t)\leq x, \varphi(t) = i)\), assuming it exists. When a differentiable density exists, the system of partial differential equation which describes the evolution of the densities \(\bs f(x,t)\) is 
\begin{equation}
	\cfrac{\partial}{\partial t} \bs f(x,t) = \bs f(x,t)\bs T - \cfrac{\partial}{\partial x}\bs f(x,t)\bs C.\label{eqn: pde}
\end{equation}
The initial condition is the initial distribution of the fluid queue, \(f_i(x,0)\). Often a differentiable density function does not exist and therefore the partial differential equation (\ref{eqn: pde}) is not well-defined. For example, for a fluid queue with a regulated boundary, if the initial distribution of the fluid queue is a point mass at any point \(x_0\geq 0\) and in phase \(i\in\calS_+\cup\calS_0\), then a density function \(f_i(x,t)\) will not exist for any finite \(t\). Specifically, a point mass will persist along the ray \(x_0+c_it\), \(t\geq 0\). In such situations, it is the \emph{weak solution} to (\ref{eqn: pde}) that we seek. 

Boundary conditions may also be imposed on (\ref{eqn: pde}). 

Discretisation methods approximate the operator on the right-hand side of (\ref{eqn: pde}) by a matrix, in our case, by the generator of a QBD-RAP. 
