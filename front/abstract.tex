%!TEX root = ../thesis.tex
\chapter{Abstract}
\label{ch:abstract}
A fluid-fluid queue is a stochastic fluid queue, where the driving process is a fluid queue itself. Fluid queues provide a model for a single continuous performance measure of a system in the presence of a random environment. Fluid queues have found a wide variety of applications including risk processes, telecommunications, and environmental modelling, among others. Given the success of fluid queues it is believed that the extension to fluid-fluid queues, which enable us to track two continuous performance measures of a system, will also find success. \cite{bo2014} provide an analysis of fluid-fluid queues and derive operator-analytic expressions for the first-return operator, and stationary distribution of a fluid-fluid queue.

This thesis provides approximations to fluid queues so that we can approximate the operators in \cite{bo2014}. It investigates three main approximation schemes; the DG scheme (Chapter~\ref{ch:galerkin}) which is a popular finite-element scheme, the uniformisation scheme of \cite{bo2013} which approximates a fluid queue by a continuous-time Markov chain (specifically, a quasi-birth-and-death-process (QBD)), and the QBD-RAP scheme which is a generalisation of a QBD to allow matrix exponential inter event times. The QBD-RAP scheme is novel; we describe the construction of the scheme in Chapter~\ref{sec: construction and modelling}, and provide an analysis to show that it is convergent in Chapters~\ref{sec: conv} and \ref{ch: global results}. We demonstrate the effectiveness of the approximation schemes in Chapter~\ref{sec: numerics}, focussing on problems with discontinuous solutions. In general, we find that the DG scheme performs remarkably well for smooth problems, but can produce oscillations and negative probability estimates in the presence of discontinuities, the QBD-RAP approximation performs well in the presence of discontinuities, but does not perform as well as the DG scheme for smooth problems, the uniformisation scheme produces reliable approximations in the presence of discontinuities, but converges slowest for all problems. 


