%!TEX root = ../thesis.tex
\chapter{Declaration}
\label{ch:declaration}

This thesis contains work taken from the publication \cite{blnos2022} which approximates the operator analytic expressions of \cite{bo2014} via the DG method. I am a co-author on this paper. The main conceptualisation and initial writing of the manuscript was done by Vikram Sunkara, Giang Nguyen and Nigel Bean. I extended the manuscript in the following ways.
\begin{itemize}
    \item Significant additions to Section~2 to introduce the partition induced by the approximation scheme to the theoretical operators of \cite{bo2014} to help elucidate the connection between the theoretical operators and the approximation counterparts. 
    \item The addition of Section~4.4 which introduces exact boundary dynamics. Prior to this boundary conditions were approximated by a small interval in the DG scheme (in this thesis, I generalised the boundary conditions further).
    \item The addition and proof of Corollary~4.1.
    \item The formal definition and derivation of the approximation to the operator \(\mathbb R\) in Section~5.1. 
    \item Introduction of the concept of operator kernels to the paper (used throughout Section~5.5).
    \item The final version of the code (Vikram did the initial version).
    \item Extension of the numerical experiments to consider higher-order bases, smaller cell-widths, elucidating issues around truncation errors, and computing confidence intervals. 
    \item Addition of Appendix~A on the properties of the DG approximation to the generator of the fluid queue.
    \item Additional details in Appendix~B. 
    \item I also made significant contributions to the writing (and rewriting) the manuscript, submission, and responding to reviewers' comments. 
\end{itemize}
To clarify which parts of this thesis are related to the paper \cite{blnos2022}, the following sections were taken, almost verbatim except for minor editing, notational changes so that the notation matches that of the thesis, and the addition more general boundary conditions,
\begin{itemize}
    \item Section~\ref{sec: ffq intro},
    \item Chapter~\ref{ch:galerkin}, except for Sections~\ref{sec: other applications} and \ref{sec: limiting and linearity},
    \item Appendix~\ref{appendix:example},
    \item Appendix~\ref{sec:properties}.
\end{itemize}